\chapter{Part One}

\section{I} %% Book One, Part One, Chapter I

``\textit{Eh bien, mon prince. G\^enes et Lucques ne sont plus des
  apanages, des} estates, \textit{de la famille Buonaparte. Non, je
  vous pr\'eviens, que si vous ne me dites pas, que nous avons la
  guerre, si vous vous permettez encore de pallier toutes les
  infamies, toutes les atrocit\'es de cet Antichrist (ma parole, j'y
  crois) --- je ne vous connais plus, vous n'\^etes plus mon ami, vous
  n'\^etes plus} my devoted slave, \textit{comme vous
  dites.}\footnote{Well, my prince. Genoa and Lucca are no more than
  possessions, estates, of the Buonaparte family. No, I'm warning you,
  if you do not tell me we are going to war, if you again permit
  yourself to excuse all the infamies, all the atrocities of that
  Antichrist (I truly believe that is what he is) --- then I don't
  know you any longer, you are no longer my friend, you are are no
  longer my devoted slave, as you say.} Well, hello, hello. \textit{Je
  vois que je vous fair peur,}\footnote{I see that I've frightened
  you.} sit down and tell me everything.''

So spoke\todo{In this way began...?}\ the well-known Anna Pavlovna
Scherer, maid of honor and confidante of Empress Maria Feodorovna, in
July 1805, upon greeting the important and
high-ranking\todo{chinovnogo}\ Prince Vasily, who was the first to
arrive at her reception. Anna Pavlovna had had a cough for several
days, she had the \textit{grippe}, as she said (\textit{grippe} was
then a new word, used only by rarified people). In notes sent out that
morning, carried by a footman in red, she had written to everyone
without distinction:

``\textit{Si vous n'avez rien de mieux \`a faire, M.~le comte} (or
\textit{mon prince}), \textit{et si la perspective de passer la
  soir\'ee chez une pauvre malade ne vous effraye pas trop, je serai
  charm\'ee de vous voir chez moi entre 7 et 10 heures. Annette
  Scherer.}''\footnote{If you have nothing better to do, Count (or
  Prince), if the thought of passing the evening with a poor sick
  woman does not frighten you too much, I would be charmed to see you
  between 7 and 10. Annette Scherer.}

\textit{``Dieu, quelle virulente sortie!''}\footnote{My god, what a
  vicious attack!} answered the prince upon entering, not at all
rattled by this greeting. He was wearing a tailored dress uniform,
stockings, boots, and stars, with a bright expression on his plain
face.

He spoke that refined French in which our forefathers not only spoke,
but thought, and with the quiet, patronizing\todo{?}\ intonations
peculiar to a significant person who has seen many years in society
and at court. He approached Anna Pavlovna, kissed her hand, presented
his bald head to her, scented\todo{?}\ and gleaming, and calmly sat on
the sofa.

``\textit{Avant tout dites moi, comment vous allez, ch\`ere
  amie?}\footnote{First tell me, how are you, my dear friend?} Set my
mind at ease,'' he said, not changing his voice and tone, in which
decorum and affected sympathy barely covered indifference and even
mockery.

``How can one feel healthy\ldots\ when one is suffering morally? Do
you really think it's possible to remain calm in our time?'' Anna
Pavlovna said. ``You'll be here for the whole reception, I hope?''

``And the English ambassador's party? Today is Wednesday. I must put
in an appearance there,'' the prince said. ``My daughter is coming for
me, and we'll go there together.''

``I thought that today's party was canceled. \textit{Je vous avoue que
  toutes ces f\^etes et tous ces feux d'artifice commencent \`a
  devenir insipides.}''\footnote{I must confess, all these parties and
  all these fireworks are starting to become dull.}

``If they had known you wanted it, they would have canceled the
party,'' the prince said, saying by habit, like a wound clock, things
he did not even want to be believed.

\textit{``Ne tourmentez pas. Eh bien, qu'a-t-on d\'ecid\'e par rapport
  \`a la d\'ep\^eche de Novosilzoff? Vous savez
  tout.''}\footnote{Don't torment me. Well, what was decided in
  regards to Novosiltsov's dispatch? You know everything.}

``What can I tell you?'' the prince said in a cold, bored
tone. \textit{``Qu'a-t-on d\'ecid\'e? On a d\'ecid\'e que Buonaparte a
  br\^ul\'e ses vaisseaux, et je crois que nous sommes en train de
  br\^uler les notres.''}\footnote{What was decided? It was decided
  that Buonaparte has burned his ships, and I believe that we are
  about to burn ours.}

Prince Vasily always spoke lazily, like an actor playing his role in
an old play. Anna Pavlovna Scherer, on the contrary, despite her forty
years, was absolutely filled with liveliness and activity.

To be an enthusiast had become her position in society, and sometimes,
even when she did not want to herself, she played the enthusiast to
satisfy the expectations of people who knew her. The restrained smile
playing constantly on Anna Pavlovna's face, although it did not fit
her aging features, expressed a constant awareness, like that of a
spoiled child's, of this endearing weakness, which she did not want,
could not, and did not find it necessary to overcome.\todo{fix.}

In the midst of their conversation about political matters, Anna
Pavlovna became heated.

``Oh, don't talk to me about Austria! Perhaps I don't understand
anything, but Austria never wanted and does not want war. They will
betray us. Russia alone should save Europe. Our benefactor knows his
high calling and will be true to it. That is one thing I believe for
certain. The greatest role in the world awaits our kind and glorious
sovereign, and he is so virtuous and good that God will not abandon
him, and he will fulfill his calling to crush the hydra of revolution,
which is even more terrible now in the face of that murderer and
villain. We alone must avenge the blood of that just man. Who else can
we rely on, I ask you\ldots{}? England with its commercial spirit
won't comprehend and can't comprehend the heights of Emperor
Aleksandr's soul. They refused to clear\todo{?}\ Malta. They want to
wait and see, they're searching for an ulterior motive to our
actions. What did they say to Novosiltsov\ldots{}?  Nothing. They
don't comprehend, they can't comprehend our emperor's selflessness. He
wants nothing for himself and wants everything for the good of the
world. And what have they promised? Nothing. And what they've promised
won't happen! Prussia already decided that Bonaparte is invincible and
that Europe can do nothing against him\ldots{} And I don't believe one
word from Hardenberg or Haugwitz. \textit{Cette fameuse neutralit\'e
  prussienne, ce n'est qu'un pi\`ege.}\footnote{That famous Prussian
  neutrality is nothing but a trap.} I believe in God alone and in the
lofty fate of our dear emperor. He will save Europe\ldots{}!'' She
stopped suddenly, smiling in self-mockery at her vehemence.

``I think,'' the prince said, smiling, ``that if they had sent you
instead of our dear Wintzingerode, you would have taken the Prussian
king's consent in a stroke. You are so eloquent. Would you give me
some tea?''

``In a moment. \textit{A propos,}'' she added, again smiling, ``today
I'm having two very interesting people over, \textit{le vicomte de
  Mortemart, il est alli\'e aux Montmorency par les
  Rohans,}\footnote{The viscount of Mortemart, he is related to
  Montmorency through the Rohans.} one of the best families of
France. He is one of the good emigrants, one of the true ones. And
then \textit{l'abb\'e Morio:}\footnote{Father Morio.} do you know that
profound mind? He was received by the sovereign. Do you know him?''

``Ah! I'll be glad to meet him,'' the prince said. ``Tell me,'' he
added, as if just remembering something, entirely offhandedly, when in
fact what he was asking about was the main reason for his visit, ``is
it true that \textit{l'imp\'eratrice-m\`ere} wants Baron Funke named
first secretary in Vienna? \textit{C'est un pauvre sire, ce baron, \`a
  ce qu'il para\^it.}''\footnote{The empress-mother\ldots\ He is a
  poor creature, that baron, so it seems.} Prince Vasily had hoped to
have his son appointed to the position, which others were trying to
obtain for the baron through Empress Maria Feodorovna. 

Anna Pavlovna half-closed her eyes to show that neither she, nor
anyone else, could judge what the empress wanted or found pleasing.

\textit{``Monsieur le baron de Funke a \'et\'e recommand\'e \`a
  l'imp\'eratrice-m\`ere par sa s\oe ur,''} she said with a sad, dry
tone.\footnote{Baron Funke was recommended to the empress-mother by
  her sister.} At the same time that Anna Pavlovna named the empress,
her face suddenly took on a deep and earnest expression of devotion
and high esteem, coupled with sadness, as it did every time she
referred to her great patroness in conversation. She said that Her
Majesty was pleased to show Baron Funke \textit{beaucoup d'estime} and
her gaze once again showed a twitch of sadness.\footnote{Much
  respect.}

The prince, indifferent, fell quiet. Anna Pavlovna, with her feminine
cunning and quick rhythm, honed at court, wanted to chasten the prince
for having the impudence to speak his mind about a person who had been
recommended to the empress, and at the same time to console him.

\textit{``Mais \`a propos de votre famille,''} she said, ``you must
know that your daughter, since she has come out, \textit{fait les
  d\'elices de tout le monde. On la trouve belle, comme le
  jour.}''\footnote{But about your own family\ldots\ gives delight to
  everyone. People find her as beautiful as the day.}

The prince bowed to show his esteem and gratitude.

``I often think,'' Anna Pavlovna continued after a minute's silence,
moving closer to the prince and smiling tenderly at him, as if to show
him that the political and society conversations were over, and she
now spoke from the heart, ``I often think about how unfairly happiness
in life is distributed. Why should fate have given you two fine
children (excluding Anatoly, your youngest, I don't like him),'' she
added peremptorily, raising her brows, ``such delightful children? And
honestly, you value them less than anyone and don't deserve them
because of it.''

And she smiled her ecstatic smile.

\textit{``Que voulez-vous? Lafater aurait dit que je n'ai pas la bosse
  de la paternit\'e,''} the prince said.\footnote{What can I do?
  Lafater would say I lack the paternal bump.}

``Enough joking. I wanted to talk seriously with you. You know, I'm
dissatisfied with your youngest son. Just between us,'' her face took
on a sad expression, ``they were talking about him at Her Majesty's,
and she pities you.''

The prince did not answer, but she remained silent, giving him a
significant look, waiting for an answer. Prince Vasily winced.

``What can I do?'' he said finally. ``You know that I've done
everything a father can do to bring them up well, and both boys came
out \textit{des imb\'eciles}.\footnote{Idiots.} Ippolit, at least, is
a quiet idiot, but Anatoly is an unquiet one. That's the whole
difference,'' he said, more unnaturally and animatedly than usual, at
the same time sharply revealing in the lines forming around his mouth
something unexpectedly coarse and unpleasant.

``And why are children born to people like you? If you were not a
father, you would be beyond reproach,'' Anna Pavlovna said, raising
her eyes thoughtfully.

``\textit{Je suis votre} devoted slave, \textit{et \`a vous seule je
  puis l'avouer.} My children --- \textit{ce sont les entraves de mon
  existence.} It is my cross. Or so I explain it to
myself. \textit{Que voulez vous\ldots{}}?'' \footnote{I am your
  devoted slave, and I can confess this to you alone. My children ---
  they are my life's chains\ldots\ What can I do?}

Anna Pavlovna thought for a moment.

``You have never thought about finding a wife for your prodigal son
Anatoly. They say,'' she said, ``that old maids \textit{ont la manie
  des mariages}. I still haven't felt that weakness in myself, but I
have a \textit{petite personne} who is very unhappy with her father,
\textit{une parente \`a nous, une princesse}
Bolkonskaya.''\footnote{Have the habit of making marriages\ldots{}
  Girl\ldots{} A relative of mine, a princess Bolkonskaya\ldots}
Prince Vasily did not answer, although with a quickness of
consideration and memory peculiar to society people he showed with a
nod of the head that he had taken this information into consideration.

``No, do you know Anatole costs me 40,000 a year,'' he said, obviously
unable to stop the doleful course of his thoughts. He fell silent.

``What will happen after five years if it keeps going on like this?
\textit{Voil\`a l'avantage d'\^etre p\`ere.}\footnote{There you have
  the benefit of being a father.} Is she rich, your
princess?''\todo{If what keeps going on like this??}

``The father is very rich and stingy. He lives in the country. You
know, he's the famous Prince Bolkonsky, who was dismissed by the late
emperor and was called the `Prussian king.' He's a very intelligent
man, but peculiar and severe. \textit{La pauvre petite est
  malheureuse, comme les pierres.}\footnote{The poor girl is miserable
  as sin.} She has a brother, he was just married to Lise Meinen and
is one of Kutuzov's adjutants. He will be here tonight.''

\textit{``Ecoutez, ch\`ere Annette,''} the prince said, suddenly
taking his interlocutor by the hand and for some reason pulling down
on it, ``\textit{Arrangez-moi cette affaire et je suis votre} most
devoted slave \textit{\`a tout jamais}\ldots{} `slev,' \textit{comme
  mon} steward \textit{m'\'ecrit des} dispatches:
ee-vee.\footnote{Arrange this matter for me and I will be your most
  devoted slave forever\ldots{} `Slev,' as my steward writes to me in
  his dispatches.} She's from a good family and rich. That's all that
I need.''
% Explaining "ee-vee" / "pokoj-yer-p": http://languagehat.com/linguistic-tolstoy/

And with the same free and familiar, gracious movements that set him
apart, he took the maid of honor's hand, kissed it, and, having
kissed it, dropped the maid of honor's hand and collapsed on the
cushions and looked to the side.

\textit{``Attendez,''} Anna Pavlovna said, thinking for a
moment. ``Tonight I'll talk to Lise (\textit{la femme du jeune}
Bolkonsky). And perhaps it can be settled. \textit{Ce sera dans votre
  famille, que je ferai mon apprentissage de vielle
  fille.}''\footnote{Young Bolkonsky's wife\ldots{} I can use your
  family as my apprenticeship to become an old maid.}

\section{II} %% Book One, Part One, Chapter II

Anna Pavlovna's drawing room began to fill a little. The high
aristocracy of Petersburg was arriving, people entirely different from
one another by age and character, but identical in the stratum of
society they inhabited; Prince Vasily's daughter arrived, the great
beauty Helene, coming to get her father for the ambassador's
party. She wore a ball gown and the monogram that marked her as a maid
of honor. Arriving at the same time was \textit{la femme la plus
  s\'eduisante de P\'etersbourg}, the little princess Bolkonskaya, who
was married the previous winter and no longer went out in `great'
society because she was pregnant, but who still attended smaller
receptions.\footnote{The most captivating woman in Petersburg.} Prince
Ippolit, Prince Vasily's son, arrived with Mortemart, who was his
guest; Father Morio and others also arrived.

``Have you seen\ldots'' or ``Have you been introduced to \textit{ma
  tante}?'' Anna Pavlovna said to the arriving guests and in complete
earnest guided them to a little old woman in large bows\todo{?}\ who
had surfaced from another room as soon as the guests started arriving,
told her their names, slowly moving her eyes from the guest to
\textit{ma tante}, and then walked away.\footnote{My aunt.}

All the guests went through this rite of greeting her aunt, who was
unknown to everyone, interesting to no one, and totally
unnecessary. Anna Pavlovna watched their greetings with sad,\todo{?}\
solemn interest, quietly approving them. \textit{Ma tante} spoke to
each guest in the same expressions about her health and the health of
Her Majesty, which was now, thank God, much better. Each person, upon
approaching her, tried not to hurry out for propriety's sake, then
walked away with a feeling of relief after fulfilling a heavy
obligation, never returning to her for the rest of the evening.

The young Princess Bolkonskaya arrived with her work\todo{?}\ in a
velvet bag stitched in gold. Her upper lip, pretty, with slightly dark
little whiskers, did not quite cover her teeth, but was more endearing
because of it when it opened and even more endearing when it stretched
to meet her lower one. As is often the case with alluring women, her
flaw --- the too-short lip and half-open mouth --- seemed to complete
her particular, entirely unique beauty. Everyone enjoyed looking at
this pretty mother-to-be, full of health and life, so easily enduring
her condition. Old men and bored, somber young men felt themselves
becoming more like her just by sitting and talking with her for a
little while. Whoever talked with her and saw the bright little smile
and gleaming white teeth she showed constantly, with every word,
thought he must be especially amiable today. And everyone one of them
thought the same thing.

The little princess, waddling, walked around the table in short, quick
steps with her work bag on her arm and, happily straightening her
dress, sat on the sofa near the silver samovar, as if everything she
did was \textit{partie de plaisir} for her and everyone around
her.\footnote{A little treat.}

\textit{``J'ai apport\'e mon ouvrage,''} she said, opening her handbag
and addressing everyone at once.\footnote{I brought my work.}

``See here, \textit{Annette, ne me jouez pas un mauvais tour},'' she
said to their hostess. \textit{``Vous m'avez \'ecrit, que c'\'etait
  une toute petite soir\'ee; voyez, comme je suis
  attif\'ee.''}\footnote{Annette, I hope you're not playing a dirty
  trick on me. You wrote to me that this would be an entirely small
  reception; you see how I'm gotten up.}

And she opened her arms to show her refined grey dress, all in lace,
belted with a wide ribbon a little below the bust.

\textit{``Soyez tranquille, Lise, vous serez toujours la plus
  jolie,''} Anna Pavlovna answered.\footnote{Don't worry, Lise, you
  will always be the prettiest.}

\textit{``Vous savez, mon mari m'abandonne,''} she continued in the
same tone, turning to a general, \textit{``il va se faire tuer. Dites
  moi, pourquoi cette vilaine guerre,''} she said to Prince Vasily
and, not waiting for an answer, turned to Prince Vasily's daughter,
the beautiful Helene.\footnote{You know, my husband is abandoning
  me. He's going to get himself killed. Tell me, why are they fighting
  this nasty war?}

\textit{``Quelle d\'elicieuse personne, que cette petite princesse!''}
Prince Vasily said quietly to Anna Pavlovna.\footnote{What a delicious
  person that little princess is!}

Soon after the little princess arrived, a massive, heavy young man
with short-clipped hair entered, wearing glasses, light-colored pants
in the then-current fashion, a high jabot, and a brown tail-coat. This
heavy young man was the illegitimate son of the celebrated Catherinian
grandee, Count Bezukhov, who was then dying in Moscow. He had still
not served anywhere, having just arrived from abroad, where he was
brought up, and this was his first time out in society. Anna Pavlovna
greeted him with a bow she reserved for people of the lowest hierarchy
in her salon. But notwithstanding this lowest kind of greeting, upon
Pierre's appearance, Anna Pavlovna's face expressed anxiety and fear
of the kind usually felt at the appearance of something huge and out
of place. Although Pierre truly was a little larger than the other men
in the room, this fear could only be ascribed to the glance, at once
intelligent and at the same time shy, observant and natural, that
distinguished him from everyone else in the drawing room.

\textit{``C'est bien aimable \`a vous,} monsieur Pierre,
\textit{d'\^etre venu voir une pauvre malade,''} Anna Pavlovna said to
him, exchanging fearful glances with her aunt, to whom she guided
him.\footnote{It is so kind of you, Mr.~Pierre, to come see a poor
  sick woman.} Pierre mumbled something unintellible and continued to
search the room with his eyes. He smiled joyfully, happily at the
little princess as to\todo{?}\ a close friend and approached the
aunt. Anna Pavlovna's fear was not in vain, because Pierre walked away
from the aunt without listening to all of her speech about Her
Majesty's health. Anna Pavlovna, fearful,\todo{Really?}\ stopped him
with the words:

  ``Do you know Father Morio? He is a very interesting person\ldots{}''
  she said.

``Yes, I've heard about his plan for eternal peace, and it is very
interesting, though hardly possible\ldots{}''

``Oh, you think so?'' Anna Pavlovna said, just in order to say
something and again return to her duties as hostess, but Pierre was
impolite in reverse. Earlier he had left without listening to his
interlocutor's words to the end; now, through his own words, he
stopped an interlocutor\todo{hate this word}\ who needed to leave
him. Bending his head and setting apart his large feet, he began to
prove to Anna Pavlovna why he believed the priest's plan was a chimera.

``We will talk later,'' Anna Pavlovna said, smiling.

And, getting rid of the young man who did not know how life was lived,
she returned to her duties as hostess and continued to listen in and
look in, ready to help wherever the conversation lulled. Like the
owner of a textile factory, who, having assigned his workers their
places, walks around the place, noting idleness or the unaccustomed,
scraping, too-loud noise of a spindle needing attention, moving
hastily, puts it back in the right track\todo{?}\ --- so Anna
Pavlovna, walking around her drawing-room, would approach a circle
that had fallen silent or was talking too much and, with a single word
or the movement of someone from one circle to another, would get the
conversational machine running evenly and properly again. But in the
midst of her worries, one could still see a particular fear for
Pierre. She carefully watched him as he approached Mortemart to see
what was said around him, and then walked away to another circle,
where the priest was speaking. For Pierre, who had been raised abroad,
this reception of Anna Pavlovna's was the first he had seen in
Russia. He knew that the entire intelligentsia of Petersburg was
assembled here, and his eyes flitted about like a child in a toy
shop. He kept worrying he would miss an intelligent
conversation. Looking at the confident and refined faces gathered
here, he kept waiting for something especially intelligent. Finally,
he approached Morio. The conversation seemed interesting to him, and
he stopped, waiting for a chance to express his opinions, as young
people like to do.

\section{III} %% Book One, Part One, Chapter III

Anna Pavlovna's reception had gotten underway. The spindles were
humming all around the room, turning with regularity and never falling
silent. With the exception of \textit{ma tante}, who sat by herself
with only another old woman with a thin, worry-worn face, somewhat
strange in this sparkling society, the assembled society divided into
three circles. In one, more male, the priest was the center; in
another, younger, were the beautiful Princess Helene, Prince Vasily's
daughter, and the pretty, rosy-cheeked little princess Bolkonskaya, a
little chubby\todo{?}\ for her young age. In the third were Mortemart
and Anna Pavlovna.

The viscount was attractive, with soft features and manners,\todo{?}\
a young man, who obviously considered himself a celebrity, but, due to
good breeding, offered himself for the good of the society in which he
found himself. Anna Pavlovna was obviously providing him as a treat to
her guests. Like a good maitre-d'hotel offers a piece of beef as
something preternaturally wonderfully when you would not want to touch
it had you seen it in the dirty kitchen, so Anna Pavlovna served her
guests that night first the viscount, then the priest, like something
preternaturally refined. In Mortemart's circle, people were talking at
that moment about the slaughter of the Duke of Enghien. The viscount
told them how the Duke of Enghien died because of his own generosity,
and that Bonaparte had peculiar\todo{?}\ reasons for his animosity.

\textit{``Ah! voyons. Contez-nous cela, vicomte,''} Anna Pavlovna
said, overjoyed that that phrase came out sounding somehow \textit{\`a
  la Louis XV, ``contez-nous cela, vicomte.''}\footnote{Ah, come
  now. Tell us about it, viscount.}\todo{?}

The viscount bowed to show his obedience and smiled politely. Anna
Pavlovna started a circle around the viscount and invited everyone to
listen to his story.

\textit{``Le vicomte a \'et\'e personellement connu de monseigneur,''}
Anna Pavlovna whispered to one person. \textit{``Le vicomte est un
  parfait conteur,''} she told another. \textit{``Comme on voit
  l'homme de la bonne compagnie,''} she said to a third; and the
viscount was given to the assembled society in the most refined, most
sympathetic light, like roast beef on a hot platter, garnished with
greens.\footnote{The viscount was personally acquainted with the
  duke. The viscount is a perfect storyteller. Now we'll see a true
  man of high society.}

The viscount was ready to begin his story and smiled subtly.

``Move over here, \textit{ch\`ere H\'el\`ene},'' Anna Pavlovna said to
the beautiful princess, who was sitting a ways off, forming the center
of another circle.

Princess Helene smiled; she got up with the same unchanging smile,
characteristic of a woman of a perfectly beautiful woman, with which
she had entered the drawing room. Her white ball gown, embellished
with ivy and moss, rustled slightly, and, her white shoulders and her
lustrous hair and diamonds glistening, she walked past the parting
crowd of men and walked straight toward Anna Pavlovna, not looking at
anyone but smiling at everyone and giving each one the opportunity to
admire her figure, her full shoulders, her chest and back, which were
very exposed, according to the fashion of the time, somehow bringing
the brilliance of a ball along with her. Helene was so lovely that not
even the shadow of coquetry could be found in her; quite the opposite,
she seemed almost ashamed of her unquestionable and too-strong and
overwhelming beauty. It was as if she wanted to diminish the power of
her own beauty.

\textit{``Quelle belle personne!''} said everyone who saw
her.\footnote{What a beautiful person!} As if struck by something
extraordinary, the viscount shrugged his shoulders and lowered his
eyes while she sat down before him and turned her luminous, unchanging
smile on him.

\textit{``Madame, je crains pour mes moyen devant un pareil
  auditoire,''} he said, bowing his head with a smile.\footnote{Madam,
  I fear for my abilities before such an audience.}

The princess leaned her full arm on a little table and did not find it
necessary to say anything. She waited smiling. During the telling of
the story, she sat straight, now and then looking at her own full,
beautiful arm, lying lightly on the table, then at her even more
beautiful chest, on which she adjusted her diamond necklace; a few
times she adjusted the folds of her own dress and, when the story made
an impression on her, looked at Anna Pavlovna and immediately took on
the same expression that was on the maid of honor's face, and then
returned calmly to her beaming smile. The little princess followed
Helene from the tea table.

\textit{``Attendez moi, je vais prendre mon ouvrage,''} she told
them. \textit{``Voyons, \`a quoi pensez-vous?''} she said, turning to
Prince Ippolit: \textit{``apportez-moi mon
  ridicule.''}\footnote{Wait, I will go get my work. Come now, what
  are you thinking about? Bring me my work bag.}

The princess, smiling and talking with everyone, suddenly moved to the
other circle and, having seated herself, happily straightened her
dress.

``Now I am fine,'' she told them and, asking the viscount to begin,
took up her work.

Prince Ippolit brought her work bag, followed after her, and, moving a
chair closer to her, sat down near her.

\textit{Le charmant Hippolyte} was striking in his resemblance to his
beautiful sister and even more so because, despite their resemblance,
he was strikingly unattractive. His facial features were the same as
his sister's, but she was lit with a merry, self-satisfied, young,
unchanging smile and an extraordinary, classical physical beauty; the
brother, on the other hand, had a face obscured by idiocy, which
expressed a never-changing, self-confident displeasure at everything,
and his body was lean and weak. His eyes, nose, mouth --- all of them
were pressed into one ill-defined and bored grimace, and his arms and
legs were always taking on unnatural positions.

\textit{``Ce n'est pas une histoire de revenants?''} he said, seating
himself near the princess and hastily bringing his lorgnette to his
eyes, as if he could not begin to speak without that
instrument.\footnote{This isn't a ghost story, is it?}

\textit{``Mais non, mon cher,''} the surprised storyteller said,
shrugging his shoulders.\footnote{Well, no, my dear man.}

\textit{``C'est que je d\'eteste les histoires de revenants,''} Prince
Ippolit said with a tone that made clear that he had said the
words, and only later understood what they meant.\footnote{It's just
  that I hate ghost stories.}

Because of the self-confidence with which he spoke, no one could
understand whether what he said was very intelligent, or very
stupid. He was wearing a dark green tail-coat, trousers the color of
\textit{cuisse de nymphe effray\'ee}, as he himself said, stockings,
and boots.\textit{The body of a frightened nymph.}

\textit{Vicomte} relatedly very nicely the then-circulating story
about how the Duke of Enghien secretly traveled to Paris for a
rendezvous with Mlle.~George, and how he met Bonaparte there, who had
also been enjoying the favors of the celebrated actress, and how
Napoleon, meeting the duke there, he fell into one of the fits to
which he was prone, and was completely in the duke's power, which the
duke did not take advantage of, and despite this, Bonaparte
subsequently responded to his magnanimity by having the duke killed.

The story was very nice and interesting, especially at the part where
the rivals recognize one another, and the ladies seemed to be in a
state of excitement.

\textit{``Charmant,''} Anna Pavlovna said with a questioning glance at
the little princess.

\textit{``Charmant,''} the little princess whispered, sticking the
needle in her work, as if to show that the charm and fascination of
the story prevented her from continuing her work.

The viscount clearlh valued this silent praise and, smiling
gratefully, was about to continue; but at that moment Anna Pavlovna,
who was still keeping an eye on the young man who frightened her,
noticed that he was talking too heatedly and loudly with the priest and
rushed to alleviate the dangerous situation. Pierre had, in fact,
managed to strike up a conversation with the priest about political
equilibrium, and the priest, clearly fascinated by the simple-hearted
heatedness of the young man, explained his beloved idea before
him. Both were listening and speaking too naturally and animatedly,
and Anna Pavlovna did not like it.

``The means are European equilibrium and the \textit{droit des
  gens},'' the priest said. ``One needs only for a powerful government,
like Russia, famous for its barbarism, to become the disinterested
head of a union having as its goal equilibrium in Europe, and it would
save the world!''

``How will you find that equilibrium?'' Pierre was about to ask; but
at that moment, Anna Pavlovna approached them and, giving Pierre a
strict glance, asked the Italian how he was holding up in the face of
the climate there. The Italian's face suddenly changed and took on an
offensively false, sweet expression, which was clearly his habit when
talking with women.

``I am so charmed by the delights of the mind and the learning of
society here, especially among your women, who have been so kind as to
accept me, that I haven't had a moment to think about the climate,''
he said.

Not letting the priest and Pierre leave her, Anna Pavlovna united them
with the larger circle so she could more easily observe them.

At that moment, a new face appeared in the drawing room. The new face
was young Prince Andrei Bolkonsky, husband of the little
princess. Prince Bolkonsky was an extremely attractive young man, not
tall, with defined and dry features.\todo{?}\ Everything about his
appearance, from his tired, bored glance to his quiet, measured pace,
exhibited the starkest contrast to his lively little wife. It was
clear that he was not only acquainted with everyone in the drawing
room, but he was so tired of them that to look at them or listen to
them bored him. It seemed that, of all the faces he found so wearying
there, the face of his pretty wife bored him most of all. With a
grimace that ruined his handsome looks, he turned away from her. He
kissed Anna Pavlovna's hand and, glowering, looked over the society
gathered there.

\textit{``Vous vous enr\^olez pour la guerre, mon prince?''} Anna
Pavlovna said.\footnote{You're going to war, Prince?}

\textit{``Le g\'en\'eral Koutouzoff,''} Bolkonsky said, stressing the
final syllable \textit{zoff}, like a Frenchman, \textit{``a bien voulu
  de moi pour aide-de-camp\ldots{}''}\textit{Kutuzov has agreed to
  take me on as an aide-de-camp.}

\textit{``Et Lise, votre femme?''}\textit{And Lise, your wife?}

``She's going to the country.''

``Aren't you ashamed to take your delightful wife away from us?''

``Andr\'e,'' his wife said, turning to her husband and speaking in the
same coquettish tone with which she addressed even strangers, ``what a
story the viscount just told us about Mlle.~George and Bonaparte!''

Prince Andrei furrowed his brow and turned away. Pierre, who from the
time Prince Andrei entered the drawing room had not taken his happy,
friendly eyes off him, approached him and took him by the arm. Prince
Andrei grimaced without turning around, expressing irritation that
someone was pulling on his arm, but, seeing Pierre's smiling face,
gave an unexpectedly friendly and pleasant smile.

``Well, now\ldots{}! You've found your way into high society!'' he
said to Pierre.

``I knew you would be here,'' Pierre answered. ``I'll come over for
supper,'' he added softly, not wanting to interrupt the viscount, who
was continuing his story. ``Can I?''

``No, absolutely not,'' Prince Andrei said, laughing, telling Pierre
with a press of his hand that he did not even need to ask that. He
wanted to say more, but at that moment, Prince Vasily and his daughter
got up, and the men stood up to let them pass.

``If you'll forgive me, my dear viscount,'' Prince Vasily said to the
Frenchman, affectionately pulling his sleeve down toward the chair to
keep him from getting up. ``This unfortunate party at the ambassador's
deprives me of the pleasure and is interrupting you. I am so sad to
leave your exquisite reception,'' he said to Anna Pavlovna.

His daughter, Princess Helene, lightly holding up the folds of her
dress, made her way past the chairs, a smile shining even more
brightly on her beautiful face. Pierre watched with rapturous, almost
frightened eyes as the great beauty passed by him.

``Very beautiful,'' Prince Andrei said.

``Very,'' Pierre said.

Passing by, Prince Vasily grasped Pierre by the arm and turned to Anna
Pavlovna.

``Educate this bear for me,'' he said. ``He's been living with me for
a month, and this is the first time I've seen him in society. Nothing
is so necessary to a young man as the company of intelligent women.''

\section{IV} %% Book One, Part One, Chapter IV

Anna Pavlovna smiled and promised to look after Pierre, who she knew
was related to Prince Vasily by his\todo{?}\ father.

An old woman, who had been sitting with \textit{ma tante}, hastily
stood up and caught Prince Vasily in the hall. All trace of her
earlier feigned interest disappeared from her face. Her kind,
worry-worn face now expressed only worry and fear.

``What will you say, Prince, about my Boris?'' she said, catching him
in the hall. (She spoke the name Boris with special emphasis on the
\textit{o}.) I can't stay in Petersburg any longer. Tell me, what news
can I bring back to my poor boy?''

Despite the fact that Prince Vasily was listening reluctantly, almost
impolitely, to the old woman, and even showed impatience, she smiled
at him affectionately, pathetically, and took him by the arm to keep
him from leaving.

``A word to the sovereign costs you nothing, and he'll be transferred
directly into the Guards,'' she begged.

``Believe me, I will do everything I can, Princess,'' Prince Vasily
answered, ``but it is difficult for me to ask the sovereign; I would
suggest you appeal to Rumyantsev through Prince Golitsyn: that would
make more sense.''

The old woman was called Princess Drubetskaya, one of the best names
in Russia, but she was poor, had long since left society, and had lost
her earlier connections. She had come that night to obtain a
commission in the Guards for her only son. She had invited herself to
Anna Pavlovna's reception and come to her drawing room only to see
Prince Vasily, for that reason only she had listened to the viscount's
story. She was frightened by Prince Vasily's words; her one-beautiful
face expressed bitter animosity, but that went only for only a
minute. Soon she smiled again and seized Prince Vasily's arm tighter.

``Listen, Prince,'' she said, ``I have never asked you for anything
and will never ask you again, I've never once reminded you of my
father's friendship toward you. But now, for God's sake I beg of you,
do this for my son, and I'll consider you my benefactor,'' she added
hastily. ``No, don't be angry, just promise me. I asked Golitsyn, and
he turned me down. \textit{Soyez le bon enfant que vous avez
  \'et\'e},'' she said, trying to smile, with tears in her
eyes.\footnote{Be the friend you once were.}

``Papa, we're late,'' said Princess Helene, who was waiting by the
door, turning her pretty head on her classical-looking shoulders.

But influence in society represents capital, which one must guard
carefully, lest it disappear. Prince Vasily knew this, and, once he
grasped that if he asked favors for everyone who asked him, he would
soon be unable to ask for himself, he rarely made use of his
influence. In Princess Drubetskaya's case, however, he felt, after her
renewed appeal, something akin to a pang of conscience. She had
reminded him of something that was true: he owed his first
opportunities in the civil service to her father. Furthermore, he had
seen enough of her methods to realize that she was one of those women,
especially mothers, who, once they had taken something in their heads,
will not rest until they see their desires realized, and failing that,
are prepared for daily, hourly harassment and even the occasional
scene. This last reason caused him to waver.

``\textit{Ch\`ere} Anna Mikhaylovna,'' he said with his constant
familiarity and boredom, ``it is nearly impossible for me to do what
you want; but to show you that I love you and value the memory of your
late father, I will do the impossible; your son will be transferred to
the Guards, I give you my word. Are you satisfied?''

``My dear, oh, my benefactor! I expected nothing less from you; I knew
how kind you were.''

He wanted to leave.

``Wait, two words. \textit{Une fois pass\'e aux
  gardes\ldots{}}''\footnote{Once he's been transferred to the
  Guards\ldots{}} She stopped short: ``You are well acquainted with
Mikhail Ilarionovich Kutuzov, recommend Boris to him as an
adjutant. Then I could rest easy, and then\ldots{}''

Prince Vasily smiled.

``That I cannot promise. I can't tell you how beseiged Kutuzov has
been since he was named commander-in-chief. He himself said to me that
every lady in Moscow is trying to set up her children as adjutants.''

``No, promise, I won't let you go, my dear benefactor.''

``Pap\'a,'' the great beauty repeated in the same tone, ``we're
late.''

``Well, \textit{au revoir}, goodbye. You see?''

``Then you'll tell the sovereign about this tomorrow?''

``Certainly, but I can promise nothing about Kutuzov.''

``No, promise, promise, Basile,'' Anna Mikhaylovna shouted after him,
smiling like a young coquette, which must have been natural to her
once, but which no longer fit her time-worn face.

She had clearly forgotten her years and brought all her old womanly
ways to bear out of habit. But as soon as he left, her face again took
on the cold, feigned expression it had carried before.  She returned
to the circle where the viscount was continuing to tell his story, and
she again pretended to listen, waiting until it was time to leave, her
task completed.

``But what do you think of this latest comedy \textit{du sacre de
  Milan}?'' Anna Pavlovna said. \textit{``Et la nouvelle com\'edie des
  peuples de G\^enes et de Lucques, qui viennent pr\'esenter leurs
  v\oe ux \`a M.~Buonaparte. M.~Buonaparte assis sur un tr\^one, et
  exau\c{c}ant les v\oe ux des nations! Adorable! Non, mais c'est \`a
  devenir folle! On dirait, que le monde entier a perdu la
  t\^ete.''}\footnote{Of the coronation in Milan. And the latest
  comedy of the people of Genoa and Lucca, who are making their
  desires known to Mr.~Buonaparte. Mr.~Buonaparte sat on a throne, and
  will fulfill the desires of the nations! Delightful! No, this is
  becoming crazy! It seems the whole world has lost its head.}

Prince Andrei laughed, looking Anna Pavlovna straight in the face.

\textit{``Dieu me la donne, gare \`a qui la touche,''} he said,
quoting Bonaparte's words when he crowned himself. \textit{``On dit
  qu'il a \'et\'e tr\`es beau en pronon\c{c}ant ces paroles,''} he
added, and repeated the words again in Italian: \textit{``Dio mi la
  dona, guai a chi la tocca.''}\footnote{God has given it to me, woe
  to him who touches it.}

\textit{``J'esp\`ere enfin,''} Anna Pavlovna continued, \textit{``que
  \c{c}a a \'et\'e la goutte d'eau qui fera d\'eborder le verre. Les
  souverains ne peuvent plus supporter cet homme, qui menace
  tout.''}\footnote{I hope that this is, at last, the straw that
  breaks the camel's back. The sovereigns cannot suffer this man any
  longer, who threatens everything.}

\textit{``Les souverains? Je ne parle pas de la Russie,''} the
viscount said politely but despairingly: \textit{``Les souverains,
  madame! Qu'ont ils fait pour Louis XVII, pour la reine, pour madame
  Elisabeth? Rien,}'' he continued, becoming animated. \textit{``Et
  croyez-moi, ils subissent la punition pour leur trahison de la cause
  des Bourbons. Les souverains? Ils envoient des ambassadeurs
  complimenter l'uruspateur.''}\footnote{The sovereigns? I won't speak
  of Russia. The sovereigns, madam! What did they do for Louis XVII,
  for the queen, for Madame Elisabeth? Nothing. And believe me, they
  will suffer punishment for their betrayal of the Bourbon cause. The
  sovereigns? They are sending ambassadors to congratulate the
  usurper.}

And, sighing contemptuously, he again changed position. Prince
Ippolit, who had long been looking at the viscount with his
lorgnette, turned his whole body suddenly toward the little princess
upon hearing these words and, asking her for her needle, began to show
her the Cond\'e shield, drawing on the table with the needle.

\textit{``B\^aton de gueules, engr\^el\'e de gueules d'azur --- maison
  Cond\'e,''} he said.\footnote{A bend gules, azure engrailed with
  gules --- House Cond\'e. \textit{(Ippolit's description of the arms
    of House Cond\'e does not seem to be accurate, though it is
    close. The correct blazon should be: azure, three fleurs-de-lys
    or, a bendlet couped gules, a bordure gules. --- Trans.)}}

The princess listened, smiling.

``If Bonaparte remains on the throne of France for another year,'' the
viscount said, continuing the conversation with the look of someone,
as the best-informed on this matter, simply following the path of his
own thoughts, not listening to anyone else,\todo{yikes}\ ``then things
will have gone too far. Through intrigue, force, exiles, executions,
society --- I mean good society, French society --- will be destroyed
forever, and then\ldots{}''

He shrugged his shoulders and spread his hands. Pierre was about to
say something: the conversation interested him, but Anna Pavlovna, who
had been watching for this, interrupted him.

``Emperor Aleksandr,'' she said with the sadness that accompanied all
her speeches about the imperial family, ``has proclaimed that he will
allow the French people themselves to choose their form of
government. And I think there is no question that, once free of the
usurper, the entire nation will throw itself into the arms of the
lawful king,'' Anna Pavlovna said, trying to be kind to the emigrant
and royalist.

``That's questionable,'' Prince Andrei said. ``\textit{Monsieur le
  vicomte} is entirely correct to suggest that things have already
gone too far. I think that it will be difficult to return to the old
ways.''

``From what I've heard,'' Pierre cut in, ``almost the entire nobility
has gone over to Bonaparte's side.''

``It's the Bonapartists who says so,'' the viscount said, not looking
at Pierre. ``At the moment, it's difficult to know the opinion of
French society.''

\textit{``Bonaparte l'a dit,''} Prince Andrei said with a
grin.\footnote{Bonaparte has said what it is.}

(It was obvious that he did not like the viscount, and that, although
he was not looking at him, his speeches were directed at him.)

\textit{``Je leur ai montr\'e le chemin de la gloire, ils n'en ont pas
  voulu; je leur ai ouvert mes antichambres, ils se sont
  pr\'ecipit\'es en foule\ldots{}''} he said after a brief silence,
again repeating the words of Napoleon. \textit{``Je ne sais pas \`a
  quel point il a eu le droit de le dire.''}\footnote{``I have shown
  them the path of glory, they didn't want to take it; I opened my
  antechambers to them, they rushed in in an unruly mob.'' I don't
  know how much right he had to say that.}\todo{Any scholarship on
  this Napoleon quote?}

\textit{``Aucun,''} was the viscount's retort. ``After the murder of
the duke, even the people most partial to him stopped seeing him as a
hero. \textit{Si m\^eme \c{c}a \'et\'e un h\'eros pour certaines
  gens},'' the viscount said, turning to Anna Pavlovna,
\textit{``depuis l'assassinat du duc il y a un martyr de plus dans le
  ciel, un h\'eros de moins sur la terre.''}\footnote{None. Even if he
  had been a hero to certain people, after the duke's assassination,
  there was one more martyr in heaven and one fewer hero on earth.}

Anna Pavlovna and the others could not even respond to the viscount's
words with a smile before Pierre again burst into the conversation,
and Anna Pavlovna, although she had a foreboding that he would say
something improper, could not stop him.

``The execution of the Duke of Enghien,'' Pierre said, ``was a
political necessity; and I see greatness of spirit in the fact that
Napoleon was not afraid to take responsibility for that act on himself
alone.''

\textit{``Dieu! mon Dieu!''} Anna Pavlovna uttered with a frightened
whisper.

\textit{``Comment, M.~Pierre, vous trouvez que l'assassinat est
  granduer d'\^ame,''} the little princess said, smiling and drawing
her work toward her.\footnote{How do you find greatness of spirit in
  murder.}

\textit{``Ah! Oh!''} came various voices from around the room.

``Capital!'' Prince Ippolit said in English and started hitting his
knee with his palm. The viscount merely shrugged his shoulders.

Pierre looked at his audience triumphantly over his glasses.

``I say that,'' he continued recklessly, ``because the Bourbons ran
away during the revolution, leaving the people to anarchy; and
Napoleon alone was able to understand the revolution, defeat it, and
so for the sake of the general welfare can't stop on account of one
man's life.''

``Wouldn't you like to come over to this table?'' Anna Pavlovna
said. But Pierre continued his speech without answering.

``No,'' he said, becoming more and more animated, ``Napoleon is a
great man because he made himself higher than the revolution, getting
rid of its corruption, retaining everything good --- the equality of
citizens, freedom of speech and the press --- and only then took
power.''

``Yes, if, having taken power, he had not used it to commit murder but
instead given it to the lawful king,'' the viscount said, ``then I
would have called him a great man.''

``He couldn't have done that. The people gave power to him precisely
so he could save them from the Bourbons, and because the people saw a
great man in him. The revolution was a great pursuit,'' Monsieur
Pierre continued, showing with this reckless and provocative opening
statement his great inexperience and his desire to get all his ideas
out at once.

``A great pursuit --- revolution and regicide\ldots{}? After
that\ldots{} wouldn't you like to come over to this table?'' Anna
Pavlovna repeated.

\textit{``Contrat social,''} the viscount said with a curt
smile.\footnote{The social contract.}

``I'm not talking about regicide. I'm talking about ideas.''

``Yes, the ideas of robbery, murder, and regicide,'' an\todo{the?}\ ironic voice
again interrupted.

``There were extremes, of course, but the meaning of the whole
endeavor isn't in those, the meaning is in the rights of man, in
emancipation from prejudices, in the equality of citizens; and
Napoleon retained all those ideas in their full force.''

``Freedom and equality,'' the viscount said contemptuously, as if he
had finally decided to prove to this raw youth the stupidity of his
words, ``are very loud words that have long since lost their
power. Who does not love freedom and equality? Even our Savior
preached freedom and equality. But did people really become happier
after the revolution? Quite the opposite. We wanted freedom, but
Bonaparte destroyed it.''

Prince Andrei smilingly at Pierre, then the viscount, then their
hostess. During the first minute of Pierre's outburst, Anna Pavlovna
was horrified, despite her long experience in society; but when she
saw that, despite Pierre's sacrilegious words, the viscount remained
calm, and when she had decided she should not cut off their speeches,
she mustered all her strength and, uniting with the viscount, fell
upon the young speaker.

\textit{``Mais, mon cher m-r Pierre,''} Anna Pavlovna said, ``how do
you explain a great man who can execute a duke, any man, in fact,
without trial and without a crime?''\footnote{But, my dear Mr.~Pierre.}

``I would ask,'' the viscount said, ``how \textit{monsieur} can
explain 18 Brumaire? Was it not a deception? \textit{C'est un
  escamotage, qui ne ressemble nullement \`a la mani\`ere d'agir d'un
  grand homme.}''\footnote{It was a trick, which in no way resembles
  the way a great man acts.}

``And what about the prisoners in Africa that he killed?'' the little
princess said, ``that was horrible!'' And she shrugged her shoulders.

\textit{``C'est un roturier, vous aurez beau dire,''} Prince Ippolit
said.\footnote{He's a commoner, you might well say.}

Monsieur Pierre did not know whom to answer, looked around at
everyone, and smiled. His smile was not like others' smiles, emerging
fluidly from their unsmiling faces. With him, on the contrary, when he
smiled, his face suddenly, in a single moment, dropped its serious and
even sullen expression and another one appeared --- childish, kind,
even a bit stupid and almost begging for forgiveness.

The viscount, who was meeting him for the first time, realized that
this Jacobin was not as frightening as his words made him
seem. Everyone fell silent.

``Would you like him to answer all of you at once?'' Prince Andrei
said. ``Besides, when it comes to the acts of a political person, one
must distinguish between the acts of a private person, a commander, or
an emperor. Or so it seems to me.''

``Yes, yes, of course,'' Pierre joined in, pleased by this
assistance.\todo{?}

``It can't be denied,'' Prince Andrei continued, ``Napoleon was a
great man at the Bridge of Arcole, at the hospital at Jaffa, where he
gave his hand to the plague victims there, but\ldots{} but there are
other acts that are difficult to justify.''

Prince Andrei, clearly hoping to smooth over Pierre's clumsy speech,
sat up, preparing to leave and making a sign to his wife.

Suddenly Prince Ippolit got up and, making signs with his hands for
everyone to stay in their places and asking them to sit, said:

\textit{``Ah! aujourd'hui on m'a racont\'e une anecdote moscovite,
  charmante: il faut que je vous en r\'egale. Vous m'excusez, vicomte,
  il faut que je raconte en russe. Autrement on ne sentira pas le sel
  de l'histoire.''}\footnote{Ah! Today I heard a charming story from
  Moscow: I must regale you with it. Excuse me, viscount, I must tell
  it in Russian. It won't make sense otherwise.}

Here Prince Ippolit thought for a minute, clearly with some
difficulty.

``She said\ldots{} yes, she said: `you, girl (\textit{\`a la femme de
  chambre}), put your \textit{livr\'ee} on and you go with me, on
carriage, \textit{faire des visites}.' ''\footnote{To the girl\ldots{}
  livery\ldots{} to make visits.}

Here Prince Ippolit snorted and burst out laughing well before his
audience, which left a bad impression of him as a storyteller. Still,
many of them, including the old woman and Anna Pavlovna, smiled.

``She went. Suddenly was a strong wind. Girl loses her hat, and long
her hair, they fly all over\ldots{}''

Here he could no longer hold himself together and started laughing
again in short bursts, in between saying:

``And everyone in society discover\ldots{}''

With this the story ended. Although it was not clear why he had to
tell the story or why he had to tell it in Russian, Anna Pavlovna and
the others nevertheless appreciated Prince Ippolit's sense of civility
in high society, putting a pleasant and well-timed end to Pierre's
unpleasant and graceless outburst. The conversation after this story
fell to shallow, insignificant talk about the most recent ball, the
one coming up next, a play, about when and where who could be
seen.\todo{?}

\section{V} %% Book One, Part One, Chapter V

Having thanked Anna Pavlovna for her \textit{charmante soir\'ee}, the
guests began to go their separate ways.\footnote{Charming reception.}

Pierre was awkward. Fat, taller than average, broad, with huge red
hands, he did not know how to make an entrance, as they say, and knew
even less how to make an exit, that is, to say something especially
pleasant before leaving. Furthermore, he was scattered. Upon getting
up, instead of his own hat he took a tricorne with a general's plume
and held it, tugging on the feather, until the general asked for him
to return it. But for all his confusion and inability to make an
entrance and speak well once there, he was redeemed by his
kindheartedness, simplicity, and humility. Anna Pavlovna turned to him
and, with Christian meekness forgiving him for his outburst, nodded to
him and said:

``I hope we'll see more of you, but I also hope you'll change your
views, my dear Monsieur Pierre,'' she said.

When she said this to him, he did not answer, but merely bowed and
again showed his childish smile to everyone, communicating nothing,
except perhaps, \textit{Views are what they are, but you see what a
  nice, kind fellow I am.} And everyone, Anna Pavlovna included, felt
that unwittingly.

Prince Andrei went into the hall and, presenting his shoulders to the
footman, who put on his cloak, listened indifferently to his wife
chattering with Prince Ippolit, who had also come out to the
hall. Prince Ippolit was standing by the pregnant, attractive princess
and looking directly at her with his lorgnette.

``Go, Annette, you'll catch cold,'' the little princess said, bidding
farewell to Anna Pavlovna. \textit{``C'est arr\^ete,''} she added
quietly.\footnote{It's decided.}

Anna Pavlovna had already managed to discuss the match she hoped to
make between Anatole and the little princess's sister-in-law.

``I'm relying on you, my dear friend,'' Anna Pavlovna said, also
quietly, ``write to her and tell me \textit{comment le p\`ere
  envisagera la chose. Au revoir,}'' and she left the
hall.\footnote{How her father sees the matter. Goodbye.}

Prince Ippolit approached the little princess and, bringing his face
close to her, started to say something to her in a half-whisper.

Two footmen, one the princess's, the other his, waiting for them to
stop talking, stood with a shawl and a redingote and listened to their
conversation in French, incomprehensible to them, as if they
understood what was being said but did not want to show it. The
princess, as always, smiled as she spoke and laughed while listening.

``I'm so glad I didn't go to the ambassador's,'' Prince Ippolit said:
``boring\ldots{} It was a wonderful reception, wasn't it, wonderful?''

``They're saying the ball will be very good,'' the princess answered,
her little whiskered lip jerking up. ``All the beautiful women in
society will be there.''

``Not all of them, because you won't be there; not all of them,''
Prince Ippolit said, laughing happily, and, taking the shawl from the
footman, pushed him aside and started to put it on the
princess. Either out of clumsiness, or on purpose (no one would have
been able to make out which), he left his hands on her long after the
shawl had been put on, as if embracing the young woman.

She separated from him graciously, but still smiling, turned around,
and looked at her husband. Prince Andrei's eyes were closed: so he
seemed weary and drowsy.

``Are you ready?'' he asked his wife, looking her over.

Prince Ippolit hastily put on his redingote, which, according to the
new fashion, fell below his knees, and, still caught in the sleeves,
ran out onto the porch with the princess, who was being seated in the
carriage by her footman.

\textit{``Princesse, au revoir,''} he shouted, tripping over his
tongue as well as his feet.\footnote{Princess, goodbye.}

The princess, gathering her dress, sat in the darkness of the
carriage; her husband adjusted his saber; Prince Ippolit, under the
pretense of trying to help, got in everyone's way.

``If you would be-e-e so kind, sir,'' Prince Andrei addressed Prince
Ippolit coldly and unpleasantly in Russian, barring his way.

``I'll be waiting for you, Pierre,'' came Prince Andrei's voice again,
this time affectionately and tenderly.

The postillion moved, and the carriage wheels began to clatter. Prince
Ippolit gave a burst of laughter, standing on the porch and waiting
for the viscount, whom he had promised to bring home.

\textit{``Eh bien, mon cher, votre petite princesse est tr\`es bien,
  tr\`es bien,''} the viscount said, sitting in the carriage with
Ippolit. \textit{``Mais tr\`es bien.''} He kissed the tips of his
fingers. \textit{``Et tout-\`a-fait fran\c{c}aise.''}\footnote{Well,
  my dear, your little princess is very nice, very nice. Oh, very
  nice. And just like a French girl.}

Ippolit snorted and laughed.

\textit{``Et savez-vous que vous \^etes terrible avec votre petit air
  innocent,''} the viscount continued. \textit{``Je plains le pauvre
  mari, ce petit officier, qui se donne des airs de prince
  r\'egnant.''}\footnote{And you know, you are terrible, playing
  innocent. I pity her poor husband, that little officer, who was
  putting on airs, like a prince on a throne.}

Ippolit snorted again and said through laughter:

\textit{``Et vous disiez, que les dames russes ne valaient pas les
  dames fran\c{c}aises. Il faut savoir s'y prendre.''}\footnote{And
  you said Russian women didn't stand up to French women. You have to
  know how to go about it.}

Pierre, who arrived first at Prince Andrei's, went into the prince's
study as if he were a member of the family and, out of habit, lay
right down on the sofa, picked up the first book that fell to hand (it
was Caesar's \textit{Commentaries}) and started to read from the
middle, leaning his elbow on the sofa.

``What did you do to \textit{m-lle Scherer}? She's fallen seriously
ill,'' Prince Andrei said, walking into the study and rubbing his
small, white hands.

Pierre swung his whole body around, making the sofa creak, turned his
animated face toward Prince Andrei, smiled, and waved his hand.

``No, that priest was very interesting, but he has the whole thing all
wrong\ldots{} I think eternal peace is possible, but I don't know how
to explain it\ldots{} Just not through political equilibrium\ldots{}''

Prince Andrei was clearly not interested in abstract conversations of
that kind.

``You can't go around, \textit{mon cher}, saying everything as soon as
you think it. Well, have you decided on anything, at long last? The
Guards Cavalry or the diplomatic service?'' Prince Andrei asked after
a minute's silence.

Pierre sat on the sofa, pressing his leg underneath him.

``As you can imagine, I still don't know. I don't like the one or the
other.''

``But you have to decide on something, yes? Your father is waiting.''

Pierre had been sent abroad at the age of ten with a priest as a
tutor, remaining there until the age of twenty. When he returned to
Moscow, his father dismissed the priest and told the young man: ``Now
you go to Petersburg, get the lay of the land, and choose
something. I'll agree to anything. Here's a letter to Prince Vasily,
and here is some money. Write to me about everything, I'll help you
with anything.'' Pierre had been choosing a career for three months
already and had accomplished nothing. He had talked about his choice
with Prince Andrei. Pierre wiped his brow.

``But he must be a Mason,'' he said, meaning the priest he had seen at
the reception.

``That's all nonsense,'' Prince Andrei stopped him again, ``let's talk
about the matter at hand. Have you seen the Horse Guards?''

``No, I haven't, but I just thought of something, and I want to tell
you. We're going to be at war with Napoleon. If we were fighting for
freedom, I'd understand, I'd be the first to take up military service;
but to help England and Austria against the greatest man in the
world\ldots{} it isn't right\ldots{}''

Prince Andrei merely shrugged his shoulders at Pierre's childish
words. He made a show of not answering so stupid a concern; but in
fact it would have been difficult to answer such a naive question any
other way than how Prince Andrei did.

``If everyone fought only for principles they believed in, there would
be no war,'' he said.

``That would be wonderful,'' Pierre said.

Prince Andrei laughed.

``It very well might be wonderful, but it will never happen\ldots{}''

``Well, why are you going to war?'' Pierre asked.

``Why? I don't know. I have to. Furthermore, I'm going\ldots{}'' He
stopped. ``I'm going because the life I have here, this life --- it
doesn't feel right!''

\section{VI} %% Book One, Part One, Chapter VI

In the next room, a woman's dress rustled. Prince Andrei shook
himself\todo{ugh}, as if waking up, and his face took on the same
expression it had had\todo{ugh}\ in Anna Pavlovna's drawing
room. Pierre put his feet down off the sofa. The princess entered. She
was in a different dress, a housedress, but just as elegant and fresh
as the one she had been wearing. Prince Andrei got up, 