\message{ !name(book1-1.tex)}
\message{ !name(book1-1.tex) !offset(-2) }
\chapter{Part One}

\section{ }

``\textit{Eh bien, mon prince. G\^enes et Lucques ne sont plus des
  apanages, des} estates, \textit{de la famille Buonaparte. Non, je
  vous pr\'eviens, que si vous ne me dites pas, que nous avons la
  guerre, si vous vous permettez encore de pallier toutes les
  infamies, toutes les atrocit\'es de cet Antichrist (ma parole, j'y
  crois) --- je ne vous connais plus, vous n'\^etes plus mon ami, vous
  n'\^etes plus} my devoted slave, \textit{comme vous
  dites.}\footnote{Well, my prince. Genoa and Lucca are no more than
  possessions, estates of the Buonaparte family. No, I'm warning you,
  if you do not tell me we are going to war, if you again permit
  yourself to excuse all the infamies, all the atrocities of that
  Antichrist (I truly believe that is what he is) --- then I don't
  know you any longer, you are no longer my friend, you are are no
  longer my devoted slave, as you say.} Well, hello, hello. \textit{Je
  vois que je vous fair peur,}\footnote{I see that I've frightened
  you.} sit down and tell me everything.''

So spoke\todo{In this way began...?}\ the well-known Anna Pavlovna
Scherer, lady-in-waiting and confidante of Empress Maria Feodorovna,
in July 1805, meeting the important and high-ranking\todo{chinovnogo}\
Prince Vasily, who was the first to arrive at her reception. Anna
Pavlovna had had a cough for several days, she had the
\textit{grippe}, as she said (\textit{grippe} was then a new word,
used only by rarified people). In notes sent out that morning, carried
by a footman in red, was written to everyone without distinction:

``\textit{Si vous n'avez rien de mieux \`a faire, M.~le comte} (or
\textit{mon prince}), \textit{et si la perspective de passer la
  soir\'ee chez une pauvre malade ne vous effraye pas trop, je serai
  charm\'ee de vous voir chez moi entre 7 et 10 heures. Annette
  Scherer.}''\footnote{If you have nothing better to do, Count (or
  Prince), if the thought of passing the evening with a poor sick
  woman does not scare you too much, I would be charmed to see you
  between 7 and 10. Annette Scherer.}

``\textit{Dieu, quelle virulente sortie!}''\footnote{My god, what a
  vicious attack!} answered the prince upon entering, not at all
rattled by this greeting. He was wearing a tailored dress uniform, in
stockings, boots, and stars, with a bright expression on his plain
face.


\message{ !name(book1-1.tex) !offset(-49) }
