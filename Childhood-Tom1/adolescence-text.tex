% Copyright 2015 Christopher A. Tessone
% Creative Commons Attribution-NonCommercial-ShareAlike 4.0
% International License
% http://creativecommons.org/licenses/by-nc-sa/4.0/
\makeoddhead{modruled}{}{\scshape Adolescence}{}
\markboth{Adolescence}{}

\chapter{A Trip in Stages} %c1

Two carriages have again been brought around to the front steps of the house at Petrovskoye; in one, a coach, are sitting Mimi, Katyenka, Lyubochka, a lady's maid, and the steward \emph{himself}, Yakov; the other is a britzka, in which Volodya and I are riding with a footman, Vasily, who has recently been taken into service from the fields.

Papa, who is supposed to arrive in Moscow a few days after us, is standing on the front steps without a hat, making the sign of the cross over the britzka and the window of the coach.

``Well, Christ be with you! Move along!'' Yakov and the coachmen (we are using our own horses) take off their hats and cross themselves. ``Giddy-up! Go with God!'' The body of the coach and the britzka begin to bounce up and down on the uneven road, and birch trees fly past us along the wide lane, one after another. I am not the least bit sad: my mind's eye is directed not at what I am leaving behind, but at what is waiting for me. Every step we take away from the objects that are tied to heavy remembrances filling my imagination until then makes those remembrances lose their strength and quickly replaces them with a comforting feeling of consciousness of a life full of strength, freshness, and hope.

I have rarely spent a few days so---I will not say happily: I was still ashamed to indulge in happiness---but so pleasantly and well as the four days of our journey. The whole time, I did not see before my eyes the closed door of mother's room, which I could not walk past without shuddering, nor the shut-up piano, which no one approached and everyone looked at with a kind of pain, nor any mourning clothes (all of us were wearing simple traveling clothes), nor any of the other things that reminded me vividly of our irrevocable loss and kept me on my guard against every appearance of real life out of fear I would somehow insult \emph{her} memory. Here, on the contrary, there were constantly new, picturesque places and things that drew my attention and entertained me, and all that nature in bloom inspired comforting thoughts in my soul---satisfaction with the present and bright hope for the future.

Early, early in the morning, Vasily---merciless and, as always happens with people in a new position, overzealous---pulls off my covers and announces that it is time to go and everything is already prepared. No matter how much I huddle in, get angry at him, or try to trick him so that I can continue my sweet morning dream for just a quarter of an hour more, I see from the decisive look on Vasily's face that he is implacable and is prepared to pull my covers off another twenty times, so I jump up and run to the yard to wash up.

On the porch, the samovar is already boiling, and it is turning as red as a lobster because Mitka the post-boy is fanning the flames: it is damp and foggy in the yard, as if steam were rising from the fragrant manure; the bright, happy light of the sun is lighting up the eastern part of the sky and the thatched roofs of the spacious sheds lining the yard, which shine from the dew that covers them. Beneath them, our horses can be seen tied up near the feed troughs, and their measured chewing can be heard. A shaggy horse named Zhuchka, nestled down since before dawn in front of a heap of dry manure, stretches lazily and, swishing his tail, sets out for the other side of the yard with a little trot. The fussy proprietress opens the creaking gates, chases the daydreaming cows out onto the street, where you can already hear the stomping, mooing, and bleating of the herd, and shares a word or two with a drowsy neighbor. Filipp, the sleeves of his shirt rolled up, is turning a crank to hoist a bucket out of a deep well, clear water splashing out of it. He pours it into an oak trough where ducks have woken up and are paddling around in a puddle; and I look with pleasure at Filipp's large face and full beard, and his thick veins and muscles, which stand out on his strong, naked arms when he exerts himself.

On the other side of the partition, where Mimi and the girls slept, and across which we talked during the evening, movement is heard. Masha keeps running past more and more often, trying to hide various objects from our curious glances with her dress, and finally the door opens and we are called to drink tea.

Vasily, in a fit of overzealousness, is incessantly running into the room, carrying one thing out, then another, winking at us, and begging Marya Ivanovna in every way possible to leave earlier. The horses are packed and expressing their impatience, shaking their bells from time to time; the suitcases, trunks, large cases and small little cases are all stowed again, and we try to get seated in our places. But every time, we find a mountain in the britzka where our seats are supposed to be, so that we cannot understand how it was all stowed the day before and how we are supposed to sit down now; one walnut tea chest in particular, with a triangular lid, puts me in a foul mood as they hand it into the britzka and put it underneath me. But Vasily says that everything will settle down soon, and I am forced to trust him.

The sun has just now risen above an unbroken mass of white clouds covering the east, and the surrounding area became suffused with a calm, joyful light. Everything around me is so splendid, and my heart is light and calm\ldots{} The road winds on ahead like a wild, broad ribbon between dry stubble and glittering fields of dewy green; here and there along the road we see a sad-looking willow or a young birch with fine, sticky leaves throwing a long, unmoving shadow over the dry clay of the ruts and the fine green grass of the road\ldots{} The monotonous noise of the wheels and bells does not drown out the songs of the skylarks who make winding circles right above the road. The smell of moth-eaten cloth, dust, and something sour that distinguishes our britzka covers the smell of the morning, and I feel a comforting anxiety in my heart, the desire to do something---a sign of true delight.

I was not able to pray at the posting station; but since I have noticed more than once that, on any day in which I forget, because of one circumstance or another, to fulfill my religious responsibilities, some misfortune always happens to me, I try to correct my mistake: I take off my peaked cap, turning toward the corner of the britzka, read my prayers, and cross myself under my jacket so that no one can see it. But a thousand different objects draw away my attention, and I absent-mindedly repeat the same prayers several times in a row.

Over there on a footpath winding near the road, some slowly moving figures can be seen: they are pilgrim women. Their heads are wrapped in dirty scarves, birchbark rucksacks on their backs, legs swathed in dirty, ragged strips of cloth and shod in heavy bast shoes.\footnote{The strips wrapping their legs and feet are \textit{onuchi}, long strips of cloth similar to puttees that formed part of the typical dress of Russian peasants. --- Trans.} Swinging their walking sticks at regular intervals and scarcely looking at us, they move forward with a slow, heavy step, one after the other, and I am occupied by these questions: Where are they going, and why? Will their journey continue for a long time, and how soon will the long shadows they are casting on the road meet up with the shadow of the willow, which they must pass? Here a carriage and four pulled by post horses comes quickly toward us. Two seconds and faces appear at a distance of five feet, looking at us curiously and affably, and now they have already flashed past, and it seems strange that those faces have nothing in common with me, and that I may never see them again.

