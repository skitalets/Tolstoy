% Copyright 2015 Christopher A. Tessone
% Creative Commons Attribution-NonCommercial-ShareAlike 4.0
% International License
% http://creativecommons.org/licenses/by-nc-sa/4.0/
\makeoddhead{modruled}{}{\scshape Adolescence}{}
\markboth{Adolescence}{}

\chapter{A Trip in Stages} %c1

Two carriages have again been brought around to the front steps of the house at Petrovskoye; in one, a coach, are sitting Mimi, Katyenka, Lyubochka, a lady's maid, and the steward \emph{himself}, Yakov; the other is a britzka, in which Volodya and I are riding with a footman, Vasily, who has recently been taken into service from the fields.

Papa, who is supposed to arrive in Moscow a few days after us, is standing on the front steps without a hat, making the sign of the cross over the britzka and the window of the coach.

``Well, Christ be with you! Move along!'' Yakov and the coachmen (we are using our own horses) take off their hats and cross themselves. ``Giddy-up! Go with God!'' The body of the coach and the britzka begin to bounce up and down on the uneven road, and birch trees fly past us along the wide lane, one after another. I am not the least bit sad: my mind's eye is directed not at what I am leaving behind, but at what is waiting for me. Every step we take away from the objects that are tied to heavy remembrances filling my imagination until then makes those remembrances lose their strength and quickly replaces them with a comforting feeling of consciousness of a life full of strength, freshness, and hope.

I have rarely spent a few days so---I will not say happily: I was still ashamed to indulge in happiness---but so pleasantly and well as the four days of our journey. The whole time, I did not see before my eyes the closed door of mother's room, which I could not walk past without shuddering, nor the shut-up piano, which no one approached and everyone looked at with a kind of pain, nor any mourning clothes (all of us were wearing simple traveling clothes), nor any of the other things that reminded me vividly of our irrevocable loss and kept me on my guard against every appearance of real life out of fear I would somehow insult \emph{her} memory. Here, on the contrary, there were constantly new, picturesque places and things that drew my attention and entertained me, and all that nature in bloom inspired comforting thoughts in my soul---satisfaction with the present and bright hope for the future.

