\part*{Childhood}
\makeoddhead{modruled}{}{\scshape Childhood}{}
\markboth{Childhood}{}

\chapter{Karl Ivanych, the Tutor} %c1

On the 12th of August, 18\ldots{}, exactly three days after my birthday, when I turned ten and received some marvelous gifts, Karl Ivanych woke me up at seven o'clock in the morning, hitting the wall just above my head with a flyswatter made from sugar paper and a stick. He did this so carelessly that he hit the icon of my angel, which was hanging from the oak headboard, and the dead fly fell right on my head.\footnote{By angel, Tolstoy actually means the narrator's saintly namesake. --- \textit{Trans.}} I stuck my nose out from under the cover, stopped the swinging icon with my hand, knocked the dead fly onto the floor, and threw an angry look at Karl Ivanych, my eyes still clouded with sleep. He was wearing a motley cotton dressing-gown---belted with a belt from the same material---a red knit cap with a tassel, and soft goat-skin \todo{?} boots, and he continued to walk along the walls, looking for targets and slapping them with his flyswatter.

``Fine,'' I thought, ``I \emph{am} small, but why does he keep bothering me? Why doesn't he swat flies by Volodya's bed? There are so many over there! No, Volodya is older than me; I'm the youngest: that's why it's me he torments. That's what he thinks about all day,'' I whispered, ``how to make trouble for me. He can see very well that he woke me up and frightened me, but he acts like he doesn't notice\ldots{} what a disgusting person! His robe, his little hat, and that tassel---disgusting!'' %nikolai

At the same time that I was inwardly expressing my annoyance with Karl Ivanych, he went over to his bed, looked at his watch, which was hanging over it in a beaded slipper, hung the flyswatter up on its nail, and turned to us, obviously in a very pleasant mood.

\textit{``Auf, Kinder, auf!\ldots{} s'ist Zeit. Die Mutter ist schon im Saal,''} he shouted in his kind German voice, then he went over to me, sat by my feet, and took his snuff box out of his pocket.\footnote{Up, children, up! 'Tis time. Mother is already in the hall. \textit{Ger.}} I pretended I was still sleeping. Karl Ivanych took snuff, wiped his nose, snapped his fingers, and then started in on me. Laughing, he started to tickle the soles of my feet. \textit{``Nu, nun, Faulenzer,''} he said.\footnote{Well, now, lazy boy! \textit{Ger.}} %karl

As much I dreaded the tickling, I did not jump out of bed and did not answer him, instead hiding my head deeper under the pillows, kicking my feet as hard as I could and doing everything I could to keep from laughing.

``He's so kind and loves us so much, and I was thinking such awful things about him!'' %nikolai

I was annoyed both at myself and at Karl Ivanych, wanting to laugh and wanting to cry: my nerves were unsettled.

\textit{``Ach, lassen sie, Karl Ivanych!''}~I shouted with tears in my eyes, sticking my head out from under the pillows.\footnote{Ah, leave me alone, Karl Ivanych! \textit{Ger.}} %nikolai

Karl Ivanych seemed surprised, left the soles of my feet alone, and started to ask with concern: what was it? had I had a nightmare?\ldots{} My tears came all the more abundantly because of the sympathy with which he tried to guess the reason for them, and because of his kind German face: I was ashamed, and I could not understand how just a minute before I could dislike Karl Ivanych and find his robe, hat, and tassel so disgusting; now, on the contrary, all of this seemed extremely endearing, and even the tassel seemed like clear proof of his kindness. I told him that I was crying because I had had a nightmare that \textit{maman} had died and they were taking her to be buried.\footnote{Tolstoy uses French \textit{papa} and \textit{maman} throughout for the narrator's parents. Because they are used so extensively, I have left them untranslated but also unitalicized except for the first occurrence. --- \textit{Trans.}} I made all this up, because I absolutely could not remember what I had dreamed about that night; but when Karl Ivanych started to comfort and console me, touched by my story, I started thinking that I had really had that nightmare, and my tears started flowing for another reason. 

When Karl Ivanych left me, and I started to pull my stockings onto my little feet, sitting up in bed, my tears had subsided a bit, but the dark thoughts brought on by my made-up dream did not leave me. Uncle Nikolai came in---a small, neat little man, always serious, exact, respectable, \todo{?} and a great friend of Karl Ivanych.\footnote{Uncle Nikolai is a servant, not a relative. Uncle is used here as a title to indicate that he is responsible for looking after the children. --- \textit{Trans.}} He brought our clothes and shoes: boots for Volodya, for me an absolutely intolerable pair of children's shoes with bows. I was ashamed to cry in front of him; besides, the morning sun was shining through the window, and Volodya, doing an imitation of Marya Ivanovna (our sisters' governess), was laughing so happily and sonorously, standing over the wash stand, that even serious old Nikolai, a towel over his shoulder, with soap in one hand and a washbasin in the other, said, smiling:

``That's enough, Vladimir Petrovich, kindly wash up.'' %unclenikolai

I cheered up completely.

\textit{``Sind sie bald fertig?''} Karl Ivanych's voice came from the classroom.\footnote{Are you almost ready? \textit{Ger.}} %karl

His voice was strict and no longer had the same tone of kindness that had touched me and brought me to tears. In the classroom, Karl Ivanych was an entirely different person: he was a schoolmaster. I dressed quickly, washed up, and, with a brush in my hand, smoothing my wet hair, I answered his call.

Karl Ivanych, with glasses on his nose and a book in his hand, sat in his usual place, between the door and the window. To the left of the door were two little shelves: one was ours, the children's, the other was Karl Ivanych's, his \emph{personal} shelf. On ours were all sorts of books---educational and non-educational: some were standing up, others were lying down. Only the two large volumes of the \textit{Histoire des voyages} in red covers were standing straight up next to the wall;\footnote{History of the Voyages. \textit{Fr.} The reference is to the \textit{Histoire g\'en\'erale des voyages}, a collection of accounts of voyages to the Australia and the Dutch East Indies collected and translated into French by Abb\'e Pr\'evost. --- \textit{Trans.}} then there were long, thick, large, and small books---covers without books and books without covers; we used to push and shove them in there before recess, when we were told to put the library, as Karl Ivanych loudly called that shelf, in order. The collection of books on his \emph{personal} shelf, if not as large as the one on ours, was even more varied in composition. From those books, I remember three: a German brochure on the manuring of cabbage gardens, without a cover, one volume containing a history of the Seven Years' War in parchment, burnt at one corner, and a full course on hydrostatics. Karl Ivanych spent the greater part of his time reading, even ruining his eyes in the process; but besides those books and the \textit{Northern Bee}, he read nothing else.

Among the items lying on Karl Ivanych's shelf, one reminds me of him more than anything else. It was a little circle of cardboard, set on a wooden base to which the cardboard circle was attached by way of a few pins. On this circle, a picture was pasted, a caricature of some lady and her hairdresser. Karl Ivanych was very good at pasting things together and had devised this device himself and made it to protect his weak eyes from bright light.

I can see him before me now, a long figure in a cotton dressing-gown with sparse grey hair sticking out from his little red hat. He sits next to a little table, on which the circle with the hairdresser is standing, casting a shadow over his face; in one hand, he holds a book, the other rests on the arm of the chair; near him lies his watch, with a painted hunter on the dial, a checkered handkerchief, a round black snuff box; a green eyeglass case, a set of tongs on a tray. \todo{I think this is right, not translated by the Maudes!} Every item was so sedately, so exactly placed that by their good order alone one could infer that Karl Ivanych's conscience was clean and his soul at peace.

At times, after we had had enough of running around downstairs in the hall, I would sneak up to the classroom on tip-toe---Karl Ivanych would be sitting there alone in his chair, reading one of his favorite books with a peaceful and sublime expression. Sometimes, I would find him in a moment when he was not reading: his glasses would be at the very end of his aquiline nose, his blue eyes, half-closed, would be staring off into the distance with a peculiar expression on his face, and his lips would be smiling sadly. It would be quiet; the only sounds would be his even breathing and the ticking of his watch with the hunter.

At times, he would not notice me, and I would stand at the door and think: poor, poor old man! There are so many of us, we are playing, we are enjoying ourselves, and he is all on his own, and he has no one to caress him. It is true what he says, that he is an orphan. And the story of his life is so terrible! I remember when he told it to Nikolai---it would be terrible to be in his position! And I would feel such pity for him that at times I would come up to him, grab his hand, and say, \textit{``Lieber Karl Ivanych!''}\footnote{Dear Karl Ivanych! \textit{Ger.}} He loved it when I talked to him like that; he would always caress me, and I could tell that he was deeply moved.

On the second wall hung maps, nearly all of them tattered, but skillfully mended by Karl Ivanych's hand. On the third wall, in the middle of which was the door leading downstairs, two rulers hung on one side: one, nicked all over, was ours, the other, a new one, was his \emph{personal} ruler, used more for discipline than for drawing lines; on the other side was a blackboard on which our serious offences were marked with circles and our minor ones with little crosses. To the left of the board was a corner where we had to kneel for punishment.

How I remember that corner! I remember the flap of the stove, the vent in that flap, and the noise that it made when you moved it. At times, I would be kneeling there in the corner, kneeling, kneeling until my knees and back hurt, and I would think: Karl Ivanych has forgotten about me: I suppose he is sitting there peacefully in his soft chair and reading his hydrostatics---and what about me? And I would start to open and shut the vent or pick plaster off of the wall; but if too large a piece suddenly fell on the ground with a loud noise, the fear alone was really worse than any punishment. I would glance over at Karl Ivanych---and he would be sitting quietly with a book in his hand, as if he had not noticed a thing.

In the middle of the room stood a table covered with ragged oilcloth, the corners visible underneath nicked by our pen knives. Around the table were a few stools, unfinished but worn smooth from long use. The final wall was taken up by three little windows. This was the view from them: right under the windows was a road, every pothole, every pebble, every rut of which was long since familiar and dear to me; beyond the road was a little park full of trimmed lilac, behind which a wicker fence could be seen; through the trees, a meadow was visible, with a barn to one side of that and the forest opposite; far away in the forest, the warden's little hut was visible. From the window, to the right, part of the terrace was visible, where the grown-ups would usually sit until dinner. At times, while Karl Ivanych was correcting our dictation, I would look out that side, see mother's black hair, someone's back, and hear vague conversations and laughter from there; then I would get annoyed that I could not be there and think: ``When will I be a grown-up, be finished with studying, and get to sit with the people that I love instead of doing dialogues?'' My annoyance would turn into sadness and, God knows why, or what I was thinking about, but I could not hear a word of Karl Ivanych's complaints about my mistakes. %nikolai

Karl Ivanych took off his robe, put on his blue tail-coat, which had high, gathered shoulders, straightened his tie in front of the mirror, and guided us downstairs to say good morning to mother.

\chapter{Maman} %c2

Mother sat in the drawing room and was pouring tea; she held the teapot with one hand, and with the other held the tap of the samovar; water was overflowing from the teapot and spilling onto the tray. But although she was looking at it intently, she did not notice this, nor did she notice that we had come in.

So many memories of the past emerge when you try to resurrect the features of a loved one; you can only see them vaguely through those memories, which cloud your vision like tears. Those are the tears of the imagination. When I try to remember mother as she was at that time, I see only her brown eyes, which always expressed kindness and love in equal measure, the mole a little below the little curly hairs on her neck, her embroidered white collar, her dry, tender hand, which caressed me so often and which I often kissed; but the overall picture eludes me.

To the left of the sofa stood an old English piano; in front of the piano sat my dark-looking sister, Lyubochka, her pink fingers, just washed in cold water, playing Clementi's etudes with visible effort. She was eleven years old; she went around in a short linen dress and white, lace-trimmed bloomers and could manage octaves only in arpeggio. Near her, half-turned away, sat Marya Ivanovna in a bonnet with pink ribbons, in a light blue Caraco jacket, and her red, angry face took on an even stricter expression as soon as Karl Ivanych walked in. She looked at him menacingly and, not answering his bow, continued to count, tapping her foot: \textit{un, deux, trois, un, deux, trois,} louder and more insistently than before.

Karl Ivanych, not paying even the slightest attention to this as usual, went directly to mother's hand with a German greeting. She came to her senses, shook her head, as if trying to drive away unpleasant thoughts with this gesture, gave her hand to Karl Ivanych and kissed his wrinkled temple at the same time he kissed her hand:

\textit{``Ich danke, lieber Karl Ivanych,''} and, continuing to speak in German, she asked:\footnote{My thanks, dear Karl Ivanych. \textit{Ger.}} %maman

``Did the children sleep well?'' %maman

Karl Ivanych was deaf in one ear, and now, because of the noise from the piano, he could hear nothing at all. He stooped closer to the sofa, leaned with one arm on the table, supporting himself with one leg, and, with a smile that seemed to me the height of sophistication, lifted his hat a little off his head and said:

``You will forgive me, Natalya Nikolayevna?'' %karl

Karl Ivanych, so that his bare head did not catch cold, never took off his little red hat, but he asked forgiveness for this each time he came into the drawing room.

``Put it back on, Karl Ivanych\ldots{} I am asking you, did the children sleep well?'' asked maman fairly loudly, moving closer to him. %maman

But he again heard nothing, covered his bald spot with the little red hat, and smiled even more sweetly.

``Stop for a minute, Mimi,'' said maman to Marya Ivanovna with a smile: ``we can't hear anything.'' %maman

Although her face was already pretty, when mother smiled she became incomparably more beautiful, and everything around her seemed to become brighter. If in the heaviest moments of my life I could have just caught a glimpse of that smile, then I would never have known what sorrow is. It seems to me that what we call beauty lies in the smile alone: if a smile adds charm to a face, then the face is very fine; if it does not change it, then the face is ordinary; and if the smile ruins it, then it is truly ugly.

Having greeted me, maman took my head in both hands and tipped it back, then looked at me fixedly and said:

``Did you cry today?'' %maman

I did not answer. She kissed my eyes and asked in German:

``What were you crying about?'' %maman

When she talked with me in this friendly way, she always spoke in that language, which she had mastered to perfection.

``It was a dream, maman, that's why I was crying,'' I said, remembering in all its detail my made-up dream and shuddering involuntarily at the thought. %nikolai

Karl Ivanych confirmed my words but was silent about the dream. Changing the subject to the weather---a conversation in which Mimi also took part---maman put six pieces of sugar on the tray for some of the more valued servants, stood up, and went over to the lace frame that stood by the window.

``Well, go to \textit{papa} now, children, and tell him he needs to come see me before he goes out to the barn.'' %maman

The music, the counting, and the menacing glances began again, and we went to papa. Walking through another room, which had been called the footman's pantry since my grandfather's day, we went into his study.

\chapter{Papa} %c3

He was standing by his desk and, pointing at some envelopes, papers, and piles of money, spoke heatedly and explained something angrily to the steward, Yakov Mikhaylov, who was standing in his usual place between the door and the barometer, his hands placed behind his back, and was quickly wriggling his fingers back and forth.

The angrier papa got, the faster Yakov's fingers moved: likewise, when papa stopped, the fingers stopped as well; but when Yakov himself started talking, the fingers became extremely agitated and twitched desperately in every direction. It seemed to me that one could guess his secret thoughts by their movements; his face was always calm---it expressed an awareness of his dignity, but also of his subordinate position, in other words: I am right, but do as you please!

Seeing us, papa merely said:

``Just a moment, wait.'' %papa

And he nodded his head at the door, so that one of us would shut it.

``Oh, merciful heavens! What is with you today, Yakov?'' he continued saying to the steward, twitching his shoulder (he had a habit of doing that). ``This envelope, with 800 rubles enclosed\ldots{}'' %papa

Yakov moved beads on the counting frame, sliding over 800, and turned his gaze on some indeterminate point, waiting for what would come next.

``\ldots{} for housekeeping expenses in my absence. Understand? You should get 1,000 rubles from the mill\ldots{} Yes or no? You should get 8,000 back from the Treasury bonds; by your estimates we can sell 7,000 pounds of hay---I will put it out for 45 kopecks---so you should get 3,000; so how much money do you get in the end? 12,000\ldots{} Yes or no?'' %papa

``Yes, exactly, sir,'' said Yakov. %yakov

But from the quickness with which his fingers moved, I could tell he wanted to object; papa stopped him:

``Well, from that money you will send 10,000 to the Council for Petrovskoye. Now, the money in the bank,'' continued papa, while Yakov mixed up the count and put up 21,000 instead of 12,000, ``you will bring that to me and show it as an expense as of today's date.'' Yakov mixed up the count and turned over the counting-frame, apparently showing by this movement that the 21,000 would also be spent. ``The money in this envelope you will send in my name to the address on the outside.'' %yakov

I was standing close to the desk and looked at the inscription. On the envelope was written: ``To Karl Ivanovich Mauer.''

Apparently noticing that I had read something that I did not need to see, papa put his hand on my shoulder and showed me away from the desk with an easy movement. I did not know whether it was a caress or a reproach; in any case, I kissed the large, sinewy hand that was lying on my shoulder.

``Yes, sir,'' said Yakov. ``And what orders are there for the Khaborovka money?'' %yakov

Khabarovka was mother's village.

``Leave it in the bank and do not use it for anything, under any circumstances, without my orders.'' %papa

Yakov was silent for a few seconds; then suddenly his fingers began to whirl around with double the speed, and, changing his expression from the one of dull obedience with which he listened to his master's orders to his more characteristic expression of sharp cunning, he moved the counting frame toward him and began to speak:

``Permit me to report, Pyotr Aleksandrych, that it will not be possible to pay even the Council on time as you wish. You were so kind as to say,'' he continued after a pause, ``that we should receive money from the bonds, from the mill, and from the hay\ldots{}'' Counting out all of these items, he slid beads over to show each one. ``But I am afraid we may be mistaken in our estimates,'' he added, falling silent for a moment and looking gravely at papa. %yakov

``Why?'' %papa

``If you would be so kind as to see: regarding the mill, the miller has come to me twice already to ask for a deferment and swore by Christ our Lord that he did not have the money\ldots{} And he is actually here now: perhaps you would like to speak with him yourself?'' %yakov

``What does he have to say?'' asked papa, shaking his head to show that he did not want to speak with the miller. %papa

``Isn't it obvious? He says there was nothing to mill, and what little money there was, he had to put it all into the dam. So, if we take the mill away from him, \emph{sir}, do you think we will make our numbers then? You were so kind as to speak about the bonds, well, I believe I already reported to you that once we put our money in them, we could not get it back soon. The other day, I sent a cart of flour to Ivan Afanasyich in the city, along with a note about this matter: he answered the same as before---I am happy to make an effort for Pyotr Aleksandrych, but the matter is not in my hands---and that everything suggests it will be two months or more until our check comes. You were so kind as to speak about the hay, let us assume that it can sell for 3,000\ldots{}'' %yakov

He slid 3,000 over on the counting frame and was silent for a moment, looking first at the counting frame, then into papa's eyes, with an expression that said:

``You can see for yourself how little this is! And we'll take a loss on the hay in any case if we sell it right now, you know that yourself\ldots{}'' %yakov

It was clear that had a large reserve of such arguments left; it must have been for this reason that papa interrupted him.

``I am not going to change my orders,'' he said, ``but if there is, in fact, any delay in receiving this money, then there's nothing that can be done, take whatever is needed from the Khaborovka money.'' %papa

``Yes, sir.'' %yakov

It was obvious from Yakov's fingers and the expression on his face that this last order gave him great pleasure.

Yakov was a serf, an extremely diligent and devoted person; like all good stewards, he was tightfisted in the extreme on his master's behalf and had the strangest conceptions about his master's interests. He was perpetually looking after profit for his master's estate at the expense of his mistress's estate, always trying to prove that it was unavoidable to spend the revenues from her property on Petrovskoye (the village in which we lived). At the present moment, he was triumphant, because he had succeeded perfectly in this.

Having greeted us, papa told us what lazybones we had become out in the country, that we were no longer little, and that it was time for us to study seriously.

``You already know, I think, that I am going to Moscow tonight, and I am taking you with me,'' he said. ``You are going to live at your grandmother's, and maman and the girls will stay here. And you know that only one thing will make me pleased---and that is to hear that you are studying hard and that they are satisfied with you there.'' %papa

Although the preparations that we had noticed for a few days had prepared us for something unusual to happen, this news nevertheless affected us terribly. Volodya turned red and with a quivering voice gave him mother's message.

``Then this is what my dream was predicting!'' I thought. ``Please God, don't let anything worse happen.'' %nikolai

I was very, very sad for mother, but, at the same time, the thought that we had become grown-ups made me happy.

``If we are going today, then surely there won't be any classes: that is nice!'' I thought. ``But I am sad for Karl Ivanych. They will probably let him go, or else they wouldn't have made that envelope for him\ldots{} It would be better to have to study for a century but not go away, not leave mother behind, and not treat poor Karl Ivanych so badly. He is already so unhappy!'' %nikolai

These thoughts flashed through my head; I did not move from the spot and looked fixedly at the black bows on my shoes.

Having said a few words with Karl Ivanych about the falling barometer and having ordered Yakov not to feed the dogs, so he could ride out after dinner on a farewell hunt with the young hounds, papa sent us to study, contrary to my expectations, consoling us, however, with a promise to take us on the hunt.

On the way upstairs, I ran out onto the terrace. Near the doors, lying in the sun and squinting her eyes, lay father's favorite borzoi dog, Milka.

``Milochka,'' I said, caressing her and kissing her snout, ``we are going today; goodbye! We'll never see each other again.'' %nikolai

I was deeply affected by this and cried.

\chapter{Classes} %c4

Karl Ivanych was already in a bad mood. This was obvious from his furrowed brow, and from the way he had thrown his frock coat in the bureau, \todo{?} and how he angrily belted up his robe, and how forcefully he had dug his nail into the place in the dialogue book where the passage that we were supposed to memorize ended. Volodya made a decent effort at studying; I, on the other hand, was so upset that I could not do anything at all. For a long time, I sat gazing dumbly at the dialogue book, but because of the tears welling up in my eyes at the thought of our imminent parting, I could not read; when the time came for me to repeat them to Karl Ivanych, who listened to me, squinting (this was a bad sign), I came to the place where one person says, \textit{``Wo kommen sie her?''} and the other answers: \textit{``ich komme vom Kaffe-Hause,''} and I could not hold back my tears any longer, unable because of my sobbing to pronounce the words: \textit{``Haben sie die Zeitung nicht gelesen?''}\footnote{Where are you coming from? I am coming from the cafe. Have you read the newspaper? \textit{Ger.}} When we turned to penmanship, I left so many blots on the page because of the tears I let fall that it was as if I was writing with water on parcel paper.

Karl Ivanych got angry, made me kneel in the corner, said over and over that I was being obstinate, that I was playing a farce (this was his favorite phrase), threatened me with the ruler, and demanded that I ask his pardon when I could barely get out a word for crying; finally, seemingly feeling the injustice of what he was doing, he went into Nikolai's room and slammed the door.

From the classroom, I could hear the conversation in Uncle Nikolai's room.

``You heard, Nikolai, that the children are going to Moscow?'' said Karl Ivanych, entering the room. %karl

``I heard, of course.'' %unclenikolai

Nikolai must have been about to stand up, because Karl Ivanych said, ``Sit, Nikolai!'' and shut the door right after that. I left the corner and went over to the door to eavesdrop.

``No matter what nice things you do to people, no matter how attached you are becoming, you obviously cannot be expecting gratitude, eh, Nikolai?'' said Karl Ivanych with feeling. %karl

Nikolai, sitting by the window, repairing shoes, nodded his head.

``I am living in this house twelve years and can say before God, Nikolai,'' continued Karl Ivanych, lifting his eyes and his snuff box toward the ceiling, ``that I have loved them and helped them more than as if they were my own children. You remember, Nikolai, when Volodyenka had the fever, you remember how I sat for nine days by his bed, not closing my eyes. Yes: then I was nice, kind Karl Ivanych, then I was needed; but now,'' he added, smiling ironically. ``Now \emph{the children have become grown-ups: they need to study seriously.} As if they don't study anything here, right, Nikolai?'' %karl

``Of course they do,'' said Nikolai, putting down his awl and spreading out the thread with both hands. %unclenikolai

``Well, now I have become not needed, they need to throw me out; and where are the promises? where is the gratitude? I love and respect Natalya Nikolayevna, Nikolai,'' he said, placing his hand on his chest, ``but what does she have to do with it\ldots{}? They don't care about her desires in this house, that's it.'' At this he threw a scrap of leather onto the floor with a significant gesture. ``I know whose tricks these are and why I have become not needed: because I am not always flattering and conniving like \emph{certain people}. I am used to always and in front of everyone telling the truth,'' he said proudly. ``Forget about them, for God's sake! They are not getting any richer because I will not be here, and I will find my bite of bread, if God is merciful\ldots{} Don't you think, Nikolai?'' %karl

Nikolai lifted his head and looked at Karl Ivanych as if to ascertain whether he really could find his bite of bread somewhere else---but he said nothing.

Karl Ivanych went on in this vein for a long time: he talked about how they knew how to value his services much better in some general's house where he had lived before (it was very painful for me to hear that), talked about Saxony, about his parents, about his friend, the tailor Sch\"onheit, etc., etc.

I empathized with him in his sorrow, and it pained me that father and Karl Ivanych, whom I loved almost equally, did not understand one another; I again went into the corner, got down on my knees, and debated with myself about how to re-establish harmony between them.

Returning to the classroom, Karl Ivanych told me to stand up and prepare my notebook for writing from dictation. When everything was ready, he lowered himself majestically into his chair and, in a voice that seemed to come from a great depth, began to dictate the following: \textit{``Von al-len Lei-den-schaf-ten die grau-samste ist\ldots{} haben sie geschrieben?''}\footnote{Of all the passions, the cruelest is\ldots{} Have you written that? \textit{Ger.}} Here he stopped, slowly took snuff, and continued with new strength: \textit{``die grausamste ist die Un-dank-bar-keit\ldots{}''}\footnote{The cruelest is ingratitude\ldots{} \textit{Ger.}} Waiting for the continuation, having written the last word, I looked at him. %karl

\textit{``Punctum,''} he said with a barely noticeable smile and gestured for us to give him our notebooks.\footnote{Period. \textit{Lat.}} %karl

He read this aphorism several times with different intonations and with an expression of the greatest pleasure, expressing as it did his heartfelt thought about the situation; then he gave us a history lesson and sat down by the window. His face was not sullen as before; it expressed the contentment of a person who had gotten suitable revenge for a wrong committed against him.

It was a quarter before the hour; but Karl Ivanych, it seemed, was not thinking about letting us go: he was constantly giving us new lessons. Boredom and our appetites increased in equal measure. With the greatest impatience I followed the signs that suggested dinner was near. First a servant girl with a washcloth coming to wash the plates, then the sound of dishes in the dining room, the table being moved and the chairs begin dragged into place, then Mimi with Lyubochka and Katyenka (Katyenka was Mimi's twelve year old daughter) coming in from the garden; but Foka was nowhere to be seen---Foka, the butler, always came in and announced that the food was ready. Only then could we throw down our books and, paying no attention to Karl Ivanych, run downstairs.

Just then we heard steps on the stairs; but it was not Foka! I had learned his gait and could always recognize the creaking of his boots. The door opened, and a figure appeared that was completely unknown to me.

\chapter{The Holy Fool} %c5

A man of about fifty entered the room; he had a pale, pockmarked, elongated face, long, grey hair, and a sparse, reddish beard. He was so tall that, in order to pass through the door, he not only had to bend his head down but bend at the waist as well. He wore something tattered that looked like a kaftan worn over a cassock; in his hand, he held an enormous staff.\footnote{The Russian kaftan was a coat or gown with long sleeves and a high collar, which buttoned down the front, very different from the garments that go by the name in other countries. --- \textit{Trans.}} Entering the room, he knocked on the floor as hard as he could with it, and, his brows crooked and his mouth open outrageously wide, he roared with the most frightening and unnatural laughter. One of his eyes was crooked, and the white pupil of that eye danced incessantly and gave his already unattractive face an even more disgusting \todo{?} appearance.

``Aha! You're caught!'' he cried, running up to Volodya with short little steps, and grabbed him by the head, thoroughly inspecting it---then, with a perfectly serious expression he stepped back, went up to the table, and began to blow under the oilcloth and make the sign of the cross. ``Oh, misery! Oh, pain\ldots{}! My dearest ones\ldots{} they're flying away,'' he said, his voice trembling from tears, looking passionately and intently at Volodya, and started to wipe away his tears, which were indeed falling, with his sleeve. %grisha

His voice was hoarse and rough, his movements hurried and erratic, his speech nonsensical and rambling (he never used pronouns), but the cadence was so affecting, and his yellow, deformed face sometimes took on such an open, sorrowful expression, that, listening to him, it was impossible not to feel a mixture of pity, fear, and sorrow.

This was Grisha, a holy fool and pilgrim.

Where was he from? Who were his parents? What compelled him to choose the pilgrim's life that he led? No one knew. All I know is that since 1815 \todo{?} he was known as a holy fool, who winter and summer went around barefoot, visiting monasteries, giving little images of saints \todo{better than icons?} to those he came to love, and saying mysterious things, which certain people took as prophecies, that no one had ever known him in any other condition, that every now and then he would go to see grandmother, and that some said he was the unhappy son of rich parents and was pure of spirit, while others said he was just a peasant and an idler.

Finally, the long awaited and punctual Foka appeared, and we went downstairs. Grisha, sobbing and continuing to speak all kinds of absurdities, came after us and knocking his crutch against the steps of the stairs. Papa and maman were walking around the drawing room arm-in-arm and talking quietly about something. Marya Ivanovna sat sedately on one of the chairs that were arranged symmetrically at right angles to the sofa, and, with a strict but restrained voice gave instruction to the girls, who were next to her. \todo{look over that sentence again} As soon as Karl Ivanych entered the room, she looked at him and immediately turned away, and her face took on an expression that can be described as saying: I do not notice you, Karl Ivanych. By the girls' eyes it was obvious that they wanted badly to give us some very important piece of news; but to jump up from their places and go over to us would have been a violation of Mimi's rules. We first had to go over to her, say, ``Bonjour, Mimi!'' \todo{punct.} shuffle our feet, and only then were we allowed to enter into conversations.

What an intolerable individual that Mimi was! It always seemed that no topic was permissible around her: she found everything indecent. On top of that, she was forever nagging us: \textit{``parlez donc fran\c cais,''} and it was our bad luck that dinner was precisely where we wanted to chatter away in Russian; and just as you were digging into a dish and wanted to be left alone, then, without fail, it would be, \textit{``mangez donc avec du pain,''} or, \textit{``comment ce que vous tenez votre fourchette?''}\footnote{Speak French \ldots{} eat with bread \ldots{} how are you holding your fork? \textit{Fr.}} \todo{Added graf break here, not in original.} %mimi

``What business is it of hers?'' I would think. ``Let her teach her girls, we have Karl Ivanych for that.'' I shared his hatred for \emph{certain people} completely.

``Ask mama if you can take us along on the hunt,'' said Katyenka in a whisper, pulling on my jacket to stop me as the grown-ups were going ahead into the dining room. %katyenka

``Alright, we'll try.'' %nikolai

Grisha ate in the dining room, but at a special little table; he did not raise his eyes from his plate, sighed from time to time, made frightening faces, and said, as if talking to himself, ``A pity! She flew away\ldots{} The dove will fly away into the heavens\ldots{} oh, the stone on the grave!'' and so on. %grisha

Maman had been out of sorts since morning; Grisha's words and actions noticeably deepened her sad disposition. 

``Oh, yes, I almost forgot to ask you about one thing,'' she said, handing father a dish filled with soup. %maman

``What was it?'' %papa

``Tell them to lock up your frightening dogs, please; they very nearly bit poor Grisha when he came through the yard. They could attack the children, too.'' %maman

Hearing that they were talking about him, Grisha turned to the table, started showing everyone the tattered ends of his clothes, and said, while chewing:

``Wanted them to tear me to pieces\ldots{} God would not allow it. A sin to let the dogs torment me! A great sin! Don't beat the big man,\footnote{He called all men this, without distinction. --- \textit{LNT.}} what's a beating? God forgives\ldots{} Not in these days.'' %grisha

``What is he saying?'' asked papa, examining him with a fixed and severe gaze. ``I don't understand anything.'' %papa

``I understand him,'' answered maman: ``He told me that some hunter set his dogs on him on purpose, so he's saying, `He wanted them to tear me to pieces, but God would not allow it, and asks you not to punish him for it.''' %maman

``Ah! So that's it!'' said papa. ``How does he know that I want to punish that hunter?'' He then continued in French: ``You know that I am not very enthusiastic about these holy men, but that one especially bothers me, and he ought to\ldots{}'' %papa

``Ah, don't say that, my friend,'' maman interrupted him, as if frightened by something, ``what could you know about him?'' %maman

``Well, I suppose I've had the chance to study their sort of people---so many of them come to you---they're all alike. Always the same story\ldots{}'' %papa

It was clear that mother had an entirely different opinion about that and did not want to argue.

``Pass me that pie, please,'' she said. ``Well, are things going well today?'' %maman

``No, it makes me angry,'' papa continued, picking up the pie, but holding it at such a distance that maman could not reach it, ``no, it makes me angry when I see intelligent and educated people being taken in by lies.'' %papa

And he struck the table with his fork.

``I asked you to pass me the pie,'' she repeated, reaching out her hand.

``And it is wonderful,'' continued papa, moving his hand away, ``that the people like that get picked up by the police. The only good they do is upset the weak nerves of certain individuals,'' he added with a smile, noting that the conversation was bothering mother a great deal, and gave her the pie.

``I'll say only this: it's difficult to believe that a person who, in spite of his sixty years, walks around winter and summer barefoot and wears seventy pounds of chains under his clothing, never taking them off, and who more than once has turned down offers to live peacefully with all his needs met---it's difficult to believe that a person would do all of that just out of laziness.'' %maman

``Concerning the prophecies,'' she added with a sigh, falling silent for a moment, ``\textit{je suis pay\'ee pour y croire};\footnote{I have paid to be believed. \textit{Fr.}} \todo{check this French translation.} I'm sure I've told you how Kiryusha prophesied my late papa's death to the day and the hour.'' %maman

``Ah, what have you done to me!'' said papa, smiling and placing his hand by his mouth on the side where Mimi was sitting. (When he did this, I always listened with rapt attention, expecting something funny.) ``Why did you remind me of his feet? I just looked at them and now I won't eat anything.'' %papa

Dinner was coming to an end. Lyubochka and Katyenka were winking at us incessantly, turning in their chairs, and generally expressing serious excitement. Their winking meant: ``why aren't you asking if you can take us along on the hunt?'' I jabbed Volodya with my elbow, Volodya jabbed back, and finally we decided: first with a timid voice, then fairly firmly and loudly, he explained that, since we were going to leave that day, we would like it if the girls could go on the hunt with us in the brake. \todo{anachronism? or do I need to say shooting brake?} After a short conference among the grown-ups, the question was decided in our favor and---what was all the more pleasant---maman said that she herself would come with us.

\chapter{Preparation for the Hunt} %c6

During dessert, Yakov was called in and ordered to bring up the brake, the dogs, and the saddle horses---all in the greatest detail, calling each horse by name. Volodya's horse had a limp; papa gave orders for a hunter to be saddled for him. That word, ``hunter horse,'' sounded strange in maman's ears: it seemed to her that a hunter horse must be some kind of wild beast, and that it would certainly bolt and kill Volodya. Despite assurances from papa and Volodya, who with astonishing bravado said that it was nothing to him and that he loved it when a horse bolted, poor maman repeated over and over that she would be suffering the entire trip.

Dinner came to an end; the grown-ups went into the study to drink coffee, and we ran into the garden, shuffling our feet on the paths, which were covered with fallen, yellow leaves, and talking. We began conversations about the fact that Volodya would be riding a hunter horse, about how it was shameful that Lyubochka ran more slowly than Katyenka, about how interesting it would be to see Grisha's chains, and so on; about the fact that we would soon be parting not a word was said. Our conversation was interrupted by the clatter of the approaching brake, a house boy sitting on every spring. Behind the brake came the hunters with the dogs, and behind the hunters, the coachman, Ignat, on the horse designated for Volodya, leading my ancient nag. At first we rush to the fence, where we could see all of these interesting things, then with a shriek and a rush of feet, we ran upstairs to dress, and in fact to dress as much as possible like the hunters. One of the most important means for doing this was to stuff our trousers into our boots. Without the slightest delay, we got to work on this, hurrying to get done with it and run out to the front steps to revel in the sight of the dogs, the horses, and conversation with the hunters.

The day was hot. Strangely shaped white clouds had been gathering on the horizon since morning; then a little breeze began to drive them closer and closer, so that they covered the sun from time to time. No matter how close the clouds came or how dark they got, they were clearly not fated to become a storm and ruin our pleasure one last time. Toward evening, they started to break apart again: some turned white, elongated, and ran out to the horizon; others, just above our heads, turned into translucent white fluff; a single large black cloud remaind in the east. Karl Ivanych always knew where clouds like that would go; he explained that that cloud would go to Maslovka, that there would be no rain, and that the weather would be fine.

Foka, despite his advancing years, ran down the stairs very deftly and quickly, shouting, ``Let's go!'' and, setting his legs apart, stood firmly in the center of the porch, right between the spot where the coachman was supposed to bring up the brake and the threshold, positioned like a man who did not need to be reminded of his duty. The ladies came down and, after a short debate over who would sit on which side and hold on to whom (although it seemed to me there was no need to hold on to anyone), they found seats, opened their parasols, and were off. When the brake started moving, maman, pointing at the ``hunter horse,'' asked the coachman in a quivering voice:

``Is that the horse for Vladimir Petrovich?'' %maman

And when the coachman answered in the affirmative, she waved her hand dismissively and turned away. I felt extremely impatient: I mounted my horse, looked between its ears, and performed various evolutions in the yard.

``If you would be so kind, don't crush the dogs,'' one of the hunters said to me. %hunter

``Calm yourself: it's not my first time,'' I answered proudly. %nikolai

Volodya sat on the ``hunter horse,'' not without a small shudder, despite his strength of character, and, stroking its head, he asked several times:

``Is it gentle?'' %volodya

He looked very fine on a horse---just like a grown-up. His thighs sat so well in the saddle that I was jealous---especially because, as far as I could judge by my shadow, I looked less splendid by far.

Then we heard papa's steps on the stairs; the whipper-in drove together the hounds, which had scattered; the hunters with their borzois called theirs in and started to mount. The huntsman guided his horse up to the steps; the dogs from papa's pack, which had previously lain in various picturesque poses near the horse, ran toward him. After him came Milka, wearing a beaded collar and shaking her bell. She ran happily out as always to greet the kennel dogs: she played with one group of them, searched some for fleas, and smelled and growled at the others.

Papa mounted his horse, and we rode off.

\chapter{The Hunt} %c7

The whip, who was called the Turk, \todo{or the Ibrik? strange} rode in front of the rest on a blue roan with a hooked nose. He was wearing a fur hat and had an enormous horn slung over his shoulder and a knife on his belt. From this person's dark, grim appearance, one could easily think that he was on his way to a battle to the death, not a hunt. Near the hind legs of his horse, in a motley, excited tangle, ran the leashed hounds. A pitiful fate awaited any dog that decided to fall back. It required great effort to pull back its leash-mate, and when it had succeeded, one of the whippers-in riding behind would certainly slap her with the quirt, scolding her: ``back into the pack!'' Riding out through the gates, papa ordered the hunters and us to ride along the road while he himself turned into the rye field.

The grain harvest was in full swing. The endless, bright yellow field was enclosed on only one side by a tall, deep-blue forest, which at that moment seemed to me a far-off, mystical place, beyond which the earth must end, or some undiscovered lands begin. In the tall, thick rye, the bent back of a reaper woman could be seen in one of the cleared strips, the ears of grain flapping as she moved them back and forth between her fingers, another woman bent over a cradle, the sheaves scattered all over the stubble alongside the cornflowers. On the other side, men were standing in carts wearing only their long shirts, stacking the sheaves and shaking their dust onto the scorching-hot field. The foreman, in boots and a coat thrown over his shoulders, a tally stick in his hand, noticed papa from afar, took off his felt hat, wiped his red hair and beard with a towel, and shouted at the women. The chestnut horse that papa was riding trotted playfully, from time to time dropping its head to its chest to tug at the reins, and using its tail to brush away the flies that were hungrily clinging to it. Two borzoi dogs, curling their tails raptly like sickles and stepping high, jumped gracefully through the tall stubble behind the horses' feet; Milka ran forward and, curling up her head, waited for a treat. The hum of everyone's voices, the clatter of the horses and the carts, the happy singing of the quail, the buzzing of the insects that crowded the air, unmoving, the smell of the wormwood, \todo{?} straw, and horse sweat, a thousand different hues and shadows that were bathed in the light of the burning sun across the light yellow stubble, the deep blue expanse of the forest and the pale violet clouds, the white spider webs that floated in the air or lay on the stubble---all of this I saw, heard, and felt.

Approaching the Kalinovo Forest, \todo{?} we found the brake already there and, beyond all expectation, a one-horse cart as well, in the middle of which sat a server. Underneath the straw, we could see a samovar, a tub with a mold for making ice cream, and still more enticing little bundles and boxes. There was no mistaking it: it was tea in the open air, ice cream, and fruits. At the sight of the cart, we noisily expressed our happiness, because drinking tea in the forest on the grass, in a place where no one had ever, ever drunk tea, seemed like the greatest delight.

The Turk approached the island of grass, stopped, listened attentively to papa's lengthy instructions about how to divide the group and where to go (however, he never held to those instructions, but did as he pleased), unleashed the dogs, slowly tied the leashes to his saddle, mounted the house, and, whistling, disappeared into the young birches. The unleashed hounds first expressed their pleasure with a wag of their tails, shook themselves, straightened, and then with a slow little trot, sniffing and wagging their tails, ran off in different directions.

``Do you have a handkerchief?'' asked papa. %papa

I took mine out of my pocket and showed it to him.

``Well, tie your handkerchief on that grey dog\ldots{}'' %papa

``Zhiran?'' I said, with a knowing air. %nikolai

``Yes; and go run along the road. When you get to the clearing, stop and look: don't come back to me without a hare!'' %papa

I wrapped my handkerchief around Zhiran's furry neck and threw myself headlong toward my assigned place. Papa laughed and shouted after me:

``Quickly, quickly, or you'll be late.'' %papa

Zhiran kept stopping incessantly, raising his ears and listening to the hunters' calls. I did not have the strength to draw him from his place, and I began to yell: ``Hark! Hark!'' Then Zhiran burst forward with such strength that I could barely hold onto him and fell more than once before we reached our place. Choosing a shady, level place by the roots of a tall oak, I lay down on the grass, sat Zhiran down next to me, and started to wait. My imagination, as often happens on these occasions, ran far ahead of reality: I imagined myself chasing down my third hare while the first hound was just howling in the forest. Turka's voice rang out, loud and animated, throughout the forest; the hound yelped, and her voice was heard more and more often; a different, deeper voice joined it, then a third, a fourth\ldots{} The voices broke off, then broke in on top of one another. The sounds grew progressively stronger and more persistent and finally blended together into one ringing, rumbling din.  \textit{The island was clamorous, and the hounds were boiling over.}\footnote{The ``island'' refers to the unleashed pack of hounds chasing after the hare. --- \textit{Trans.}}

Hearing this, I froze in place. Fixing my eyes on the edge of the forest, I grinned foolishly; sweat poured off of me, and although the drops, running down my chin, tickled me, I did not wipe them off. It seemed to me that no moment could be more decisive than this one. This attitude of intensity was too unnatural to continue for very long. The hounds rumbled along the very edge of the forest, then gradually moved away from me; there was no hare. I began to look around. With Zhiran, it was the same story: at first he was yelping and straining to be let loose, then he lay down beside me, put his muzzle on my knees, and calmed down.

Near the exposed roots of the oak under which I was sitting, along the dry, grey earth, among the dry, fallen oak leaves, the acorns, the shrunken, moss-covered twigs, the yellow-green moss, and the sparse, thin, green grasses, moved swarms upon swarms of ants. One after another, they hurried along the well-worn paths they had blazed: some of them with heavy loads, others traveling light. I picked up a long stick and blocked their way. You should have seen how some of them, not heeding the danger, crawled underneath, others climbed over; and a few, especially those who were carrying loads, were completely lost and did not know what to do: they kept stopping, looking for a way around, or they turned back, or they climbed up the stick to my hand and, it seemed, planned to climb under the sleeve of my jacket. I was distracted from these observations by a butterfly, with yellow wings, which hovered alluringly in front of me. As soon as I turned my attention toward it, it flew about two steps away from me, hovered over a wilted white clover blossom and landed on it. I do not know whether it was being warmed by the sun, or if it was taking nectar from that blossom---I only knew that it seemed to feel very fine. From time to time, it flapped its wings and pressed closer to the blossom, then finally it froze entirely. I laid my head on both hands and watched it with pleasure.

Suddenly, Zhiran howled and took off with such strength that I nearly fell. I looked around. At the edge of the forest, with one ear raised and the other pressed against its head, ran a hare. The blood rushed to my head, and I forgot everything at that moment: I shouted something in a furious voice, released the dog, and started to run. But no sooner had I done that than I began to regret it: the hare hunkered down, hopped once, and then it disappeared.

But I was even more ashamed when, chasing after the hounds and driving them on at the top of his voice, Turka appeared from behind the bushes! He saw my mistake (which lay in not \emph{waiting for the right moment}) and, glanced at me with contempt, said only: ``Ah, sir!'' But you had to hear how it was said! It would have been easier for me if he had hung me on the horn of his saddle like a hare.

I stood on that spot for a long time, despairing and not calling back the dog, just saying to myself as I slapped my thighs:

``My God, what did I do!'' %nikolai

I heard the hounds chasing further along, and the sounds of the island on the other side pursuing the hare, and Turka with his enormous horn calling the dogs---but I did not move from that spot\ldots{}

\chapter{Games} %c8

The hunt came to an end. In the shadows of the young birch trees, we spread a rug, and the whole group sat in a circle on the rug. The server, Gavrilo, having trampled the lush, green grass around him, wiped down the plates and took leaf-wrapped plums and pears out of a box. The sun shone through the green branches of the young birches and threw dancing circles of light onto the patterns of the rug, on my legs, and even on Gavrilo's bald, sweaty head. A light breeze, passing through the leaves of the trees, my hair, and my sweaty face, refreshed me considerably.

Once they gave us our ice cream and fruits, there was nothing to do on the rug, and we got up and went to play, in spite of the slanting, burning rays of the sun.

``Well, what game?'' said Lyubochka, squinting from the sun and hopping on the grass. ``Let's play Robinson.'' %lyubochka

``No, that's boring,'' said Volodya, dropping lazily to the grass and chewing on some leaves. ``You always want to play Robinson! If you want to play something, let's build a gazebo instead.'' %volodya

Volodya was clearly puffing himself up: he was obviously so proud of himself for riding on a hunter horse and was pretending to be very tired. It may also have been that he had too much reason and too little power of imagination to truly enjoy playing Robinson. The game consisted of acting out scenes from \textit{Robinson Suisse}, which we had all read not long before.\footnote{That is, \textit{The Swiss Family Robinson} by Johann David Wyss. --- \textit{Trans.}}

``Oh, please\ldots{} Why don't you want to something to make us happy?'' the girls badgered him. ``You can be \textit{Charles} or \textit{Ernest}, or Father --- whichever one you want?'' said Katyenka, trying to pull him up from the ground by the sleeve of his jacket. %katyenka

``I really don't want to---it's boring!'' said Volodya, stretching out with a self-satisfied smile on his face.

``I wish we were sitting at home then, if no one wants to play,'' Lyubochka said through tears. %lyubochka

She was an awful crybaby.

``Well, let's go; just don't cry, please: I can't stand it!'' %volodya

Volodya's condescension gave us very little pleasure; on the contrary, his lazy and bored attitude took all the charm out of the game. When we sat on the ground and, pretending we were going fishing, began to row as hard as we could, Volodya sat with his arms crossed in a pose that had nothing in common with a fisherman's pose. I pointed this out to him; but he answered that, no matter how much or how little we waved our arms, we would not gain or lose anything and anyhow would not go far. Unfortunately, I had to agree with him. When, pretending that I was going hunting, with a walking stick over my shoulder, I set off for the forest, Volodya lay on his back, threw his hands behind his head, and told me that he was walking, too. Actions and words like that threw cold water on our games and were extremely unpleasant, all the more so because in our hearts we could not but agree that Volodya was acting reasonably.

Even I knew I could not kill a bird with a walking stick, or even shoot at one. It was a game. By that reasoning, you could not ride in a coach made of chairs; but Volodya would remember, I think, how on long winter evenings we used to cover an armchair with scarves, making a carriage out of it, with one of us the driver, the other the footman, the girls in the middle, and three chairs serving as the three horses---and we would go off down the road. What adventures we used to have on that road! And how happily and quickly those winter evenings went by! Judging by reality, no games are possible. And if no games are possible, then what is left?

\chapter{First Love, or Something Like It} %c9

Pretending that she was picking some kind of American fruits from the tree, Lyubochka plucked a worm of enormous size off one of the leaves, dropped it onto the ground with horror, raised her arms up and leapt back, as if afraid something would gush out of it. The game stopped; all of us, heads together, fell on the ground to look at this rare sight.

I looked over Katyenka's shoulder while she tried to lift the worm up on a leaf, which she had placed in its way.

I noticed that lots of girls had the habit of shrugging their shoulders, trying to get the open neck of their dress back into place with the motion. I also remember how Mimi would always get angry at that motion and say: \textit{``c'est un geste de femme de chambre.''}\footnote{That is a chamber maid's gesture. \textit{Fr.}} Bent over the worm, Katyenka made that very same motion, and at the same time the wind lifted her scarf from her little white neck. At the time she made that motion, her shoulder was two fingers away from my lips. I was not looking at the worm any more, I was looking as hard as I could at Katyenka's shoulder, and I kissed it. She did not turn, but I noticed that her neck and ears went red. Volodya, not even lifting his head, said contemptuously:

``Oh, what tenderness!'' %volodya

Tears came to my eyes.

I could not take my eyes off of Katyenka. I had long since gotten used to seeing her fresh little face and blond hair, and I had always loved them; but now I looked at her face more carefully and fell even more in love. When we came back to the grown-ups, papa, to my great joy, announced that, because maman had asked, our departure would be put off until the next day.

We rode back together in the brake. Volodya and I, trying to exceed one another in bravado and horseriding skill, kept prancing next to it. My shadow was longer than before and, judging by it alone, I concluded that I looked the part of an attractive horseman; but this self-satisfied feeling was soon demolished by the following circumstance. Trying to charm everyone sitting in the brake once and for all, I fell back a bit, then, with the help of the whip and my foot, I spurred on my horse, took on an easy, graceful attitude, and tried to rush past them like a whirlwind, on the side where Katyenka was sitting. I just had to decide which was better: to gallop by silently, or to shout something? But my intolerable horse, coming even with the draught horses, despite my every effort, came to a stop so unexpectedly that I came out of my saddle, hitting the horse's neck and nearly flying off of my mount.

\chapter{What Kind of Person Was My Father?} %c10

He was a person of the previous century and had in common with the youth of that century an elusive character that was chivalrous, intrepid, self-assured, courteous, and dissolute. He looked with contempt on people of the present century, and this view came as much from inborn pride as from a secret annoyance that he had not been able to have in our century the influence or the successes that he had had in his own. His two great passions in life we cards and women; he won in the course of his life several millions and had affairs with a countless number of women of all classes.

His large, stately stature, strange walk (tiny little steps), habit of twitching his shoulder, little eyes, always smiling, large, aquiline nose, odd lips, which fit together awkwardly but were pleasant-looking, deficiencies in pronunciation (a lisp), and a bald spot covering nearly his entire head: this was my father's outer appearance, which he knew how to use not only to pass as, and be, a person \textit{\`a bonnes fortunes,} but to please everyone without exception---people of all classes and stations, especially those he particularly wanted to please.

He knew how to take the upper hand in relations with anyone. Never a person of \emph{the very highest society}, he always associated with people of that circle, and was respected by them. He knew that utmost measure of pride and conceit which, without offending others, elevated him in the opinion of society. He was original, but not always, and employed his originality as a means of substituting in various circumstances for breeding or wealth. Nothing on earth could excite any feeling of surprise in him: whatever brilliant situation he found himself in, it seemed that he was born for it. He knew so well how to hide from others and keep at a distance from himself that dark side of life, so well known to all of us, filled with minor annoyances and troubles, that it was difficult not to envy him. He was a connoisseur of all things that give comfort and delight, and knew how to use them. Brilliant connections were his hobby, and he had them in part through my mother's relations, and in part through the comrades of his youth, with whom he was angry for going far up the ranks, while he always remained a retired lieutenant in the Guards. He, like all former military men, did not know how to dress according to fashion; he did, however, dress with originality and refinement. His clothes were always very wide and free, with splendid shirts, large, turned-back cuffs and collars\ldots{} However, it all fit his large stature, strong build, bald head, and calm, self-assured movements. He was sensitive and even prone to crying. Often, when reading aloud, when he came to a passage filled with pathos, his voice would begin to tremble, tears would start to show, and he would drop the book, annoyed. He loved music, and sang various songs, accompanying himself on the piano: romances by his friend A---, gypsy songs, and motifs from operas; but he did not love serious music and, paying no attention to general opinion, would say openly that Beethoven's sonatas aroused boredom and sleep in him, and that he liked nothing better than to hear ``Do not wake me, young lad'' as sung by Semyonova or ``Not alone'' as sung by the gypsy Tanyusha. His nature was one of those that needed a public for a good cause. \todo{?} And he only considered good what the public called good. God only knows whether he had any moral convictions whatsoever. His life was so full of passions of every kind that he had no time to form them, and he was so happy in life that he saw no need of them.

In old age he formed a permanent view on things and unchangeable rules---but solely on a practical basis: those actions and that form of life that provided him happiness or pleasure, he considered good, and found that always and everywhere, people should do exactly those things. He captivated people when he spoke, and that ability, I think, strengthened the flexibility of his rules: he was capable of describing one and the same action as either the dearest little prank, or the lowest dirty trick.

\chapter{Business in the Study and the Drawing Room} %c11

Dusk had already fallen when we arrived home. Maman sat down at the piano and we, the children, brought paper, pencils, paints, and settled in to draw all around the round table. I had only dark-blue paint; but, despite this, I decided to draw the hunt. Having made lively depictions of a blue boy riding a blue horse, and blue dogs, I felt uncertain whether I could draw a blue hare, and I ran to papa in his study to consult him about this. Papa was reading something and answered my question, ``Is there such a thing as a blue hare?'' without raising his head: ``there is, my friend, there is.'' Returning to the round table, I made my depiction of a blue hare, then found it necessary to turn the blue hare into a bush. I did not like the bush, either: I made it into a tree, the tree a haystack, the haystack a cloud, and finally I had so stained the paper with blue paint that I tore it up in annoyance and went to doze in the Voltaire armchair.

Maman played Field's second concerto---he had been her teacher. I dozed, and in my imagination sprang up bright memories, light and limpid. She played a Beethoven sonata, full of pathos, and I experienced memories of something sad, heavy, and dark. Maman played these two pieces often; and so, I remember well the feeling they awoke in me. That feeling was like a memory; but a memory of what? It seemed like you were remembering something that had never been.

 Across from me was the door into the study, and I saw Yakov enter along with some people in kaftans and beards. The door was closed immediately after they went in. ``Well, the business has started!'' I thought. It seemed to me that there must be no greater matters in the world that what transpired in that study; I was confirmed in this thought by the fact that everyone usually approached the doors of the study whispering and on tiptoe; papa's loud voice and the smell of his cigar came through, which always, I do not know why, attracted me greatly. Half-asleep, I was suddenly struck by the loud creak of boots in the footman's pantry. Karl Ivanych, on tiptoe, but with a dark and decisive face, with some kind of notes in his hand, approached the door and knocked lightly. He was let in, and the door again slammed shut.

``I hope nothing unfortunate will happen,'' I thought, ``Karl Ivanych is aggravated: he looks capable of anything\ldots{}'' %nikolai

I dozed off again.

Nothing unfortunate did happen; after an hour, I came to be awoken by the same creaking of boots. \todo{``came to be'' here to reflect the tone of ``cherez chas vremeni,'' possibly a better way out there to do that.} Karl Ivanych, wiping away tears, which I noticed on his cheeks, with his handkerchief, came out through the door and, mumbling something to himself under his breath, went upstairs. Papa came out after him and went into the drawing room.

``Do you know what I've just decided?'' he said in a happy voice, laying his hand on maman's shoulder. %papa

``What, my friend?'' %maman

``I'm going to take Karl Ivanych along with the children. There's room in the britzka.\footnote{A britzka was a long, hooded carriage, similar to a phaeton or chaise. Its greater length allowed for the inclusion of a bed or desk in the body of the carriage. --- \textit{Trans.}} They are used to him, and he is quite attached to them, it seems; and 700 rubles a year is practically nothing, \textit{et puis au fond c'est un tr\`es bon diable.}''\footnote{And in essence he's a very good little devil. \textit{Fr.}} %papa

I could never get why papa called Karl Ivanych names.

``I am very glad,'' maman said, ``for the children, for him: he is a good old man.'' %maman

``If you had seen how touched he was when I told him to keep the 500 rubles as a gift\ldots{} but what is most amusing of all is the accounting he brought me. That is worth seeing,'' he added with a smile, giving her the note, written in Karl Ivanych's own hand, ``it's a delight!'' %papa

Here are the contents of that note:

% sic passim --- Karl Ivanych's broken written Russian
``For the childeren two of fishing rod --- 70 kopecks.

``Color paper, with a gold borders, plester, and a dummy for making boxes, in gifts --- 6 r.~55 k.

``Pant to Nikolai --- 4 rubles.

``Promised by Pyotr Aleksantrovich's from Moscow in the year 18--- a gold watch of 140 rubles.

``In sum it would result for Karl Mauer to receive besides his salaries --- 159 rubles 79 kopecks.'' %karl

Reading this note, in which Karl Ivanych demands to be paid all the money spent by him on gifts, and even to be paid for a promised gift, anyone would think that Karl Ivanych was nothing more than an insensitive and self-interested egoist---and anyone would be wrong.

Entering the study, notes in hand, and with a prepared speech in his head, he had planned to lay out eloquently before papa all the injustices he had suffered in our house; but when he began to speak in the same touching voice and with the same sensitive intonations with which he usually gave us dictation, his eloquence had its strongest effect on his own self; and so, arriving at the place where he said, ``however sad it would be for me to part ways with the children,'' he fell apart completely, his voice trembled, and he was forced to take checkered handkerchief out of his pocket. %karl

``Yes, Pyotr Aleksandrych,'' he said through tears (that part was definitely not in the prepared speech), ``I'm so used to the children that I don't know what I'll do without them. It should be better if I serve you without salary,'' he added, wiping away his tears with one hand and present his accounting with the other. %karl

That Karl Ivanych was speaking sincerely at that moment, I can affirm myself, because I know his kind heart; but how exactly his accounting accorded with his words remains a mystery to me.

``If you are sad, then it would make me even sadder to part ways with you,'' said papa, patting him on the shoulder, ``I've just changed my mind.''

Not long before supper, Grisha came into the room. From the very day he came into our house, he never ceased sighing and weeping, which, for many people who believed in his ability to prophesy, portended some kind of misfortune for our house. He started to say goodbye and informed us that he would move on tomorrow. I winked at Volodya and went through the door.

``What?'' %volodya

``If you want to see Grisha's chains, then let's go right now to the men's quarters upstairs---Grisha sleeps in the second room---we can sit in the storage room and see everything.'' %nikolai

``Excellent! Wait here, I'll call the girls.'' %volodya

The girls came running, and we went upstairs. Not without some argument over who would go into the dark storage room first, we found our seats and started to wait.

\chapter{Grisha} %c12

It felt eerie to us in the dark; we pressed close to one another and said nothing. Almost right behind us came Grisha, walking quietly. In one hand, he held his staff, in the other, a tallow candle in a brass candleholder. Not one of us breathed.

``Lord Jesus Christ! Most Holy Mother of God! Father, Son, and Holy Ghost\ldots{}'' he said over and over, breathing in air, with different intonations and abbreviations particular only to those who frequently repeat those words.

Setting his staff in the corner with a prayer and looking over the bed, he began to undress. Unfastening his ancient black belt, he slowly took off his tattered nankeen coat, smoothing it thoroughly and hanging it on the back of the chair. His face no longer expressed hastiness and stupidity; on the contrary, he was calm, pensive, and even majestic. His movements were slow and deliberate.

Remaining in just his shirt, he quietly lowered himself onto the bed, crossed himself from all sides, and he straightened the chains under his long shirt, obviously---because he winced---with some effort. Having sat for a little and looked attentively at his shirt, torn through in places, he stood up, lifted the candle up to the level of the icon case with a prayer, crossed himself before the several icons that hung there, and turned the flame of the candle upside-down. It went out with a snap.

Through the windows, which faced the forest, the nearly-full moon penetrated. The long, white figure of the holy fool was lit on one side by the pale, silver lunar rays, and on the other was a black shadow; together with the shadows from the window frames, it fell on the floor, walls, and reached to the ceiling. Out in the courtyard, the guard was knocking on his iron slab.

Having laid his enormous arms on his chest and lowered his head, his breathing incessant and hard, Grisha silently stood before the icons, then with difficulty lowered himself to his knees and began to pray.

At first, he quietly said well-known prayers, just stressing certain words, then repeating them, but louder and with great animation. He began to speak his words, with visible effort trying to express himself in Church Slavonic. His words were incoherent, but touching. He prayed for all his benefactors (that is what he called them when they took him in), among them for mother, for us, he prayed for himself, so that God would forgive him his grave sins, repeated: ``O God, forgive our enemies!'' then groaning, he got up, and, repeating over and over the same words, fell to the ground and again got up, despite the weight of his chains, which gave out a dry, sharp sound, striking the earth.

Volodya pinched my leg painfully; but I did not even look around: I merely ran my hand over the spot and continued, with a feeling of childish surprise, pity, and awe, to follow all Grisha's movements and words.

Instead of the fun and laughter that I had been counting on, entering the storage room, I trembled and felt my heart sink.

Grisha remained for a long time in that state of religious ecstacy and improvized prayers. First he would repeat a few times in a row: ``Lord, have mercy,'' but each time with new strength and feeling; then he would say, ``grant me your pardon, O Lord, teach me what to do\ldots{} teach me what to do, O Lord!'' with such feeling, that it was as if he expected an immediate answer to his words; then we would hear only mournful wailing\ldots{} He got up on his knees, laid his arms on his chest, and fell silent. %grisha

I slowly stuck my head out from behind the door and did not breathe. Grisha did not stir; heavy sighs escaped from his chest; in the cloudy pupil of his crooked eye, lit by the moon, stood a tear. 

``Thy will be done!'' he shouted suddenly with unsurpassable feeling, fell with his forehead on the earth, and wept like a baby. %grisha

Much water has flowed since that time, many memories of bygone days have lost their meaning for me and become vague dreams, and even the pilgrim Grisha has long since completed his final pilgrimage; but the impression that he made on me, and the feeling he awoke in me, will never die in my memory.

O great Christian, Grisha! Your faith was so strong that you felt the closeness of God, your love so great that the words flowed on their own from your lips---you did not subject them to reason\ldots{} And what praise you brought to His majesty when, not finding words, you fell to the earth in tears\ldots{}!

The feeling of deep emotion with which I listened to Grisha could not last long, because, first of all, my curiosity was more than satisfied, and second, my leg was worn out from sitting in one place and I wanted to join the whispering and racket of the group, which I heard behind me in the dark storage room. Someone grabbed me by the hand and said in a whisper, ``Whose hand is this?'' It was completely dark in the storage room; but by the contact and the whispering voice alone, right in my ear, I could tell right away it was Katyenka.

Absolutely unconsciously, I grabbed her arm, clad in short sleeves, by the elbow and pressed in toward her with my lips. Katyenka was truly surprised at my action and jerked back her hand: with that motion, she knocked over a broken chair that stood in the storage room. Grisha lifted his head, looked around quietly, and, saying a prayer, began to make the sign of the cross toward all corners of the room. We ran out of the storage room noisily, whispering to one another.

\chapter{Natalya Savishna} %c13

Halfway through the previous century, through the courtyards of the village of Khabarovka ran a barefooted but happy, fat, and red-cheeked girl in a shabby dress, \textit{Natashka}. Because of a request by her father, the clarinetist Savva, and long service, my grandfather brought her \emph{upstairs}---into the number of our grandmother's female servants. The lady's maid \textit{Natashka} distinguished herself in her position by her gentle disposition and her diligence. When my mother was born and needed a nanny, they entrusted that responsibility to \textit{Natashka}. In this new profession as well, she won praises and great rewards for her industry, loyalty, and affection for the young mistress. But the powdered head, stockings, and buckles of the young, sharp footman Foka, who in his service had frequent dealings with Natalya, captured her coarse but loving heart. She even decided to go herself to grandfather to ask for permission to marry Foka. Grandfather took her desire for ingratitude, became angry, and sent poor Natalya to the cattle yard in a village on the steppe as punishment. After 6 months, however, since no one could replace Natalya, she was returned to the courtyard and her previous position. Having returned from exile in her work clothes, she appeared to grandfather, fell before his feet, and begged him to return to her his favor and kindness and forget the foolishness that had come over her and which, she vowed, would never return. And indeed, she kept her word.

From that time, Natashka became Natalya Savishna and wore a matron's bonnet; the entire reserve of love she kept inside her, she transferred onto her young miss.

When a governess replaced her at my mother's side, she received the keys to the pantry, and the linens and all the provisions were given into her hands. She fulfilled these new duties with the same diligence and love. She lived her whole life for the good of the family, and saw waste, spoilage, and misuse everywhere, and used all her powers to try to fight the same.

When maman got married, desiring to show Natalya Savishna her thanks for twenty years of her labors and affection, she called her in and, expressing in the most complimentary terms all her appreciation and love for her, handed her a sheet of stamped paper on which was written Natalya Savishna's manumission, and said that, regardless of whether she would continue to serve in our house or not, she would always receive an annual pension of 300 rubles. Natalya Savishna silently listened to all of this, then, taking the document in her hands, glanced at it with spite, mumbled something through her teeth, and ran out of the room, slamming the door. Not understanding the reason for this strange action, maman, waiting a bit, went into Natalya Savishna's room. She was sitting with teary eyes on her trunk, working a handkerchief in her fingers, and was gazing fixedly at the scraps of her shredded manumission, which were scattered on the floor in front of her.

``What's with you, my dear, dear Natalya Savishna?'' asked maman, taking her by the hand. %maman

``Nothing, precious girl,'' she answered, ``I guess I've disgusted you somehow, since you're running me out of the house\ldots{} Well, I'll go.'' %natalya

She snatched away her hand and, barely holding back from crying, wanted to leave the room. Maman held her back, embraced her, and they both broke down crying.

As early as I can remember anything, I remember Natalya Savishna, her love and affection; but only now can I appreciate them---at the time, I did not have even a thought for how rare and marvelous a creature that old woman was. Not only did she not talk about herself, it seemed she did not think about herself: her entire life was love and self-sacrifice. I became so used to her selfless, tender love for us that I could not even imagine it could be any other way, I was not grateful to her in the least, and I never asked myself the questions: I wonder, is she happy? is she satisfied?

I used to run out of our lesson, under the pretext of unavoidable need, to her room, settle in, and start to daydream aloud, not in the least bothered by her presence. She was always busy with something: either she was mending a stocking, or rummaging in the trunks that filled her room, or taking inventory of the linens and, listening to whatever rubbish I was talking, ``so when I'm a general, I'm going to marry a marvelous beauty, buy myself a chestnut horse, build a glass house, and send for Karl Ivanych's relatives from Saxony,'' and so forth, she would say: ``yes, precious boy, yes.'' Usually, when I was getting up and preparing to leave, she would open a light blue trunk, inside the lid of which---as I now remember it---were glued: a painted representation of some kind of hussar, the picture from a jar of pomade, and one of Volodya's drawings---she would take a piece of incense out of that trunk, light it, and waving it around, would say:

``This is Ochakov incense, precious. When your late grandpa---God rest his soul---went to fight the Turk, he brought this back with him. This here is one of my last pieces,'' she would add with a sigh. %natalya

In the trunks that filled her room was absolutely everything. Whatever might be necessary, people usually said, ``You should ask Natalya Savishna,'' and, truly, after rummaging a bit, she would find the needed object and would say, ``Here, it's a good thing I saved this back.'' In these trunks were thousands of objects of the sort no one in the house except her worried about or knew. %natalya

One time I was angry with her. Here is what happened. At dinner, pouring myself some kvass, I dropped the carafe and spilled on the table-cloth.

``Go call Natalya Savishna, so she can enjoy what her little favorite just did,'' said maman.

Natalya Savishna came in, and, seeing the puddle that I had made, shook her head; then maman whispered something in her ear, and she, giving me a threatening look, left the room.

After dinner, I was going going into the hall, skipping, in a cheerful mood, when suddenly Natalya Savishna jumped out from behind the door with the table-cloth in her hand, grabbed me, and, regardless of desperate resistance on my part, started rubbing the wet spot in my face, saying, ``Don't you dare stain the table-cloth, don't you dare stain the table-cloth!'' This offended me so much that I started to howl from fury. %natalya

``How could she!'' I said to myself, walking around the hall and choking back tears, ``Natalya Savishna, just \emph{Natalya}, talking to me like that, ordering me around, and even hitting me in the face with that wet table-cloth like a servant boy. No, it's horrible!''\footnote{In Tolstoy's Russian text, Natalya Savishna commands Nikolai using the informal form of \textit{you} \textit{(ty)} rather than the formal \textit{(vy)}. Servants typically used the formal even with their masters' children. It's this informality that offends Nikolai in the original. --- \textit{Trans.}}

When Natalya Savishna saw that I was starting to slobber, she immediately ran off, and I, continuing to walk around, thought about how I could pay back that impudent \emph{Natalya} for the abuse she had caused me.

After a few minutes, Natalya Savishna returned, came up to me timidly, and started to admonish me.

``Enough, my precious boy, don't cry\ldots{} Forgive me, I'm an old fool\ldots{} It's my fault\ldots{} Forgive me, now, dearie\ldots{} Here, take it.'' %natalya

She pulled out a cone out of her scarf, made of red paper, which had two caramels and a dried fig in it, and handed them to me with trembling hand. I did not have sufficient strength to look the kind old woman in the eye; I took the present, turning away, and my tears flowed even more abundantly, no longer out of fury now, but out of love and shame.

\chapter{Parting} %c14

On the next day, after the events I described, at twelve o'clock in the morning, a barouche and a britzka stood by the porch. Nikolai was dressed for the road, that is, his trousers were tucked into his boots and his belt was wrapped tight-tight around his old frock coat. He stood in the britzka and was putting overcoats and pillows away underneath the seating; when it seemed too high to him, he would sit down on the pillows and, hopping up and down, crammed them down.

``Do me a little favor, Nikolai Dmitrych, would you find a place for the master's little case,'' said papa's breathless valet, leaning out of the carriage, ``it's quite small\ldots{}'' %valet

``I wish you had said so earlier, Mikhei Ivanych,'' answered Nikolai rapidly and with annoyance, throwing some parcel onto the bottom of the britzka with all his might. ``Oh God, my head is spinning already, and here you come with your little cases,'' he added, lifting his peaked cap and wiping big drops of sweat from his tanned forehead. %nikolai

The servant men in their frock coats, kaftans, and shirts, without hats, the women, in their work clothes, striped scarves, with babies in their arms, and the barefooted kids all stood by the front steps, looking at the carriages and talking among themselves. One of the drivers---a hunched old man in a winter hat and a rough, hooded cloak---held the barouche pole in his hand, pushed it this way and that, and thought deeply and carefully about how the carriage would ride; the other---a conspicuous young fellow wearing only a long white shirt with Turkey red gussets, in a black felt hat that looked like a dinner plate, which he kept knocking first toward one ear, then the other, scratching his blond curls---laid his own cloak on the coach box, tossed the reins there as well, and, flipping a little, braided whip back and forth, looked first at his boots, then at the coachmen who were greasing the britzka. One of them, straining, held the carriage; \todo{close?} the other, bent over the wheel, thoroughly greased the axle and bushing---in fact, so that no leftover tar was wasted on the brush, he even dabbed some on the bottom, making circles. \todo{?} The worn-out post horses, of all different colors, stood near a railing and swatted at flies with their tails. Some of them, spreading their shaggy, swollen legs, squinted their eyes and dozed off; others, out of boredom, scratched one another or nibbled on leaves or the tough stalks of dark-green ferns that grew near the front steps. A few of the borzoi dogs---some breathed heavily, lying in the sun, others walked in the shade under the barouche and the britzka and licked the fat near the axles. Everywhere in the air was a kind of dusty haze, the horizon was a grey-violet color; but there was not a cloud in the sky. A strong western wind lifted columns of dust from the roads and paths, bent the tops of the tall linden trees and birches in the garden, and carried the fallen yellow leaves far away. I sat by the window and waited impatiently for the end of all the preparations.

When everyone had gathered in the drawing room around the round table, so that we could spend a few minutes together one last time, it never even crossed my mind that a sad moment awaited us. Completely empty thoughts ran through my mind. I asked myself various questions: which driver would go in the britzka and which one in the barouche? who would go with papa, and who with Karl Ivanych? and why did they want to make certain I was wrapped up in a scarf and a padded coat?

I thought, ``Am I a little baby? I'm sure I won't freeze. I just wish it would all be over: let's just get in and go.'' %nikolai

``Who should I give the note about the children's clothing to?'' said Natalya Savishna to maman, coming into the room with teary eyes and a note in her hand. %natalya

``Give it to Nikolai, then you come and say goodbye to the children.''

The old woman was about to say something, then suddenly stopped, covered her face with her scarf, and, waving her hand, left the room. My heart ached a bit, when I saw that movement; but my impatience to leave was stronger than that feeling, and I continued to listen with perfect indifference to father's conversation with mother. They were talking about things that obviously did not interest the one or the other: what needed to be bought for the house? what needed to be said to Princess Sophie and Madame Julie? and would the road be good?

Foka entered and, in precisely the same voice he used to announce, ``The food is ready,'' stopped by the lintel and said, ``The horses are ready.'' I noticed that maman winced and went white at the news, as if it were unexpected for her.

Foka was ordered to close all the doors in the room. This amused me greatly, ``as if we're all hiding from someone.''

When everyone was seated, Foka also sat down on the edge of a chair; but as soon as he had done that, the door squeaked and everyone turned to look. It was Natalya Savishna, hastily coming in, and, without raising her eyes, she took shelter near the door on the same chair as Foka. I can see before me now like it was yesterday Foka's bald head and wrinkled, immobile face, and the bent, kind figure in the bonnet, with grey hair visible underneath. They were pressed together on the same chair, and both of them were uneasy.

I continued to be carefree and impatient. The ten seconds that we spent sitting with the door closed seemed to be an entire hour. Finally, we stood, crossed ourselves, and started to say goodbye. Papa embraced maman and kissed her a few times.

``Enough, my dear,'' said papa, ``we're not parting forever, after all.'' %papa

``It's still sad!'' said maman with a voice quivering from tears. %maman

When I heard that voice, and saw her trembling lips and her eyes full of tears, I forgot about everything and was sad myself, pained and so afraid, that I would rather have run away than say goodbye to her. I understood in that moment that, while embracing my father, she had already said her goodbye to us.

She began to kiss Volodya and make the sign of the cross over him so many times, that---supposing she would turn to me next---I shoved myself forward; but she kept blessing him and pressing him to her chest over and over. Finally, I embraced her and, clinging to her, cried and cried, thinking of nothing else but my sorrow.

When we went to get in, the annoying servants came into the front room to say goodbye. Their ``your hand, please,'' noisy kisses on our shoulders, and the smell of lard from their heads awoke in me a feeling very close to distress at the presence of these irritating people. Under the influence of this feeling, I kissed Natalya Savishna's cap with extreme coldness when she, all in tears, said goodbye to me.

It is strange that I can see before me now all the faces of the servants and could describe them in the finest detail; but maman's face and posture absolutely escape the grasp of my imagination: possibly because during all that time I could not once get up the courage to look at her. It seemed to me that, had I done that, her and my sorrowfulness would grow to impossible heights.

I threw myself before everyone else into the barouche and settled on the rear seat. The top was up and blocked my view, but some kind of instinct told me that maman was still there.

``Should I look at her, or no\ldots{}? Oh, just one last time!'' I said to myself and leaned out of the barouche toward the front steps. At that time, maman, thinking the same thing, went up to the opposite side of the barouche and called me by name. Hearing her voice behind me, I turned toward her, but so quickly that we knocked heads; she smiled sadly and kissed me hard, hard for the last time.

When we had driven a few yards, I decided to look at her. The wind had picked up the light blue scarf that was tied around her head; her head lowered and her face covered with her hands, she was slowing climbing the front steps. Foka was supporting her.

Papa sat next to me and said nothing; I was choking back tears, and something was pinching my throat such that I was afraid I might suffocate\ldots{} Driving out onto the main road, we saw a white handkerchief, which someone was waving from the balcony. I continued to cry, and the thought that my tears showed my sensitive nature gave me pleasure and joy.

After we had gone a verst or so, I settled in more comfortably and began to look with persistent attention at the nearest object in front of my eyes---the rear part of the trace-horse, which was running on my side.\footnote{A verst was a unit of length roughly two-thirds of a mile (3,500 feet) long. ---\textit{Trans.}} I watched how that trace-horse, a pinto, waved its tail, how it knocked one leg against the other, how the driver touched it with the braided whip and its legs started to leap together; I watched how the back band leapt on it, and the rings after the back band, and watched until the back band became covered in lather near the tail. I began to look around: at the wavy fields of ripe grain, at the dark fallows, where I could see a plow, a peasant, a horse with a foal, at the verst markers, I even looked at the coach box to see which driver was riding with us; and my face was not even dry from my tears before my thoughts were far from my mother, with whom I just had parted, perhaps forever. But every memory brought me back to thoughts of her. I remembered the mushroom that I had found the day before in the birch lane, remembered how Lyubochka and Katyenka had fought over who would get to pick it, remembered how they had cried when saying goodbye to us.

How I was sad to leave them! sad for Natalya Savishna, and the birch alley, and sad for Foka! Even that awful Mimi---I was sad to leave her. Sad, sad to leave everything! And poor maman? Tears welled up again in my eyes; but not for long.

\chapter{Childhood} %c15

O happy, happy, forever-lost days of childhood! How could I not love, not cherish my memories of that time? These memories revive and elevate my soul and serve as a source of the greatest enjoyment for me.

When tired of running about, at times I would sit at the tea table, on my high chair; it is late, and I have long since drunk my cup of milk with sugar, sleep is beginning to shut my eyes, but I do not move from my place, I sit and listen. How could I not listen? Maman is talking with someone, and the sounds of her voice are so sweet, so inviting. Those sounds alone say so much to my heart! With eyes dimmed with sleep, I look fixedly at her face and, suddenly, she becomes small, so small---her face is no larger than a button; but it is still clearly visible: I see how she glances at me and smiles. I love seeing her so tiny. I squint my eyes even further, and she becomes no larger than when I see myself, a boy, reflected in others' pupils; but then I stir---and the spell is destroyed; I narrow my eyes, turn around, try every which way to bring it back, but in vain.

I get up, get my legs together, and settle comfortably into my chair.

``You'll fall asleep again, Nikolenka,'' maman says to me, ``you'd better go upstairs.'' %maman

``I don't want to sleep, mama,'' I answer her, and obscure but sweet visions fill my imagination, healthy, childish sleep begins to shut my eyelids, and after a minute, I lose consciousness and sleep until someone wakes me. I would feel, half-waking, someone's tender hand shaking me; from the touch alone I could recognize it and, still asleep, involuntarily grab it and press it hard, hard to my lips. %nikolai

Everyone has already gone; one candle is burning in the drawing room; maman said she would wake me up herself; it is she who sits in the chair on which I am sleeping, her marvelous hand stroking my hair, and above my ear comes that lovely, familiar voice:

``Get up, dear soul: it's time to go to sleep.'' %maman

No one's indifferent gazes constrain her: she is not afraid to pour out all her tenderness and love on me. I do not stir, but kiss her hand all the harder.

``Get up, my angel.'' %maman

With her other hand, she takes me by the collar and, her fingers moving quickly, she tickles me. It is quiet in the room, half-dark: my nerves are excited by the tickling and from being woken up; mama sits right next to me; she touches me; I smell her scent and hear her voice. All of this makes me jump up, throw my arms around her neck, press my head to her chest, and, gasping, say:

``Oh, lovely, lovely mama, how I love you!'' %nikolai

She smiles her sad, charming smile, takes my head in both her hands, kisses my forehead, and sits me on her knees.

``So you love me very much, do you?'' She is silent for a minute, then says, ``see that you always love me, never forget me. If your mama is gone from this world, you won't forget her? You won't forget, Nikolenka?'' %maman

She kissed me even more tenderly.

``Enough! Let's not talk about this, my dearest mama, my soul!'' I cry, kissing her knees, and tears flow in streams from my eyes---tears of love and delight. %nikolai

After this, when I would arrive upstairs and kneel before the icons, in my cotton dressing-gown, what a marvelous feeling I would experience, saying: Lord, bring papa and mama to salvation. Repeating my prayers---which my mouth, in early childhood, mumbled for the first time before my beloved mother---love toward her and love toward God became strangely mixed in one feeling.

After my prayers, I would wrap myself up in the covers; my soul would feel weightless, bright, and comforted; one dream would come right after another---but what were they about? They were elusive, but filled with pure love and hopes for joy and happiness. I would remember Karl Ivanych and his bitter fate---the only person that I knew to be unhappy---and I would feel so miserable, and love him so much, that tears would flow from my eyes, and I would think, ``God grant him happiness, give me the opportunity to help him, to ease his sorrow; I'm ready to sacrifice anything for him.'' Then I would bury my favorite porcelain toy---a little bunny or doggy---in the corner of a down pillow and admire how nice, comfortable, and warm it was for it to lie there. I would also pray for God to give happiness to everyone, for everyone to be contented, and for there to be good weather for going out the next day, then I would turn onto my other side, my thoughts and dreams would get all mixed up, I would laugh, and fall asleep quietly, peacefully, with my face still wet with tears.

Will that freshness, that carefree feeling, the need for love and strength of faith I possessed in childhood ever return? What time could be better than when the two greatest virtues---innocent joy and a boundless need for love were the only impulses in life?

Where are those heated prayers? Where is that best of gifts---those pure tears of tender emotion? My guardian angel would fly down, wipe away those tears with a smile, and send down sweet visions to my unspoiled, childish imagination.

Has life really left such heavy tracks on my heart that those tears and delights are gone from me for ever and ever? Am I really left with memories alone?

\chapter{Verses} %c16

Almost a month after we moved to Moscow, I was sitting upstairs at my grandmother's house, at a big table, and was writing; opposite me sat the drawing teacher making some final corrections to a drawing in black pencil of a Turk's head wearing a turban. Volodya stood behind the teacher, craning his neck, and watched him from over his shoulder. The head was Volodya's first work in black pencil and that very day, on grandmother's angel day, he was supposed to bring it to her.\footnote{An angel day, also called the name day, was a person's baptismal anniversary. ``Angel'' references the character's grandmother's namesake, see note XX above. --- \textit{Trans.}} \todo{Create reference to first footnote in ch. 1.}

``So you're not going to lay down shadows here?'' said Volodya to the teacher, standing on tip-toe and pointing at the Turk's neck. %volodya

``No, it's not necessary,'' said the teacher, putting away the pencils and drawing-pen in a box with a sliding lid, ``now it looks splendid. Don't lay a finger on it anymore. Well, and you, Nikolenka,'' he added, getting up and continuing to look sidelong at the Turk, ``will you finally reveal to us your secret? What are you bringing your grandmother? To tell the truth, you'd best bring a head, too. Goodbye, gentlemen,'' he said, took up his hat and ticket, and left.

In that moment, I also thought a head would be better than what I had been working on. When they told us that it would soon be grandmother's name day and that we should prepare a present for the day, the thought crossed my mind to write some verses for her for the occasion, and I put down two verses right away, with rhymes, hoping to put down the rest just as quickly. I absolutely cannot remember how I got that idea, so strange for a child, but I remember that I liked it a great deal, and that I answered every question on the subject by saying that I would certainly bring grandmother a present, but that I would not tell anyone what it was going to consist of.

Against all my expectations, it turned out that, besides the two verses I had thought up in the heat of the moment, I could not, despite all my efforts, compose anything further. I began to read the verses that were in our books; but neither Dmitriyev nor Derzhavin could help me;\footnote{Ivan Ivanovich Dmitriyev and Gavri\"il Romanovich Derzhavin were renowned poets and civil servants in the late 18th and early 19th centuries. --- \textit{Trans.}} on the contrary, they further convinced me of my lack of ability. Knowing that Karl Ivanych loved to copy down little verses, I started to rummage around in his papers and, among some German poems, I found one, Russian, apparently from his own pen.

\begin{verse}
Madame L\ldots{}\ldots{} Petrovskaya, 1828, June 3.

Remember here,\\
Remember there,\\
Remember me ever.\\
From this today and to forever\\
Remember always to my grave\\
How I truly love to you gave.

Karl Mauer.
\end{verse}

I liked this poem, written in a pretty, round hand, on a thin piece of postal paper, for the touching feeling that permeated it; I learned it by heart at once and decided to take it as a pattern. Things went much more easily. On the name day, a congratulatory poem of twelve verses was ready, and, sitting at the table in the classroom, I copied it onto vellum.

Two sheets of paper were already ruined\ldots{} but not because I had thought to change anything in them: the verses seemed very fine to me; but from the third line, their ends began to turn more and more toward the top, so that from a distance it was obvious that they were written crookedly and were no good for anything.

The third sheet was just as crooked as the previous ones; but I decided not to copy anymore. In my poem, I congratulated grandmother, wished her many years of health, and concluded with this:

\begin{verse}
We'll love you this year and another,\\
And praise you like our own true mother.
\end{verse}

It seemed like it might not come out too badly, but this last verse offended my ear in some strange way.

``And praise you like-our fam-ily's mo-ther,'' I repeated under my breath. ``What rhyme could I use in place of \emph{mother}? Brother? Smother\ldots{}? Oh, it will work! It's better than Karl Ivanych's!'' %nikolai

So I wrote down the final verse. Then I read the whole work aloud in the bedroom, with feeling and gestures. There were lines that had no meter at all, but I did not stop for them; the last one struck me even more strongly and unpleasantly. I sat down on the bed to think\ldots{}

``Why did I write: \textit{like our own true mother}? Our actual mother was not there, after all, it was hardly necessary to mention her; I loved my grandmother, it was true, respected her, but all the same, she was not like my mother\ldots{} why had I written that, why had I lied? It \emph{was} poetry, but it still was not necessary.'' %nikolai

At the same time, the tailor came in carrying a new jacket.

``Well, so be it!'' I said with intense impatience, tossed the verses under my pillow with annoyance, and ran to try on my Moscow clothes. %nikolai

My Moscow clothes turned out to be very fine: a brown jacket with bronze buttons, tailored to fit tightly---nothing like the ones they made for us in the village, with room to grow---black pants, also narrow, and it was a marvel how well they brought out my muscles and broke over my boots.

``Finally I have pants with trouser stripes on them, real ones!'' I dreamed, beside myself with joy, looking at my legs from all angles. Although I felt uncomfortable, pressed tightly into the new clothes, I hid that from everyone and said that, on the contrary, I was very much at ease, and that if there was any shortcoming in the clothing, it was only that it was a little too roomy. After that, I stood for a long time in front of the mirror coming my abundantly pomaded hair; but no matter how I tried, I could not smooth the tufts that stuck up from the back of my head: as soon as I stopped pressing on them with the brush to test whether they would obey, they stood up and stuck out in different directions, making my face look very funny.

Karl Ivanych was getting dressed in the other room, and someone brought his blue tail coat and some other white articles of clothing to him through the classroom. At the door leading downstairs, I could hear the voice of one of grandmother's maids; I went out to see what she needed. She held a heavily starched shirt-front over her arm and told me that she was bringing it to Karl Ivanych, and that she had not slept the night before because she was getting it washed in time. I took the shirt-front and asked whether grandmother had gotten up.

``Come now! She has already had her coffee and the archpriest has arrived. What a handsome youngster you are!'' she added with a smile, looking over my new clothes. %maid

This remark made me blush; I turned around on one foot, snapped my fingers, and hopped, wanting to make her feel that she truly knew just how much of a handsome youngster I was.

When I brought the shirt-front to Karl Ivanych, it was already unnecessary: he had put on another and, bending over in front of a little mirror that stood on the table, he was holding the puffy bow of his tie with both hands and was trying to see whether his smoothly shaven chin would fit freely through it or not. Having adjusted our clothes from all sides and having asked Nikolai to do the same for him, he led us to grandmother. I find it funny to remember how strongly all three of us smelled of pomade while we began to go down the staircase.

Karl Ivanych had in his hands a little box of his own making, Volodya had his drawing, and I my verses; each of us had a greeting on the tip of our tongues to accompany our gift. At the moment when Karl Ivanych opened the door to the hall, the priest was putting on his \textit{riza}, and the first sounds of prayer rang out.\footnote{\textit{Riza} is one name for the upper vestment worn by a priest during an Orthodox worship service.}

Grandmother was already in the hall: hunched and leaning on the back of a chair, she stood by the wall in pious prayer; papa stood near her. He turned to us and smiled, noticing how we hurriedly hid the gifts that we had prepared behind our backs and, trying to remain unnoticed, stood right by the door. The whole effect of surprise that we were counting on was lost.

When everyone began to approach the cross, I suddenly felt myself under the heavy influence of an insurmountable, stupifying shyness and, feeling that I would never have enough courage to bring my gift, I hid behind Karl Ivanych's back; he greeted grandmother with carefully chosen expressions, passed his little box from his right hand to his left, delivered it to the \textit{honor\'ee}\todo{this is a bit dicey, no French in the original}, and walked back several steps to make way for Volodya. Grandmother seemed to be enraptured by the little box, which had gold borders pasted on it, and expressed her gratitude with the most tender smile. It was obvious, however, that she did not know where to put the little box, and that was probably why she offered it to papa to see how incredibly artfully it was made.

Having satisfied his curiosity, papa handed it to the archpriest, who seemed to greatly enjoy the little thing: he shook his head and looked with curiosity now at the little box, now at the master who could make such a splendid thing. Volodya brought up his Turk and also won the most flattering compliments from all sides. Then came my turn: grandmother turned to me with an approving smile.

Those who have experienced shyness know that it increases in direct relationship to time, while decisiveness decreases in inverse relationship, that is: the longer the condition continues, the more insurmountable it becomes and the less decisiveness remains.

The last of my bravery and decisiveness left me while Karl Ivanych and Volodya were bringing up their presents, and my shyness surpassed all bounds: I felt my blood rushing incessantly to my head as one blush gave way to another and big drops of sweat formed on my forehead and nose. My ears burned, I felt a shudder and perspiration over my entire body, and I shifted from one foot to the other and could not move from my place.

``Well, show me what you have, Nikolenka---a little box or a drawing?'' papa said to me. There was nothing to be done: with a trembling hand I held up the fateful sheet, all crumpled; but my voice completely refused to serve me, and I stopped in front of grandmother, silent. I was driven out of my mind by the thought that, instead of the expected drawing, in front of everyone they were going to read my good-for-nothing verses and the words: \textit{like our own true mother}, which would clearly show that I had never loved her and had forgotten her. How can I convey my sufferings when grandmother began to read my poem aloud and when she stopped, lost, in the middle of a verse to smile at papa in what seemed to me then to be a mocking way, when she spoke it not at all the way I wanted it to be read, and when, because of her weak eyesight, not reading to the end, she gave the paper to papa and asked him to read it to her from the beginning? It seemed to me that she did it because she was tired of reading such bad and crookedly written verses, and so that papa could himself read the last verse, which so clearly proved my callousness. I waited for him to smack me on the nose with the verses and say, ``Awful little boy, never forget your mother\ldots{} Here's something to think about!'' But nothing like that happened; on the contrary, when everything had been read, grandmother said: \textit{``charmant!''} and kissed me on the forehead. %papa

The little box, the drawing, and the verses were laid next to two cambric shawls and a snuff box with a portrait of maman on the pullout table of the Voltaire chair in which grandmother always sat.

``Princess Varvara Ilyinichna,'' announced one of the two enormous footmen who rode on grandmother's carriage. %footman

Grandmother was lost in thought, looking at the portrait that was set in the tortoise-shell snuff box, and did not answer.

``Should I admit her, Your Excellency?'' repeated the footman.

\chapter{Princess Kornakova} %c17

``Admit her,'' said grandmother, settling deeper in her chair.

The princess was a woman of about forty-five, small, frail, dried-up and peevish, with greyish-green, unpleasant eyes, whose expression clearly contradicted the unnaturally sweet set of her little mouth. Light red hair showed from beneath her velvet hat with its ostrich feather; her brows and eyelashes were even brighter and redder against the unhealthy color of her face. Despite this, thanks to her natural movements, tiny little hands, and her particularly dry features, her appearance overall had something noble and energetic about it.

The princess talked a great deal, and, because of her talkative nature, belonged to that category of people who always talk as if they are being contradicted, even though no one has said a word to them: she would raise her voice, then, gradually lowering it, she would suddenly begin talking with new liveliness and would look around at those individuals present who were not taking part in the conversation, as if trying to buck herself up with a look.\todo{ugh.}

Despite the fact that the princess kissed grandmother's hand and called her \textit{ma bonne tante} incessantly,\footnote{My good aunt. \textit{Fr.}} I noticed that grandmother was displeased with her: she raised her brows in a sort of unusual way, listening to her story about how it was impossible for Prince Mikhaylo to come and congratulate grandmother, despite the strongest possible desire; and, answering the princess's French speech in Russian, she said, stretching her words more than usual:

``I am very grateful to you, my dear, for your attentiveness: and as for Prince Mikhaylo not coming, well, what can one say\ldots{} he always has masses of things to attend to; and how could one say it would be a pleasure for him to sit with an old woman?'' %grandmother

And, not giving the princess time to rebut her words, she continued:

``So, my dear, how are your little children?'' %grandmother

``Yes, thank God, \textit{ma tante}, they're growing up, studying, playing little tricks\ldots{} especially Etienne, my oldest, he's becoming such a rascal that there's no peace and harmony in the house; though he is intelligent---\textit{un gar\c con, qui promet.}\footnote{A boy with promise. \textit{Fr.}} You can only imagine, \textit{mon cousin,}'' she continued, turning to papa alone, because grandmother, not the least interested in the princess's children but wishing to brag about her grandsons, took my verses with care from under the little box and began to unfold them: ``you can only imagine, \textit{mon cousin}, what he did the other day\ldots{}'' %kornakova

And the princess, leaning toward papa, began telling him the story with great animation. Having finished her story, which I did not hear, she began laughing at once and, looking searchingly into papa's face, she said:

``What do you think of the boy, \textit{mon cousin}?\todo{correct italics?} He deserved to be whipped; but the idea of the trick\todo{or is it the lie he told to cover?} was so intelligent and amusing that I forgave him, \textit{mon cousin.}'' %kornakova

And the princess, fixing her eyes on grandmother and saying nothing, continued to smile.

``Do you really \textit{beat} your children, my dear?'' asked grandmother, raising her eyebrows meaningfully and placing special emphasis on the word \textit{beat}. %grandmother

``Ah, \textit{ma bonne tante,}'' casting a quick glance at papa, the princess answered in a sweet little voice, ``I know your opinion on that count; but permit me not to agree with you on this one thing: however much I think about it, however much I read or take advice on this subject, experience still brings me back to the certainty that it is necessary to use fear on children. To make anything of a child, fear is necessary\ldots{} isn't that so, \textit{mon cousin}? And what, \textit{je vous demande un peu,}\footnote{I ask you. \textit{Fr.}} do children fear more than the rod?'' %kornakova

With this, she looked at us searchingly, and, I admit, I was somehow uncomfortable in that moment.

``Whatever you say, a boy is still a baby until 12 or even 14 years old; now a girl, that's another matter.'' %kornakova

``How fortunate,'' I thought, ``that I am not her son.'' %nikolai

``Yes, that is splendid, my dear,'' said grandmother, folding up my verses and laying them under the little box, as if after this she did not consider the princess worthy of hearing such a composition, ``that is very good, only tell me, please, just what kind of delicate feelings can you expect from your children after that?'' %grandmother

And, considering that argument irrefutable, grandmother added, to end the conversation:

``However, everyone is entitled to their opinion on that count.'' %grandmother

The princess did not answer, but only smiled condescendingly, expressing by this that she forgave these strange prejudices in an individual whom she esteemed so greatly.

``Ah, now introduce me to your young people,'' she said, looking at us and smiling invitingly. %kornakova

We stood up and, turning our eyes to the princess's face, we had no idea what we needed to do to show that we had been introduced.

``Kiss the princess's hand,'' said papa. %papa

``I beg you to love your old aunt,'' she said, kissing Volodya's hair, ``although I'm a distant relative, I count by the ties of friendship, not by degrees of relation,'' she added, directing this predominantly to grandmother; but grandmother continued to be displeased with her and answered:

``Eh! My dear, is that really what passes for family nowadays?'' %grandmother

``This is my young man of the world,'' said papa, pointing at Volodya, ``and that is the poet,'' he added while I was kissing the small, dry little hand of the princess and imagining with extreme clarity a rod in that hand, beneath the rod, a bench, and so on, and so on. %papa

``Which?'' asked the princess, holding me by the hand. %kornakova

``That one, the little one with the curls,'' answered papa, smiling cheerily.

``What did my curls ever do to him\ldots{}is there really nothing else for him to talk about?'' I thought and walked away to the corner.

I had the strangest ideas about beauty---I even considered Karl Ivanych the handsomest man in the world; but I knew very well that I was not good looking, and in that I was not mistaken one bit; and so every comment about my physical appearance offended me.

I remember very well a time when, at dinner---I was six years old then---they were talking about my physical appearance, and maman tried to find something good to say about my face: she said that I had intelligent eyes, a pleasant smile, and, finally, ceding the argument to my father and plain fact, was forced to admit that I was ugly; and then, when I thanked her for dinner, she patted me on the cheek and said:

``You remember this, Nikolenka, no one is ever going to love you for your face; so you have to try to be an intelligent and kind boy.'' %maman

Those words not only convinced me that I was not handsome, but that I should certainly be a kind and intelligent boy.

Despite this, I frequently found myself experiencing moments of despair: I imagined that there was no happiness on earth for a person with such a wide nose, fat lips, and small, grey eyes like mine; I asked God to perform a miracle---turn me into a handsome boy, and everything that I had at present, everything that I would have in the future, I would give it all for a handsome face.

\chapter{Prince Ivan Ivanych} %c18

Once the princess had heard the verses and heaped their composer with praises, grandmother softened, began to talk with her in French, stopped calling her \textit{my dear} and using the formal \textit{you}, and invited her to bring all her children to stay the night, to which the princess agreed and, sitting a little longer, went home.

So many guests arrived that day with congratulations that all morning, in the courtyard near the porch, there were several carriages parked at all times.

\textit{``Bonjour, ch\`ere cousine,''} said one of the guests, entering the room and kissing grandmother's hand. %princeivan

He was a man of about seventy, tall, in a military uniform with large epaulets, at the collar of which a large white cross could be seen, the expression on his face calm and open. I found the freedom and simplicity of his movements striking. Despite the fact that he had only a half-circle of thinning hair around the crown of his head and the set of his upper lip clearly demonstrated a lack of teeth, his face was remarkably good-looking.

Prince Ivan Ivanych, at the end of the previous century,\todo{misplaced} thanks to his noble character, attractive physical appearance, remarkable bravery, noble and strong blood, and especially to good fortune, made a brilliant career at a very young age. He continued to serve and very quickly his ambition was satisfied, such that he could wish for nothing further in that regard. From early youth, he held himself as if ready to occupy the very same brilliant position in society in which fate subsequently placed him; consequently, although in his brilliant and vain life, as in all others, he met with misfortunes, disappointments, and afflictions, he never once betrayed his lofty way of thinking, or his calm character, or the fundamental rules of religion and morality, and he gained the esteem of others not so much on the basis of his brilliant position as on the basis of his consistency and steadfastness. He was no great mind, but, thanks to his position, which allowed him to look on all the vain troubles of life from above, his way of thinking was elevated. He was kind and sensitive, but cold and somewhat haughty in his behavior. This occurred because, having been placed in a position where he could be useful to many, through coldness he sought to protect himself from incessant requests and favors by people who only wanted to use his influence. This coldness was softened, however, by the condescending courteousness of a man of \emph{very great society}. He was educated and well read; but his education stopped at what he had acquired in his youth, that is, at the end of the previous century. He had read everything that had been written during the 18th century in France, a time known for philosophy and eloquence, knew thoroughly all the best works of French literature, meaning he could (and loved) to cite frequently passages from Racine, Corneille, Boileau, Moliere, Montaigne, Fenelon; he possessed brilliant knowledge of mythology had profitably studied, in French translations, the ancient masterpieces of epic poetry, possessed satisfactory knowledge of history, obtained from Segur; but he possessed no understanding of mathematics further than arithmetic, nor of physics, nor of contemporary literature: in conversation he could hold silent decently or say a few general phrases about Goethe, Schiller, or Byron, but he had never read them. Despite this French-classical education, of which there now remain very few examples, his conversation was simple, and this simplicity equally hid his ignorance of a few things and set a pleasant tone and tolerance of others. He was a great enemy of every kind of originality, saying that originality was a trick for people of a bad sort. Society was essential for him, wherever he lived; in Moscow or abroad, he always lived with identical openness and on certain days entertained the entire city. He was on such terms with everyone in the city that an invitation card from him could serve as a passport into every drawing room, many young and beautiful ladies eagerly presented their rosy cheeks to him to be kissed with a kind of fatherly feeling, and others, apparently important and upstanding people, felt indescribable pleasure when invited by the prince.

Few people like grandmother remained for the prince, who were of the same circle as him, with a similar upbringing, views on things, and of the same age; therefore he especially valued his ancient, friendly ties with her and always showed her the greatest esteem.

I could not get enough of looking at the prince: the esteem that everyone showed him, the large epaulets, the particular joy that grandmother expressed upon seeing him, and the fact that he alone, apparently, did not fear her, conversed with her perfectly freely, and even had the courage to call her \textit{ma cousine}, inspired in me esteem for him that equalled what I felt for grandmother. When shown my verses, he called me over and said:

``Who can tell, \textit{ma cousine}, perhaps he will be another Derzhavin.''\footnote{Gavrila Romanovich Derzhavin (1743--1816) was a highly-esteemed classical Russian poet of the 18th and early 19th centuries.} %princeivan

At this, he pinched my cheek so painfully that, if I did not cry out, it was only because I had managed to guess that it was kindly meant.

The guests all began to leave, papa and Volodya stepped out: in the drawing room, the prince, grandmother, and I remained.

``Why did your lovely Natalya Nikolayevna not come?'' Prince Ivan Ivanych suddenly asked, after a minute's silence. %princeivan

\todo{Graf below has screwed up spacing after it.}

``Ah, \textit{mon cher},'' grandmother answered, lowering her voice and laying her hand on the sleeve of his uniform, ``she most likely would have come had she been free to do as she wished. She wrote to me that Pierre suggested to her that they come, but that she decided against it herself, because they evidently had no income at all this year; and she writes, `Besides, I have no reason to bring the whole household to Moscow this year. Lyubochka is still too small; and as far as the boys, who are coming to live with you, are concerned, I am much more at ease than if they were with me.' Isn't that splendid!'' grandmother continued in a tone that clearly demonstrated that she did not find it splendid at all. ``It was long since time to send the boys here, so that they can learn something and get used to society; what kind of education could they get in the country, after all? The oldest is thirteen years old, and the other eleven. You may have noticed, \textit{mon cousin}, that they are absolutely wild\ldots{}don't even know how to enter a room.'' %grandmother

``Still, I cannot understand,'' the prince answered, ``why do we hear these constant complaints about distressed circumstances? \emph{He} has a very good situation, and Natasha's Khabarovka, where you and I, in days of yore, played in the theatre, I know it like the back of my hand---a marvelous estate! It should always bring a splendid income.'' %princeivan

``I will tell you, as a true friend,'' grandmother interrupted him, a sad expression on her face. ``It seems to me that these are all just excuses for \emph{him} to live here alone, ramble about the clubs and various dinners, and do God knows what else; and she suspects nothing. You know that angelic kindness---she believes \emph{him} in everything. He assured her that the children needed to be brought to Moscow, but that she should stay in the country with that stupid governess---she believed him; if you told her that children needed to be whipped like Princess Varvara Ilyinichna whips hers and I think she would agree to it on the spot,'' said grandmother, turning in her chair, with a look of utter contempt on her face. ``Yes, my friend,'' grandmother continued after a minute's silence, taking one of her two handkerchiefs in her hand to wipe away a tear that appeared, ``I often think that \emph{he} can neither value nor understand her, and, despite all her kindness and love toward him and the pains she takes to hide her woe---I know all that very well---she cannot be happy with him; and mark my words, if he does not\ldots{}'' %grandmother

Grandmother covered her face with the handkerchief.

\textit{``Eh! ma bonne amie,''} said the prince reproachfully, ``I see you have not become any more sensible---eternally grieving and crying over imagined woe. You should be ashamed! I've known \emph{him} a long time and know him to be an attentive, kind, and splendid husband, and most importantly---the noblest of men, \textit{un parfait honn\^ete homme}.''\footnote{A perfectly honest man. \textit{Fr.}} %princeivan

Having heard this conversation by accident, which I was not supposed to hear, I found my way out of the room on tip-toe, seriously agitated.

\chapter{The Ivins} %c19

``Volodya! Volodya! The Ivins!'' I shouted, looking out the window and seeing three boys in blue bekesha coats with beaver-skin lapels who, following after their dandyish young governor, crossed from the sidewalk on the opposite side of the street from our house.\footnote{A bekesha was a long men's coat with a gathered waist.}

The Ivins were related to us and were almost the same age as us; soon after our arrival in Moscow, we met them and became friends.

The second Ivin---Seryozha---was a dark-looking, curly-haired boy with a hard, turned-up little nose, pretty, very fresh lips, which only rarely covered his upper row of white teeth, protruding slightly, dark-blue, splendid eyes, and an uncommonly pert expression on his face. He never smiled, but either looked perfectly serious or laughed with all his heart, a ringing, distinct, and extremely captivating laugh.\todo{?} I found his original good looks striking from first glance. I felt an insurmountable attraction to him. Just seeing him was enough to make me happy; and for a time all my heart's strength was focused on that one desire: when I had to pass three or four days without seeing him, I began to miss him, and I became sad to the point of tears. All my dreams, sleeping and waking, were of him: lying down to sleep, I hoped to dream of him: closing my eyes, I saw him before me and cherished the apparition as the purest delight. I would not have trusted anyone in the world with those feelings, so highly did I prize them. Perhaps because he was tired of feeling my restless eyes, incessantly turned toward him, or simply did not feel any liking for me, he was noticeably more interested in playing with and talking to Volodya than me; but I was nevertheless satisfied, wanted nothing more, demanded nothing, and was always ready to sacrifice anything for him. Besides the passionate attraction that he inspired in me, his presence excited in me, to no less a degree, another feeling---fear of bothering him, offending him somehow, being unlikeable: perhaps because his face looked haughty, or because, despising my own physical appearance, I overvalued the virtues of beauty, or, what is most likely of all, because I felt as much fear toward him as I did love, a certain sign of love. The first time that Seryozha talked to me, I so lost my composure from this unexpected happiness that I turned white, then red, and could not answer him. He had the bad habit, when he was lost in thought, of fixing his eyes on one point and blinked incessantly, twitching his nose and brows at the same time. Everyone found that this habit spoiled his looks, but I found it so endearing that I unconsciously picked up the habit, and, a few days after I met him, grandmother asked whether my eyes were hurting me, since I was flapping them open and shut like an owl. No word was said between us about love; but he felt his power over me and unknowingly, but tyrannically, used it in our childish relations; I, no matter how much I wished to tell him everything that was on my heart, was too afraid of him to dare to be open; I tried to seem indifferent and submitted to him uncomplainingly. Occasionally, his influence seemed heavy to me, intolerable; but it was not in my power to get out from under him.

I find it sad to remember this fresh, splendid feeling of selfless and limitless love, which died unexpressed and unrequited.

It is strange---I wonder why, when I was a child, I tried to seem like a grown-up, while now that I am one no longer, I often wish I could be like one. So many times that desire---not to seem like a little child in my relations with Seryozha---stopped a feeling that was yearning to find expression and forced me to hide my true feelings. Not only did I not dare to kiss him, which I sometimes wanted to do very much, take him by the hand, say how happy I was to see him, but I did not even dare to call him Seryozha---it was always Sergei: that was how things were between us. Every expression of sensitivity demonstrated childishness and proved me to be a \emph{little boy}. Not having gone through those bitter trials that drive adults to be careful and cold in their relations with others, we denied ourselves the pure pleasure of tender, childish affection only out of a strange desire to pretend to be \emph{grown up}.

I met the Ivins while they were still in the servants' room, greeted them, and then flew back to grandmother: I told her that the Ivins had arrived as if the news should bring her complete happiness in life. Then, not taking my eyes off Seryozha, I went after him into the drawing room and followed his every movement. In the time it took grandmother to say that he had grown a great deal and direct her penetrating eyes at him, I felt the feeling of fear and hope that an artist must experience when awaiting a verdict on his work from an esteemed judge.

The Ivins' young governor, Herr Frost, with grandmother's permission, went with us to the front garden, sat on the green bench there, arranged his legs, the perfect picture of a gentleman, placing his cane, topped with a bronze knob, between them, and, with the look of a person who was very satisfied with his actions, lit a cigar.

Herr Frost was a German, but a German cut from an entirely different cloth than our kind Karl Ivanych: first, he spoke very correct Russian, and French with a very bad accent, and enjoyed the reputation, especially among women, of a very learned person; second, he had a reddish moustache and wore a ruby stick-pin in his black satin scarf, the ends of which were shoved under his suspenders, and light blue pants made of a shot fabric, with trouser stripes; third, he was young, had an attractive, self-satisfied appearance, and uncommonly visible, muscular legs. It was obvious that this last virtue he prized especially: he thought their effects to be irresistible in relation to individuals of the female sex, and it must have been with this goal in mind that he tried to display his legs in the most visible place---standing or sitting, he always brought his calves into motion.\todo{needs work} He was the very type of a young Russian German who wanted to be a good fellow\todo{ugh} and a ladies' man.

We enjoyed ourselves in the front garden. A game of robbers was going very well; but one thing threatened to ruin everything. Seryozha was a robber: chasing after some passing riders, he stumbled and hit his knee against a tree at top speed, so sharply that I thought that he would break into pieces. Despite the fact that I was a gendarme and my duty was to catch him, I approach him and began to ask with concern whether he was hurt. Seryozha got angry at me: he made a fist, stomped his feet, and in a voice that clearly demonstrated that he was badly hurt, shouted at me:

``Well, so what? You've ruined the game now! Well, why aren't you trying to catch me? Why aren't you trying to catch me?'' he repeated several times, looking sidelong at Volodya and the older Ivin, who were pretending to be the passing riders, hopping and running along the path, and he suddenly screamed and ran off to catch them with a loud laugh.

I can hardly relate how amazed, captivated I was by this heroic act: despite the frightful pain, he not only did not cry, but gave no sign that he was hurt, and did not forget the game for a minute.

Soon after this, when Ilyinka Grap had also joined our company, and we were going upstairs to dinner, Seryozha had another chance to captivate and amaze me with his astonishing fortitude and strength of character.

Ilyinka Grap was the son of a poor foreigner who at one point lived in my grandfather's house, was somehow obligated to him, and now considered it his solemn duty to send his son, as frequently as possible, to stay with us. If he supposed that an acquaintance with us would afford his son some kind of honor or pleasure, then he was absolutely mistaken in that regard, because we not only were not friendly with Ilyinka, but we paid attention to him only when we wanted to laugh at him. Ilyinka Grap was a boy about thirteen years old, skinny, tall, pale, with an ugly mug like a bird's, and good-natured, submission expression. He was very poorly dressed, but at the same time pomaded his hair so abundantly that we were certain that on a sunny day Grap's pomade would melt on his head and pour down the back of his jacket. When I remember him now, I find that he was a very obliging, quiet, and kind boy; at the time, he seemed to me such a detestable creature that it was not worth feeling sorry for him or even thinking about him.

When the game of robbers came to an end, we went upstairs, began to \emph{busy ourselves} and show off to one another with various school-boy tricks. Ilyinka looked at us with a timid, surprised smile, and when we proposed that he give it a try, he refused, saying that he did not have the strength. Seryozha was surprisingly nice; he took off his jacket---his face and eyes burned---he roared with laughter incessantly and came up with new pranks to play: he jumped over three chairs placed in a row, rolled a wheel through the entire room, stood on his head on top of copies of Tatishchev's lexicon, which we had placed in the middle of the room as a kind of pedestal, and at the same time played tricks with his legs that killed us, making it impossible for us to hold back from laughing. After the last trick, he thought for a minute, blinked his eyes, and suddenly, with an absolutely serious face, went up to Ilyinka: ``try to do it; it's really not hard.'' Grap, noting that everyone's attention was on him, turned red and in a barely audible voice assured him that he could not possibly do it. %seryozha

``What's this all about, why doesn't he want to show us anything? What a girl he is\ldots{} He definitely needs to stand on his head!'' %seryozha

And Seryozha took him by the hand.

``Definitely, definitely, stand him on his head!'' we all shouted, surrounding Ilyinka, who at that moment was frightened and turning white, grabbed him by the hand and pulled him toward the lexicons.

``Leave me alone, I'll do it myself! You'll rip my jacket!'' shouted the unhappy victim. But his shouts of despair further emboldened us; we were dying of laughter; the green jacket was straining at every seam.

Volodya and the eldest Ivin pushed his head down and placed him on the lexicons; Seryozha and I grabbed the poor boy by his thin little legs, which he waved in various directions, rolled up his pant legs to the knees, and, with loud laughter, flung them upwards; the youngest Ivin kept his whole body in balance.

It so happened that, after some loud laughter, we all suddenly fell silent, and the room became so quiet that only the unhappy Grap's heavy breathing could be heard. At that moment, I was not entirely convinced that everything was happy and funny.

``Now that is a good fellow,'' said Seryozha, giving him a slap of the hand. %seryozha

Ilyinka was silent and, trying to pull away, threw his legs in various directions. In one of these desperate movements, he hit Seryozha in the eye with his heel so hard that Seryozha let go of his legs right away, grabbed his eye, from which tears were flowing involuntarily, and pushed Ilyinka with all his strength. Ilyinka, no longer held up by us, fell in a kind of lifeless way to the ground and through his tears could only get out:

``What are you tormenting me for?'' %ilyinka

The pitiable figure of poor Ilyinka, with his tear-stained face, disheveled hair, and rolled-up pants with unpolished boot-tops visible underneath, was striking to us; we were all silent and tried to force smiles.

Seryozha came to his senses first.

``What an old woman, a whiner,'' he said, touching him gently with his foot. ``You can't take a joke\ldots{} Well, enough, get up.'' %seryozha

``I already told you that you're nasty,'' Ilyinka said angrily, and, turning away, he began to sob loudly. %ilyinka

``Oh! Now he's gonna\todo{?} stomp his feet, cry, and moan!'' shouted Seryozha, grabbing a lexicon in his hands and waving it over the head of the unhappy boy, who did not even think to protect himself and just covered his head with his hands.

``Here you go! Here you go! Let's drop him if he can't take a joke\ldots{} Let's go downstairs,'' said Seryozha, laughing unnaturally. %seryozha

I looked at the poor young man with concern; he was lying on the floor and hiding his face in the lexicons, crying so much that it seemed he might die of the convulsions that shook his whole body.

``Hey, Seryozha!'' I said to him, ``why did you do that?'' %nikolai

``Here's another one! I didn't start crying today when I hit my leg and almost broke the bone, did I?'' %seryozha

``Yes, that's true,'' I thought. ``Ilyinka is nothing more than a crybaby, and Seryozha here---he's a good fellow\ldots{}what a good fellow!'' %nikolai

I could not even imagine that the poor young man was probably crying not so much from physical paint as from the thought that five boys, who he might even have liked, decided without any reason to hate him and persecute him.

I absolutely could not explain to myself the cruelty of my own actions. Why did I not go up to him, not protect him, and not comfort him? Where did my feelings of sympathy go, which would sometimes cause me to cry my eyes out at the sight of a baby jackdaw thrown out of its nest, or a puppy that someone was carrying to throw over the fence, or a chicken that the cook's assistant was carrying in for soup?

Could those splendid feelings really be drowned out by my love for Seryozha and my desire to seem to be a good fellow like he was? How despicable were that love and that desire to be a good fellow! They caused the only black spots on the pages of my memories of childhood.

\chapter{The Guests Arrive} %c20

Judging by how especially busy things were in the staging area off the dining room, by the bright lighting, which gave a new, festive air to all the long-familiar objects in the drawing room and the hall, and judging especially from the fact that Prince Ivan Ivanych would not have sent his musicians in vain, no small number of guests was expected toward evening.

I ran to the window at the sound of every passing carriage, placed my palms against my temples and the glass, and watched the street with impatient curiosity. From the darkness that at first hid all the objects in the window I could see only a little: across the way, a long-familiar little shop with a lamp showing, at the corner, a large house with two windows lit at the bottom, in the middle of the street some Vanka with two fares would pass, or an empty barouche returning home slowly;\footnote{Vanka, a nickname for Ivan, refers to the driver of a cheap carriage driver for hire, analogous to jarvey in English. --- \textit{Trans.}} but now a coach was driving up to the front steps, and I, in full certainty that it was the Ivins, who had promised to arrive early, ran out to meet them in the front room. Instead of the Ivins, two individuals of the female sex appeared behind the liveried arm opening the door: one large, wearing a dark blue women's coat with a sable collar, the other small, muffled up in a green shawl under which only her tiny little feet could be seen in fur boots. Paying no attention at all to my presence in the front room, although I had considered it my duty, upon the appearance of these individuals, to bow to them, the small one silently approached the large one and stood in front of her. The large one unwound the scarf that covered the small one's entire head, unbuttoned her coat, and, when the liveried servant took these items for safekeeping and took off the fur boots, the muffled-up individual began a marvelous twelve-year-old girl in a short, open muslin dress, white bloomers, and tiny little black shoes.\todo{needs work} On her little white neck was a black velvet ribbon; her head was covered in dark blond curls, which suited her splendid-looking face from the front, and her naked little shoulders from the back, and I would not have believed anyone, not even Karl Ivanych, had they told me that they curled that way because they had been twisted in little pieces of the \textit{Moscow Gazette} since morning and were heated with curling irons. It seemed that she had been born with that curly head of hair.

The most striking feature of her face was her the uncommonly large size of her prominent, half-closed eyes, which formed a strange yet pleasant contrast with her tiny little mouth. Her lips were pressed together, and her eyes looked so serious that the general expression of her face was such that you did not expect a smile from it, and a smile from that expression was all the more fascinating.

Trying to remain unnoticed, I slipped through the door into the hall and considered it necessary to stroll back and forth, pretending that I was lost in thought and had no idea that guests were arriving. When some guests got to the midpoint of the hall, I acted as I was coming to my senses, bowed to them excessively, and and announced that grandmother was in the drawing room. Mrs.~Valakhina, whose face I liked greatly, especially because I found in it a strong resemblance to her daughter Sonyechka's face, nodded her head toward me benevolently.

It seemed that grandmother was very happy to see Sonyechka: she called her closer, fixed a ringlet of hair on her head, which had fallen over her forehead, and, staring fixedly at her face, said, \textit{``Quelle charmante enfant!''}\footnote{What a charming child! \textit{Fr.}} Sonyechka smiled, turned red, and looked so lovely that I also turned red looking at her.

``I hope you won't be bored here with me, my friend,'' grandmother said, lifting her little face by her chin. ``Please enjoy yourself and dance as much as you can. Here we already have one lady and two gentlemen,'' she added, turning to Mrs.~Valakhina and reaching her hand out to me.\todo{Actually making contact but this sounds a bit better.}

I found this chance for intimacy so pleasant that it made me turn red yet again.

Feeling that my shyness was getting worse, and hearing the noise of another carriage approaching, I considered it necessary to get away. In the front room, I found Princess Kornakova with her son and an improbable quantity of daughters. The daughters all had more or less the same face---similar to the princess and ugly; for this reason, not a single one of them caught my attention. Removing their coats and furs,\todo{Not great.} they all started talking suddenly in high-pitched little voices, laughing and making a fuss about something---probably the fact that there were so many of them. Etienne was a boy of about fifteen years old, tall, fleshy, with an emaciated countenance, sunken eyes with dark circles, and arms and legs enormous for his age; he was clumsy, had an unpleasant and uneven voice, but he seemed very satisfied with himself and was exactly what I thought, based on my own understanding, a boy who was beaten with rods had to be.

We stood facing one another for a fairly long time and, not saying a word, looked one another over closely; then, drawing closer, we both seemed about to kiss the other, but then, looking into one another's eyes again, for some reason we thought better of it. When all his sisters' dresses swished past us, I asked whether it was cramped in the coach, just to have something to start the conversation.

``I don't know,'' he answered me carelessly, ``I never ride inside a carriage, after all, because as soon as I sit down, I start to feel sick, and mommy knows that. When we go anywhere in the evening, I always sit on the box---it's much more enjoyable: you can see everything, Filipp lets me drive, sometimes I even get to hold the whip. Then sometimes the other riders get, you know\ldots{}'' he added with a significant gesture. ``It's splendid!'' %etienne

``Your Excellency,'' said a footman, entering the front room, ``Filipp is asking where you were so kind as to place the whip?''\todo{May need to be more flexible about izvoljat'...} %footman

``Where I placed it? I gave it to him, of course.'' %etienne

``He says that you did not give it to him.'' %footman

``Well, then I hung it on the lamp.'' %etienne

``Filipp says that it is also not on the lamp, so perhaps it would be better if you said that you took it and lost it, or else Filipp will have to answer for your nonsense with his own money,'' the exasperated footman continued, becoming more and more emboldened. %footman

The footman, who by all appearances was a respectable, gloomy person, seemed to be taking Filipp's side hotly and intended to get to the bottom of the matter, whatever it took. Out of an involuntary feeling of tactfulness, I moved to the side; but the other footmen present behaved completely differently: they stepped in closer, watching the old servant with approval.

``Well, if I lost it, then I lost it,'' said Etienne, evading the possibility of further explanations. ``Whatever the whip costs him, I'll pay it. This is so funny it kills me!'' he added, coming up to me and leading me into the drawing room. %etienne

``No, forgive me, master, how are you going to pay? I know what paying means with you: you've owed Marya Vasilyevna a twenty-kopeck piece for eight months now, you've been in debt to me for two years running, Petrushka\ldots{}'' %footman

``You'll be silent!'' shouted the young prince, turning white from anger. ``Or I'll tell everything.'' %etienne

``I'll tell everything, I'll tell everything!'' the footman uttered. ``This isn't good, Your Excellency!'' he added with particular expressiveness at the same time that we were entering the hall, and left to take the coats to the chest.

``That's it, that's it!'' we heard someone's approving voice behind us in the front room.

Grandmother had an unusual gift for employing second-person pronouns (singular and plural) with a certain tone and in certain circumstances to express her own opinion about people. Although she used the formal and informal \textit{you} contrary to generally accepted custom, in her mouth, these nuances took on a completely different meaning.\footnote{That is, she used the formal \textit{you} for intimates and the informal \textit{you} when trying to create distance between her and someone else. --- \textit{Trans.}} \todo{Fix? T. a bit opaque throughout... May need a footnote} When the young prince approached her, she said a few words to him, addressing him formally, and looked at him with an expression of such disdain that, had I been in his place, I would have lost my composure; but Etienne was a boy seemingly not of that \emph{mentality:} not only did he pay no attention whatsoever to grandmother's welcome, but he ignored her as an individual, making his bow to all of society gathered in the room, if not deftly then at least with some ease. Sonyechka occupied all of my attention: I remember that when Volodya, Etienne, and I were talking in the hall in a place where Sonyechka was visible, and she could see and hear us, I spoke with pleasure; when I happened to say some phrase that, as far as I understood, was funny or bold, I delivered it loudly and looked toward the door into the drawing room; when we moved to another place, where we could be neither heard nor seen from the drawing room, I was silent and found no pleasure in the conversation. 

The drawing room and the hall were beginning to fill with guests; among their number, as is always the case at children's parties, were a few older children who did not want to pass up the chance to have fun and dance, as if only to give pleasure to the mistress of the house.\todo{clunky}

When the Ivins arrived, instead of the pleasure that I normally experienced when meeting Seryozha, I felt a kind of strange annoyance with him for the fact that he would see Sonyechka and show himself to her.

\chapter{Before the Mazurka} %c21

``Hey! Looks like you're going to have dancing,'' Seryozha said, coming in from the drawing room and taking a new pair of kid gloves out of his pocket. ``I'll have to put on my gloves.'' %seryozha

``What should we do? We don't have any gloves,'' I thought. ``I'll have to go upstairs and look for some.'' %nikolai

But although I dug through all the chests of drawers, in one of them I found only our green mittens for traveling, and in the other only one kid glove, which could not possibly fit me: first, because it was extremely old and dirty, second, because it was too big for me, but most importantly, because the middle finger was missing from it, probably cut out long ago by Karl Ivanych for his injured hand. I put the remains of the glove on my hand anyway and looked fixedly at the place on my middle finger that was always covered in ink stains.

``If Natalya Savishna were here, she would have had some gloves for sure. I can't go downstairs looking like this, because if someone asks me why I'm not dancing, what can I say? And I can't stay here, either, because I'll certainly be missed. What can I do?'' I said, waving my hands. %nikolai

``What are you doing here?'' said Volodya, entering the room. ``ask a lady\ldots{} It's starting now.'' %volodya

``Volodya,'' I said to him in a tone that expressed a state close to desperation, showing him my hand with two fingers shoved through the dirty glove. ``Volodya, you never thought about this!'' %nikolai

``About what?'' he said impatiently. ``Oh, gloves,'' he added with complete indifference, noticing my hand. ``We definitely don't have any; we'll have to ask grandmother\ldots{} What do you think she'll say?'' And, not giving the matter any more thought, he ran downstairs. %volodya

The equanimity with which he responded to circumstances which seemed very important to me calmed me, and I hurried into the drawing room, completely forgetting about the monstrous glove, which I was still wearing on my left hand.

Carefully approaching grandmother's chair and reaching my hand out to her gown, I said to her in a whisper:

``Grandmother! What can we wear? We don't have any gloves!'' %nikolai

``What, my friend?'' %grandmother

``We don't have any gloves,'' I repeated, moving closer and closer to her and putting both my hands on the arm of her chair.

``And what is this?'' she said, suddenly grabbing my left hand. \textit{``Voyez, ma ch\`ere,''} she continued, turning to Mrs.~Valakhina, \textit{``voyez comme ce jeune homme s'est fair \'el\'egant pour danser avec votre fille.''}\footnote{Look, my dear. Look how this young man has made himself elegant to dance with your daughter. \textit{Fr.}} %grandmother

Grandmother held me firmly by the hand and seriously but questioningly looked at all those present until the curiosity of all the guests was satisfied and laughter filled the room. 

I would have been very pained had Seryozha seen me at that moment, when I was wincing from shame and trying in vain to tear away my hand, but for Sonyechka to see it, laughing so hard that tears welled up in her eyes and all her curls bounced around her blushing face, I was not even a little ashamed. I understood that her laughter was too loud and natural to be mocking; on the contrary, the fact that we were laughing together and looking at one another seemed to bring me closer to her. This episode with the glove, although it could end badly, had the advantage of putting me a free footing in a circle that had always seemed to me most frightening of all---the circle of the drawing room; I no longer felt even the slightest shyness there in the hall.

The suffering of shy people comes from uncertainty about the opinion others have formed of them; as soon as that opinion is expressed clearly---whatever it happens to be---the suffering ceases.

How lovely Sonyechka Valakhina was when she danced a French quadrille across from me with that clumsy young prince! How lovely her smile was when she gave me her hand in the \textit{cha\^ine}! How lovely when her reddish\todo{was dark-blond above...} curls jumped in time to the music, and when she so innocently did her \textit{jet\'e-assembl\'e} with her tiny little feet! In the fifth figure, when my lady ran to the other side and when I, waiting for the beat, prepared to dance my solo, Sonyechka set her lips in a serious face and began to look away. But she worried about me in vain: I bravely did the \textit{chass\'e en avant, chass\'e en arri\`ere, glissade},\todo{These all probably need footnotes or one big omnibus footnote on the quadrille.} and when I approached her, I showed her the glove with the two fingers jutting out in a playful gesture. She burst out laughing and started mincing around on the parquet floor. I also remember how, when we formed the circle and all took each other's hands, she bent her head and, not taking her hand out of mine, scratched her little nose against her glove. I can see it all in front of my eyes, and I can still hear the sounds of the quadrille from ``The Maiden of the Danube'' that played over all of this.

The second quadrille came, which I danced with Sonyechka. Sitting down next to her, I felt extremely awkward and absolutely did not know what to talk about with her. When my silence had gone on far too long, I began to fear that she would take me for a fool and decided to disabuse her of that notion no matter what. \textit{``Vous \^etes habitante de Moscou?''} I said to her and, after an affirmative answer, \textit{``Et moi, je n'ai encore jamais fr\'equent\'e la capitale,''} counting especially on the effect of the word \textit{fr\'equenter}.\footnote{Are you a resident of Moscow? As for me, I have never visited the capital. \textit{Fr.}} I felt, however, that although I had made a brilliant start and had fully proven my great knowledge of the French language, I was in no condition to continue in the same spirit. It was still a while until our turn would come to dance, and the silence returned: I looked at her anxiously, wanting to know what impression I had made, and expecting some help from her. ``Where did you find such a hilarious glove?'' she asked me suddenly; and that question gave me great pleasure and relief. I explained that the glove belonged to Karl Ivanych and expanded, even a little ironically, about Karl Ivanych as an individual, about how funny he could be when he took off his little red hat, and about how once he fell off a horse while wearing his green bekesha---right into a puddle---and so on. The quadrille went by quickly. It was all going very well; but why had I ridiculed Karl Ivanych? Would I really have lost Sonyechka's good opinion had I described him with the love and esteem that I felt for him? %nikolai

When the quadrille ended, Sonyechka said \textit{merci} to me with such a lovely expression on her face that I truly felt I had earned her gratitude. I was in ecstacy, beside myself with happiness, and could barely recognize myself: where did I find this courage, confidence, and even impudence? ``Not one thing can get in my way!'' I thought, strolling about the hall without a care. ``I'm ready for anything!'' %nikolai

Seryozha asked me to be his \textit{vis-\`a-vis} in the next dance. ``Alright,'' I said. ``I don't have a lady, but I'll find one.'' Throwing a glance over the hall, I noticed that all the ladies were taken except one large girl standing by the door to the drawing room. A tall young man approached her with the intent, I inferred, of inviting her; he was two steps away from her, and I was on the opposite end of the hall. In the blink of an eye, graciously sliding across the parquet floor, I flew over the distance that divided us and, shuffling my foot, invited her to the \textit{contredanse} in a firm voice. The large girl gave me her hand with a patronizing smile, and the young man was left without a lady. %nikolai

I was so conscious of my strength that I paid no attention to the young man's annoyance; but afterward I found out that that young man had asked who the disheveled boy was who slipped past him and took his lady out from under his nose.

\chapter{The Mazurka} %c22

The young man out of whose hand I took my lady was part of the first couple to dance the mazurka. He jumped up from his place, holding his lady by the hand, and instead of doing a \textit{pas de Basques} like Mimi taught us, he just danced straight ahead quickly; running to the corner, he stopped, moved his legs apart, knocked his heel, turned, and, hopping, danced further.

Since I did not have a lady for the mazurka, I sat behind grandmother's high-backed chair and observed what was going on.

``What is he doing that for?'' I thought, debating with myself. ``That's nothing like what Mimi taught us, after all: she assured us that people dance the mazurka on tip-toe, moving their legs fluidly and making circles; and here it turns out they dance it completely differently. There are the Ivins, and Etienne, all dancing, and they aren't doing any \textit{pas de Basques}; and our Volodya has picked up the new style, too. Not bad! And how lovely is Sonyechka?! There she goes\ldots{}'' I was enjoying myself a great deal. %nikolai

The mazurka was coming to an end: a few older men and women were approaching grandmother to bid her farewell and leave; the footmen, trying to avoid the dancers, were carrying equipment into the back room; grandmother was noticeably tired, speaking to everyone very slowly and, it seemed, unwillingly; the musicians lazily began the same motif again for the thirtieth time. The large girl with whom I had danced, making figures,\todo{?} noticed me and, smiling treacherously---likely wanting to please grandmother---led Sonyechka to me along with one of the countless princesses. \textit{``Rose ou hortie?''} she said to me.\footnote{Rose or nettle? \textit{Fr.}} %largegirl

``Ah, you're here!'' grandmother said, turning around in her chair. ``Go, my friend, go now.'' %grandmother

Although at that moment I would rather have hidden my head under grandmother's chair than come out from behind it, how could I refuse? I stood up, said \textit{``rose,''} and glanced timidly at Sonyechka. I had not yet come to my senses when someone's hand in a white glove appeared in mine, and a princess with the most pleasant smile set off with me in tow, not at all suspecting that I did not know the first thing about what to do with my feet.

I knew that ``pas de Basques'' were out of place, improper, and could even embarrass me; but the familiar sounds of the mazurka, operating on my sense of hearing, communicated a certain direction to my acoustic nerves, which, in turn, passed that direction on to my feet; and these last members, absolutely involuntarily and to the surprise of all those watching, began to elaborate fatal, circuitous, and fluid \textit{pas} on tip-toe. While we were going in a straight line, I was able to get by somehow, but at the turn I noticed that, if I did not take the necessary measures, I would certainly end up too far in front. To avoid that unpleasantness, I stopped with the intention of doing the same ``wheel'' that the young man had done in the first round. But at the very moment when I moved my legs apart and was trying to hop, the princess, hurriedly making a circle around me, looked at my feet with an expression of slow-witted curiosity and surprise. That look killed me. I lost my way so badly that, instead of dancing, I started stomping my feet in place out of time to the music, in the strangest and most absurd way possible, and finally stopped completely. Everyone was looking at me: some of them with surprise, some with curiosity, some with ridicule, some with sympathy; only grandmother was looking at me indifferently.

\textit{``Il ne fallait pas danser, si vous ne savez pas!''} said papa's angry voice in my ear and, pushing me out of the way gently, he took my lady's hand, led her through the next turn in the old style, to the loud approval of everyone watching, and brought her back to her place.\footnote{You shouldn't be dancing if you don't know how! \textit{Fr.}} The mazurka ended there. %papa

``Lord! Why are you punishing me like this!'' %nikolai

~

\centerline{* * *}

~

Everyone despises me and will always despise me\ldots{} The road to everything is closed to me: to friendship, love, honors\ldots{} All is lost!! Why did Volodya make signs to me, which everyone could see but which could not help me? Why did that disgusting princess have to look at my legs like that? Why did Sonyechka\ldots{}she is lovely; but why did she smile at that moment? Why did papa turn red and grab me by the hand? Did he really feel shame for me, too? Oh, it is terrible! If only maman had been there, she would not have turned red because of her Nikolenka\ldots{} And my imagination ran away with that lovely image. I remembered the meadow in front of our house, the tall linden trees, the clear pond over which swallows hovered, the blue sky and the white, transparent clouds that stopped overhead, the fragrant stacks of fresh hay, and more peaceful, iridescent memories passed through my imagination, so upset with what had happened.

\chapter{After the Mazurka} %c23

At supper, the young man who had danced in the first round sat at our table, the children's table, and pay me special attention, which would have flattered my vanity, if I could have felt anything after the misfortune that had occurred. But the young man, as it turned out, wanted to cheer me up if at all possible: he tried to be playful, called me a good fellow, and, as soon as no grown-ups were looking, he poured wine into my glass from various bottles and made me drink it. Torward the end of supper, when the butler poured me just a quarter of a glass of champagne from a bottle that was wrapped in a napkin, and when the young man insisted that he pour me a full one and made me drink it in one gulp, I felt a pleasant warmth throughout my whole body, a particular fondness toward my cheerful protector, and burst out laughing at something.

Suddenly, the sounds of the grossvater dance rang out from the hall, and everyone started to get up from the table. My friendship with the young man ended right there: he went away to the grown-ups, and I, not daring to follow him, approached Mrs.~Valakhina with curiosity, trying to overhear what she and her daughter were saying.

``Just another half hour,'' Sonyechka said earnestly. %sonyechka

``We really can't, my angel.'' %valakhina

``Just for me, please?'' she said, coaxing her. %sonyechka

``Is it really going to be fun for you if it makes me ill tomorrow?'' Valakhina said and was careless enough to smile. %valakhina

``Oh, you'll let me! We can stay?'' Sonyechka said, jumping out of happiness. %sonyechka

``What am I supposed to do with you? Go, go, dance\ldots{} Here is a gentleman for you,'' she said, pointing at me. %valakhina

Sonyechka gave me her hand, and we ran into the hall.

The wine I had drunk, and Sonyechka's presence and cheerfulness, made me completely forget the unhappy incident of the mazurka. I made my legs do the most amusing things: imitating horses, I trotted a little, proudly lifting my legs high, then stomped them in place like a ram angry at a dog, laughing with all my heart at the same time and not worrying even a little about what impression I was leaving on those watching. Sonyechka also laughed without stopping: she laughed at how we were spinning around, hand in hand, laughed when she looked at some old nobleman who, slowly lifting his feet, stepped over the scarf looking as if it required great effort, and she was dying of laughter when I jumped up nearly to the ceiling to show her my deftness.

Going through grandmother's study, I glanced at myself in the mirror: my face was all sweaty, my hair wild, the tufts in the back sticking up worse than ever; but the overall expression of my face was so cheerful, good, and healthy, that I liked myself very much.

``If I always looked like I do now,'' I thought, ``I could get people to like me.'' %nikolai

But when I glanced at the gorgeous face of my lady again, besides\todo{Get rid of besides??} the expression of cheerfulness, health, and lack of care that I liked so much in it, I saw so much refined beauty that I became annoyed with myself, and I understood how stupid it was for me to hope to attract the attention of such a marvelous creature.

I could not hope for my feelings to be returned, and I did not even think about that: my soul was overflowing with happiness even without it. I did not understand that, beyond the feeling of love that was filling my soul with delight, it was possible to demand an even greater happiness and wish for something more than to have that feeling go on without end. I was happy even with what I had. My heart was beating like a dove's, blood was rushing incessantly into it, and I wanted to cry.

As we walked along the corridor, past the dark storage room under the staircase, I glanced into it and thought: what happiness it woudl be if I could live a century with her in that dark storage room! And for no one to know we were living there.

``Today has been very fun, don't you think?'' I said in a quiet, trembling voice, and quickened my pace, frightened not such much by what I had said as by what I was getting ready to say. %nikolai

``Yes\ldots{}very!'' she answered, turning her little head toward me with such an open and kind expression that I could not stop being afraid. %sonyechka

``Especially after supper\ldots{} But if you only knew how much I regret it\ldots{}'' I wanted to say ``how sad it makes me,'' but did not dare, ``that you're leaving soon and we won't see each other again.'' %nikolai

``Why wouldn't we see each other?'' she said, staring fixedly at the toes of her shoes and running her finger over the latticed screens that we were passing. ``Every Tuesday and Friday, mother and I ride on Tverskaya Street. Do you really not go out?'' %sonyechka

``We'll definitely have to ask permission for Tuesday, and if they don't let me go, I'll run away alone---without my hat. I know the way.'' %nikolai

``Do you know what? With some boys who come to visit us, I decide not to be formal and speak to them more intimately. Do you want to do that?'' she added, shaking her little head and looking me right in the eyes.\footnote{Sonyechka means here that she uses the informal \textit{you} with some of the boys she knows. --- \textit{Trans.}} %sonyechka

At that moment, we were walking into the hall and another, lively part of the grossvater was beginning. ``Let's go\ldots{}miss,'' I said at a moment when the music and noise could drown out my words.

``Let's go, \emph{sis}, not let's go, miss,'' Sonyechka corrected me and laughed.\todo{Other plays on words that would work?} %sonyechka

The grossvater ended, and I had not succeeded in saying a single phrase addressing her informally, although I did not cease thinking up phrases that included the pronoun \textit{you} several times. I lacked the courage for it. ``Do you want to?'' ``Let's you and I\ldots{}'' sounded in my ears and brought on a kind of inebriation: I saw nothing and no one except Sonyechka. I saw her locks gathered up and lain behind her ears to open up parts of her brow and temples that I had not yet seen; I saw how she was wrapped up in her green shawl so tightly that only the end of her little nose was visible; I noticed how, if she had not made a little opening around her mouth with her rosy little fingers, she would certainly have suffocated, and I saw how, descending the staircase with her mother, she quickly turned back toward us, nodded her little head, and disappeared behind the door.

Volodya, the Ivins, the young prince, myself, we were all in love with Sonyechka and, standing on the staircase, we followed her with our eyes. To whom in particular she was nodding, I do not know; but in that moment, I was firmly convinced that it was done for me.

Saying goodbye to the Ivins, I spoke with Seryozha very freely, even a little coldly, and shook his hand. If he understood that from that day he had lost my love and his power over me, then he likely regretted it, although he tried to seem completely indifferent.

For the first time in my life, I was unfaithful in love, and for the first time I experienced the sweetness of that feeling. I was delighted to exchange the worn-out feeling of habitual devotion for the fresh feeling of love filled with mystery and uncertainty. Moreover, to fall in love and out of love at the same time means to love twice as strongly as before.

\chapter{In Bed} %c24

``How could I have loved Seryozha so long, and so passionately?'' I considered, lying in bed. ``No! He never understood, could never value my love, never deserved it\ldots{} And Sonyechka? What a delight! `Do you want to?' `You start.' Speaking so intimately with me.'' %nikolai

I jumped up on all fours, vividly imagining her little face, covered my head with the blanket, wrapped myself in it all around, and, when there were no openings left in it, lay back down and, feeling pleasantly warm, sank into sweet dreams and memories. Fixing my gaze on the lining of the quilted blanket, I saw her as clearly as I had an hour before; in my mind, I had a conversation with her, and that conversation, although there was no sense to it at all, brought me indescribable delight, because that familiar pronoun \textit{you} occurred in it incessantly.

These dreams were so clear that I could not fall asleep because of the sweet excitement they caused me, and I wanted to share my excess happiness with someone. 

``She's so lovely!'' I said, almost out loud, turning hard onto my other side. ``Volodya! Are you sleeping?'' %nikolai

``No,'' he answered me in a sleep voice. ``What is it?'' %volodya

``I'm in love, Volodya! Absolutely in love with Sonyechka.'' %nikolai

``Well, so what?'' he answered me, stretching.

``Ah, Volodya! You can't imagine what is happening to me\ldots{} I was just lying here under the blanket and saw her so clearly, so clearly, had a conversation with her, it was simply astonishing. And you know what? When I lie here and think of her, God knows why, but it makes me so sad and I want to start crying so badly.'' %nikolai

Volodya stirred.

``I would only wish for one thing,'' I continued, ``and that's to always be with her, always see her, and nothing more. Are you in love? Admit it, tell me the truth, Volodya.'' %nikolai

It is strange that I wanted us both to be in love with Sonyechka, and for us to tell each other about it.

``What does it matter to you?'' Volodya said, turning his face toward me. ``Maybe I am.'' %volodya

``You don't want to sleep, you were pretending!'' I shouted, noticing how brilliant his eyes looked---he had not been thinking about sleep at all---and I threw off my blanket. ``Let's discuss her instead. She's a delight, isn't it true? Such a delight that if she told me, 'Nikolasha! Jump out the window, or throw yourself into a fire,' well, upon my word, I would!'' I said. ``I'll jump right now, and happily. Ah, what a delight!'' I added, vividly imagining her before me, and, to get as much enjoyment as possible from the image, I threw myself impulsively onto my other side and thrust my head under the pillow. ``I want to cry so badly, Volodya.'' %nikolai

``What an idiot!'' he said, smiling and then falling silent for a while. ``I'm not like that at all, not like you. I think that if I could, I'd first want to sit next to her and talk\ldots{}'' %volodya

``Ah! Then you are in love?'' I interrupted him. %nikolai

``Then,'' Volodya continued, smiling tenderly, ``then I would kiss her little fingers, her eyes, her lips, her little nose, her feet---I would kiss her all over.'' %volodya

``That's stupid!'' I cried from under the pillow. %nikolai

``You don't understand anything,'' Volodya said contemptuously. %volodya

``No, I understand, \emph{you} don't understand, and what you're saying is stupid,'' I said through tears. %nikolai

``Oh, there's nothing to cry about. But she is a real girl!''\todo{Ugh.} %volodya

\chapter{The Letter}

On April 16th, almost six months after the day I just described, father came upstairs to see us during class and announced that we would leave that night for the country. My heart felt pinched for a moment at this news, and my thoughts immediately ran to mother.

The cause for this unexpected trip was the following letter:

\begin{quotation}
\begin{flushright}
Petrovskoye. April 12th.
\end{flushright}

Just today, at nine o'clock in the evening, I received your kind letter of April 3rd and, according to my usual habit, I am replying right away. Fyodor brought it yesterday from the city, but because it was so late, he gave it to Mimi this morning. Mimi, under the pretense that I was not well and was out of sorts, kept it from me for a whole day. I actually did have a small fever and, to tell you the truth, for four days I have felt unwell somehow and have not gotten out of bed.

Please, don't worry, my friend: I feel well enough, and if Ivan Vasilyich will let me, tomorrow I think I will get up.

On Friday of last week, I went for a drive with the children; but near the gates onto the main road, by that little bridge that always used to make me terrified, the horses became stuck in the mud. The day was splendid, and I had the idea of walking on foot to the main road while they pulled out the barouche. When I got as far as the chapel, I was very tired and sat down to rest, and because a half hour went by while some people gathered to pull out the carriage, I got cold, especially my legs, because I was wearing shoes with thin soles, and I had wet them through. After lunch, I felt a chill and a fever, but continued walking out of routine, and after tea sat down with Lyubochka to play piano for four hands. (You won't recognize her: she has made such progress!) But imagine my surprise, when I noticed that I could not keep time. I tried several times to count the time, but everything was getting mixed up in my head, and I heard a strange noise in my ears. I would count: one, two, three, then suddenly: eight, fifteen, and the most important thing was that I saw I was off and could not get things right. Finally, Mimi came to help me and put me to bed almost by force. There you are, my friend, a detailed account of how fell ill, and how it's my own fault. The next day, I had a fairly strong fever, and our kind old Ivan Vasilyich came; he has been staying with us ever since and promises to let me out into God's creation again soon. What a marvelous old man, that Ivan Vasilyich! When I had the fever and delirium, he sat next to my bed all night, not shutting his eyes the whole time, and now, since he knows I'm writing, he's sitting with the girls in the sofa room, and I can hear from my bedroom how he is telling them German folk tales, and how they are dying of laughter listening to him.

``La belle Flamande,'' as you call her, is staying here as my guest for the second week, because her mother went away somewhere to visit, and all the pains she is taking prove her sincere affection for me.\footnote{The beautiful Flemish woman. \textit{Fr.}} She is confiding all her most intimate secrets to me. With her splendid face, kind heart, and youth, she might turn into a splendid girl in all respects, if only she were in good hands; but in the company she keeps now, judging by her stories, she will certainly come to ruin. The thought occurred to me that if I did not have so many children, I would do well to take her myself.

Lyubochka wanted to write to you herself, but she has already shredded three sheets of paper and says, ``I know what how papa likes to ridicule people: if I make even one mistake, he'll show it to everyone.'' Katyenka is just as lovely as ever, and Mimi is just as kind and boring.

Now let me talk about something serious: you wrote me that things are not going well for you this winter, and that it's essential that you take the Khabarovka money. It's strange to me that you are even asking for my consent for that. Doesn't everything that belongs to me belong to you just as much?

You are so kind, my dear friend, for hiding the true state of things from me out of fear that you'll cause me distress; but I've surmised that you must have lost a great deal gambling, and I'm not the least distressed by that, I swear; so as long as things can be fixed, please, don't think much about it and don't torment yourself for no reason. Not only am I not used to counting on your winnings for the children, but, forgive me, I don't even count on your entire estate. When you win, I find as little joy in that as I do distress when you lose; I'm only distressed by your unhappy passion toward playing, which takes a part of your tender affection away from me and forces me to say such bitter truths to you as I am now, and God knows how painful that is for me! I never cease to pray to Him for one thing, that he would save us from\ldots{}not from poverty (what is poverty?), but from some terrible condition where the interests of the children, which I must protect, will come into conflict with ours. So far, God has answer my prayer: you haven't crossed the line where we must sacrifice the fortune that is not ours anymore, but is our children's, or\ldots{}even to think about it is frightening, but that terrible misfortune has always threatened us. Yes, it is a heavy cross that the Lord has sent us!

You wrote me about the children as well, and you returned to our old argument: you've asked me to agree to put them in an educational institution. You know my prejudice against that kind of upbringing\ldots{}

I don't know, my dear frend, whether you will agree with me; but in any case, I beg you, out of your love for me, give me your promise that, while I am still alive and after my death, if it pleases God to separate us, that that will never happen.

You wrote me that it's essential that you go up to Petersburg on business. Christ be with you, my friend, go and return quickly. We all find it boring here without you! The spring is just marvelous: they've taken out\todo{Some winter ones, I think?} the balcony doors already, the path to the greenhouse was completely dry four days ago, the peach trees are all in bloom, only there is still snow here and there, the swallows have flown back, and today Lyubochka brought me the first spring flowers. The doctor says that in three days I will be completely well and I will be able to breathe fresh air and warm myself in the April sun. Farewell, dear friend, don't worry about my illness, please, nor about your losses; finish your business as soon as you can and come back with the children for the entire summer. I'm making plans for how to spend it that will make you marvel, and the only thing missing to make them come true is you.
\end{quotation}

The last part of the letter was written in French in a cursive and uneven hand, on a different scrap of paper. I'm translating it word for word:

\begin{quotation}
Don't believe what I wrote about my illness; no one suspects how serious it is. I only know one thing, which is that I can't get out of bed anymore. Don't waste a single minute, come immediately and bring the children. Maybe I can manage to embrace you one last time and give them my blessing: that is my last wish. I know this is a terrible blow I'm inflicting on you; but all the same, sooner or later, from me or from others, you would have experienced something like this; we will try to get through this misfortune with steadfastness and hope in God's mercy. We will obey His will.

Don't think that what I am writing is the delirium of a sick imagination; on the contrary, my thoughts are extremely clear at this moment, and I am completely calm. Don't try to give yourself false comfort with the hope that these are the false, vague premonitions of a fearful heart. No, I feel it, I know it---and therefore I know that God chose to reveal it to me---that I have only a short while left to live.

Will my love for you and the children end with my life? I have come to understand that that is not possible. I feel too strongly in this moment to think that that feeling, without which I cannot understand existence, could ever be destroyed. My soul cannot exist without love for all of you: and I know that it will exist eternally, if only because a feeling like my love could not have come about if it were going to come to an end some day.

I won't be with you anymore; but I am absolutely certain that my love will never leave you, and that thought is so comforting to my heart that I await my coming death calmy and without fear.

I am calm, and God knows that I always looked on death, as I do now, as a passage into a better life; but why are these tears pressing on me\ldots{}? Why take a loving mother away from her children? Why inflict such a heavy, unexpected blow on you? Why must \emph{I} die, when your love has made my life endlessly happy?

Let His holy will be done.

I can't write without crying anymore. Perhaps I won't see you. Thank you, my priceless friend, for all the happiness that you surrounded me with in this life; once I'm over there, I will ask God to reward you. Farewell, dear friend; remember that I won't be here, but my love will never leave you, not at any time or in any place. Farewell, Volodya, my angel, farewell, Benjamin---my Nikolenka.

Please don't let them ever forget me!!
\end{quotation}

With the letter was a little note in French from Mimi, with the following contents:

\begin{quotation}
The sorrowful premonitions she is telling you about have been more than confirmed by the words of the doctor. Last night, she told me to take this letter to the post right away. Thinking that she had said this in a delirium, I waited until the next morning and decided to unseal it. As soon as I unsealed it, Natalya Nikolayevna asked me what I had done with the letter and ordered me to burn it if I hadn't sent it. She kept talking about it and warned me that it would kill you. Don't put off your trip home if you want to see your angel, because she has not left us yet. Forgive my scrawl. I haven't slept for three nights. You know how I love her!
\end{quotation}

Natalya Savishna, who had spent the whole night of April 11th in mother's bedroom, told me that, having written the first part of the letter, maman laid it on the table next to her and rested a little.

``I admit,'' Natalya Savishna said, ``I myself dozed off in the chair and the dropped the stocking I was working on. Only I heard in my sleep---sometime about one o'clock---that she was talking, like;\todo{too much?} I opened my eyes, and there I see: my dearie, sitting there in bed, her arms crossed like this, and tears were coming down in three streams. `So it's all finished?' she said and then covered her face with her hands. %natalya

``I jumped up and started to ask her what was the matter with her? %natalya

```Ah, Natalya Savishna, if you only knew who I just saw.' %natalya

``No matter how much I asked, she wouldn't say anything else to me, she just told me to bring over the little table, she wrote something else, told me to seal the letter in front of her and send it right away. After that, everything got worse and worse.'' %natalya

\chapter{What Awaited Us in the Country} %c26

On April 18th, we found ourselves getting out of the traveling barouche near the front steps of the house at Petrovskoye. Leaving Moscow, papa was pensive, and when Volodya asked him whether maman was sick, he looked at him with sadness and nodded his head silently. During the journey, he calmed down noticeably; but, the nearer we came to home, the more his face took on a sorrowful expression and, when, getting out of the barouche, he asked Foka, who had run out of the house, breathless, ``Where is Natalya Nikolayevna?'' his voice became unsteady, and there were tears in his eyes. Kind old Foka, glancing at us furtively, lowered his eyes and, opening the door into the front room, turned back and answered:

``The mistress has not deigned to leave her bedroom for six days now.'' %foka

Milka, who I later learned had not ceased howling mournfully since the day when maman fell ill, threw himself at father happily---jumped on him, yelped, licked his hands; but he pushed her away and went into the drawing room, from there into the sofa room, which had a door leading right into the bedroom. The closer he came to that room, the more noticeably anxious he became, which could be seen from all the movements of his body; entering the sofa room, he walked on tip-toe, barely taking a breath, and cross himself before he decided to take the handle of the closed door. During this time, Mimi ran out into the corridor, disheveled and tear-stained. ``Ah, Pyotr Aleksandrych!'' she said in a whisper, with an expression of true despair, and then, noticing that papa was turning the latch, she added, barely audibly, ``You can't go through here---you'll have to go through the girls' room.'' %mimi

Oh, how heavy all of this was on my childish imagination, so inclined toward sorrow by a frightening premonition of what was to come!

We went into the girls' room: in the corridor, we came across the imbecile Akim, who always amused us with the funny faces he made; but at that moment, not only did he not seem funny to me, but nothing struck me as so painful as the sight of his foolish, unconcerned face. In the girls' room, the two girl servants, who were working on something, started to get up to bow to us and had such sorrowful expressions that I became frightened. Walking through Mimi's room, papa opened the door to the bedroom, and we went in. To the right of the door were two windows hung with scarves;\todo{not great. kerchiefs literally.} Natalya Savishna sat next to one of them, with glasses on her nose, and was knitting a stocking. She did not kiss any of us as she usually did, but stood up, looked at us through her glasses, and her tears flowed heavily. I did not at all like the fact that, upon first seeing us, everyone was beginning to cry, when before they had been completely calm.

To the left of the door stood some screens, and behind the screens was the bed, the little table, a small cabinet covered with medicines, and a large chair, in which the doctor had dozed off; near the bed stood a young, very blond girl of remarkable beauty in a white morning house-dress who, having rolled up her sleeves a little, was applying ice to maman's head, which I could not see at that moment. The girl was \textit{la belle Flamande}, about whom maman had written and who would subsequently play such an important role in the life of our entire family. As soon as we came in, she took one hand away from maman's head and straightened the folds of her house-dress over her chest, then said in a whisper, ``She is out.'' %flamande

I was in deep sorrow, but I automatically noticed all the particulars of that moment. It was almost dark in the room, hot, and it smelled like a combination of mint, eau de cologne, camomile, and Hoffmann's drops.\footnote{Hoffmann's drops were a mixture of three parts alcohol and one part ether, taken orally as a curative in the nineteenth century, especially by women. --- \textit{Trans.}} This smell struck me so strongly that whenever I smell it, or even simply remember it, my imagination momentarily takes me back to that gloomy, stuffy room and reproduces all the little details of that terrible moment.

Maman's eyes were open, but she saw nothing\ldots{} Oh, I will never forget that frightening look! It expressed such suffering!

We were led out.

When I later asked Natalya Savishna about mother's last moments, this is what she said to me:

``When they led you out, she tossed and turned for a long time after that, my dear, dearie, like something was strangling her right there; then her head slipped off the pillows and she nodded off, so quietly, so calmly, just like an angel from heaven. I just went out to see why they weren't bringing her food---I came back and my sweet girl, she had thrown everything off and was waving for your father to come in; he bent down next to her, and you could see she didn't even have the strength to say what she wanted to say: as soon as she opened her lips she would start moaning. `My God! Lord! The children! The children!' I wanted to run and get all of you, but Ivan Vasilyich stopped me and said, `It'll upset them even worse, better not.' After that, all she could do was raise her hand and let it drop again. And what she meant by that, only God knows. I think she was giving each of you her blessing; obviously God wouldn't let her (at the very end) see her children one last time. Then she tried to sit up, my dearie, with her hands like this, and suddenly said, and in a voice I can't even bear to remember, `Mother of God, don't leave them!' Here the pain had gotten to her heart, you could see it in her eyes how much she was suffering, the poor girl; she fell back on the pillows, bit the sheet; and the tears, precious boy, the tears kept coming.'' %natalya

``What then?'' I asked.

Natalya Savishna could not say any more: she turned away and cried bitterly.

Maman died in terrible suffering.

\chapter{Sorrow} %c27

The next day, late in the evening, I wanted to look at her again: overcoming an involuntary feeling of fear, I quietly opened the door and walked into the hall on tip-toe.

In the middle of the room, on the table, stood the coffin, with burned-out candles all around it in tall silver candlesticks; in the far corner sat a deacon reading the Psalter in a quiet, monotonous voice.

I stopped by the door and started to watch; but my eyes were so raw from crying and my nerves so unsettled that I could not make out anything; everything mixed together in a strange way; the light, the brocade, the velvet, the large candlesticks, the rose-colored pillow trimmed in lace, the crown, the cap with ribbons, and something else that looked transparent with a waxy color.\todo{Note about the paper crown at the panikhida?} I knelt on the chair to look carefully at her face; but in the place where it was supposed to be, that same pale-yellow, transparent object appeared. I could not believe that it could be her face. I started to look at it fixedly, and little by little I began to recognize in it her familiar features that were so dear to me. I gasped when I was finally sure that it was her; but why had her eyes fallen? Why was she so strangely pale, and where had the black-looking spot on one cheek, just under the transparent skin, come from? Why was the expression across her whole face so strict and cold? Why were her lips so pale and the set of her mouth so splendid, so stately and expressing an unearthly calmness, which caused a cold shudder to run down my back and scalp when I looked into that face?

I watched and felt an incomprehensible, insurmountable power drawing my eyes toward the lifeless face. I could not take my eyes off it, and my imagination painted a picture for me of blossoming life and happiness. I forgot that the dead body that lay before me and which I watched senselessly, like an object that had nothing in common with my memories, was \emph{her}. I imagined her in various situations: alive, happy, smiling; then suddenly some feature in the pale face would strike me when my eyes stopped on it: I would remember the terrible reality, then gasp, but I did not stop looking. And again my dreams would replace reality, and again my sense of reality would destroy the dreams. Finally, my imagination wore out, it stopped deceiving me; my sense of reality also disappeared, and I lost myself completely. I do not know how much time I spent in that state, I do not even know what happened; I know only that for a time I lost consciousness of my own existence and experienced an exalted, indescribably pleasant, and sad delight.

Perhaps, flying away to that better world, her beautiful soul was looking down with sadness on the one in which she had left us; she saw my grief, pitied it, and on the wings of love, with a heavenly smile of pity, came down to earth to comfort and bless me.

The door squeaked and another deacon came in to replace the first one. This noise woke me up, and the first thought that came to me was that, since I was not crying and was sitting on the chair in a pose that was not particularly touching, the deacon might take me for an unfeeling boy who had climbed up on the chair out of mere compassion, or curiosity: I crossed myself, bowed, and started crying.

Remembering my impressions of that time now, I find that only that single minute of forgetting myself was true sorrow. Before and after the burial, I never stopped crying and felt sad, but I am ashamed to remember that sadness, because there was a vain, self-serving feeling mixed in: whether the desire to show that I was more grieved than everyone else, or worrying about the effect I was having on others, or an idle curiosity that caused me to observe Mimi's house-dress and the faces of those present. I detested myself for not experiencing that single feeling of sadness exclusively, and tried to hide all the others; because of that, my grief was insincere and unnatural. Moreover, I experienced a kind of delight in knowing that I was unhappy, tried to stimulate that sense of unhappiness in fact, and more than anything else, that egotistical feeling drowned out sincere grief in me.

Having slept that entire night soundly and calmly, as is always the case after being truly distressed, I awoke with my tears dried and my nerves calmed. At ten o'clock, they called us to the funeral service, which was to be celebrated before the body was carried out. The room was full of servants and peasants, all of them in tears, who had come to say farewell to their mistress. During the service, I cried the proper amount, crossed myself, and bowed deeply toward the ground, but I was not praying in my heart and was fairly cool-headed; I was worried about my new short coat, which they had made me wear, and was bothered by the fit under the arms, thought about how I could avoid getting my pants too dirty, and furtively made observations of all those present. Father stood at the head of the coffin, was as pale as a handkerchief, and was holding back tears with noticeable effort. His tall figure---in a black tail coat, his face pale and expressive, his movements graceful and self-assured as always when he crossed himself, bowed, picked up earth with his hand, took the candle from the hands of the priest, or approached the coffin---was extremely affecting; but, I do not know why, I did not like the fact that he could seem so affecting at that moment. Mimi stood leaning against the wall and, it seemed, could barely keep on her feet; her dress was crumpled and covered in lint, her cap pushed off to the side; her swollen eyes were beautiful, her head shook; she did not stop sobbing and making sounds that tore at the heart and kept covering her face with her handkerchief and hands. It seemed to me that she was doing it to cover her face from onlookers to rest for a moment from her feigned sobbing. I remembered how, the day before, she had told father that maman's death was such a terrible blow, which she could never hope to get over, that she had lost her everything, that that angel (which was what she called maman) right before her death and not forgotten her and had expressed a desire to provide for her and Katyenka's future forever. She shed bitter tears while telling him this, and, perhaps, her feeling of sorrow was real, but it was not plain and pure. Lyubochka, in a black dress stitched with mourning badges, all wet from tears, hung her head and looked infrequently at the coffin, and her face expressed only a childish sort of fear. Katyenka stood near mother and, despite her drawn-out face, she looked as pink as she always was. Volodya's open nature was open even in sorrow: he would stand, lost in thought and stare at some object, then his mouth would suddenly begin to frown, and he would hastily cross himself and bow. All the others at the funeral, mere bystanders, I found intolerable. The comforting phrases they told father---that things were better for her there, that she was not made for this world---made me very annoyed.

What right did they have to talk about her and cry over her? A few of them, in talking about us, called us \textit{orphans}. As if without them no one would have known that children who have no mother are called by that name! They truly seemed to enjoy being the first to give us that name, just like people usually fight to be the first to call a newly married girl \textit{Madame}. 

In the far corner of the hall, almost hidden behind the open door of the staging area off the dining room, a gray, hunched-over old lady stood on her knees. Her hands folded and her eyes raised to the sky, she was not crying, but praying. Her soul was seeking God, and she asked Him to reunite her with the person she loved more than anyone else on earth, firmly hoping that it would be soon.

``That's who truly loved her!'' I thought, and I felt ashamed of myself. %nikolai

The funeral service ended; the face of the deceased was visible, and all those present, except us, began to approach the coffin and give her a kiss.

One of the last to come up to say farewell to the deceased was a peasant woman with a pretty five-year-old girl in her arms, who, God knows why, she had brought with her. At that moment, I had accidentally dropped my wet handkerchief and wanted to pick it up; but as soon as I bent over, I was struck by a strange, piercing cry, full of such terror that, if I live to be a hundred, I will never forget it, and when I remember, a cold shudder always runs through my body. I raised my head---that same peasant woman was standing on a stool near the coffin and holding the girl in her arms with great difficulty; the girl was waving her little hands, throwing back her frightened face, and staring wide-eyed at the face of the deceased, screaming in a strange, frantic voice. I gave a shout in a voice which I suspect was even more terrifying than the one that had struck me, and ran out of the room.

Only at that moment did I understand what was causing the strong, heavy smell that, mixing with the smell of incense, had filled the room; and the thought that that face, which only a few days before had been filled with beauty and tenderness, the face of the person I loved more than anyone else on earth, could arouse terror, that thought opened to me for the first time the bitter truth about what had happened and filled my heart with despair.

\chapter{Some Final Sad Remembrances} %c28

Maman was already gone, and our life took its usually course: we went to sleep and got up at the same times and in the same rooms; morning and evening tea, dinner, supper---everything at its usual time; tables, chairs stood in the same places; nothing in the house and in our manner of living changed; except she was not there\ldots{}

It seemed to me that after such misfortunate, everything ought to change completely; our ordinary way of life seemed to me an insult to her memory and reminded me too vividly of her absence.

On the day before the burial, after dinner, I wanted to sleep, and I went into Natalya Savishna's room, expecting to settle in on her bed, on the soft feather mattress under the quilted blanket. When I came in, Natalya Savishna was lying in her bed and seemed to be sleeping; having heard the sound of my steps, she sat up, threw off her wool shawl, which was covering her head to keep flies away, and, straightening her cap, she sat on the edge of the bed. 

Since it had previously been common for me to come sleep in her room after dinner, she guessed why I had come and said to me, getting up from the bed:

``What? You've come to rest a little, right, dearie? Lie down.'' %natalya

``What do you mean, Natalya Savishna?'' I said, holding her by the hand. ``I'm not here for that at all\ldots{} I came to\ldots{} I mean, you're tired yourself: better you should lie down.'' %nikolai

``No, my precious boy, I've had enough sleep,'' she said to me (I knew she had not slept for three days). ``I don't feel like sleeping anyway,'' she added with a deep sigh. %natalya

I wanted to talk to Natalya Savishna about our misfortune; I knew her sincerity and love, and for that reason it would have been a comfort to cry with her.

``Natalya Savishna,'' I said, falling silent for a bit and sitting down on the bed, ``did you expect this?'' %nikolai

The old woman looked at me with bewilderment and curiosity, apparently not understanding why I was asking her about this.

``Who could have expected it?'' I repeated. %nikolai

``Ah, precious boy,'' she said, throwing me a look of tender compassion, ``it's not about what we expected, and I can't believe it even now. It's long since time for me to lay my old bones down to rest; look what I've had to put up with from living so long: the old master---your grandfather, eternal rest be unto him, Prince Nikolai Mikhaylovich---my two brothers, my sister Annushka, I've buried them all, and they were all younger than me, precious boy, and now it looks like I've had to outlive her, too, for my sins. His holy will! He must have taken her because she was deserving, and he needs good ones up there.'' %natalya

This simple, striking idea was comforting to me, and I moved closer to Natalya Savishna. She crossed her arms over her chest and looked up; her fallen, damp eyes expressed a great, but calm sort of grief. She hoped firmly that God would not keep her long from the one on whom, for so many years, all the strength of her love had been focused.

``Yes, my precious boy, I nursed her and swaddled her for a long time, and she called me Nasha. She would run over to me sometimes, grab me by her little hands, and start kissing me and saying: %natalya

```My Nashik, my pretty one, my little turkey.' %natalya

``And I would joke with her and say: %natalya

```It's not true, precious girl, you don't love me; you watch, you'll grow up big, you'll get married, and you'll forget about your Nasha.' She would think about it and say, `No, I won't get married then if I can't take Nasha along with me; I won't ever leave my Nasha behind.' And here she has left me and didn't wait for me. But she loved me, your late mother! Then again, who didn't she love, honestly! Yes, precious boy, you should never forget your mother; she wasn't just a person, she was an angel from heaven. When her soul finally gets to the heavenly kingdom, she will love you from there, too, and she'll be so pleased with you there.'' %natalya

``Why are you talking about when she finally gets to the heavenly kingdom?'' I asked. ``I think she's already there now.'' %nikolai

``No, precious boy,'' Natalya Savishna said, lowering her voice and sitting closer to me on the bed. ``Her soul is here now.'' %natalya

And she pointed upwards. She was talking almost in a whisper and with such feeling and conviction that I automatically raise my eyes upwards, looking at the cornices and searching for something. ``Before the soul of a righteous person goes to paradise, it must go through forty trials, precious boy, over forty days, and it often stays in the house\ldots{}'' %natalya

She kept on speaking in that vein for a long time, and she spoke with simplicity and confidence, as if she was talking about the most ordinary things, which she had seen herself and about which no one could possibly have the slightest doubt. I listened to her, holding my breath, and although I did not quite understand what she was saying, I believe her completely.

``Yes, precious boy, she's here now, watching us, listening, maybe, to what we're saying,'' concluded Natalya Savishna. %natalya

And, lowering her head, she fell silent. She needed a handkerchief to wipe away the tears that had fallen; she stood up, looked me right in the face, and said with a voice trembling from emotion:

``The Lord has moved me a lot of steps closer to Himself with this. What do I have left? Who do I have to live for? Who to love?'' \todo{whom?} %natalya 

``Don't you love us, though?'' I said reproachfully, barely holding back tears. %nikolai

``God knows how much I love you, all my dearies, but to love someone like I loved her\ldots{}I've never loved anyone like that, and I can't love anyone that much.'' %natalya

She could not speak anymore, turned away from me, and began sobbing loudly.

I had forgotten about sleep by then; we sat silently across from one another and cried.

Foka came into the room; noticing our situation and, probably not wanting to disturb us, he stopped by the doors, silent and looking at us timidly.

``What do you need, Fokasha?'' Natalya Savishna asked, wiping her face with the handkerchief. %natalya

``A pound and a half of raisins, four pounds of sugar, and three pounds of rice for the rice pudding, please.''\footnote{The dish mentioned here is \textit{kut'ja}, traditionally made around Christmas and Epiphany, as well as for burials and memorials. --- \textit{Trans.}} %foka

``Just a minute, just a minute, dear,'' Natalya Savishna said, hurriedly taking snuff and moving with quick steps toward a trunk. The last traces of grief brought on by our conversation disappeared when she turned to her duties, which she considered extremely important.

``Why four pounds?'' she said peevishly, getting the sugar and weighing it out on the balance. ``Three and a half would be enough.'' %natalya

And she took a few pieces off of the scales.

``And what do you think of that, I just let go of eight pounds of grain yesterday, and now you're asking for more. You'll do what you like, Foka Demidych, but I will not give you any rice. That Vanka is happy that all is chaos and confusion in the house: maybe he thinks no one will notice. No, I'm not going to indulge anyone out of the master's stores. Have you ever seen such a thing---eight pounds?'' %natalya

``What can I do?'' he said. ``It was all used up.'' %foka

``Well, here, take it, here! Let him have it!'' %natalya

I was struck by that transition from the touching emotion with which she had spoke with me to peevishness and bean counting. Thinking about it later, I understood that, despite what was going on in her heart, she still had enough presence of mind to do her work, and the strength of habit drew her toward her usual tasks. Sorrow was affecting her so strongly that she did not even need to hide the fact that she was able to worry about extraneous subjects; she would not even have understood how someone could think it strange of her.

Vanity is a feeling completely incompatible with true sorrow, and yet that feeling is so deeply inculcated in human nature that only rarely can even the strongest sorrow banish it completely. Vanity in sorrow shows up in the desire to seem either grief-stricken, or unhappy, or hard; and these low desires, which we will not admit, but which almost never leave us---even in the worst grief---take away its strength, dignity, and sincerity.\todo{possible to follow?} Natalya Savishna was so deeply struck by her misfortune that not a single desire remained in her heart, and she was living only on habit.

Having given Foka the provisions he had asked for and reminded him about the pie that needed to be made to feed the clergyman, she let him go, took the stocking, and again sat down near me.

Our conversation started from the same place, and we again cried and again wiped away our tears.

My conversations with Natalya Savishna continued every day; her quiet tears and calm, pious speeches gave me comfort and relief.

But we were soon parted: three days after the burial, everyone in the house left for Moscow, and I was destined never to see her again.

Grandmother received the awful news of what had happened only when we returned, and her sorrow was unusually strong. We were not allowed to see her, because she was in a daze for an entire week, the doctors feared for her life, especially because she would not only not take any medication, but would not speak with anyone, did not sleep, and would not take any food. Sometimes, sitting alone in the room, in her armchair, she would suddenly begin to laugh, then sob without shedding tears, she experienced convulsions, and she would shout senseless or terrible words in a frantic voice. It was the first real sorrow she had ever felt, and that sorrow drove her to despair. She needed to blame someone for her misfortune, and she said frightening things, threatened someone with uncommon force, would jump out of her armchair, walk back and forth through the room taking long strides, and then fall unconscious.

One time I did go into her room: she was sitting in her armchair as usual, and she seemed to be calm; but I was struck by how she was staring. Her eyes were wide open, but her gaze was undefined and vacant: she was looking right at me, but did not seem to see me. Her lips slowly began to smile, and she said in a tender, touching voice: ``Come here, my friend, come closer, my angel.'' I thought she was addressing me, and I came closer, but she was not looking at me. ``Ah, if you knew, my soul, how I've been suffering, and how happy I am that you've come, my girl\ldots{}'' I understood that she imagining that she could see maman, and I stopped. ``They told me you were gone,'' she continued, furrowing her brow, ``what rubbish! How could you die before me?'' And she roared with a frightening, hysterical laughter.

Only people who are capable of loving deeply can experience deep sorrow; but that need to love serves to counteract the sorrow and heals them. Because of that, humanity's moral nature is more tenacious than physical nature. Sorrow never kills.

After a week, grandmother could cry, and she got better. Her first thought, when she came back to her senses, was us, and her love for us increased. We barely left her armchair; she would cry quietly, talk about maman, and caress us tenderly.

It would not have occurred to anyone to think, looking at grandmother's grief, that she was exaggerating it, and her expression of that grief was strong and touching; but I sympathized more with Natalya Savishna, I do not know why, and to this day I am convinced that no one loved and missed maman as sincerely and purely as that simple-hearted and loving creature.

With my mother's death, the happy time of childhood ended for me, and a new period began---the period of adolescence; but since my memories about Natalya Savishna, whom I never saw again and who had such a strong and positive influence on my direction and the development of my sensitivity toward others, belongs to the first period, I will see a few words about her and her death.

After our departure, as people who had remained in the country later told me, she became very bored with nothing to do. Although all the trunks remained in her hands, and she never ceased digging in them, repacking, weighing out, unpacking; she missed the news and fuss of the master's country house when inhabited by the master and his family, which she was used to from childhood. Sorrow, a change in the manner of her life, and the absence of worries soon brought on the infirmities of old age, to which she had a tendency. Exactly a year after my mother passed on, she began suffering from an edema, and she took to bed.

I think it was hard for Natalya Savishna to live, and even harder to die, alone, in the big, empty house at Petrovskoye, without her relatives, without friends. Everyone in the house loved and respected Natalya Savishna; but she was not friends with anyone and was proud of that. She thought that in her position as the housekeeper, enjoying the trust of her masters and having at hand all those trunks with valuable goods, friendship with someone would certainly have driven her to favoritism and criminal leniency; for that reason, or perhaps because she had nothing in common with the other servants, she kept her distance from everyone and said that she had neither any godfathers nor godsons in the house, and would not indulge anyone out of the master's stores.

Entrusting all her feelings in warm prayers to God, she sought and found consolation; but sometimes, in moments of weakness, to which we are all liable, when the concern and tears of a living being bring us the best consolation, she would put her pug on the bed with her (which would lick her hands, staring at her with its yellow eyes), speak to it, and cry quietly, petting it. When the pug began to howl mournfully, she would try to calm it down and would say, ``Enough, I don't need you to tell me I'm going to die soon.'' %natalya

A month before her death, she took some white calico fabric, white muslin, and pink ribbons out of her trunk; with the help of a girl, she sewed herself a white dress, a cap, and saw to everything that was needed for her burial in the smallest details. She also sorted out all the master's trunks and gave everything over to the steward's wife with an incredibly clear inventory: then she took out two silk dressed, an ancient shawl that grandmother had given her at some point, grandfather's military uniform, embroidered in gold, which had also been given to her personally. Thanks to her careful diligence, the embroidery and braid on the uniform were completely fresh, and the fabric was untouched by moths.

Before she passed on, she expressed her desire\todo{clunky} that one of the dresses---the pink one---be given to Volodya for a dressing-gown or a beshmet,\footnote{A beshmet was a kind of Russian kaftan with a high collar. --- \textit{Trans.}} and the puce one with a windowpane pattern was for me, for the same use; the shawl was for Lyubochka. She willed the uniform to whichever of us became an officer first. The rest of her estate and money---with the exception of forty rubles, which she set aside for her burial and services of remembrance---she gave to her brother. Her brother, who had long since gotten his freedom, lived in some distant province and lived a dissipated life; because of that, she had no dealings of any kind with him in life. 

When Natalya Savishna's brother appeared to receive the inheritance, and the deceased's entire estate turned out to be twenty-five rubles in notes, he did not want to believe it and said that it was impossible for an old woman, who had lived sixty years in a rich house, had everything she needed at hand, was stingy her entire life, and held on to every rag, could leave nothing behind her. But it was in fact so.

Natalya Savishna suffered from her illness for two months and endured her suffering with true Christian patience: she did not grumble, did not complain, but only prayed to God, as was her habit. The hour before her death, she made her confession with quiet happiness, took the the sacrament, and received extreme unction.

She asked all the servants for forgiveness for any wrongs she may have done them, and asked her confessor, Father Vasil,\todo{Sic!} to tell us that she did not know how to thank us for our kindness, and asked us to forgive her if, because of her stupidity, she caused any of us pain, ``but I was never a thief, and I can say I never profited off a single thread from the master.'' That was one quality that she valued in herself.

Having put on the house-dress and cap she had prepared and propped her elbows on the pillows, she continued talking to the priest until the very end, remembered that she had not left anything for the poor, got out ten rubles, and asked that they be given out in the parish; then, crossing herself, she lay down and breathed her last, with a happy smile, pronouncing the name of God.

She left life with no regrets, was not afraid of death, and took it as a blessing. They often say that about people, but it is so rarely actually like that! Natalya Savishna could not be afraid of death, because she died with a steadfast faith and having fulfilled the rule of the Gospel. Her whole life had been pure, selfless love and self-sacrifice.

So what if her beliefs could have been more lofty, her life directed at higher aims; is her pure soul really any less worthy of love and wonderment?\todo{?}

She performed the best and greatest act in this life---she died without regret and without fear.

They buried her, according to her wishes, not far from the chapel that stood over mother's grave. The little hillock, overgrown with nettles and burdock, under which she lies, is surrounded by a black fence, and I never forget to come over to that fence from the chapel and bow all the way to the ground.

Sometimes I stop silently between the chapel and the black fence. In my heart, heavy remembrances suddenly awaken in my heart. The thought comes to me: did Providence really unit me with these two creatures only to make me miss them forever\ldots{}?
