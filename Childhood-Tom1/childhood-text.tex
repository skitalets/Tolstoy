\part{Childhood}
\makeoddhead{modruled}{}{\scshape Childhood}{}
\markboth{Childhood}{}

\chapter{Karl Ivanych, the Tutor}

On the 12th of August, 18\ldots{}, exactly the third day after my birthday, when I turned ten and received such marvelous gifts, Karl Ivanych woke me up at seven o'clock in the morning, hitting the wall just above my head with a flyswatter made from sugar paper and a stick. He did this so carelessly that he hit the icon of my angel\todo{that is, patron saint---footnote needed}, which was hanging from the oak headboard, and the dead fly fell right on my head. I stuck my nose out from under the cover, stopped the icon, which was still swinging, with my hand, knocked the dead fly onto the floor, and threw an angry look at Karl Ivanych, my eyes still clouded with sleep. He was wearing a motley cotton dressing-gown, belted with a belt from the same material, a red knit cap with a tassel, and soft goat-skin \todo{?} boots, and he continued to walk along the walls, looking for targets and slapping with his flyswatter.

``Fine,'' I thought, ``I \emph{am} small, but why does he keep bothering me? Why doesn't he swat flies by Volodya's bed? There are so many over there! No, Volodya is older than me; I'm the youngest: that's why it's me he torments. That's what he thinks about all day,'' I whispered, ``how to make trouble for me. He can see very well that he woke me up and frightened me, but he acts like he doesn't notice\ldots{} what a disgusting person! His robe, his little hat, and that tassel---disgusting!''

At the same time I was mentally expressing my annoyance with Karl Ivanych, he went over to his bed, looked at his watch, which was hanging over it in a beaded slipper, hung the flyswatter up on its nail, and, obviously in a very pleasant mood, turned to us.

\textit{``Auf, Kinder, auf!\ldots{} s'ist Zeit. Die Mutter ist schon im Saal,''} he shouted in his kind German voice, then he went over to me, sat by my feet, and took his snuff box out of his pocket.\footnote{Up, children, up! 'Tis time. Mother is already in the drawing room. \textit{(Ger.)}} I pretended I was still sleeping. Karl Ivanych took snuff, wiped his nose, snapped his fingers, and then started in on me. Laughing, he started to tickle the soles of my feet. \textit{``Nu, nun, Faulenzer,''} he said.\footnote{Well, now, lazy boy! \textit{(Ger.)}}

However much I dreaded the tickling, I did not jump out of bed and did not answer him, instead hiding my head deeper under the pillows, kicking my feet as hard as I could and doing everything I could to keep from laughing.

``He's so kind and loves us so much, and I was thinking such awful things about him!''

I was annoyed both at myself and at Karl Ivanych, wanting to laugh and wanting to cry: my nerves were unsettled.

\textit{``Ach, lassen sie, Karl Ivanych!''}~I shouted with tears in my eyes, sticking my head out from under the pillows.\footnote{Leave me alone, Karl Ivanych! \textit{(Ger.)}}

Karl Ivanych seemed surprised, left the soles of my feet in peace, and started asking with concern: what was it? had I had a nightmare?\ldots{} his kind German face and the sympathy with which he tried to guess the reason for my tears made them flow all the more abundantly: I was ashamed, and I could not understand how just a minute before I could not love Karl Ivanych and find his robe, hat, and tassel so disgusting; now, on the contrary, all of this seemed extremely endearing, and even the tassel seemed like a clear indication of his kindness. I told him that I was crying because I had had a nightmare---as if \textit{maman} had died and they were taking her to the grave. I made all this up, because I absolutely could not remember what I had dreamed about that night; but when Karl Ivanych, touched by my story, started to comfort and soothe me, I started thinking I had indeed had that nightmare, and my tears started flowing for another reason. 

When Karl Ivanych left me, and I, sitting up a little in bed, started to pull my stockings \todo{?} onto my little legs, my tears had subsided a little, but the dark thoughts brought on by my imagined dream did not leave me. Uncle \todo{need a footnote...} Nikolai came in---a small, neat little man, always serious, exact, respectable, \todo{?} and a great friend of Karl Ivanych. He brought our dresses and shoes: boots for Volodya, for me still an intolerable pair of children's shoes with bows. \todo{specific term for shoes?} I was ashamed to cry in front of him; the morning sun was shining through the window besides, and Volodya, mimicking Marya Ivanovna (our sisters' governess), was laughing so happily and sonorously, standing over the wash stand, that even serious old Nikolai, a towel over his shoulder, with soap in one hand and a basin in the other, smiling, said:

``That's enough, Vladimir Petrovich, kindly wash up.''

I cheered up completely.

\textit{``Sind sie bald fertig?''} Karl Ivanych's voice came from the classroom.\footnote{Are you almost ready? \textit{(Ger.)}}

His voice was strict and no longer had the same tone of kindness that had touched me and brought me to tears. In the classroom, Karl Ivanych was an entirely different person: he was a schoolmaster. I dressed quickly, washed up, and, with a brush in my hand, smoothing my wet hair, I answered his call.

Karl Ivanych, with glasses on his nose and a book in his hand, sat in his usual place, between the door and the window. To the left of the door were two little shelves: one was ours, the children's, the other was Karl Ivanych's, his \emph{personal} shelf. On ours were all sorts of books---educational and non-educational: some were standing up, others were lying down. Only the two large volumes of \textit{Histoire des voyages} in red covers rested sedately against the wall;\footnote{History of the Voyages. \textit{(Fr.)}} \todo{Voltaire?} then there were long, thick, large, and small books---covers without books and books without covers; we used to push and shove them in there before recess, when we were told to put the library, as Karl Ivanych loudly called that shelf, in order. The collection of books on his \emph{personal} shelf, if not so large as the one on ours, was even more varied in composition. From those books, I remember three: a German brochure on the manuring of cabbage gardens---without a cover, one volume containing a history of the Seven Years' War---in parchment, burnt at one corner, and a full course on hydrostatics. Karl Ivanych spent the greater part of his time reading, even ruining his eyes in the process; but other than those books and the \textit{Northern Bee}, he read nothing else.

Among the items lying on Karl Ivanych's shelf, I am constantly reminded of one in particular. It was a little circle of cardboard, set on a wooden base to which the cardboard circle was attached by way of a few pins. On this circle, a picture was pasted, a caricature of a lady and a hairdresser. Karl Ivanych was very good at pasting things together and had devised the circle himself and made it to protect his weak eyes from bright light.\todo{Run on ok?}

I can see him before me now a long figure in a cotton dressing-gown with sparse grey hair sticking out from his little red hat. He sits by a little table, on which the circle with the hairdresser is standing, casting a shadow over his face; in one hand he holds a book, the other rests on the arm of the chair; near him lies his watch, with a painted hunter on the dial, a checkered handkerchief, a round black snuff box; a green eyeglass case, a set of tongs on a tray. \todo{I think this is right, not translated by the Maudes!} Every item was so sedately, so exactly placed that by their good order alone one could conclude that Karl Ivanych's conscience was clean and his soul at peace.

At times, after we had had enough of running around downstairs in the drawing room, I would sneak up to the classroom on tiptoe---Karl Ivanych would be sitting there alone in his chair, reading one of his favorite books with a peaceful and sublime expression. Sometimes I would find him in a moment when he was not reading: his glasses would be at the very end of his aquiline nose, his blue eyes, half-closed, would be staring off into the distance with a peculiar gaze, and his lips would be smiling sadly. It would be quiet; the only sounds would be his even breathing and the ticking of his watch with the hunter.

At times he would not notice me, and I would stand at the door and think: poor, poor old man! There are so many of us, we are playing, we are enjoying ourselves, and he is all by his lonesome, \todo{neologism, ok?} and he has no one to show him affection. It is true what he says, that he is an orphan. And the story of his life is so terrible! I remember when he told it to Nikolai---it would be terrible to be in his position! And I feel such pity for him that I would at times come up to him, grab his hand, and say, \textit{``Lieber Karl Ivanych!''} He loved it when I talked to him like that; he would always look at me with affection, and I could tell that he was deeply moved.

On the second wall hung maps, nearly all of them tattered, but skillfully mended by Karl Ivanych's hand. On the third wall, in the middle of which was the door leading downstairs, on one side hung two rulers: one, ragged, was ours, the other, a new one, was his \emph{personal} ruler, used more as a spur to hard work than for drawing lines; on the other side was a blackboard on which our serious offences were marked with circles and our minor ones with little crosses. To the left of the board was a corner, where they made us kneel.

How I remember that corner! I remember the flap of the stove, the vent in that flap, and the noise that it made when you moved it. At times I would be kneeling there in the corner, kneeling, kneeling, so that my knees and back hurt, and I would think: Karl Ivanych has forgotten about me: I suppose he is sitting there peacefully in his soft chair and reading his hydrostatics---and what about me? And I would start to open and shut the vent or pick plaster off from the wall; but if too large a piece suddenly fell on the ground with a loud noise, well, the fear alone was worse than any punishment. I would glance over at Karl Ivanych---and he would be sitting quietly with a book in his hand, as if he had not noticed a thing.

In the middle of the room stood a table covered with ragged oilcloth, the corners visible underneath ragged with the cuts of our pen knives. Around the table were a few stools, unfinished but worn smooth from long use. The final wall was taken up by three little windows. This was the view from them: right under the windows was a road, every potholes, every pebble, every rut of which was long since known and dear to me; beyond the road was a little park full of trimmed lilac, behind which a wicker fence could be seen; through the trees, a meadow was visible, with a barn to one side of that and a forest opposite; far away in the forest, the warden's little hut was visible. From the window, to the right, part of the terrace was visible, where the grown-ups would usually sit until dinner. At times, while Karl Ivanych was correcting our dictation, I would look out that side, see mother's black hair, someone's back, and hear vague conversations and laughter from there; then I would get annoyed that I could not be there and think: ``when will I be a grown-up, stop studying, and be able to sit with the people I love instead of doing dialogues?'' My annoyance would turn into sadness and, God knows why, or what I was thinking about, but I would not hear a word of Karl Ivanych, who was angry about my mistakes.

Karl Ivanych took off his robe, put on his blue tail coat with pleats and folds \todo{?} on the shoulders, straightened his tie in front of the mirror, and guided us downstairs to say good morning to mother.

