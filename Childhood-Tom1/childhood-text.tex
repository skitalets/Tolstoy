\part*{Childhood}
\makeoddhead{modruled}{}{\scshape Childhood}{}
\markboth{Childhood}{}

\chapter{Karl Ivanych, the Tutor}

On the 12th of August, 18\ldots{}, exactly the third day after my birthday, when I turned ten and received such marvelous gifts, Karl Ivanych woke me up at seven o'clock in the morning, hitting the wall just above my head with a flyswatter made from sugar paper and a stick. He did this so carelessly that he hit the icon of my angel\todo{that is, patron saint---footnote needed}, which was hanging from the oak headboard, and the dead fly fell right on my head. I stuck my nose out from under the cover, stopped the icon, which was still swinging, with my hand, knocked the dead fly onto the floor, and threw an angry look at Karl Ivanych, my eyes still clouded with sleep. He was wearing a motley cotton dressing-gown, belted with a belt from the same material, a red knit cap with a tassel, and soft goat-skin \todo{?} boots, and he continued to walk along the walls, looking for targets and slapping with his flyswatter.

``Fine,'' I thought, ``I \emph{am} small, but why does he keep bothering me? Why doesn't he swat flies by Volodya's bed? There are so many over there! No, Volodya is older than me; I'm the youngest: that's why it's me he torments. That's what he thinks about all day,'' I whispered, ``how to make trouble for me. He can see very well that he woke me up and frightened me, but he acts like he doesn't notice\ldots{} what a disgusting person! His robe, his little hat, and that tassel---disgusting!''

At the same time I was mentally expressing my annoyance with Karl Ivanych, he went over to his bed, looked at his watch, which was hanging over it in a beaded slipper, hung the flyswatter up on its nail, and, obviously in a very pleasant mood, turned to us.

\textit{``Auf, Kinder, auf!\ldots{} s'ist Zeit. Die Mutter ist schon im Saal,''} he shouted in his kind German voice, then he went over to me, sat by my feet, and took his snuff box out of his pocket.\footnote{Up, children, up! 'Tis time. Mother is already in the hall. \textit{(Ger.)}} I pretended I was still sleeping. Karl Ivanych took snuff, wiped his nose, snapped his fingers, and then started in on me. Laughing, he started to tickle the soles of my feet. \textit{``Nu, nun, Faulenzer,''} he said.\footnote{Well, now, lazy boy! \textit{(Ger.)}}

However much I dreaded the tickling, I did not jump out of bed and did not answer him, instead hiding my head deeper under the pillows, kicking my feet as hard as I could and doing everything I could to keep from laughing.

``He's so kind and loves us so much, and I was thinking such awful things about him!''

I was annoyed both at myself and at Karl Ivanych, wanting to laugh and wanting to cry: my nerves were unsettled.

\textit{``Ach, lassen sie, Karl Ivanych!''}~I shouted with tears in my eyes, sticking my head out from under the pillows.\footnote{Leave me alone, Karl Ivanych! \textit{(Ger.)}}

Karl Ivanych seemed surprised, left the soles of my feet in peace, and started asking with concern: what was it? had I had a nightmare?\ldots{} his kind German face and the sympathy with which he tried to guess the reason for my tears made them flow all the more abundantly: I was ashamed, and I could not understand how just a minute before I could not love Karl Ivanych and find his robe, hat, and tassel so disgusting; now, on the contrary, all of this seemed extremely endearing, and even the tassel seemed like a clear indication of his kindness. I told him that I was crying because I had had a nightmare---as if \textit{maman} had died and they were taking her to the grave. I made all this up, because I absolutely could not remember what I had dreamed about that night; but when Karl Ivanych, touched by my story, started to comfort and soothe me, I started thinking I had indeed had that nightmare, and my tears started flowing for another reason. 

When Karl Ivanych left me, and I, sitting up a little in bed, started to pull my stockings \todo{?} onto my little legs, my tears had subsided a little, but the dark thoughts brought on by my imagined dream did not leave me. Uncle \todo{need a footnote...} Nikolai came in---a small, neat little man, always serious, exact, respectable, \todo{?} and a great friend of Karl Ivanych. He brought our dresses and shoes: boots for Volodya, for me still an intolerable pair of children's shoes with bows. \todo{specific term for shoes?} I was ashamed to cry in front of him; the morning sun was shining through the window besides, and Volodya, mimicking Marya Ivanovna (our sisters' governess), was laughing so happily and sonorously, standing over the wash stand, that even serious old Nikolai, a towel over his shoulder, with soap in one hand and a basin in the other, smiling, said:

``That's enough, Vladimir Petrovich, kindly wash up.''

I cheered up completely.

\textit{``Sind sie bald fertig?''} Karl Ivanych's voice came from the classroom.\footnote{Are you almost ready? \textit{(Ger.)}}

His voice was strict and no longer had the same tone of kindness that had touched me and brought me to tears. In the classroom, Karl Ivanych was an entirely different person: he was a schoolmaster. I dressed quickly, washed up, and, with a brush in my hand, smoothing my wet hair, I answered his call.

Karl Ivanych, with glasses on his nose and a book in his hand, sat in his usual place, between the door and the window. To the left of the door were two little shelves: one was ours, the children's, the other was Karl Ivanych's, his \emph{personal} shelf. On ours were all sorts of books---educational and non-educational: some were standing up, others were lying down. Only the two large volumes of \textit{Histoire des voyages} in red covers rested sedately against the wall;\footnote{History of the Voyages. \textit{(Fr.)}} \todo{Voltaire?} then there were long, thick, large, and small books---covers without books and books without covers; we used to push and shove them in there before recess, when we were told to put the library, as Karl Ivanych loudly called that shelf, in order. The collection of books on his \emph{personal} shelf, if not so large as the one on ours, was even more varied in composition. From those books, I remember three: a German brochure on the manuring of cabbage gardens---without a cover, one volume containing a history of the Seven Years' War---in parchment, burnt at one corner, and a full course on hydrostatics. Karl Ivanych spent the greater part of his time reading, even ruining his eyes in the process; but other than those books and the \textit{Northern Bee}, he read nothing else.

Among the items lying on Karl Ivanych's shelf, I am constantly reminded of one in particular. It was a little circle of cardboard, set on a wooden base to which the cardboard circle was attached by way of a few pins. On this circle, a picture was pasted, a caricature of a lady and a hairdresser. Karl Ivanych was very good at pasting things together and had devised the circle himself and made it to protect his weak eyes from bright light.\todo{Run on ok?}

I can see him before me now a long figure in a cotton dressing-gown with sparse grey hair sticking out from his little red hat. He sits by a little table, on which the circle with the hairdresser is standing, casting a shadow over his face; in one hand he holds a book, the other rests on the arm of the chair; near him lies his watch, with a painted hunter on the dial, a checkered handkerchief, a round black snuff box; a green eyeglass case, a set of tongs on a tray. \todo{I think this is right, not translated by the Maudes!} Every item was so sedately, so exactly placed that by their good order alone one could conclude that Karl Ivanych's conscience was clean and his soul at peace.

At times, after we had had enough of running around downstairs in the hall, I would sneak up to the classroom on tiptoe---Karl Ivanych would be sitting there alone in his chair, reading one of his favorite books with a peaceful and sublime expression. Sometimes I would find him in a moment when he was not reading: his glasses would be at the very end of his aquiline nose, his blue eyes, half-closed, would be staring off into the distance with a peculiar gaze, and his lips would be smiling sadly. It would be quiet; the only sounds would be his even breathing and the ticking of his watch with the hunter.

At times he would not notice me, and I would stand at the door and think: poor, poor old man! There are so many of us, we are playing, we are enjoying ourselves, and he is all by his lonesome, \todo{neologism, ok?} and he has no one to show him affection. It is true what he says, that he is an orphan. And the story of his life is so terrible! I remember when he told it to Nikolai---it would be terrible to be in his position! And I feel such pity for him that I would at times come up to him, grab his hand, and say, \textit{``Lieber Karl Ivanych!''} He loved it when I talked to him like that; he would always look at me with affection, and I could tell that he was deeply moved.

On the second wall hung maps, nearly all of them tattered, but skillfully mended by Karl Ivanych's hand. On the third wall, in the middle of which was the door leading downstairs, on one side hung two rulers: one, ragged, was ours, the other, a new one, was his \emph{personal} ruler, used more as a spur to hard work than for drawing lines; on the other side was a blackboard on which our serious offences were marked with circles and our minor ones with little crosses. To the left of the board was a corner, where they made us kneel.

How I remember that corner! I remember the flap of the stove, the vent in that flap, and the noise that it made when you moved it. At times I would be kneeling there in the corner, kneeling, kneeling, so that my knees and back hurt, and I would think: Karl Ivanych has forgotten about me: I suppose he is sitting there peacefully in his soft chair and reading his hydrostatics---and what about me? And I would start to open and shut the vent or pick plaster off from the wall; but if too large a piece suddenly fell on the ground with a loud noise, well, the fear alone was worse than any punishment. I would glance over at Karl Ivanych---and he would be sitting quietly with a book in his hand, as if he had not noticed a thing.

In the middle of the room stood a table covered with ragged oilcloth, the corners visible underneath ragged with the cuts of our pen knives. Around the table were a few stools, unfinished but worn smooth from long use. The final wall was taken up by three little windows. This was the view from them: right under the windows was a road, every potholes, every pebble, every rut of which was long since known and dear to me; beyond the road was a little park full of trimmed lilac, behind which a wicker fence could be seen; through the trees, a meadow was visible, with a barn to one side of that and a forest opposite; far away in the forest, the warden's little hut was visible. From the window, to the right, part of the terrace was visible, where the grown-ups would usually sit until dinner. At times, while Karl Ivanych was correcting our dictation, I would look out that side, see mother's black hair, someone's back, and hear vague conversations and laughter from there; then I would get annoyed that I could not be there and think: ``when will I be a grown-up, stop studying, and be able to sit with the people I love instead of doing dialogues?'' My annoyance would turn into sadness and, God knows why, or what I was thinking about, but I would not hear a word of Karl Ivanych, who was angry about my mistakes.

Karl Ivanych took off his robe, put on his blue tail coat with pleats and folds \todo{?} on the shoulders, straightened his tie in front of the mirror, and guided us downstairs to say good morning to mother.

\chapter{Maman}

Mother sat in the drawing room and was pouring tea; with one hand she held the teapot, and with the other the tap of the samovar, from which water was flowing through the top of the teapot and onto the tray. But although she was staring intently, she did not notice this, nor did she notice that we had come in.

So many memories of the past surface when you try to resurrect features of a loved one that you can see them only vaguely through those memories, clouding your vision like tears. Those are tears of the imagination. When I try to remember mother as she was at that time, I see only her brown eyes, which always expressed kindness and love equally, the mole on her neck, a little below the place where her little curly hairs began, her embroidered white collar, her dry, tender hand, which caressed me so often and which I often kissed; but the overall picture slips eludes me.

To the left of the sofa stood an old English piano; in front of the piano sat my dark-looking sister, Lyubochka, her pink fingers, just washed with cold water, playing Clementi's etudes with visible effort. She was eleven years old; she went around in a short linen dress and white, lace-trimmed pants and could manage octaves only in arpeggio. Near her, half-turned away, sat Marya Ivanovna in a cap with pink ribbons, in a light blue caraco jacket and with a red, angry face, which took on an even stricter expression as soon as Karl Ivanych walked in. She looked at him menacingly and, not answering his bow, continued, tapping her foot, to count: \textit{un, deux, trois, un, deux, trois} louder and more demandingly than before.

Karl Ivanych, not paying even the slightest attention to this, as usual, went directly to mother's hand with a German greeting. She came to her senses, shook her head, as if by this gesture trying to drive away unpleasant thoughts, gave her hand to Karl Ivanych and kissed him on his wrinkled temple, at the same time he kissed her hand:

\textit{``Ich danke, lieber Karl Ivanych,''} and, continuing to speak in German, she asked:\footnote{My thanks, dear Karl Ivanych. \textit{(Ger.)}}

``Did the children sleep well?''

Karl Ivanych was deaf in one ear, and now, because of the noise from the piano, he could hear nothing at all. He stooped closer to the sofa, leaned with one arm on the table, supporting himself with one leg, and, with a smile that seemed to me the height of sophistication, lifted his hat a little off his head and said:

``You will forgive me, Natalya Nikolayevna?''

Karl Ivanych, so that his bare head did not catch cold, never took off his little red hat, but he asked forgiveness for this each time he came into the drawing room.

``Put it back on, Karl Ivanych\ldots{} I am asking you, did the children sleep well?'' asked \textit{maman} fairly loudly, moving closer to him.

But he again heard nothing, covered his bald spot with the little red hat, and smiled even more sweetly.

``Stop for a minute, Mimi,'' said \textit{maman} to Marya Ivanovna with a smile: ``we can't hear anything.''

When mother smiled, though her face was already pretty, she became incomparably more beautiful, and everything around her seemed to become brighter. If in the heaviest moments of my life I could have just caught a glimpse of that smile, then I would never have known what sorrow is. It seems to me that the smile alone holds what we call beauty: if a smile adds charm to a face, then the face is very fine; if it does not change, then it is ordinary; and if it ruins the face, then it is ugly.

Having greeted me, \textit{maman} took my head in both hands and leaned it back, then looked at me fixedly and said:

``Did you cry today?''

I did not answer. She kissed my eyes and asked in German:

``What were you crying about?''

When she talked with me in this friendly way, she always spoke in that language, which she had mastered to perfection.

``It was a dream, \textit{maman}, that's why I was crying,'' I said, remembering in all its detail my made-up dream and shuddering involuntarily at the thought.

Karl Ivanych confirmed my words but was silent about the dream. Changing the subject to the weather---a conversation which Mimi also took part in---\textit{maman} put six pieces of sugar for some of the more valued servants, stood up and went over to the lace frame that stood by the window.

``Well, go to \textit{papa} now, children, and tell him he needs to come see me before he goes out to the barn.''

The music, the counting, and the menacing glances began again, and we went to \textit{papa}. Walking through another room, which from my grandfather's days was still called the footmen's pantry, we went into his office.

\chapter{Papa}

He was standing by his desk and, pointing at some envelopes, papers, and piles of money, speaking heatedly, and explaining something angrily to the steward, Yakov Mikhaylov, who was standing in his usual place between the door and the barometer, his hands placed behind his back, and was quickly moving his fingers back and forth.

The angrier \textit{papa} got, the faster the fingers moved: likewise, when \textit{papa} stopped, the fingers stopped as well; but when Yakov himself started talking, the fingers became extremely agitated and flew around desperately in every direction. It seemed to me that one could guess Yakov's secret thoughts by their movements; his face was always calm---it expressed an awareness of the merits of his case, but also of his subordinate position, in other words: I am right, however, as you like it!

Seeing us, \textit{papa} merely said:

``Wait, just a moment.''

And he nodded his head at the door, so that one of us would shut it.

``Oh, merciful heavens! What is with you today, Yakov?'' he continued saying to the steward, twitching his shoulder (he had a habit of doing that). ``This envelope, with 800 rubles enclosed\ldots{}''

Yakov moved beads on the counting frame, sliding over 800, and turned his gaze on some indeterminate point, waiting for what would come next.

``\ldots{} for housekeeping expenses in my absence. Understand? You should get 1,000 rubles from the mill\ldots{} Yes or no? You should get 8,000 back from the Treasury bonds; by your estimates we can sell 7,000 pounds of hay---I will put it out for 45 kopecks---so you should get 3,000; so how much money do you get in the end? 12,000\ldots{} Yes or no?''

``Yes, exactly, sir,'' said Yakov.

But from the quickness with which his fingers moved, I could tell he wanted to object; \textit{papa} stopped him:

``Well, from that money you will send 10,000 to the Council for Petrovskoye. Then the money that is at the bank,'' continued \textit{papa}, while Yakov mixed up the count and put up 21,000 instead of 12,000, ``you will bring to me and show as an expense as of this date.'' Yakov mixed up the count and turned over the counting-frame, apparently showing by this movement that the 21,000 would be spent. ``This envelope with the money you will send in my name to the address listed.''

I was standing close to the desk and looked at the inscription. On the envelope was written: ``To Karl Ivanovich Mauer.''

Apparently noticing that I had read something that I did not need to see, \textit{papa} put his hand on my shoulder and with an easy movement showed me away from the desk. I did not know whether it was a caress or a reproach; in any case, I kissed the large, sinewy hand that was lying on my shoulder.

``Yes, sir,'' said Yakov. ``And what orders are there for the Khaborovka money?''

Khabarovka was mother's village.

``Leave it in the bank and do not use them for anything, under any circumstances, without my orders.''

Yakov was silent for a few seconds; then suddenly his fingers began to whirl around with double the speed, and he, changing his expression from the one of dull obedience with which he listened to his master's orders to his more characteristic expression of sharp cunning, moved the counting frame toward him and began to speak:

``Permit me to report, Pyotr Aleksandrych, that it will not be possible to pay even the Council on time as you wish. You were so kind as to say,'' he continued after a pause, ``that we should receive money from the bonds, from the mill, and from the hay\ldots{}'' Counting out all of these items, he slid beads over to show each one. ``But I am afraid we may be mistaken in our estimates,'' he added, falling silent for a while and looking gravely at \textit{papa}.

``Why?''

``If you would be so kind as to see: regarding the mill, the miller has come to me twice already to ask for a deferment and swore by Christ our Lord that he did not have the money\ldots{} And he is actually here now: perhaps you would like to speak with him yourself?''

``What does he have to say?'' asked \textit{papa}, shaking his head to show that he did not want to speak with the miller.

``Isn't it clear? He says there was nothing to mill, and what little money there was, he had to put it all into the dam. So, if we take him off the mill, \emph{sir}, do you think we will make our estimates then? You were so kind as to speak about the bonds, well, I believe I already reported to you that once we put our money in them, we could not get it back soon. The other day, I sent a cart of flour to Ivan Afanasyich in the city, along with a note about this matter: he answered the same as before, I am happy to make an effort for Pyotr Aleksandrych, but the matter is not in my hands, and that everything suggests it will be two months or more until our check comes. You were so kind as to speak about the hay, let us assume that it can sell for 3,000\ldots{}''

He slid 3,000 over on the counting frame and was silent for a minute, looking first at the counting frame, then into \textit{papa's} eyes, with an expression that said:

``You can see for yourself how little this is! And we'll take a loss on the hay in any case if we sell it right now, you know that yourself\ldots{}''

It was clear that had a large reserve of such arguments left; it must have been for this reason that \textit{papa} interrupted him.

``I am not going to change my orders,'' he said, ``but if there is, in fact, any delay in receiving this money, then there's nothing that can be done, take whatever is needed from the Khaborovka money.''

``Yes, sir.''

By Yakov's fingers and the expression on his face, it was obvious that this last order gave him great pleasure.

Yakov was a slave, an extremely diligent and devoted person; like all good stewards, he was tightfisted in the extreme on his master's behalf and had the strangest conceptions about his master's interests. He was perpetually looking after profit for his master's estate at the expense of his mistress's estate, always trying to prove that it was unavoidable to spend the revenues from her property on Petrovskoye (the village in which we lived). At the present moment, he was triumphant, because he had succeeded perfectly in this.

Having greeted us, \textit{papa} told us what lazybones we had become, that we were no longer little, and that it was time for us to study seriously.

``You already know, I think, that I am going to Moscow tonight, and I am taking you with me,'' he said. ``You are going to live at your grandmother's, and \textit{maman} and the girls will stay here. And you know that only one thing will make me pleased---and that is to hear that you are studying hard and that they are satisfied with you there.''

Although the preparations that had been obvious for a few days had prepared us for something unusual to happen, this news nevertheless affected us terribly. Volodya turned red and with a quivering voice gave him mother's message.

``Then this is what my dream was predicting!'' I thought. ``Please God, don't let anything worse happen.''

I was very, very sad for mother, but, at the same time, the thought that we had become grownups made me happy.

``If we are going today, then surely there won't be any classes: that is nice!'' I thought. ``But I am sad for Karl Ivanych. They will probably let him go, or else they wouldn't have made that envelope for him\ldots{} It would be better to have to study for a year but not go away, not leave mother behind, and not treat poor Karl Ivanych so badly. He is already so unhappy!''

These thoughts flashed through my head; I did not move from the spot and looked fixedly at the black bows on my shoes.

Having said a few words with Karl Ivanych about the falling barometer and having ordered Yakov not to feed the dogs, so he could ride out after dinner on a fairwell hunt with the young hounds, \textit{papa}, against my expectations, sent us to study, consoling us, however, with a promise to take us on the hunt.

On the way upstairs I ran out onto the terrace. Near the doors, lying in the son and squinting her eyes, lay father's favorite borzoi, Milka.

``Milochka,'' I said, caressing her and kissing her snout, ``we are going today; goodbye! We'll never see each other again.''

I was deeply affected by this and cried.
 