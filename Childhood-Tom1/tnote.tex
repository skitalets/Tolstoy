\chapter{Translator's Note}

\textit{Childhood}, a semi-autobiographical work published in 1852 in the literary journal \textit{Sovremennik} (\textit{The Contemporary}), is Count Lev Nikolayevich Tolstoy's first novel. Although the novella clearly shows Tolstoy's youth and inexperience as a writer in its simple plot and occasionally rough style, the author's brilliant, physically grounded characterization and keen insight into people are also evident.

The two novellas that followed this one, \textit{Adolescence} and \textit{Youth}, round out a tetrology that Tolstoy never finished. While the three books are frequently published (and taught) together, the later books lack the innocence and earnestness of the first. Lev Nikolayevich thought so as well---he writes in the introduction to his ``Remembrances'':

\begin{quotation}
\begin{tolerant}{1000}
I regretted the fact that I had written it: it was not very good, too literary, insincerely written. [\ldots{}] I especially did not like the last two parts, \textit{Adolescence} and \textit{Youth}, in which, besides an awkward mixture of truth and fiction, there was insincerity as well: the desire to present as good and important things that I did not then consider good and important---my democratic bent.
\end{tolerant}
\end{quotation}

While there are numerous recent translations of Tolstoy, including a few of the early novellas, I nevertheless hope this translation contributes something valuable. It represents an engagement with the novella \textit{de novo}, not based on any existing translation. Because I am placing my work under a permissive license, I hope it will improve access to high quality translations overall, replacing legacy 19th and early 20th century translations where they are currently used.

\begin{tolerant}{1000}
I am grateful for the recent, massive effort exerted to bringing Lev Nikolayevich's complete works online. Thousands of volunteers from nearly 50 countries were involved, making original works, drafts and edits, critical apparatus from the Soviet era, and the rest easily available. \textit{Childhood} was translated from the 1935 edition of the \textit{Complete Works} as preserved in the Lenin Library and scanned by the Tolstoy.ru team.
\end{tolerant}

Finally, I ask one favor of the attentive reader: if you find errors, omissions, or opaque language in this translation, please drop me an email at the address below. I plan to continue refining the translation and pushing out updates to those using electronic versions of the work.

\begin{flushright}
Chris Tessone\\
chris@tessone.net
\end{flushright}
