% Copyright 2015-2016 Christopher A. Tessone
% Creative Commons Attribution-NonCommercial-ShareAlike 4.0
% International License
% http://creativecommons.org/licenses/by-nc-sa/4.0/
\makeoddhead{modruled}{}{\scshape War and Peace}{}
\markboth{War and Peace}{}

\chapter*{Part One} %c1

\section*{I} %% Book One, Part One, Chapter I

``\textit{Eh bien, mon prince. G\^enes et Lucques ne sont plus que des apanages, des} estates, \textit{de la famille Buonaparte. Non, je vous pr\'eviens, que si vous ne me dites pas, que nous avons la guerre, si vous vous permettez encore de pallier toutes les infamies, toutes les atrocit\'es de cet Antichrist (ma parole, j'y crois) --- je ne vous connais plus, vous n'\^etes plus mon ami, vous n'\^etes plus} my devoted slave, \textit{comme vous dites.} Well, hello, hello. \textit{Je vois que je vous fair peur,} sit down and
tell me everything.''\footnote{Well then, my prince, Genoa and Lucca are nothing more than appanages, estates, of the Bonaparte family. No, I am warning you, if you do not tell me that we are going to war, if you permit yourself to make excuses for all the vile deeds, all the atrocities of that Antichrist (I give you my word, I believe that)---then I do not know you, you are no longer my friend, you are no longer my devoted slave, as you say\ldots{} I can see I have frightened you\ldots{} \textit{Fr.}} %annapavlovna

This was said in July 1805 by the famous Anna Pavlovna Scherer, maid of honor and confidante of Empress Maria Feodorovna, upon meeting the important and high-ranking Prince Vasily, who was the first to arrive at her party. Anna Pavlovna had been coughing for several days, she had \textit{la grippe}, as she said (\textit{grippe} was then a new word, used only by the elite). In notes that were sent out that morning with a red-liveried footman, she had written to everyone without distinction:

\begin{quote}
\textit{Si vous n'avez rien de mieux \`a faire, M.~le comte} (or \textit{mon prince}), \textit{et si la perspective de passer la soir\'ee chez une pauvre malade ne vous effraye pas trop, je serai charm\'ee de vous voir chez moi entre 7 et 10 heures. Annette Scherer.}\footnote{If you have nothing better to do, Count (or Prince), and if the prospect of spending the evening with a poor, sick woman does not scare you off, I would be pleased to see you at my home between 7 and 10 o'clock. Annette Scherer. \textit{Fr.}}
\end{quote}

\textit{``Dieu, quelle virulente sortie!''}\footnote{God, what a vicious attack! \textit{Fr.}} answered the prince as he entered, not put out in the least by this welcome; he was dressed in an embroidered court uniform, stockings, boots, and stars and had a bright expression on his flat face.

He spoke in that refined French in which our ancestors not only spoke, but thought, and with the quiet, patronizing intonations that are particular to a significant person who has spent a long life in society and at court. He approached Anna Pavlovna, kissed her hand, presenting his scented and gleaming bald spot to her, and calmly sat down on the sofa.

``\textit{Avant tout dites moi, comment vous allez, ch\`ere amie?}\footnote{First, tell me everything, how are you doing, dear friend? \textit{Fr.}} Set my mind at ease,'' he said, not changing his voice and tone, in which indifference and even mockery could be heard behind sympathy and politeness. %vasily

``How can one be healthy\ldots{}when one is suffering morally? Can one really remain calm in our time, if one has any feeling?'' Anna Pavlovna said. ``You are staying here the whole evening, I hope?'' %annapavlovna

``But what about the English ambassador's festival? Today is Wednesday. I have to put in an appearance there,'' the prince said. ``My daughter is coming for me and will take me there.'' %vasily

``I thought that today's festival was postponed. \textit{Je vous avoue que toutes ces f\^etes et tous ces feux d'artifice commencent \`a devenir insipides.}''\footnote{I confess that all these parties and all these fireworks are starting to become dull. \textit{Fr.}} %annapavlovna

``If they had known that you wished it, they would have postponed the festival,'' the prince said out of habit, like a wound clock, saying things that he did not expect others to believe. %vasily

\textit{``Ne me tourmentez pas. Eh bien, qu'a-t-on d\'ecid\'e par rapport \`a la d\'ep\^eche de Novosilzoff? Vous savez tout.''}\footnote{Stop tormenting me. Very well, what have they decided about the report on Novosiltsov's dispatch? You know everything. \textit{Fr.}} %annapavlovna

``What can I tell you?'' the prince said in a cold, bored
tone. \textit{``Qu'a-t-on d\'ecid\'e? On a d\'ecid\'e que Buonaparte a br\^ul\'e ses vaisseuax, et je crois que nous sommes en train de br\^uler les notres.''}\footnote{What have they decided? They have decided that Bonaparte has burned his ships, and I believe that we are about to burn ours. \textit{Fr.}} %vasily

Prince Vasily always spoke lazily, like an actor says his lines in an old play. Anna Pavlovna Scherer, on the contrary, despite her forty years, was brimming with liveliness and outburts of energy.

To be an enthusiast had become her place in society, and sometimes, when she did not even want to, she played the part of the enthusiast so as not to disappoint people's expectations. The restrained smile that constantly played on Anna Pavlovna's face, although it did not fit her aging features, expressed a constant awareness, like that of a spoiled child, of her endearing shortcoming, which she found neither desirable or necessary to correct.

In the middle of this conversation about political goings-on, Anna Pavlovna became heated.

``Ah, do not talk to me about Austria! Perhaps I do not understand anything, but Austria has never wanted war and does not want it now. They will betray us. Russia alone must be the savior of Europe. Our benefactor knows his high calling and will be true to it. That is one thing that I trust. The greatest role in the world awaits our kind and marvelous sovereign, and he is so beneficent and good that God will not leave his side, and he will fulfill his calling to crush the hydra of revolution, which is now so horribly present in the person of that villain and murderer. We alone must atone for the blood of that righteous man.\footnote{Anna Pavlovna is referring to the execution of the Duke of Enghien by Napoleon in 1804. --- Trans.} In whom can we place our hope, I ask you\ldots{}? England with its commercial spirit will not rise to the heights of Emperor Aleksandr's soul and never could rise that high. They refused to liberate Malta. They want to wait and see, they are looking for an ulterior motive for our actions. What did they say to Novosiltsov\ldots{}? Nothing. They do not understand, cannot understand the self-sacrifice of our emperor, who does not want anything for himself and does everything for the good of the world. And what have they promised? Nothing. And even what they promised, they will not do! Prussia has also declared that Bonaparte cannot be beaten and that all of Europe can do nothing against him\ldots{} And I do not believe even one word from Hardenburg or Haugwitz. \textit{Cette fameuse neutralit\'e prussienne, ce n'est qu'un pi\`ege.}\footnote{That famous Prussian neutrality, it is nothing but a trap. \textit{Fr.}} I trust in God alone and in the lofty fate of our emperor. He will save Europe\ldots{}!'' She stopped suddenly with a smile, mocking her own hot temper. %annapavlovna

``I think,'' the prince said, smiling, ``that had they sent you in place of our dear Wintzingerode, you would have seized the Prussian king's agreement in a day. You are so eloquent. Could you give me some tea?'' %vasily

``Just a moment. \textit{A propos,}'' she added, calming down again, ``two very interesting people will be here today, \textit{le vicomte de Mortemart, il est alli\'e aux Montmorency par les Rohans,} one of the best families of France.\footnote{The viscount of Mortemart, he is a relative by marriage of the Montmorencys through the Rohans\ldots{} \textit{Fr.}} He is one of the good emigrants, one of the true ones. And then \textit{l'abb\'e Morio:} do you know that deep mind? He has been received by the sovereign. Do you know him?'' %annapavlovna

``Ah! I will be happy to meet him,'' the prince said. ``Tell me,'' he added, as if he had just remembered something, and especially off-handedly at that, when, in fact, the thing he was asking about was the primary reason for his visit, ``is it true that \textit{l'imp\'eratrice-m\`ere} wishes to see Baron Funke named first secretary at Vienna? \textit{C'est un pauvre sire, ce baron, \`a ce qu'il para\^it.}''\footnote{The dowager empress\ldots{} He is a poor excuse for a man, that baron, or so it seems. \textit{Fr.}} Prince Vasily wished to see his son named to the post, which others were trying to obtain for the baron through Empress Maria Feodorovna. %vasily

Anna Pavlovna half-closed her eyes to show that neither she, nor anyone else, could pass judgment about what the empress did or did not like.

\textit{``Monsieur le baron de Funke a \'et\'e recommand\'e \`a l'imp\'eratrice-m\`ere par sa soeur,''} was all she would say, in a dry, sad tone.\footnote{Baron Funke was recommended to the dowager empress by his sister. \textit{Fr.}} When Anna Pavlovna referred to the empress, her face suddenly took on an expression of deep and earnest devotion and respect, joined with sadness, which happened every time she mentioned her lofty patroness in conversation. She told him that Her Majesty had been pleased to show Baron Funke \textit{beaucoup d'estime,} and her eyes again took on a sad look.\footnote{Great respect. \textit{Fr.}} %annapavlovna

The prince fell into an indifferent silence. Anna Pavlovna, with her characteristic courtly and feminine dexterity and quickness of tact, wanted to slap him on the wrist for daring to speak up about a person who had been recommended to the empress, and at the same time to comfort him.

\textit{``Mais \`a propos de votre famille,''} she said, ``did you know that your daughter, ever since she came out, \textit{fait les d\'elices de tout le monde. On la trouve belle, comme le jour.}''\footnote{But about your family\ldots{} [She] has become the delight of the whole world. People find her to be as beautiful as the light of day. \textit{Fr.}} %annapavlovna

The prince bowed to show his respect and appreciation.

``I often think,'' Anna Pavlovna continued after a moment's silence, moving closer to the prince and smiling at him affectionately, as if to show him that talk of politics and society was over and now an intimate conversation was beginning: ``I often think about how happiness in life is sometimes distributed unfairly. Why did fate give you two wonderful children---excluding Anatoly, your youngest, him I do not love,'' she put in peremptorily, raising her brows, ``such charming children? And indeed you value them less than anyone and for that reason are not worthy of them.'' %annapavlovna

And she smiled her ecstatic smile.

\textit{``Que voulez-vous? Lafater aurait dit que je n'ai pas la bosse de la paternit\'e,''} the prince said.\footnote{What would you have me do? Lavater would say that I do not have the bump of paternity. \textit{Fr.}} %vasily

``Stop making jokes. I wanted to speak with you seriously. Do you know, I am displeased with your youngest son. Let me say, just between us,'' her face took on a sad expression, ``they were talking about him before Her Highness and people pity you\ldots{}'' %annapavlovna

The prince did not answer, but she looked at him significantly, saying nothing and waiting for an answer. Prince Vasily frowned.

``What am I supposed to do?'' he said finally. ``You know that I have done everything for their upbringing that a father can do, and both of them turned out \textit{des imb\'eciles.}\footnote{Idiots. \textit{Fr.}} Ippolit is a fool at rest, but Anatoly is restless. That is the only difference,'' he said, smiling even more unnaturally and animatedly than usual, and at the same time showing very openly an aspect that was unexpectedly coarse and unpleasant in the wrinkles around his mouth. %vasily

``And why are children born to people like you? If you were not a father, I would be able to find no fault in you,'' Anna Pavlovna said, raising her eyes pensively. %annapavlovna

\textit{``Je suis votre} devoted slave, \textit{et \`a vous seule je puis l'avouer.} My children---\textit{ce sont les entraves de mon existence.} And my cross. That is how I explain it to myself. \textit{Que voulez vous\ldots{}?''}\footnote{I am your devoted slave, and I will confess it to you alone. My children are the bane of my existence\ldots{} What would you have me do? \textit{Fr.}} He fell silent, showing with a gesture his submission to cruel fate.

Anna Pavlovna was pensive.

``You have not thought to look for a wife for your prodigal son Anatoly. They say,'' she said, ``that old maids \textit{ont la manie des mariages.} I do not yet feel that weakness in myself, but I do have a \textit{petite personne} who is unhappy with her father, \textit{une parente \`a nous, une princesse} Bolkonskaya.''\footnote{\ldots{}have an obsession with making marriages. \ldots{}Little person\ldots{}related to us, a princess Bolkonskaya. \textit{Fr.}} Prince Vasily did not answer, although with a quickness of understanding and memory characteristic of society people, he showed with a movement of his head that he had taken in all this information. %annapavlovna

``No, did you know that that Anatoly costs me 40,000 a year,'' he said, obviously unable to hold back the aggrieved course of his thoughts. He was silent. %vasily

``How will things be in five years if this continues? \textit{Voil\`a l'avantage d'\^etre p\`ere.} Is she rich, this princess?''\footnote{The benefits of being a father. \textit{Fr.}}

``The father is very rich and stingy. He lives in the country. Do you know, it is the famous Prince Bolkonsky, who was dismissed from service under the late emperor and was called the ``Prussian king.'' He is a very intelligent man, but severe and idiosyncratic. \textit{La pauvre petite est malheureuse, comme les pierres.} She has a brother, the man who just married Lise Meinen, one of Kutuzov's adjutants. He will be here tonight.''\footnote{The poor little girl is as sad as a dog with no home. \textit{Fr.}} %annapavlovna

\textit{``Ecoutez, ch\`ere Annette,''} the prince said, suddenly taking her by the hand and pulling it downward for some reason. ``\textit{Arrangez-moi cette affaire et je suis votre} most devoted slave \textit{\`a tout jamais}---slafe \textit{comme mon} foreman \textit{m'ecrit des} dispatches: ay-eff-ee. She is from a good family and rich. That is all that I need.''\footnote{Listen, dear Annette. Arrange this matter for me and I will be your most devoted slave forever---slafe as my foreman writes to me in his dispatches. \textit{Fr.}} %vasily

And with the free and familiar, graceful movements that set him apart, he took the maid of honor by her hand, kissed it, and, having kissed it, waved the maid of honor's hand around, falling back against his armchair and looking to the side.

\textit{``Attendez,''} Anna Pavlovna said, weighing the
matter. ``Tonight I will say something to Lise (\textit{la femme du jeune} Bolkonsky). And maybe it will be settled. \textit{Ce sera dans votre famille, que je ferai mon apprentissage de vieille fille.}''\footnote{Listen. \ldots{}the young Bolkonsky's wife\ldots{} It will be with your family that I undergo my apprenticeship to become an old maid. \textit{Fr.}}

\section*{II} % Book One, Part One, Chapter Two

Anna Pavlovna's drawing room was slowly beginning to fill up. The highest nobility of Petersburg was arriving, people of the most diverse ages and characters, but identical in the circle of society in which they all lived; Prince Vasily's daughter arrived, the great beauty H\'el\`ene, who had come to collect her father and go with him to the ambassador's festival. She was wearing a ball gown and the monogram that marked her as a maid of honor. Arriving at the same time was the young little Princess Bolkonskaya, well-known as \textit{la femme la plus s\'eduisante de P\'etersbourg,} who had married the previous winter and no longer went out in `great' society due to the fact that she was pregnant, but who still went to smaller parties.\footnote{The most attractive woman in Petersburg. \textit{Fr.}} Prince Ippolit, the son of Prince Vasily, arrived with Mortemart, whom he presented to everyone; Abbot Morio and many others arrived as well.

``Have you seen \textit{ma tante} yet?'' or ``Do you know her?'' Anna Pavlovna said to guests as they arrived and very earnestly brought them over to a little old woman covered with large bows, who had surfaced from another room the moment that guests began to arrive.\footnote{My aunt. \textit{Fr.}} She introduced all the guests to her by name, slowly turning her eyes from them to \textit{ma tante,} and then left. %annapavlovna

All the guests performed this rite of greeting with the unknown, uninteresting, and unnecessary aunt. Anna Pavlovna followed their greetings with sad, solemn concern, silently endorsing them from afar. \textit{Ma tante} spoke with each one in the very same expressions about their health, about her own health, and about Her Majesty's health, which was now better, thank God. All those who were brought to her, trying not to show their haste out of politeness, finally walked away from her with a feeling of relief at having fulfilled a heavy obligation and never returned to her again the whole evening.

The young Princess Bolkonskaya arrived with her work in a velvet bag with gold embroidery. Her pretty upper lip, which was covered in fine black hairs, did not quite cover her teeth, but it was endearing when her lips parted, and was still more endearing when the upper one stretched to meet the lower. As is often the case with incredibly attractive women, her defects---the shortness of her lip and her half-open mouth---seemed to be the essence of her own beauty. Everyone enjoyed watching that pretty mother-to-be, full of health and liveliness, who was enduring her condition so well. Old men and bored, gloomy young men felt that they were becoming more like her from sitting and talking with her a little while. Whoever talked with her and saw her bright smile and white teeth gleaming at every word, which were constantly visible, thought that they must be particularly charming just then. And every one of them thought just that.

The little princess got up, went around the table with quick little steps, her little work bag in her hand, and, happily fixing her dress, she sat down on the sofa near the silver samovar, as if everything she did was \textit{partie de plaisir} for herself and everyone around her.\footnote{The purest pleasure. \textit{Fr.}} 

\textit{``J'ai apport\'e mon ouvrage,''} she said, opening her handbag and addressing everyone at once.\footnote{I have brought my work. \textit{Fr.}} %lise

``See here, Annette, \textit{ne me jouez pas un mauvais tour,}'' she said, addressing the hostess. \textit{``Vous m'avez \'ecrit, que c'\'etait une toute petite soir\'ee; voyez, comme je suis attif\'ee.''}\footnote{I hope you are not playing a trick on me. You wrote to me that this party would be quite small; look at how I am dressed. \textit{Fr.}} %lise

And she spread her arms to show her elegant grey dress, all in lace, tied with a broad ribbon just below her chest.

\textit{``Soyez tranquille, Lise, vous serez toujours la plus jolie,''} Anna Pavlovna answered.\footnote{Rest assured, Lise, you will always be the prettiest. \textit{Fr.}} %annapavlovna

\textit{``Vous savez, mon mari m'abandonne,''} she continued in the same tone, turning to a general. \textit{``Il va se faire tuer. Dites moi, pourqoui cette vilaine guerre,''} she said to Prince Vasily and, not waiting for an answer, turned to Prince Vasily's daughter, the beautiful H\'el\`ene.\footnote{You know, my husband is abandoning me. He is going to get himself killed. Tell me, why do we have this nasty war? \textit{Fr.}} %lise

\textit{``Quelle d\'elicieuse personne, que cette petite princesse!''}~Prince Vasily said quietly to Anna Pavlovna.\footnote{What a delightful person, this little princess! \textit{Fr.}} %vasily

Right behind the little princess, a massive, fat young man entered, his hair cut short, wearing glasses, light-colored trousers in the latest fashion, a high jabot, and a brown tail-coat. This fat young man was the illegitimate son of the famous Elizabethan-era magnate, Count Bezukhov, who was then dying in Moscow. He had still not entered service anywhere and had just arrived from abroad, where he had been educated, and was appearing in society for the first time. Anna Pavlovna greeted him with a bow she reserved for the lowest level of the hierarchy in her drawing room. But despite this low type of greeting, at the sight of Pierre entering the room, Anna Pavlovna's face showed anxiety and fear similar to what one expresses at the sight of something unnatural and much too huge. Although Pierre was indeed somewhat larger than the other men in the room, this fear could only have been connected with the intelligent but timid, searching and natural glance that distinguished him from everyone else in the drawing room.

\textit{``C'est bien aimable \`a vous, \emph{monsieur Pierre,} d'\^etre venu voir une pauvre malade,''} Anna Pavlovna said to him, exchanging fearful glances with her aunt, to whom she was leading him.\footnote{It is very kind of you, \emph{Mr.~Pierre,} to have come to see a poor sick woman. \textit{Fr.}} Pierre muttered something incomprehensible, his eyes continuing to search for something. He smiled happily, joyfully, bowing to the little princess as to a close acquaintance, and approached the aunt. Anna Pavlovna's fear was not in vain, because Pierre, not listening to the aunt's speech about Her Highness's health all the way through, walked away from her. Anna Pavlovna, fearful, spoke to stop him: %annapavlovna

``Do you know Abbot Morio? He is a very interesting person\ldots{}'' she said. %annapavlovna

``Yes, I've heard of his plan for eternal peace, and it is very interesting, but hardly possible\ldots{}'' %pierre

``Do you think so\ldots{}?'' Anna Pavlovna said, merely in order to have something to say, and turned again to her activities as hostess, but Pierre was now impolite in reverse. Before, he had walked away without listening to what his conversation partner had had to say; now, he stopped his partner when she needed to walk away. Head bent and long legs spread wide, he began to lay out for Anna Pavlovna why he considered the abbot's plan a chimera. %annapavlovna

``We can talk later,'' Anna Pavlovna said, smiling. %annapavlovna

And, ridding herself of the young man, who did not know how to conduct himself, she returned to her activities as hostess and continued to listen in and keep an eye on things, ready to help out at just the place where the conversation was flagging. Like the owner of a cotton mill who sets his workers in their places, walks around the establishment, noting a stopped machine or a spindle that is making an unusually loud screeching noise, moving hastily and stopping the machine or putting it on the right track---so Anna Pavlovna, walking around her drawing room, went up to a circle that had fallen silent or was talking too loudly and, with a single word or change of seating, again set the pleasant talking-machine in regular motion. But in the midst of all these cares, one could still see in her a particular fear about Pierre. She glanced at him cautiously when he went up to hear what was being said around Mortemart and then stepped away to the other circle, where the abbot was speaking. For Pierre, educated abroad, this party at Anna Pavlovna's was the first he had seen in Russia. He knew that all the intelligentsia of Petersburg was gathered here, and his head was spinning, like a child's in a toy store. He was afraid of missing anything of the intelligent conversations he could hear. Looking at the assured and refined expressions of the faces gathered here, he was waiting to hear something especially intelligent. Finally, he went up to Morio. The conversation seemed interesting to him, and he stopped, waiting for the chance to express his thoughts, as young people love to do.

\section*{III} % Book One, Part One, Chapter Three

Anna Pavlovna's party had been set in motion. The spindles ran precisely everywhere, never falling silent. Except for \textit{ma tante} --- who sat next to an elderly woman with a gaunt, tear-stained face who was somewhat foreign to the shining company assembled there --- the company had divided itself into three circles. In the first, more masculine one, the center was the abbot; in the next, younger one were the beautiful Princess H\'el\`ene, Prince Vasily's daughter, and the pretty, pink-faced Princess Bolkonskaya, too plump for her young age. In the third were Mortemart and Anna Pavlovna.

The viscount was nice-looking, with soft features and mannerisms, a young man, who obviously considered himself a person of some renown, but who, due to good breeding, humbly permitted himself to be enjoyed by the society in which he found himself. Anna Pavlovna was obviously treating her guests to him. Just as a good ma\^itre d'h\^otel can present as extraordinarily superb a piece of beef that no one would want to eat if it were seen in the dirty kitchen, Anna Pavlovna that evening served up first the viscount, then the abbot, to her guests as something extraordinarily refined. In Mortemart's circle, everyone immediately began talking about the killing of the Duke of Enghien. The viscount said that the Duke of Enghien was done in by his generosity of spirit, and that there had been particular reasons for Bonaparte's animosity.

\textit{``Ah!~voyons. Contez-nous, vicomte,''} Anna Pavlovna said, feeling with happiness that her phrase resounded with something \textit{\`a la Louis XV, ``contez-nous cela, vicomte.''}\footnote{Ah, yes! Tell us the story, viscount[\ldots{}]tell it to us, viscount. \textit{Fr.}} %annapavlovna

The viscount bowed to show his obedience and smiled politely. Anna Pavlovna formed a circle around the viscount and invited everyone to listen to his story.

\textit{``Le vicomte a \'et\'e personnellement connu de monseigneur,''} Anna Pavlovna whispered to one person. \textit{``Le vicomte est un parfait conteur,''} she said to another. \textit{``Comme on voit l'homme de la bonne compagnie,''} she said to a third; and the viscount was presented to the company in the most elegant and favorable light, like roast beef on a hot platter, garnished with greens.\footnote{The viscount was personally acquainted with the duke. The viscount is perfect at telling stories. Now you see what a man of good society looks like. \textit{Fr.}} %annapavlovna

The viscount was ready to begin his story and smiled thinly.

``Come this way, \textit{ch\`ere H\'el\`ene,}'' Anna Pavlovna said to the beautiful princess, who was sitting a little way off, forming the center of a different circle. %anapavlovna

Princess H\'el\`ene smiled; she stood up with the same unchanging smile she had had on when she entered the drawing room, the smile of a perfectly beautiful woman. With a slight rustle of her white ball gown, decorated in ivy and moss, her shoulders gleaming white and her hair and diamonds brilliant, she passed through the crowd of men, which parted, smiling at all of them but looking at no one, as if granting each one the right to admire the beauty of her figure, her round shoulders, her very exposed chest and back (following the fashion of the day), and seeming to carry with her all the luster of a ball, she went over to Anna Pavlovna. H\'el\`ene was so pretty that, not only did she not display even a shadow of coquetry, but she seemed to be ashamed of her unquestionable, overpowering, all-conquering beauty. She seemed to want to diminish the power of her beauty but could not.

\textit{``Quelle belle personne!''} said everyone who saw her.\footnote{What a beautiful woman! \textit{Fr.}} As if struck by something supernatural, the viscount shrugged his shoulders and lowered his eyes while she sat in front of him and turned that same changeless, illuminating smile on him as well.

\textit{``Madame, je crains pour mes moyens devant un pareil auditoire,''} he said, inclining his head with a smile.\footnote{Madame, an audience like this makes me uncertain of my abilities. \textit{Fr.}} %viscount

The princess leaned her bare, plump arm on a little table and did not find it necessary to say anything. She waited, smiling. During the entire story, she sat straight, looking from time to time at her plump, beautiful arm resting lightly on the table, or at her even more beautiful chest, adjusting her diamond necklace; several times, she adjusted the folds of her dress and, when the story made an impression on her, she looked over at Anna Pavlovna and took on the same expression that was on the maid of honor's face, and then returned to her beaming smile as she became calm again. The little princess, Lise, followed H\'el\`ene from the tea table.

\textit{``Attendez moi, je vais prendre mon ouvrage,''} she said. \textit{``Voyons, \`a quoi pensez-vous?''} she said, turning to Prince Ippolit, \textit{``apportez-moi mon ridicule.''}\footnote{Wait for me, I'm going to bring my work. Come now, what are you thinking about? Carry my handbag to me. \textit{Fr.}} %lise

The little princess, smiling and talking with everyone, suddenly repositioned herself and, having settled in her seat, smoothed her dress happily.

``I'm comfortable now,'' she said and, begging the viscount to begin, took up her work. %lise

Prince Ippolit brought her handbag to her, followed after her, and moved a chair close to her, sitting down next to her.

The resemblance between \textit{le charmant Hippolyte} and his beautiful sister was striking, all the more so because, despite that resemblance, he was strikingly ugly. The features of his face were the same as his sister's, but with her, everything was lit with that vivacious, self-satisfied, young, unchanging smile and the supernatural, classical beauty of her body; with her brother, on the other hand, the same face was clouded by imbecility and unchangingly expressed self-assured disgust, and his body was thin and weak. His eyes, nose, and mouth were all compressed into a vague, bored-looking grimace, and he always held his arms and legs in unnatural positions.

\textit{``Ce n'est pas une histoire de revenants?''} he said, having seated himself next to the little princess and hurriedly brought his spectacles up to his eyes, as if he could not begin speaking without the help of that instrument.\footnote{Is this a ghost story? \textit{Fr.}} %ippolit

\textit{``Mais non, mon cher,''} said the surprised story-teller, shrugging his shoulders.\footnote{Why no, my dear. \textit{Fr.}} %viscount

\textit{``C'est que je d\'eteste les histoires de revenants,''} Prince Ippolit said, in a tone that clearly showed he had said the words, and only afterward understood what they meant.\footnote{It's just that I hate ghost stories. \textit{Fr.}} %ippolit

Because of the self-assurance with which he spoke, no one could understand whether what he said was very smart or very stupid. He was in a dark-green tail-coat, in pants the color of \textit{cuisse de nymphe effray\'ee,} as he himself said, and in stockings and low boots.\footnote{A frightened nymph's thigh. \textit{Fr.}}

The \textit{vicomte} recounted very nicely a then-current anecdote about how the Duke of Enghien went to Paris in secret to see Mlle.~George, and how he met Bonaparte, who had also been enjoying the favors of the famous actress, and how, having met the duke there, Napoleon by chance had a fainting spell, which he was prone to do, and found himself in the duke's power, which the duke did not advantage of, but how Bonaparte subsequently repaid the duke's magnanimity by killing him.

The story was very nice and interesting, especially the part where the rivals suddenly recognize one another, and the ladies seemed to be very excited.

\textit{``Charmant,''} Anna Pavlvona said, looking at the little princess questioningly. %annapavlovna

\textit{``Charmant,''} the little princess whispered, sticking her needle in her work, as if to show that the charm and interest of the story prevented her from continuing her work. %lise

The viscount appreciated this unspoken praise and, smiling gratefully, started to continue; but at the same time, Anna Pavlovna, who had been keeping an eye on the frightening (to her) young man across the room, noticed that he was saying something to the abbot too loudly and heatedly, and she hurried toward the danger to help. Indeed, Pierre had managed to tie the abbot up in conversation about political equilibrium, and the abbot, apparently intrigued by the young man's simple-hearted fervor, was laying out his beloved idea for him. Both of them were too animated and were speaking and listening very naturally, and this Anna Pavlovna did not like.

``The means are European equilibrium and \textit{droit des gens,}'' the abbot said.\footnote{Law of nations. \textit{Fr.}} ``Only that one mighty government is required, like Russia, famed for its barbarism, that is willing to become, unselfishly, the head of a union with equilibrium in Europe as its goal, and it will save the world!'' %abbot

``Just how will you find that equilibrium?'' Pierre was beginning to say; but at the same time, Anna Pavlovna approached them and, looking severely at Pierre, asked the Italian how he was managing the local climate. The Italian's face suddenly changed and took on an offensively false, sweet-looking expression, which was apparently habitual for him in his conversations with women.

``I am so charmed by the delights of the education and intellect of the company in which I have the pleasure of being received, especially the feminine company, that I have not managed to think about the climate yet,'' he said. %abbot

No longer willing to leave the abbot and Pierre to themselves, Anna Pavlovna united them with the larger circle in order to supervise them more easily.

At this time, a new person appeared in the drawing room. The new person was the young Prince Andrei Bolkonsky, the husband of the little princess. Prince Bolkonsky was a very attractive young man of medium height with clearcut, dry features. Everything in his figure, starting from his tired, bored glance to his quiet, measured step, presented the starkest possible contrast with his young, vivacious wife. He was apparently not only acquainted with everyone in the drawing room, but already fed up with them, so much so that he was completely bored by having to look at them and listen to them. Of all the tiresome faces there, he seemed to be most fed up with seeing the face of his pretty wife. He turned away from her with a grimace that ruined his handsome face. He kissed Anna Pavlovna's hand and, squinting, looked over the entire company.

\textit{``Vous vous enr\^olez pour la guerre, mon prince?''} Anna Pavlovna said.\footnote{Are you signing up for the war, Prince? \textit{Fr.}} %annapavlovna

\textit{``Le g\'en\'eral Koutouzoff,''} Bolkonsky said, emphasizing the final syllable, \textit{zoff,} like a Frenchman, \textit{``a bien voulu de moi pour aide-de-camp\ldots''}\footnote{General Kutuzov has been pleased to make me one of his aides-de-camp\ldots{} \textit{Fr.}} %andrei

\textit{``Et Lise, votre femme?''}\footnote{And Lise, your wife? \textit{Fr.}} %annapavlovna

``She is going to the country.'' %andrei

``Aren't you ashamed of yourself for taking your delightful wife away from us?'' %annapavlovna

\textit{``Andr\'e,''} his wife said, speaking to her husband in the same coquettish tone with which she addressed strangers, ``what a story the viscount was just telling us about Mlle.~George and Bonaparte!'' %lise

Prince Andrei frowned and turned away. Pierre, who had not taken his friendly, joyful gaze off Prince Andrei since his entrance into the drawing room, approached him and took him by the arm. Prince Andrei, not looking around, grimaced, expressing his annoyance at whomever was grabbing him by the arm, but, having seen Pierre's smiling face, he smiled with unexpected kindness and pleasure.

``Well then! Here you are in great society!'' he said to Pierre. %andrei

``I knew that you would be here,'' Pierre answered. ``I'm coming over for dinner,'' he added quietly so as not to interrupt the viscount, who was continuing his story. ``Can I?'' %pierre

``No, absolutely not,'' Prince Andrei said, laughing, letting Pierre know by a squeeze of the hand that he did not even need to ask. He wanted to say something else, but at the same time, Prince Vasily and his daughter got up, and all the men stood to make way for them. %andrei

``Please forgive me, my dear viscount,'' Prince Vasily said to the Frenchman, gently pulling him back down into his chair by his sleeve to keep him from standing. ``This unfortunate festival at the ambassador's forces me to interrupt you and deny myself the pleasure of staying. I am very sad to leave your lovely party,'' he said to Anna Pavlovna. %vasily

His daughter, Princess H\'el\`ene, walked between the chairs, lightly holding the folds of her dress, and her smile shone even brighter on her beautiful face. Pierre watched with ecstatic, almost terrified eyes as the great beauty walked past him.

``Very beautiful,'' Prince Andrei said. %andrei

``Very,'' Pierre said. %pierre

Walking past, Prince Vasily grabbed Pierre by the arm and turned to Anna Pavlovna.

``Give this bear some culture for me,'' he said. ``Here he has been living with me for a month, and this is the first time I've seen him in society. Nothing is so necessary for a young man as the company of intelligent women.'' %vasily

\section*{IV} % Book One, Part One, Chapter Four

Anna Pavlovna smiled and promised to work with Pierre, who, she knew, was related to Prince Vasily through his father. The elderly woman, who earlier had been sitting with \textit{ma tante,} stood up hurriedly and caught Prince Vasily in the front room. All her earlier feigned interest had disappeared from her face. Her kind, tear-stained face expressed only anxiety and fear.

``What can you tell me, Prince, about my Boris?'' she said, catching him in the front room. (She spoke the name Boris with particular emphasis on the \textit{o}.) ``I cannot remain any longer in Petersburg. Tell me, what news can I bring to my poor little boy?'' %annamikhailovna

Despite the fact that Prince Vasily was listening reluctantly and almost rudely to the elderly woman and even showed impatience, she smiled at him affectionately and touchingly and took him by the hand to keep him from leaving.

``What would it cost you to say a word to the sovereign? Then he will be transferred straight to the Guards,'' she asked him. %annamikhailovna

``Trust me, I will do everything that I can, Princess,'' Prince Vasily answered, ``but it is difficult for me to ask the sovereign; I would recommend you appeal to Rumyantsev, through Prince Golytsin: that would be smarter.'' %vasily

The elderly woman carried the name of Princess Drubetskoy, the name of one of the best families of Russia, but she was poor, had long ago left society, and had lost her former connections. She had come that night to obtain a commission in the Guards for her only son. She had invited herself to the party and come to Anna Pavlovna's only to see Prince Vasily, only for that reason had she listened to the viscount's story. She was startled by Prince Vasily's words; her once-beautiful face showed bitterness, but that lasted only for a moment. She smiled again and grasped Prince Vasily more firmly by the hand.

``Listen, Prince,'' she said, ``I have never asked you for anything, will never ask you again, and have never brought up my father's friendship for you. But now, by God, I implore you, do this for my son, and I will consider you our benefactor,'' she added hurriedly. ``No, do not get angry, just promise me. I asked Golytsin, he refused. \textit{Soyez le bon enfant que vous avez \'et\'e,}'' she said, trying to smile when there were tears in her eyes.\footnote{Be the good boy you once were. \textit{Fr.}} %annamikhailovna

``Papa, we're going to be late,'' Princess H\'l\`ene said, turning her pretty head on her classical shoulders, waiting by the door. %helene

But influence in society is capital, which needs to be protected, or it will disappear. Prince Vasily knew this and, once he had understood that if he started asking for favors whenever anyone asked him, he would soon be unable to ask for himself, he rarely used his influence. On the matter of Princess Drubetskoy, however, he felt something like the prick of conscience after her new appeal: he was indebted to her father for his first steps in the civil service. Furthermore, he saw by her mannerisms that she was one of those women, mothers especially, who, once they had gotten something into their heads, would not leave it alone until their desires were realized, and, in the opposite case, were ready to harass and annoy every day, every moment, and even make scenes. Understanding this last part swayed him.

``\textit{Ch\`ere} Anna Mikhailovna,'' he said with the familiarity and boredom that was eternally in his voice, ``it is almost impossible for me to do what you want; but in order to prove to you how much I love you, and how much I honor the memory of your late father, I will do the impossible: your son will be transferred to the Guards, here is my hand. Are you satisfied?'' %vasily

``My dear, you are our benefactor! I did not expect anything else; I knew you were kind.'' %annamikhailovna

He tried to leave.

``Wait, just two words. \textit{Une fois pass\'e aux gardes\ldots{}}''\footnote{Once he has been transferred to the Guards\ldots{} \textit{Fr.}} She faltered. ``You are in well with Mikhail Ilarionovich Kutuzov, recommend Boris to be one of his adjutants. Then I could rest happy, and then\ldots{}'' %annamikhailovna

Prince Vasily smiled.

``That I will not promise. You know how besieged Kutuzov is since he was named commander-in-chief. He said to me himself that all the great ladies of Moscow are scheming to make all their children his adjutants.'' %vasily

``No, promise me, I will not let you go, my dear benefactor.''

``Pap\`a,'' the beauty repeated in the same tone, ``we're going to be late.'' %helene

``Well, \textit{au revoir,} farewell. You see?'' %vasily

``Then tomorrow you will report to the sovereign about this?'' %annamikhailovna

``Certainly, but I make no promises about Kutuzov.'' %vasily

``No, promise me, promise me, \textit{Basile,}'' Anna Mikhailovna said to his back, with the smile of a young coquette, something that must have been natural for her once upon a time, but which did not suit her venerable face.

She had apparently forgotten her years and had set in motion all her old, girlish methods, out of habit. But as soon as he left, her face again took on the same cold, feigned expression she had worn earlier. She turned back to the circle where the viscount was continuing his story, and she again pretended that she was listening, waiting until it was time to leave, since her business was now finished.

``But what do you think of this latest comedy \textit{du sacre de Milan?}'' Anna Pavlovna asked. \textit{``Et la nouvelle com\'edie des peuples de G\^enes et de Lucques, qui viennent pr\'esenter leurs vouex \`a M.~Buonaparte. M.~Buonaparte assis sur un tr\^one, et exau\c cant les voeux des nations! Adorable! Non, mais c'est \`a en devenir folle! On dirait, que le monde entier a perdu la t\^ete.''}\footnote{[This latest comedy\ldots{} of the] coronation at Milan? And the new comedy of the people of Genoa and Lucca, who are going to present their wishes to Mr.~Bonaparte. Mr.~Bonaparte sits on a throne and grants the wishes of the nations! It's adorable! No, it's getting crazy! It seems that the entire world has lost its head. \textit{Fr.}}

Prince Andrei grinned, looking Anna Pavlovna right in the eyes.

\textit{`` `Dieu me la donne, gare \'a qui la touche,' ''} he said (Bonaparte's words when the crown was placed on his head). \textit{``On dit qu'il a \'et\'e tr\`es beau en pronon\c cant ces paroles,''} he added, and repeated the same words in Italian: \textit{``Dio mi la dona, guai a chi la tocca.''}\footnote{``God has given it to me, beware anyone who would touch it.'' They say that he was quite something when he was saying the words. \textit{Fr.}}

\textit{``J'esp\`ere enfin,''} Anna Pavlovna continued, \textit{``que \c ca a \'et\'e la goutte d'eau qui fera d\'eborder le verre. Les souverains ne peuvent plus supporter cet homme, qui menace tout.''}\footnote{I hope that this is the straw that finally breaks the camel's back. The sovereigns can no longer tolerate this man, who threatens everything. \textit{Fr.}}

\textit{``Les souverains? Je ne parle pas de la Russie,''} the viscount said politely but hopelessly. \textit{Les souverains, madame! Qu'ont ils fait pour Louis XVII, pour la reine, pour madame Elisabeth? Rien,''} he continued, becoming animated. \textit{``Et croyez-moi, ils subissent la punition pour leur trahison de la cause des Bourbons. Les souverains? Ils envoient des ambassadeurs complimenter l'usurpateur.''}\footnote{The sovereigns? I am not speaking of Russia. But the sovereigns! What did they ever do for Louis XVII, for the queen, for Elisabeth? Nothing. And believe me, they will be punished for betraying the cause of the Bourbons. The sovereigns? They are sending ambassadors to pay their compliments to the usurper. \textit{Fr.}}

And he changed position, sighing with contempt. Prince Ippolit, who had long been watching the viscount through his spectacles, at these words suddenly turned toward the little princess with his entire body and, asking for her needle, began to show her the Cond\'e arms, drawing on the table with the needle. He explained the coat of arms to her with such an important air that one would think the princess had asked him about it herself.

\textit{``B\^aton de gueules, engr\^el\'e de gueules d'azur: maison Cond\'e,''} he said.\footnote{Azur, a baton engrailed gules: House Cond\'e. \textit{Fr.} Note: While the arms of the Princes of Cond\'e did feature a red (gules) baton or bendlet on a blue (azure) field, I can find no evidence of arms for any member of that house where the baton is scalloped (engrailed). The arms of the House of Bourbon, of which Cond\'e was a cadet branch, also uniformly feature fleurs-de-lys, typically three of them, which are not mentioned in Ippolit's description. --- Trans.}

The princess listened, smiling.

``If Bonaparte remains on the throne of France another year,'' the viscount continued the conversation he had begun, with the air of someone who was not listening to others but merely following his own train of thought on a matter he knew better than everyone else, ``then matters will have gone much too far. By intrigue, violence, exile, executions, society, and here I mean high society, French society, will have been forever destroyed, and then\ldots'' %viscount

He shrugged his shoulders and spread his hands. Pierre wanted to say something --- the conversation interested him --- but Anna Pavlovna, who was keeping watch over him, interrupted.

``Emperor Aleksandr,'' she said with a sadness that accompanied all her speeches about the imperial family, ``has declared that he will allow the French themselves to choose the manner by which they will be governed. And I think there can be no doubt that the entire nation, once freed from the usurper, will throw itself into the arms of the lawful king,'' Anna Pavlovna said, trying to oblige the emigrant and royalist. %annapavlovna

``That is doubtful,'' Prince Andrei said. ``\textit{Monsieur le vicomte} is absolutely right in saying that matters have already gone much too far. I think that it will be difficult to return to the old ways.'' %andrei

``As far as I have heard,'' Pierre stepped in, blushing, ``almost the entire nobilty has already gone over to the side of Bonaparte.'' %pierre

``That is just what the Bonapartists say,'' the viscount said, not looking at Pierre. ``Right now it is difficult to discover what public opinion is in France.'' %viscount

\textit{``Bonaparte l'a dit,''} Prince Andrei said with a grin.\footnote{Bonaparte said that. \textit{Fr.}} %andrei

(It was obvious that he did not like the viscount and that, although he was not looking at him, his speeches were aimed at him.)

\textit{`` `Je leur ai montr\'e le chemin de la gloire,' ''} he said after a short pause, again repeating Napoleon's words, \textit{`` `ils n'en ont pas voulu; je leur ai ouvert mes antichambres, ils se sont precipit\'es en foule\ldots{}' Je ne sais pas \`a quel point il a eu le droit de le dire.''}\footnote{``I have showed them the way to glory, and they did not want to take it; I opened my anterooms to them, and they rushed in in droves\ldots{}'' I don't know to what extent he had the right to say that. \textit{Fr.}} %andrei
	
\textit{``Aucun,''} the viscount retorted. ``After the murder of the duke, even those who were most partial to him stopped seeing him as a hero. \textit{Si m\^eme \c ca \'et\'e un h\'eros pour certaines gens,}'' the viscount said, turning to Anna Pavlovna, \textit{``depuis l'assassinat du duc il y a un martyr de plus dans le ciel, un h\'eros moins sur la terre.''} \footnote{None. If he had been a hero for certain people, after the assassination of the duke, there was one more martyr in heaven and one fewer hero on earth. \textit{Fr.}} %viscount

Before Anna Pavlovna and the others could even smile in response to the viscount's words, Pierre again burst into the conversation, and Anna Pavlovna could not stop him, even though she had a premonition that he would say something improper.

``The execution of the Duke of Enghien,'' Pierre said, ``was necessary for the state; and I see greatest of spirit specifically in the fact that Napoleon was not afraid to take the responsibility for that act himself.'' %pierre

\textit{``Dieu! mon Dieu!''} Anna Pavlovna said in a terrible whisper.

\textit{``Comment, M.~Pierre, vous trouvez que l'assassinat est grandeur d'ame,''} the little princess said, smiling and drawing her work closer to her.\footnote{Mr.~Pierre, how do you find assassination to be greatness of spirit. \textit{Fr.}}

``Ah! Oh!'' came various voices.

``Capital!'' Prince Ippolit said in English and began to slap his knee with his palm. The viscount merely shrugged his shoulders.

Pierre looked at his listeners solemnly over the top of his glasses.

``I say this only because,'' he continued in desperation, ``the Bourbons ran from the revolution, leaving the people to anarchy; and Napoleon alone was able to understand the revolution and overcome it, and so for the greater good he could not stop on account of one human life.'' %pierre

``Wouldn't you like to come over to this table?'' Anna Pavlovna said. But Pierre continued his speech, not answering her. %annapavlovna

``No,'' he said, becoming more and more animated, ``Napoleon is great because he has set himself above the revolution, put a stop to its worst abuses, retaining all the best in it --- equality of all citizens, freedom of speech and the press --- and took power only for that purpose.'' %pierre

``If upon seizing power he had given it over to the lawful king and not used it to commit murder,'' the viscount said, ``then I would have called him a great man.'' %viscount

``He couldn't have done that. The people gave him power only so that he would save them from the Bourbons, and because the people saw in him a great man. The revolution was a great undertaking,'' Monsieur Pierre continued, showing with this desperate and provocative opening that he was incredibly young and wanted to get all of his thoughts out as quickly as possible. %pierre

``Revolution and regicide --- a great undertaking? After something like that\ldots{} Now, now, wouldn't you like to come over to this table?'' Anna Pavlovna repeated. %annapavlovna

\textit{``Contrat social,''} the viscount said with a gentle smile.\footnote{The social contract. \textit{Fr.}} %viscount

``I am not talking about regicide. I am talking about ideas.'' %pierre

``Yes, the ideas of pillage, murder, and regicide,'' an ironic voice again interrupted. %unknown

``There were excesses, of course, but that's not what is important, what's important is the rights of man, emancipation from prejudice, the equality of all citizens; and Napoleon has kept these ideas with all their power.'' %pierre

``Freedom and equality,'' the viscount said with contempt, as if he had decided finally to prove to this youth the complete stupidity of his words, ``are very loud words, which have already been compromised. Who does not love freedom and equality? Our Savior Himself preached freedom and equality. Have people really become happier since the revolution? On the contrary. We wanted freedom, and Bonaparte destroyed it.'' %viscount

Prince Andrei was looking from Pierre to the viscount to their hostess with a smile. Anna Pavlovna was horrified by the first minute of Pierre's gambit, despite her long experience in society; but when she saw that, despite the sacrilegious nature of Pierre's words, the viscount was not losing his temper, and when she became convinced that it was impossible by now to suppress his words, she gathered her wits and, joining the count, attacked the speaker.

\textit{``Mais, mon cher m-r Pierre,''} Anna Pavlovna said, ``how can you explain the actions of a great person who is capable of executing a duke, or in fact any person, without accusation or trial?'' %annapavlovna

``I would like to ask,'' the viscount said, ``how \textit{monsieur} explains the \nth{18} of Brumaire? Was that not a deception? \textit{C'est un escamotage, qui ne ressemble nullement \`a la mani\`ere d'agir d'un grand homme.}''\footnote{It was a slight of hand, which looks nothing like how a great man acts. \textit{Fr.}} %viscount

``And the prisoners in Africa that he killed?'' the little princess said. ``It's terrible!'' And she shrugged her shoulders. %lise

\textit{``C'est un roturier, vous aurez beau dire,''} Prince Ippolit said.\footnote{He is a commoner, you could well say. \textit{Fr.}}

Monsieur Pierre looked around at everyone and smiled, not knowing whom to answer. His smile was not like other people's, emerging smoothly from their non-smiling face. With him, on the contrary, when he smiled, it happened suddenly, his serious and even somewhat sullen face disappearing in an instant and another appearing --- childish, kind, even a bit stupid and seemingly begging everyone's pardon.

It became clear to the viscount, who was seeing him for the first time, that this Jacobin was not as fearsome as his words suggested. Everyone fell silent.

``What, do you want him to answer everyone at once?'' Prince Andrei said. ``Besides, in the acts of a person of state one must distinguish among the acts of the private person, the commander, and the emperor. So it seems to me.'' %andrei

``Yes, yes, of course,'' Pierre joined in, overjoyed at the show of assistance. %pierre

``One cannot help but agree,'' Prince Andrei continued, ``Napoleon as a man was great at the bridge of Arcole, at the hospital at Jaffa giving his hand to the plague-ridden men there, but\ldots{} But there are other acts that are difficult to justify.'' %andrei

Prince Andrei had clearly hoped to smooth over the awkwardness of Pierre's words. He stood up and gave a sign to his wife, preparing to leave.

\begin{center}
\rule{5em}{0.4pt}
\end{center}

Prince Ippolit stood up suddenly and, waving his hands to stop everyone and asking them to sit down, said:

\textit{``Ah!~aujourd'hui on m'a racont\'e une anecdote moscovite, charmante: il faut que je vous en r\'egale. Vous m'excusez, vicomte, il faut que je raconte en russe. Autrement on ne sentira pas le sel de l'histoire.''}\footnote{Ah, today I heard a charming story about Moscow; I must amuse you with it. Excuse me, viscount, I must tell it in Russian. Otherwise you would not be able to experience the zest of the story. \textit{Fr.}} %ippolit

And Prince Ippolit began to speak in the kind of Russian spoken by Frenchmen who have lived in Russia for a year or so. Everyone stopped and listened: that was how animated and insistent Prince Ippolit was in demanding their attention to his story.

``In \textit{Moscou} there was a lady, \textit{une dame}. And she is very much a miser. She needed to have gotten two \textit{valets de pied} for carriage hers. And be very tall in the height. This is what she like. And she is having \textit{une femme de chambre}, also tall in the height. She said\ldots{}''\footnote{Moscow. A lady. Footmen. Chambermaid.} %ippolit

Here Prince Ippolit stopped to think, apparently struggling with his train of thought.

``She said\ldots{} Yes, she said: `Girl (\textit{\`a la femme de chambre}), put on a \textit{livr\'ee} and you drive with me, on the carriage, \textit{faire des visites}.' ''\footnote{Livery. Make visits. \textit{Fr.}} %ippolit

Here Prince Ippolit snorted and laughed far sooner than his listeners did, which left an impression unfavorable to the narrator. Many did smile, however, Anna Pavlovna and the elderly woman among them.

``She went. All the sudden, there became a strong wind. The girl lost her hat, and her long hairs brushed out\ldots{}'' %ippolit

Here he could no longer control himself and broke out into fits of laugher, and through his laughter he said:

``And all the world found out\ldots{}'' %ippolit

With that, the story was over. Although it was impossible to understand why he had told it, and why he felt that he absolutely had to tell it in Russian, Anna Pavlovna and the others appreciated Prince Ippolit's sense of tact, which had so agreeably brought Pierre's disagreeable and discourteous gambit to an end. The conversation after the story broke up into meaningless small talk about balls past and future, the theatre, and about whom one could see when and where.

\section*{V} % Book One, Part One, Chapter V

Having thanked Anna Pavlovna for her \textit{charmante soir\'ee,} the guests began to disperse.\footnote{Charming party. \textit{Fr.}}

Pierre was awkward. He was fat, taller than average, broad, with huge red hands, and, as they say, he did not know how to make an entrance. Even less did he know how to make an exit, that is, to say something especially pleasant before leaving. Furthermore, he was absent-minded. Getting up, he grabbed a tricorne hat with a general's plumage instead of his own hat and held it, pulling on the feathers, until the general asked him to return it. But all his absent-mindedness and inability to enter a salon and speak well in one were redeemed by the way he conveyed kindheartedness, simplicity, and humility. Anna Pavlovna returned to him and, with Christian gentleness conveying forgiveness for his gambit, nodded to him and said:

``I hope to see you again, but I also hope that you will change your opinions, Monsieur Pierre,'' she said. %annapavlovna

When she said this to him, he did not answer, only bowed and again showed everyone his smile, which meant nothing, except perhaps this: ``Options are what they are, but you see what a kind and excellent little boy I am.'' And everyone, including Anna Pavlovna, felt that whether they liked it or not.

Prince Andrei went into the front room and, presenting his shoulders to the footman who was throwing his cloak on him, listened indifferently to his wife chattering with Prince Ippolit, who had also gone into the front room. Prince Ippolit stood near the charming, pregnant princess and looked at her insistently through his spectacles.

``Go, Annette, you'll catch cold,'' the little princess said, saying farewell to Anna Pavlovna. \textit{``C'est arr\^et\'e,''} she added quietly.\footnote{It's settled. \textit{Fr.}} %lise

Anna Pavlovna had already managed to discuss with Liza the match she had taken on between Anatoly and the little princess's sister-in-law.

``I am counting on you, my dear friend,'' Anna Pavlovna said, also quietly, ``you write to her and then tell me, \textit{comment le p\`ere envisagera la chose. Au revoir,}'' and she left the front room.\footnote{How the father views the matter. \textit{Fr.}} %annapavlovna

Prince Ippolit approached the little princess and, bending his face close to her, began to say something to her in a half-whisper.

Two footmen, one the princess's, the other his, were waiting for them to finish speaking and stood with a shawl and riding-coat, listening to their French speech, which was impossible for them to understand, with expressions that suggested that they understood what was being said but did not want to show it. The princess, as always, spoke with a smile on her face and listening with laughter.

``I am so happy I didn't go to the ambassador's,'' Prince Ippolit said. ``Boring\ldots{} It was a splendid party, don't you think, splendid?'' %ippolit

``They say that the ball will be very good,'' the princess answered, curling up her hairy lip. ``All the beautiful women in society will be there.'' %lise

``Not all of them, because you won't be there; not all of them,'' Prince Ippolit said, laughing happily, and, grabbing the shawl from the footman, pushed him out of the way and started to put it on the princess. Either intentionally or out of awkwardness (no one could have sorted out which), he left his hands on her long after the shawl was on, and seemed to be embracing the young woman. 

She stepped aside graciously, but smiling the whole time, turned, and looked at her husband. Prince Andrei's eyes were closed: he seemed to be tired and near sleep.

``Are you ready?'' he asked his wife, casting a glance over her. %andrei

Prince Ippolit hurriedly put on his riding-coat, which, according to the new fashion fell to below his heels, and, getting caught in it, ran onto the front steps after the princess, whom the footman was seating in her coach.

\textit{``Princesse, au revoir,''} he shouted, his tongue caught just as his legs had been.

The princess, lifting her dress, sat in the dark of the coach; her husband adjusted his sword; Prince Ippolit, under the pretext of helping them, got in everyone's way.

``Pa-a-ardon me, good sir,'' Prince Andrei addressed Prince Ippolit dryly and unpleasantly in Russian as Ippolit preventing him from passing by. %andrei

``I'll be waiting for you, Pierre,'' the same voice, Prince Andrei's, said tenderly and fondly. %andrei

The post-boy set off, and the coach's wheels began to clatter. Prince Ippolit broke out in a fit of laughter, standing on the front steps and waiting for the viscount, whom he had promised to take home.

\begin{center}
	\rule{5em}{0.4pt}
\end{center}

\textit{``Eh bien, mon cher, votre petite princesse est tr\`es bien, tr\`es bien,''} the viscount said, having seated himself in the coach with Ippolit. \textit{``Mais tr\`es bien.''} He kissed the tips of his fingers. \textit{``Et tout-\`a-fait fran\c caise.''}\footnote{Ah yes, my dear, your little princess is very good, very good. Just very good. And she is a perfect Frenchwoman. \textit{Fr.}} %viscount

Ippolit gave a snort, laughing.

\textit{``Et savez-vous que vous \^etes terrible avec votre petit air innocent,''} the viscount continued. \textit{``Je plains le pauvre mari, ce petit officier, qui se donne des airs de prince r\'egnant.''}\footnote{And you know, you are terrible with your little innocent airs. I pity the poor husband, that little officer, who puts on airs of being a prince regnant. \textit{Fr.}} %viscount

Ippolit snorted again and said through laughter:

\textit{``Et vous disiez, que les dames russes ne valaient pas les dames fran\c caises. Il faut savoir s'y prendre.''}\footnote{And you said that Russian women can't hold their own against French women. You have to know how to take them. \textit{Fr.}}

Pierre had ridden ahead, like one of the servants. He entered Prince Andrei's study and immediately lay down on the sofa, out of habit, picked up the first book he found on the shelves (it was Caesar's \textit{Commentaries}) and began to read from the middle, propped up on his elbows.

``What have you done to Mlle.~Scherer? She is quite sick now,'' Prince Andrei said, entering the study and rubbing his small, white hands. %andrei

Pierre turned his entire body, which made the sofa creak, turned his animated face toward Prince Andrei, smiled, and waved his hand.

``No, that abbot is very interesting, although he just doesn't understand the matter\ldots{} In my opinion, eternal peace is possible, but I don't know how to say this\ldots{} Just not through political equilibrium\ldots{}'' %pierre

Prince Andrei was apparently not interested in abstract conversations.

``You must not, \textit{mon cher,} go around saying everything you think, no matter where you are. Well, anyhow, have you finally decided on something? Are you going to be a Chevalier Guard or a diplomat?'' Prince Andrei asked after a moment's silence. %andrei

Pierre sat down on the sofa, pressing his legs under him.

``If you can imagine, I still don't know. I don't like either one of them.'' %pierre

``But you have to decide, don't you? Your father is waiting.'' %andrei

Pierre had been sent abroad with a priest-tutor at the age of ten, where he remained until the age of twenty. When he returned to Moscow, his father let the priest go and told the young man: ``Now go on to Petersburg, look around, and choose something. I will agree to anything. Here is a letter to Prince Vasily, and here is some money. Write about everything you see, I am your support in everything.'' Pierre had already been trying to choose a career for three months and had accomplished nothing. Prince Andrei was talking to him about that choice. Pierre wiped his brow.

``But he must be a mason,'' he said, meaning the abbot he had seen at the party. %pierre

``You're just blathering,'' Prince Andrei stopped him again, ``let's talk about the matter at hand instead. Have you been to see the Horse Guards?'' %andrei

``No, I haven't, but here is what I've been thinking about and wanted to tell you. Right now there is a war against Napoleon. If it were a war for our own freedom, then I would understand, I would be the first to sign up for military service; but to help England and Austria against the greatest man on earth\ldots{} It's not good\ldots{}'' %pierre

Prince Andrei just shrugged his shoulders at Pierre's childish speeches. He made a show of not answering such stupidity; but in fact, it was difficult to answer such a naive question in any other way than how Prince Andrei answered him.

``If everyone went to war only for their own principles, then there would be no war,'' he said. %andrei

``And that would be splendid,'' Pierre said. %pierre

Prince Andrei laughed.

``It may well be that it would be splendid, but that will never happen\ldots{}'' %andrei

``Well, why are you going to war?'' Pierre asked. %pierre

``Why? I don't know. I have to. Furthermore, I'm going\ldots{}'' He stopped. ``I'm going because this life that I'm leading here, this life --- it's not right for me!'' %andrei

\section*{VI} % Book One, Part One, Chapter VI

In the next room, a woman's dress rustled. Prince Andrei shook himself, as if waking from a dream, and his face took on the same expression that he had worn in Anna Pavlovna's drawing room. Pierre lowered his feet from the sofa. The princess entered. She was already in a different dress, one for wearing around the house, though it was just as elegant and fresh. Prince Andrei stood up, politely moving the chair toward her.

``Why, I wonder,'' she said, speaking French as always, hurriedly and restively seating herself in the chair, ``why did Annette never get married? You are all so stupid, \textit{messieurs,} not to marry her. Forgive me, but none of you understand anything at all about women. What a troublemaker you are, Monsieur Pierre!'' %lise

``I have been fighting with your husband here; I don't understand why he wants to go to war,'' Pierre said, addressing the princess without any constraints (which are so usual in the relations between a young man and a young woman). %pierre

The princess shuddered. Apparently, Pierre had cut her to the quick.

``Oh, that is just what I've been saying!'' she said. ``I do not understand, I absolutely do not understand, why can men not live without war? Why is it that we women don't want anything, don't need anything? You be the judge of that. I've been saying to him: here he is uncle's adjutant, a brilliant situation. Everyone knows him, values him. The other day, I heard a woman at the Apraksins' ask: \textit{`c'est \c ca le fameux prince Andr\'e?' Ma parole d'honneur!}''\footnote{``Is that the famous Prince Andrei?'' My word of honor! \textit{Fr.}} She laughed. ``He's received that way everywhere. He could easily be an aide-de-camp to the emperor. You know, the sovereign spoke very graciously with him. Annette and I were talking about how it would be easy to set him up. What do you think?'' %lise

Pierre looked at Prince Andrei and, noticing that his friend did not like this conversation, did not answer.

``When are you leaving?'' he asked. %pierre

\textit{``Ah!~ne me parlez pas de ce d\'epart, ne m'en parlez pas. Je ne veux pas en entendre parler,''} the princess said in the playful, capricious tone she had used with Ippolit in the drawing room, and which obviously was not fitting for the family circle of which Pierre was, in a way, a member. ``Today, when I was thinking about how we have to cut off all these close ties\ldots{} And then, do you know, Andr\'e?'' She blinked at her husband significantly. \textit{``J'ai peur, j'ai peur!''} she whispered, her back shivering. \footnote{Oh, don't talk to me about his departure, don't talk to me about it! I don't want to hear it spoken of. I'm afraid, I'm afraid! \textit{Fr.}} %lise

Her husband watched all of this as if he were surprised at having noticed that someone else, besides Pierre and him, was in the room: however, he addressed his wife with cold civility, questioning:

``What are you afraid of, Liza? I can't understand it,'' he said. %andrei

``All men are such egoists; all of you, all of you egoists! Because of a whim, God knows why, he is dropping me, locking me up in the country alone.'' %lise

``With my father and sister, don't forget,'' Prince Andrei said quietly.

``I'll be alone all the same, without \emph{my} friends\ldots{} And he wants me not to be afraid.'' %lise

Her tone was contentious, her lip went up, giving her face a squirrel-like expression, beastly rather than happy. She was silent, as if she found it indecent to talk about her pregnancy in front of Pierre, when in fact that was the heart of the matter.

``In any case, I haven't understood \textit{de quoi vous avez peur,}'' Prince Andrei said languidly, keeping his eyes on his wife.\footnote{What you are afraid of. \textit{Fr.}} %andrei

The princess turned red and waved her hands in despair.

\textit{``Non, Andr\'e, je dis que vous avez tellement, tellement chang\'e\ldots{}''}\footnote{No, Andrei, I can say that you've changed so much, so much\ldots{} \textit{Fr.}} %lise

``Your doctor ordered you to lie down earlier,'' Prince Andrei said. ``You should go sleep.'' %andrei

The princess said nothing, and suddenly her hairy lip trembled; Prince Andrei stood up, shrugged his shoulders, and paced the room.

Pierre looked naively at him, then at the princess, surprised, and stirred as if to get up, but then changed his mind.

``What does it matter to me that Monsieur Pierre is here,'' the little princess said suddenly, and her pretty face suddenly dissolved into a teary grimace. ``I've wanted to tell you for a long time, Andr\'e: why have you changed toward me? What have I done to you? You are leaving for the army, you don't pity me at all. Why?'' %lise

``Lise!'' was all Prince Andrei said; but in that word were a plea, and a threat, and, most importantly, an assurance that she would come to regret her words; but she continued hurriedly: %andrei

``You're talking to me as if I were a sick person or a child. I see everything. Would you really have acted this way six months ago?'' %lise

``Lise, I am begging you to stop,'' Prince Andrei said, even more significantly than before. %lise

Pierre, who had become more and more agitated during this conversation, stood up and approached the princess. It seemed that he could not bear the sight of her tears and was himself ready to cry.

``Be calm, princess. It might seem that way to you, because I will tell you, I myself felt\ldots{}why he\ldots{}because\ldots{} No, forgive me, no place for an outsider here\ldots{} No, be calm\ldots{} Farewell\ldots{}'' %pierre

Prince Andrei took him by the arm to stop him.

``No, stop, Pierre. The princess is very kind and would not want to deny me the pleasure of spending the evening with you.'' %andrei

``No, he only ever thinks of himself,'' the princess said, unable to hold back angry tears. %lise

``Lise,'' Prince Anrei said dryly, raising his voice a tone to show that his patience was exhausted. %andrei

Suddenly, the angry, squirrel-like expression on the princess's beautiful face was replaced by an expression of fear, which evoked pity and attracted attention; her brow furrowed, she glanced at her husband with stunning eyes, and a shy and confiding expression appeared on her face, like that of a dog quickly but weakly wagging its lowered tail.

\textit{``Mon Dieu, mon Dieu!''} the princess said and, taking up the folds of her dress in one hand, she approached her husband and kissed his forehead.\footnote{My God, my God! \textit{Fr.}}

\textit{``Bonsoir, Lise,''} Prince Andrei said, getting up and kissing her hand politely, as if she were a stranger. %andrei

\begin{center}
	\rule{5em}{0.4pt}
\end{center}

The two friends were silent. Neither one of them wanted to begin speaking. Pierre glanced at Prince Andrei, Prince Andrei was rubbing his forehead with his little hand.

``Let's go have supper,'' he said with a sigh, standing up and heading toward the door.

They went into the dining room, which had recently been refinished in a fine, rich style. Everything, from the napkins to the silver, the glazed pottery and the crystal, carried the particular stamp of newness that is found in the households of young couples. In the middle of supper, Andrei rested his elbows on the table and began to speak with an expression of nervous anxiety, like someone who has long had something on his heart and has suddenly decided to express himself.

``Never, never get married, my friend; that's my advice to you, don't get married until you can say to yourself that you've done everything you could not to, until you've stopped loving the woman you've chosen, until you've seen her clearly; otherwise you'll be making a bitter, irreparable mistake. Get married when you're an old man, when you're not good for anything else\ldots{} Otherwise, you'll lose everything that's good and lofty about you. Everything gets wasted on trifles. Yes, yes, yes! Don't look at me so surprised. If you expect anything from yourself in the future, then you'll feel like everything is finished for you, everything is closed off, except the drawing room, where you'll be standing on the same shelf as an imperial footman and an idiot\ldots{} There!'' %andrei

He waved his hand energetically.

Pierre took off his glasses, which changed the look of his face, highlighting its kindness, and looked at his friend with surprise.

``My wife,'' Prince Andrei continued, is a splendid woman. She is one of those rare women with whom a man can rest easy when it comes to his honor; but, my God, what I wouldn't give now to be a bachelor! I'm saying this only to you, to you first, because I love you.'' %andrei

In saying this, Prince Andrei looked even less than before like the Bolkonsky who sat collapsed in a chair at Anna Pavlovna's and spoke French phrases through his teeth, squinting. Every muscle in his dry face shook with nervous animation; his eyes, in which the fire of life seemed to have gone out, now shone with a bright, radiant sheen. Obviously, the more lifeless he seemed during ordinary times, the more energetic he was in moments of vexation.

``You don't understand why I'm saying this,'' he continued. ``It's the story of life from beginning to end. You are talking about Bonaparte and his career,'' he said, although Pierre was not talking about Bonaparte. ``You talk about Bonaparte; but Bonaparte, when he was at work, he moved step by step toward his goal, he was free, he had nothing except his goal --- and he achieved it. But tie him up to a woman, and, like a chained convict, he loses every bit of his freedom. And everything you have of power, of hope, everything just feels burdensome and full of regret. Drawing rooms, gossip, balls, vanity, nothingness --- that is the enchanted circle from which you cannot escape. Now I am going to war, the greatest war that has ever been, and I know nothing and am good for nothing. \textit{Je suis tr\`es aimable et tr\`es caustique,}'' Prince Andrei continued, ``and everyone listens to me at Anna Pavlovna's. And that stupid society, without which my wife cannot live, and all those women\ldots{} If only you could know what \textit{toutes les femmes distingu\'ees} and women in general are!\footnote{I am very nice and very caustic. All these distinguished women\ldots{} \textit{Fr.}} My father is right. Egoism, vanity, dullness, nothingness in everything --- that is what women are when they show their true selves. If you look at them in society, it looks like there's something there, but it's nothing, nothing, nothing! Yes, don't get married, my dear, don't get married,'' Prince Andrei finished. %andrei

``I find it funny,'' Pierre said, that you consider \emph{yourself, yourself} to be unfit and your life to be corrupted. Everything, everything is waiting for you. And you\ldots{}'' %pierre

He did not say what came after that \emph{you,} his tone already demonstrated how highly he valued his friend and how much he expected from him in the future.

\textit{How can he say that!} thought Pierre. Pierre considered Prince Andrei the model of perfection precisely because Prince Andrei brought together, at a high degree, all the qualities that Pierre did not have and which could be most closely expressed by the concept of the power of will. Pierre was always surprised by Prince Andrei's capacity for calmly conversing with all types of people, his uncommonly good memory, how well-read he was (he read everything, knew everything, understood everything) and more than anything else his capacity to work and study. If Pierre frequently noticed in Andrei an absence of capacity for dreamy philosophizing (to which Pierre was especially inclined), he did not see any shortcoming in that, but rather strength.

In the best, simplest, and most friendly relationships, praise and flattery are as necessary as oil is necessary to make wheels turn.

\textit{``Je suis un homme fini,''} Prince Andrei said.\footnote{I am finished. \textit{Fr.}} ``Why are we talking about me? Let's talk about you,'' he said, falling silent and smiling at his own reassuring thoughts. %andrei

The same smile was reflected on Pierre's face at that same instant.

``But what is there to say about me?'' Pierre said, his mouth slackening into a carefree, happy smile. ``What am I? \textit{Je suis un b\^atard!}'' And here he blushed bright red. It was obvious that he had made a serious effort to say this. ``\textit{Sans nom, sans fortune\ldots{}} And it's fair, of course\ldots{}''\footnote{I am a bastard! Without name, without fortune\ldots{} \textit{Fr.}} But he did not say \emph{what} was fair. ``For now, I am free, and I like it. It's just that I don't know what I should start doing at all. I've badly wanted to ask your advice.'' %pierre

Prince Andrei looked at him with kind eyes. But in his glance, friendly, tender, he nevertheless expressed a consciousness of his superiority.

``You are very dear to me, especially because you are the one living person in our entire circle. You are happy. Choose whatever you want; it doesn't matter. You'll be happy everywhere, but just one thing: stop going to see the Kuragins and lead that other life. It doesn't fit you at all: all those binges, Hussar carousing, and all that\ldots{}'' %andrei

\textit{``Que voulez-vous, mon cher,''} Pierre said, shrugging his shoulders, \textit{``les femmes, mon cher, les femmes!''}\footnote{What can I do, my dear? The women, my dear, the women! \textit{Fr.}} %pierre

``I don't understand that,'' Andrei answered. ``\textit{Les femmes comme il faut,} that's a different matter; but \textit{les femmes} of Kuragin, \textit{les femmes et le vin,} I don't understand it!''\footnote{Women -- that's as it should be. But Kuragin's women, the women and the wine\ldots{} \textit{Fr.}} %andrei

Pierre was living with Prince Vasily Kuragin and took part in the dissolute life of his son, Anatole, the same one who, to reform him, was being prepared for marriage to Prince Andrei's sister.

``You know?'' Pierre said, as if he had just had an unexpected, happy thought. ``I'm serious, I have thought the same thing for a long time. Living that life, I can't decide anything or think about anything. My head hurts, I have no money. He invited me this evening, and I won't go.'' %pierre

``You give me your word of honor that you won't go?'' %andrei

``My word of honor!'' %pierre

\begin{center}
	\rule{5em}{0.4pt}
\end{center}

It was already after one o'clock in the morning when Pierre left his friend. The night was a cloudless Petersburg night in June. Pierre got into a hired barouche, intending to go home. But the closer he got, the more he felt that it was impossible to fall asleep that night, which felt more like evening or morning. He could see far down the empty streets. Dear Pierre remembered that the usual group was supposed to be gathering that night at Anatole Kuragin's to gamble, after which they typically went on a bender, which ended in one of Pierre's favorite amusements.

\textit{It would be nice to go see Kuragin,} he thought. But he at once remembered that he had given Andrei his word of honor that he would not visit Kuragin's.

But he at once wanted so passionately to experience that familiar, dissolute life once more, as often happens with people one would call lacking in character, that he decided to go. And the idea came to him at once that the word he had given meant nothing, because even before Prince Andrei, he had also given his word to Prince Anatole that he would visit him; in the end, he decided that all of these words of honor were relative, lacking any definitive sense, especially if one considered that tomorrow he could die, or something else unusual could happen, which would be neither honorable nor dishonorable. Lines of thought like this, which destroyed all his decisiveness and intentionality, came frequently to Pierre. He went to Kuragin's.

Approaching the front steps of a large house near the Horse Guards barracks, in which Anatole lived, he climbed the lighted stairs, ascending the steps, and went in through the open door. There was no one in the front room; empty bottles, cloaks, and galoshes were scattered all over; it smelled of wine, and distant shouts and conversations could be heard.

Cards and supper were already over, but the guests had not yet departed. Pierre threw off his cloak and went into the first room, where he found the remnants of supper and one footman, who, thinking no one could see him, was surreptitiously drinking the dregs from the glasses. From the third room, he could hear a racket, laughing, shouts of familiar voices, and the roaring of a bear. About eight young men were crowded around the open window, looking out anxiously. Three of them were playing with a young bear, which one of them was pulling by its chain, trying to scare another.

``I've got a hundred on Stevens!'' one shouted.

``Don't you dare help him!'' another shouted.

``I have Dolokhov!'' a third shouted. ``Kuragin, let go.''

``Hey, let Mishka go, there's a bet on.''

``All in one drink, or he loses,'' a fourth shouted.

``Yakov! Give me the bottle, Yakov!'' shouted the host himself, a tall, handsome man who was standing in the middle of the crowd in just a thin shirt that was open to the center of his chest. ``Stand back, gentlemen. Ah, here's Petrusha, my dear friend,'' he said, turning to Pierre. %kuragin

Another voice, belonging to a short man with clear, blue eyes, was especially striking for its soberness in the midst of all those drunken voices, shouted from the window, ``Come here! Open up the betting!'' It was Dolokhov, an officer of the Semyonovsky Life Guards, a notorious gambler and brawler who lived with Anatole. Pierre, smiling, looked around happily. %dolokhov

``I don't understand anything. What is going on?'' he asked. %pierre

``Wait, he isn't drunk. Give me a bottle,'' Anatole said and approached Pierre, taking a glass from the table. %kuragin

``First of all, drink.'' %kuragin

Pierre began to drink glass after glass, glancing furtively at the drunk guests, who were again massed around the window, and listening to their conversation. Anatole poured him wine and told him that Dolokhov had a bet with Stevens, a sailor who was there, that he, Dolokhov, could drink a bottle of rum while sitting on a window sill on the third floor with his legs hanging out.

``Come on, drink it all!'' Anatole said, giving the last glass to Pierre. ``Or else I won't let you watch!'' %kuragin

``No, I don't want to,'' Pierre said and approached the window, pushing Anatole away. %pierre

Dolokhov held the Englishman by the arm and articulated the conditions of the bet clearly, distinctly, primarily addressing Anatole and Pierre.

Dolokhov was a man of medium height, with curly hair and bright, light-blue eyes. He was about twenty-five years old. Like all infantry officers, he did not have a mustache, and his mouth, the most striking feature of his face, was completely visible. The lines of that mouth were remarkably finely wrought. In the center, the upper lip came down strongly in a sharp wedge over the lower, and at that corners, something like two permanent smiles formed, one on each side; and all of it together, especially taken together with his hard, impertinent, intelligent gaze, gave such a strong impression that it was impossible not to notice that face. Dolokhov was not a rich man and had no connections. And despite the fact that Anatole was spending tens of thousands, Dolokhov lived with him and managed to arrange things so that Anatole and everyone who knew them respected Dolokhov more than Anatole. Dolokhov played all the games and almost always won. However much he drank, he never lost his clarity of mind. At that time, both Kuragin and Dolokhov were celebrities in the world of rakes and souses of Petersburg.

The bottle of rum was brought in; the frame, which kept anyone from sitting on the outer slope of the window, was knocked off by two footmen, apparently hurried along and shaken by the shouts of the surrounding gentlemen.

Anatole approached the window with his confident swagger. He wanted to break something. He shoved back the footmen and pulled on the frame, but the frame would not give. He shattered the glass.

``Hey, you give it a try, you brute,'' he said, addressing Pierre. %kuragin

Pierre grabbed the cross-piece, pulled, and the oak frame broke in some places and twisted in others with a crash.

``Get the whole thing off, or they'll think I'm holding on,'' Dolokhov said. %dolokhov

``The Englishman is bragging\ldots{} Eh\ldots{}? Alright\ldots{}?'' said Anatole. %kuragin

``Alright,'' Pierre said, glancing at Dolokhov, who, having taken the bottle of rum in hand, approached the window, through which the light of the sky and the seamless merging of dusk and dawn could be seen. %pierre

Dolokhov jumped onto the window with the bottle of rum in his hand.

``Listen!'' he shouted, standing on the window sill and turning to face the room. Everyone fell silent. %dolokhov

``I will wager,'' he said in French, so that the Englishman could understand him, and he did not speak that language very well. ``I will wager fifty imperials, do you want to make it a hundred?'' he added, turning to the Englishman. %dolokhov

``No, fifty,'' the Englishman said. %english

``Alright, fifty imperials says I can drink the entire bottle of rum without taking it out of my mouth, sitting on the window, right here on this spot,'' he bent down and pointed to the sloping ledge of the wall outside the window, ``and not holding on to anything\ldots{} Yes?'' %dolokhov

``Very well,'' the Englishman said. %english

Anatole turned to the Englishman and, taking him by a button on his tail-coat and looking down on him from above (the Englishman was short in stature), he began to repeat the conditions of the bet to him in English.

``Stop!'' Dolokhov shouted, knocking the bottle against the window to get everyone's attention. ``Stop, Kuragin; listen to me. If anyone else manages to do it, I'll pay a hundred imperials. Understand?'' %dolokhov

The Englishman nodded his head but gave no sign whether he was prepared to accept this new condition or not. Anatole kept holding on to the Englishman and, despite the fact that the latter had let him know by nodding that he understood, Anatole translated all of Dolokhov's words into English for him. A young, thin boy, a Life Hussar who had lost at the tables that night, climbed onto the window, leaned out, and looked down.

``Ooh! Ooh! Ooh!'' he said, looking beyond the window at the stone footpath.

``Be careful!'' Dolokhov shouted and pulled the officer off of the window. The young man's spurs became tangled and he jumped awkwardly into the room.

Placing the bottle on the window sill so that it would be easy to get, Dolokhov carefully and calmly crawled through the window. Lowering his legs and forcing his hands against the edges of the window, he focused, sat down, lowered his arms, turned to the right, then the left, and got the bottle. Anatole brought over two candles and placed them on the window sill, although it was already completely light out. Dolokhov's curly hair and the white shirt covering his back were illuminated from both sides. Everyone stood around the window. The Englishman stood in front. Pierre was smiling and saying nothing. One of the men present, older than the others, suddenly moved forward with an angry and frightened look and wanted to grab Dolokhov by the shirt.

``Gentlemen, this is perfectly stupid; he's going to fall to his death,'' this more sensible person said.

Anatole stopped him.

``Don't touch him, you'll startle him and he'll fall. See? What then? See?'' %kuragin

Dolokhov turned, straightening himself and again forcing his hands against the window. 

``If anyone else wants to come at me,'' he said, the words barely escaping his thin, tightly compressed lips, ``then I'll throw them right down there. Well?!'' %dolokhov

Having said that ``Well?!''~he turned again, lowered his arms, took the bottle, and put it to his lips, threw back his head, and threw his free hand into the air for balance. One of the footmen, who had begun to pick up the glass, stopped in a bent-over position, not taking his eyes off of the window and Dolokhov's back. Anatole stood straight, eyes wide open. The Englishman, his lips stuck out, looked on from the side. The person who had tried to stop everything ran into a corner of the room and lay down on the sofa with his face to the wall. Pierre covered his face, and a weak smile remained, forgotten, on his face, even though that face now expressed horror and fear. Everyone was silent. Pierre took his hands away from his eyes. Dolokhov was sitting in the same position as before, only his head was bent back until the curly hair on the back of his head touched the collar of his shirt, and the hand holding the bottle was rising higher and higher, shuddering with visible effort. The bottle was clearly emptying and rising at the same time, as if it were pulling Dolokhov's head higher. ``How can it be going on so long?'' Pierre thought. It seemed to him that more than half an hour had passed. Suddenly, Dolokhov's back shifted backward, and his hand exhibited a nervous shake; that shudder was enough to move his entire body, sitting on the outer slope. His whole person shifted, and his hand and head shook with even stronger effort. One hand went up to grab for the window sill, but again lowered itself. Pierre closed his eyes again and said to himself that he would never open them again. Suddenly, he felt that everything around him was stirring. He looked up: Dolokhov was standing on the window sill, his face pale and happy.

``Empty!'' %dolokhov

He threw the bottle to the Englishman, who caught it easily. Dolokhov jumped down from the window. He smelled strongly of rum.

``Excellent! Well done! That's how to win a bet! One hell of a way to do it!'' people shouted on all sides.

The Englishman, having gotten out his purse, counted out money. Dolokhov frowned and was silent. Pierre jumped up onto the window.

``Gentlemen! Who wants to have a bet with me? I'll do the same,'' he shouted suddenly. ``And you know, I don't even need a bet. Get them to bring me a bottle. I'll do it\ldots{} Have them bring it.'' %pierre

``Come on, come on!'' Dolokhov said, smiling. %dolokhov

``What is wrong with you? Have you lost your mind? Who would let you do that? You get dizzy from climbing stairs,'' people said on all sides.

``I'll drink it, give me a bottle of rum!'' Pierre shouted, hitting a chair with a decisive and drunked gesture, and climbed through the window.

Everyone grabbed at him; but he was strong enough that he shoved anyone who got near him far back into the room.

``No, you won't catch him like that at all,'' Anatole said, ``stop, let me trick him. Listen, I'll make a bet with you, but not until tomorrow, right now we're going to ---'s.'' %kuragin

``Let's go,'' Pierre shouted, ``let's go! We'll take Mishka with us\ldots{}'' %pierre

And he grabbed for the bear, and, having embraced it and gotten it to its feet, he started to dance around the room with it.

\section*{VII} % Book One, Part One, Chapter VII

Prince Vasily kept the promise that he made at Anna Pavlovna's party to Princess Drubetskaya, who had petitioned him on behalf of her only son, Boris. A report was made about him to the sovereign, and, unlike everyone else, he was transferred to the Semyonovsk Guards Regiment as an ensign. But to be an adjutant or member of Kutuzov's staff was not where Boris was assigned, despite all of Anna Mikhailovna's efforts and intrigues. Soon after Anna Pavlovna's party, Anna Mikhailovna returned to Moscow, went directly to her wealthy relatives, the Rostovs, with whom she stayed in Moscow and with whom the Boryenka she adored (only recently brought into the army and transferred at once to become a Guards ensign) grew up and lived for years starting in childhood. The Guards had already left Petersburg on the 10th of August, and her son, remaining in Moscow to be fitted for his uniform, was supposed to catch up to her on the road to Radivilov.

At the Rostovs', both Natalyas, mother and younger daughter, were celebrating their name day. Since morning, teams of horses had been arriving and leaving ceaselessly, bringing well-wishers to the large home of the Rostovs on Povarskaya Street, well known to all of Moscow. The countess and her beautiful older daughter sat with their guests, who never ceased changing, in the drawing room.

The countess was a woman who had a thin face of an eastern type, about forty-five years old, apparently worn-out by her children, who were twelve in number. The slowness of her movements and speech, which came from weakness and lack of energy, gave her a significant air that inspired respect. Princess Anna Mikhailovna Drubetskaya, as someone who was part of the household, sat there as well, helping in the matter of receiving guests and keeping them occupied with talk. The young people were in the back rooms, not finding it necessary to participate in these visits. The count was meeting and seeing off guests, inviting everyone to dinner.

``I'm very, very grateful to you, \textit{ma ch\`ere} or \textit{mon cher},'' (he said \textit{ma ch\`ere} or \textit{mon cher} to everyone without exception, without any change in meaning, whether to those higher than him or lower than him in position), ``on my own behalf and on behalf of our dear ladies celebrating their name day. See that you come for dinner. You'll offend me if you don't, \textit{mon cher}. I'm asking you from the bottom of my heart, on behalf of the whole family, \textit{ma ch\`ere}.'' He said these words to everyone without exception or change, with the exact same expression on his full, happy, and cleanly shaven face and the exact same forceful handshake and a repetitive short bow. Having seen off one guest, the count returned to another one who was still in the drawing room; having moved up an armchair and with the look of a person who loved life and knew how to live it, setting his feet apart and placing his hands on his knees smartly, he swayed back and forth significantly, made predictions about the weather, advised them about their health, sometimes in Russian, sometimes in an awful but self-assured French, and then once more, with the look of a tired person nevertheless firm in the performance of his duties, he got up to see off the guest, adjusting his thinning hair over his bald spot, and again invited them to dinner. Sometimes, returning from the front room, he went through the conservatory and the footman's pantry into the large marble hall, where they were setting the table for eighty places, and, glancing at the waiters carrying silver and porcelain, arranging the tables and unfolding the damask tablecloths, he called over Dmitry Vasilyevich, a nobleman who managed all his affairs, and said: %countr

``Well, well, Mityenka, make sure everything comes out well. Yes, yes,'' he said, looking with pleasure at an enormous table being brought out. ``The most important thing are the table settings. Mmm-hmm\ldots{}'' And he walked away, sighing contentedly, to go back to the drawing room. %countr

``Marya Lvovna Karagina with her daughter!'' said the count's enormous traveling footman in a deep bass, coming through the doors of the drawing room. The countess thought for a moment and took a sniff from a golden snuffbox with a portrait of her husband.

``These visits are torturing me,'' she said. ``Well, she is going to be the last one I receive. Such a prude. Let her in,'' she said to the footman in a sad voice, as if she were saying, \textit{Well, finish me off!} %countessr

A tall, stout woman with a proud air and her round-faced, smiling daughter entered the drawing room, their dresses swishing.

\textit{``Ch\`ere comtesse, il y a si longtemps\ldots{} elle a \'et\'e alit\'ee la pauvre enfant\ldots{} au bal des Razoumowsky\ldots{} et la comtesse Apraksine\ldots{} j'ai et\'e si heureuse\ldots{}''} came the sounds of lively feminine voices interrupting one another and mixing with the noise of dresses and the moving of chairs.\footnote{My dear countess, it has been such a long time\ldots{} She has been bedridden, the poor child\ldots{} At the Razumovskys' ball\ldots{} and the Countess Apraksin\ldots{} I was so happy\ldots{} \textit{Fr.}} The kind of conversation began that goes on just long enough so that, at the first pause, one can stand up, noisily gather up one's dress, say, \textit{``Je suis bien charm\'ee; la sant\'e de maman\ldots{} et la comtesse Apraksine,''} and, again gathering up one's dress, pass into the front room, put on one's coat or cloak, and leave.\footnote{So, so charmed; mother's health\ldots{} and Countess Apraksin. \textit{Fr.}} The conversation was about the most important news in the city at that time --- the illness of the famous old Count Bezukhov, a wealthy man, known as being very handsome during the days of Empress Yekaterina, and his illegitimate son, Pierre, who had behaved so indecently at Anna Pavlovna Scherer's party.
