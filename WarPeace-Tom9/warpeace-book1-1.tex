% Copyright 2015 Christopher A. Tessone
% Creative Commons Attribution-NonCommercial-ShareAlike 4.0
% International License
% http://creativecommons.org/licenses/by-nc-sa/4.0/
\makeoddhead{modruled}{}{\scshape War and Peace}{}
\markboth{War and Peace}{}

\chapter*{Part One} %c1

\section*{I} %% Book One, Part One, Chapter I

``\textit{Eh bien, mon prince. G\^enes et Lucques ne sont plus que des
  apanages, des} estates, \textit{de la famille Buonaparte. Non, je
  vous pr\'eviens, que si vous ne me dites pas, que nous avons la
  guerre, si vous vous permettez encore de pallier toutes les
  infamies, toutes les atrocit\'es de cet Antichrist (ma parole, j'y
  crois) --- je ne vous connais plus, vous n'\^etes plus mon ami, vous
  n'\^etes plus} my devoted slave, \textit{comme vous dites.} Well,
hello, hello. \textit{Je vois que je vous fair peur,} sit down and
tell me everything.''\footnote{Well then, my prince, Genoa and Lucca
  are nothing more than appanages, estates, of the Bonaparte
  family. No, I am warning you, if you do not tell me that we are
  going to war, if you permit yourself to make excuses for all the
  vile deeds, all the atrocities of that Antichrist (I give you my
  word, I believe that)---then I do not know you, you are no longer my
friend, you are no longer my devoted slave, as you say\ldots{} I can
see I have frightened you\ldots{} \textit{Fr.}} %annapavlovna

This was said in July 1805 by the famous Anna Pavlovna Scherer, maid
of honor and confidante of Empress Maria Feodorovna, upon meeting the
important and high-ranking Prince Vasily, who was the first to arrive
at her party. Anna Pavlovna had been coughing for several days, she
had \textit{la grippe}, as she said (\textit{grippe} was then a new
word, used only by the elite). In notes that were sent out that
morning with a red-liveried footman, she had written to everyone
without distinction:

\begin{quote}
\textit{Si vous n'avez rien de mieux \`a faire, M.~le comte} (or
\textit{mon prince}), \textit{et si la perspective de passer la
  soir\'ee chez une pauvre malade ne vous effraye pas trop, je serai
  charm\'ee de vous voir chez moi entre 7 et 10 heures. Annette
  Scherer.}\footnote{If you have nothing better to do, Count (or
  Prince), and if the prospect of spending the evening with a poor
  sick woman does not scare you off, I would be pleased to see you at
  my home between 7 and 10 o'clock. Annette Scherer. \textit{Fr.}}
\end{quote}

\textit{``Dieu, quelle virulente sortie!''}\footnote{God, what a
  vicious attack! \textit{Fr.}} answered the prince as he entered, not
put out in the least by this welcome; he was dressed in an embroidered
court uniform, stockings, boots, and stars and had a bright expression
on his flat face.

He spoke in that refined French in which our ancestors not only spoke,
but thought, and with the quiet, patronizing intonations that are
particular to a significant person who has spent a long life in
society and at court. He approached Anna Pavlovna, kissed her hand,
presenting his scented and gleaming bald spot to her, and calmly sat
down on the sofa.

