% Copyright 2015 Christopher A. Tessone
% Creative Commons Attribution-NonCommercial-ShareAlike 4.0
% International License
% http://creativecommons.org/licenses/by-nc-sa/4.0/
\makeoddhead{modruled}{}{\scshape War and Peace}{}
\markboth{War and Peace}{}

\chapter*{Part One} %c1

\section*{I} %% Book One, Part One, Chapter I

``\textit{Eh bien, mon prince. G\^enes et Lucques ne sont plus que des
  apanages, des} estates, \textit{de la famille Buonaparte. Non, je
  vous pr\'eviens, que si vous ne me dites pas, que nous avons la
  guerre, si vous vous permettez encore de pallier toutes les
  infamies, toutes les atrocit\'es de cet Antichrist (ma parole, j'y
  crois) --- je ne vous connais plus, vous n'\^etes plus mon ami, vous
  n'\^etes plus} my devoted slave, \textit{comme vous dites.} Well,
hello, hello. \textit{Je vois que je vous fair peur,} sit down and
tell me everything.''\footnote{Well then, my prince, Genoa and Lucca
  are nothing more than appanages, estates, of the Bonaparte
  family. No, I am warning you, if you do not tell me that we are
  going to war, if you permit yourself to make excuses for all the
  vile deeds, all the atrocities of that Antichrist (I give you my
  word, I believe that)---then I do not know you, you are no longer my
friend, you are no longer my devoted slave, as you say\ldots{} I can
see I have frightened you\ldots{} \textit{Fr.}} %annapavlovna

This was said in July 1805 by the famous Anna Pavlovna Scherer, maid
of honor and confidante of Empress Maria Feodorovna, upon meeting the
important and high-ranking Prince Vasily, who was the first to arrive
at her party. Anna Pavlovna had been coughing for several days, she
had \textit{la grippe}, as she said (\textit{grippe} was then a new
word, used only by the elite). In notes that were sent out that
morning with a red-liveried footman, she had written to everyone
without distinction:

\begin{quote}
\textit{Si vous n'avez rien de mieux \`a faire, M.~le comte} (or
\textit{mon prince}), \textit{et si la perspective de passer la
  soir\'ee chez une pauvre malade ne vous effraye pas trop, je serai
  charm\'ee de vous voir chez moi entre 7 et 10 heures. Annette
  Scherer.}\footnote{If you have nothing better to do, Count (or
  Prince), and if the prospect of spending the evening with a poor
  sick woman does not scare you off, I would be pleased to see you at
  my home between 7 and 10 o'clock. Annette Scherer. \textit{Fr.}}
\end{quote}

\textit{``Dieu, quelle virulente sortie!''}\footnote{God, what a
  vicious attack! \textit{Fr.}} answered the prince as he entered, not
put out in the least by this welcome; he was dressed in an embroidered
court uniform, stockings, boots, and stars and had a bright expression
on his flat face.

He spoke in that refined French in which our ancestors not only spoke,
but thought, and with the quiet, patronizing intonations that are
particular to a significant person who has spent a long life in
society and at court. He approached Anna Pavlovna, kissed her hand,
presenting his scented and gleaming bald spot to her, and calmly sat
down on the sofa.

``\textit{Avant tout dites moi, comment vous allez, ch\`ere
  amie?}\footnote{First, tell me everything, how are you doing, dear
  friend? \textit{Fr.}} Set my mind at ease,'' he said, not changing
his voice and tone, in which indifference and even mockery could be
heard behind sympathy and good decorum. %vasily

``How can one be healthy\ldots{}when one is suffering morally? Can one
really remain calm in our time, if one has any feeling?'' Anna
Pavlovna said. ``You are staying here the whole evening, I hope?''
% annapavlovna

``But what about the English ambassador's festival? Today is
Wednesday. I have to put in an appearance there,'' the prince
said. ``My daughter is coming for me and will take me there.'' %vasily

``I thought that today's festival was postponed. \textit{Je vous avoue
  que toutes ces f\^etes et tous ces feux d'artifice commencent \`a
  devenir insipides.}''\footnote{I confess that all these parties and
  all these fireworks are beginning to starting to become
  dull. \textit{Fr.}} %annapavlovna

``If they had known that you wished it, they would have postponed the
festival,'' the prince said out of habit, like a wound clock, saying
things that he did not expect others to believe. %vasily

\textit{``Ne me tourmentez pas. Eh bien, qu'a-t-on d\'ecid\'e par
  rapport \`a la d\'ep\^eche de Novosilzoff? Vous savez
  tout.''}\footnote{Stop tormenting me. Very well, what have they
  decided about the report on Novosiltsov's dispatch? You know
  everything. \textit{Fr.}} %annapavlovna

``What can I tell you?'' the prince said in a cold, bored
tone. \textit{``Qu'a-t-on d\'ecid\'e? On a d\'ecid\'e que Buonaparte a
  br\^ul\'e ses vaisseuax, et je crois que nous sommes en train de
  br\^uler les notres.''}\footnote{What have they decided? They have
  decided that Bonaparte has burned his ships, and I believe that we
  are about to burn ours. \textit{Fr.}} %vasily

Prince Vasily always spoke lazily, like an actor says his lines in an
old play. Anna Pavlovna Scherer, on the contrary, despite her forty
years, was brimming with liveliness and outburts of energy.

To be an enthusiast had become her place in society, and sometimes,
when she did not even want to, she played the part of the enthusiast
so as not to disappoint people's expectations. The restrained smile
that constantly played on Anna Pavlovna's face, although it did not
fit her aging features, expressed a constant awareness, like that of a
spoiled child, of her endearing shortcoming, which she found neither
desirable or necessary to correct.

In the middle of this conversation about political goings-on, Anna
Pavlovna became heated.

``Ah, do not talk to me about Austria! Perhaps I do not understand
anything, but Austria has never wanted war and does not want it
now. They will betray us. Russia alone must be the savior of
Europe. Our benefactor knows his high calling and will be true to
it. That is one thing that I trust. The greatest role in the world
awaits our kind and marvelous sovereign, and he is so beneficent and
good that God will not leave his side, and he will fulfill his calling
to crush the hydra of revolution, which is now so horribly present in
the person of that villain and murderer. We alone must atone for the
blood of that righteous man.\footnote{Anna Pavlovna is referring to
  the execution of the Duke of Enghien by Napoleon in 1804. ---
  Trans.} In whom can we place our hope, I ask you\ldots{}? England
with its commercial spirit will not rise to the heights of Emperor
Aleksandr's soul and never could rise that high. They refused to
liberate Malta. They want to wait and see, they are looking for an
ulterior motive for our actions. What did they say to
Novosiltsov\ldots{}? Nothing. They do not understand, cannot
understand the self-sacrifice of our emperor, who does not want
anything for himself and does everything for the good of the
world. And what have they promised? Nothing. And even what they
promised, they will not do! Prussia has also declared that Bonaparte
cannot be beaten and that all of Europe can do nothing against
him\ldots{} And I do not believe even one word from Hardenburg or
Haugwitz. \textit{Cette fameuse neutralit\'e prussienne, ce n'est
  qu'un pi\`ege.}\footnote{That famous Prussian neutrality, it is
  nothing but a trap. \textit{Fr.}} I trust in God alone and in the
lofty fate of our emperor. He will save Europe\ldots{}!'' She stopped
suddenly with a smile, mocking her own hot temper. %annapavlovna

``I think,'' the prince said, smiling, ``that had they sent you in
place of our dear Wintzingerode, you would have seized the Prussian
king's agreement in a day. You are so eloquent. Could you give me some
tea?'' %vasily

``Just a moment. \textit{A propos,}'' she added, calming down again,
``two very interesting people will be here today, \textit{le vicomte
  de Mortemart, il est alli\'e aux Montmorency par les Rohans,} one of
the best families of France.\footnote{The viscount of Mortemart, he is
  a relative by marriage of the Montmorencys through the
  Rohans\ldots{} \textit{Fr.}} He is one of the good emigrants, one of
the true ones. And then \textit{l'abb\'e Morio:} do you know that deep
mind? He has been received by the sovereign. Do you know him?''
%annapavlovna

``Ah! I will be happy to meet him,'' the prince said. ``Tell me,'' he
added, as if he had just remembered something, and especially
off-handedly at that, when, in fact, the thing he was asking about was
the primary reason for his visit, ``is it true that
\textit{l'imp\'eratrice-m\`ere} wishes to see Baron Funke named first
secretary at Vienna? \textit{C'est un pauvre sire, ce baron, \`a ce
  qu'il para\^it.}''\footnote{The dowager empress\ldots{} He is a poor
  excuse for a man, that baron, or so it seems. \textit{Fr.}} Prince
Vasily wished to see his son named to the post, which others were
trying to obtain for the baron through Empress Maria Feodorovna. %vasily

Anna Pavlovna half-closed her eyes to show that neither she, nor
anyone else, could pass judgment about what the empress did or did not
like.

\textit{``Monsieur le baron de Funke a \'et\'e recommand\'e \`a
  l'imp\'eratrice-m\`ere par sa soeur,''} was all she would say, in a
dry, sad tone.\footnote{Baron Funke was recommended to the dowager
  empress by his sister. \textit{Fr.}} When Anna Pavlovna referred to
the empress, her face suddenly took on an expression of deep and
earnest devotion and respect, joined with sadness, which happened
every time she mentioned her lofty patroness in conversation. She told
him that Her Majesty had been pleased to show Baron Funke
\textit{beaucoup d'estime,} and her eyes again took on a sad
look.\footnote{Great respect. \textit{Fr.}} %annapavlovna

The prince fell into an indifferent silence. Anna Pavlovna, with her
characteristic courtly and feminine dexterity and quickness of tact,
wanted to slap him on the wrist for daring to speak up about a person
who had been recommended to the empress and at the same time comfort
him.

\textit{``Mais \`a propos de votre famille,''} she said, ``did you
know that your daughter, ever since she came out, \textit{fait les
  d\'elices de tout le monde. On la trouve belle, comme le
  jour.}''\footnote{But about your family\ldots{} [She] has become the
  delight of the whole world. People find her to be as beautiful as
  the light of day. \textit{Fr.}} %annapavlovna

The prince bowed to show his respect and appreciation.

``I often think,'' Anna Pavlovna continued after a moment's silence,
moving closer to the prince and smiling at him affectionately, as if
to show him that talk of politics and society was over and now an
intimate conversation was beginning: ``I often think about how
happiness in life is sometimes distributed unfairly. Why did fate give
you two wonderful children---excluding Anatoly, your youngest, him I do
not love,'' she put in peremptorily, raising her brows, ``such
charming children? And indeed you value them less than anyone and for
that reason are not worthy of them.'' %annapavlovna

And she smiled her ecstatic smile.

\textit{``Que voulez-vous? Lafater aurait dit que je n'ai pas la bosse
  de la paternit\'e,''} the prince said.\footnote{What would you have
  me do? Lavater would say that I do not have the bump of
  paternity. \textit{Fr.}} %vasily

``Stop making jokes. I wanted to speak with you seriously. Do you
know, I am displeased with your youngest son. Let me say, just between
us,'' her face took on a sad expression, ``they were talking about him
before Her Highness and people pity you\ldots{}'' %annapavlovna

The prince did not answer, but she looked at him significantly, saying
nothing and waiting for an answer. Prince Vasily frowned.

``What am I supposed to do?'' he said finally. ``You know that I have
done everything for their upbringing that a father can do, and both of
them turned out \textit{des
  imb\'eciles.}\footnote{Idiots. \textit{Fr.}} Ippolit is a fool at
rest, but Anatoly is restless. That is the only difference,'' he said,
smiling even more unnaturally and animatedly than usual, and at the
same time showing very openly an aspect that was unexpectedly coarse
and unpleasant in the wrinkles around his mouth. %vasily

``And why are children born to people like you? If you were not a
father, I would be able to find no fault in you,'' Anna Pavlovna said,
raising her eyes pensively. %annapavlovna

\textit{``Je suis votre} devoted slave, \textit{et \`a vous seule je
  puis l'avouer.} My children---\textit{ce sont les entraves de mon
  existence.} And my cross. That is how I explain it to
myself. \textit{Que voulez vous\ldots{}?''}\footnote{I am your devoted
  slave, and I will confess it to you alone. My children are the bane
  of my existence\ldots{} What would you have me do? \textit{Fr.}} He
fell silent, showing with a gesture his submission to cruel fate.

Anna Pavlovna was pensive.

``You have not thought to look for a wife for your prodigal son
Anatoly. They say,'' she said, ``that old maids \textit{ont la manie
  des mariages.} I do not yet feel that weakness in myself, but I do
have a \textit{petite personne} who is unhappy with her father,
\textit{une parente \`a nous, une princesse}
Bolkonskaya.''\footnote{\ldots{}have an obsession with making
  marriages. \ldots{}Little person\ldots{}related to us, a princess
  Bolkonskaya. \textit{Fr.}} Prince Vasily did not answer, although
with a quickness of understanding and memory characteristic of society
people, he showed with a movement of his head that he had taken in all
this information. %annapavlovna

``No, did you know that that Anatoly costs me 40,000 a year,'' he
said, obviously unable to hold back the agrieved course of his
thoughts. He was silent. %vasily

``How will things be in five years if this continues? \textit{Voil\`a
  l'avantage d'\^etre p\`ere.} Is she rich, this
princess?''\footnote{The benefits of being a father. \textit{Fr.}}

``The father is very rich and stingy. He lives in the country. Do you
know, it is the famous Prince Bolkonsky, who was dismissed from
service under the late emperor and was called the ``Prussian king.''
He is a very intelligent man, but severe and idiosyncratic. \textit{La
  pauvre petite est malheureuse, comme les pierres.} She has a
brother, the man who just married Lise Meinen, one of Kutuzov's
adjutants. He will be here tonight.''\footnote{The poor little girl is
  as sad as a dog with no home. \textit{Fr.}} %annapavlovna

\textit{``Ecoutez, ch\`ere Annette,''} the prince said, suddenly
taking her by the hand and pulling it downward for some
reason. ``\textit{Arrangez-moi cette affaire et je suis votre} most
devoted slave \textit{\`a tout jamais}---slafe \textit{comme mon}
foreman \textit{m'ecrit des} dispatches: ay-eff-ee. She is from a good
family and rich. That is all that I need.''\footnote{Listen, dear
  Annette. Arrange this matter for me and I will be your most devoted
  slave forever---slafe as my foreman writes to me in his
  dispatches. \textit{Fr.}} %vasily

And with the free and familiar, graceful movements that set him apart,
he took the maid of honor by the hand, kissed it, and, having kissed
it, waved the maid of honor's hand around, falling back against his
armchair and looking to the side.

\textit{``Attendez,''} Anna Pavlovna said, weighing the
matter. ``Tonight I will say something to Lise (\textit{la femme du
  jeune} Bolkonsky). And maybe it will be settled. \textit{Ce sera
  dans votre famille, que je ferai mon apprentissage de vieille
  fille.}''\footnote{Listen. \ldots{}the young Bolkonsky's
  wife\ldots{} It will be with your family that I undergo my
  apprenticeship to become an old maid. \textit{Fr.}}

\section*{II} % Book One, Part One, Chapter Two

