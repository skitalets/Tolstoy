% Copyright 2015 Christopher A. Tessone
% Creative Commons Attribution-NonCommercial-ShareAlike 4.0
% International License
% http://creativecommons.org/licenses/by-nc-sa/4.0/
\makeoddhead{modruled}{}{\scshape War and Peace}{}
\markboth{War and Peace}{}

\chapter*{Part One} %c1

\section*{I} %% Book One, Part One, Chapter I

``\textit{Eh bien, mon prince. G\^enes et Lucques ne sont plus que des
  apanages, des} estates, \textit{de la famille Buonaparte. Non, je
  vous pr\'eviens, que si vous ne me dites pas, que nous avons la
  guerre, si vous vous permettez encore de pallier toutes les
  infamies, toutes les atrocit\'es de cet Antichrist (ma parole, j'y
  crois) --- je ne vous connais plus, vous n'\^etes plus mon ami, vous
  n'\^etes plus} my devoted slave, \textit{comme vous dites.} Well,
hello, hello. \textit{Je vois que je vous fair peur,} sit down and
tell me everything.''\footnote{Well then, my prince, Genoa and Lucca
  are nothing more than appanages, estates, of the Bonaparte
  family. No, I am warning you, if you do not tell me that we are
  going to war, if you permit yourself to make excuses for all the
  vile deeds, all the atrocities of that Antichrist (I give you my
  word, I believe that)---then I do not know you, you are no longer my
friend, you are no longer my devoted slave, as you say\ldots{} I can
see I have frightened you\ldots{} \textit{Fr.}} %annapavlovna

This was said in July 1805 by the famous Anna Pavlovna Scherer, maid
of honor and confidante of Empress Maria Feodorovna, upon meeting the
important and high-ranking Prince Vasily, who was the first to arrive
at her party. Anna Pavlovna had been coughing for several days, she
had \textit{la grippe}, as she said (\textit{grippe} was then a new
word, used only by the elite). In notes that were sent out that
morning with a red-liveried footman, she had written to everyone
without distinction:

\begin{quote}
\textit{Si vous n'avez rien de mieux \`a faire, M.~le comte} (or
\textit{mon prince}), \textit{et si la perspective de passer la
  soir\'ee chez une pauvre malade ne vous effraye pas trop, je serai
  charm\'ee de vous voir chez moi entre 7 et 10 heures. Annette
  Scherer.}\footnote{If you have nothing better to do, Count (or
  Prince), and if the prospect of spending the evening with a poor
  sick woman does not scare you off, I would be pleased to see you at
  my home between 7 and 10 o'clock. Annette Scherer. \textit{Fr.}}
\end{quote}

\textit{``Dieu, quelle virulente sortie!''}\footnote{God, what a
  vicious attack! \textit{Fr.}} answered the prince as he entered, not
put out in the least by this welcome; he was dressed in an embroidered
court uniform, stockings, boots, and stars and had a bright expression
on his flat face.

He spoke in that refined French in which our ancestors not only spoke,
but thought, and with the quiet, patronizing intonations that are
particular to a significant person who has spent a long life in
society and at court. He approached Anna Pavlovna, kissed her hand,
presenting his scented and gleaming bald spot to her, and calmly sat
down on the sofa.

``\textit{Avant tout dites moi, comment vous allez, ch\`ere
  amie?}\footnote{First, tell me everything, how are you doing, dear
  friend? \textit{Fr.}} Set my mind at ease,'' he said, not changing
his voice and tone, in which indifference and even mockery could be
heard behind sympathy and good decorum. %vasily

``How can one be healthy\ldots{}when one is suffering morally? Can one
really remain calm in our time, if one has any feeling?'' Anna
Pavlovna said. ``You are staying here the whole evening, I hope?''
% annapavlovna

``But what about the English ambassador's festival? Today is
Wednesday. I have to put in an appearance there,'' the prince
said. ``My daughter is coming for me and will take me there.'' %vasily

``I thought that today's festival was postponed. \textit{Je vous avoue
  que toutes ces f\^etes et tous ces feux d'artifice commencent \`a
  devenir insipides.}''\footnote{I confess that all these parties and
  all these fireworks are beginning to starting to become
  dull. \textit{Fr.}} %annapavlovna

``If they had known that you wished it, they would have postponed the
festival,'' the prince said out of habit, like a wound clock, saying
things that he did not expect others to believe. %vasily

\textit{``Ne me tourmentez pas. Eh bien, qu'a-t-on d\'ecid\'e par
  rapport \`a la d\'ep\^eche de Novosilzoff? Vous savez
  tout.''}\footnote{Stop tormenting me. Very well, what have they
  decided about the report on Novosiltsov's dispatch? You know
  everything. \textit{Fr.}} %annapavlovna

``What can I tell you?'' the prince said in a cold, bored
tone. \textit{``Qu'a-t-on d\'ecid\'e? On a d\'ecid\'e que Buonaparte a
  br\^ul\'e ses vaisseuax, et je crois que nous sommes en train de
  br\^uler les notres.''}\footnote{What have they decided? They have
  decided that Bonaparte has burned his ships, and I believe that we
  are about to burn ours. \textit{Fr.}} %vasily

Prince Vasily always spoke lazily, like an actor says his lines in an
old play. Anna Pavlovna Scherer, on the contrary, despite her forty
years, was brimming with liveliness and outburts of energy.

To be an enthusiast had become her place in society, and sometimes,
when she did not even want to, she played the part of the enthusiast
so as not to disappoint people's expectations. The restrained smile
that constantly played on Anna Pavlovna's face, although it did not
fit her aging features, expressed a constant awareness, like that of a
spoiled child, of her endearing shortcoming, which she found neither
desirable or necessary to correct.

In the middle of this conversation about political goings-on, Anna
Pavlovna became heated.

``Ah, do not talk to me about Austria! Perhaps I do not understand
anything, but Austria has never wanted war and does not want it
now. They will betray us. Russia alone must be the savior of
Europe. Our benefactor knows his high calling and will be true to
it. That is one thing that I trust. The greatest role in the world
awaits our kind and marvelous sovereign, and he is so beneficent and
good that God will not leave his side, and he will fulfill his calling
to crush the hydra of revolution, which is now so horribly present in
the person of that villain and murderer. We alone must atone for the
blood of that righteous man.\footnote{Anna Pavlovna is referring to
  the execution of the Duke of Enghien by Napoleon in 1804. ---
  Trans.} In whom can we place our hope, I ask you\ldots{}? England
with its commercial spirit will not rise to the heights of Emperor
Aleksandr's soul and never could rise that high. They refused to
liberate Malta. They want to wait and see, they are looking for an
ulterior motive for our actions. What did they say to
Novosiltsov\ldots{}? Nothing. They do not understand, cannot
understand the self-sacrifice of our emperor, who does not want
anything for himself and does everything for the good of the
world. And what have they promised? Nothing. And even what they
promised, they will not do! Prussia has also declared that Bonaparte
cannot be beaten and that all of Europe can do nothing against
him\ldots{} And I do not believe even one word from Hardenburg or
Haugwitz. \textit{Cette fameuse neutralit\'e prussienne, ce n'est
  qu'un pi\`ege.}\footnote{That famous Prussian neutrality, it is
  nothing but a trap. \textit{Fr.}} I trust in God alone and in the
lofty fate of our emperor. He will save Europe\ldots{}!'' She stopped
suddenly with a smile, mocking her own hot temper. %annapavlovna

``I think,'' the prince said, smiling, ``that had they sent you in
place of our dear Wintzingerode, you would have seized the Prussian
king's agreement in a day. You are so eloquent. Could you give me some
tea?'' %vasily

