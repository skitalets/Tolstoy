% Copyright 2015 Christopher A. Tessone
% Creative Commons Attribution-NonCommercial-ShareAlike 4.0
% International License
% http://creativecommons.org/licenses/by-nc-sa/4.0/
\makeoddhead{modruled}{}{\scshape War and Peace}{}
\markboth{War and Peace}{}

\chapter*{Part One} %c1

\section*{I} %% Book One, Part One, Chapter I

``\textit{Eh bien, mon prince. G\^enes et Lucques ne sont plus que des apanages, des} estates, \textit{de la famille Buonaparte. Non, je vous pr\'eviens, que si vous ne me dites pas, que nous avons la guerre, si vous vous permettez encore de pallier toutes les infamies, toutes les atrocit\'es de cet Antichrist (ma parole, j'y crois) --- je ne vous connais plus, vous n'\^etes plus mon ami, vous n'\^etes plus} my devoted slave, \textit{comme vous dites.} Well, hello, hello. \textit{Je vois que je vous fair peur,} sit down and
tell me everything.''\footnote{Well then, my prince, Genoa and Lucca are nothing more than appanages, estates, of the Bonaparte family. No, I am warning you, if you do not tell me that we are going to war, if you permit yourself to make excuses for all the vile deeds, all the atrocities of that Antichrist (I give you my word, I believe that)---then I do not know you, you are no longer my friend, you are no longer my devoted slave, as you say\ldots{} I can see I have frightened you\ldots{} \textit{Fr.}} %annapavlovna

This was said in July 1805 by the famous Anna Pavlovna Scherer, maid of honor and confidante of Empress Maria Feodorovna, upon meeting the important and high-ranking Prince Vasily, who was the first to arrive at her party. Anna Pavlovna had been coughing for several days, she had \textit{la grippe}, as she said (\textit{grippe} was then a new word, used only by the elite). In notes that were sent out that morning with a red-liveried footman, she had written to everyone without distinction:

\begin{quote}
\textit{Si vous n'avez rien de mieux \`a faire, M.~le comte} (or \textit{mon prince}), \textit{et si la perspective de passer la soir\'ee chez une pauvre malade ne vous effraye pas trop, je serai charm\'ee de vous voir chez moi entre 7 et 10 heures. Annette Scherer.}\footnote{If you have nothing better to do, Count (or Prince), and if the prospect of spending the evening with a poor, sick woman does not scare you off, I would be pleased to see you at my home between 7 and 10 o'clock. Annette Scherer. \textit{Fr.}}
\end{quote}

\textit{``Dieu, quelle virulente sortie!''}\footnote{God, what a vicious attack! \textit{Fr.}} answered the prince as he entered, not put out in the least by this welcome; he was dressed in an embroidered court uniform, stockings, boots, and stars and had a bright expression on his flat face.

He spoke in that refined French in which our ancestors not only spoke, but thought, and with the quiet, patronizing intonations that are particular to a significant person who has spent a long life in society and at court. He approached Anna Pavlovna, kissed her hand, presenting his scented and gleaming bald spot to her, and calmly sat down on the sofa.

``\textit{Avant tout dites moi, comment vous allez, ch\`ere amie?}\footnote{First, tell me everything, how are you doing, dear friend? \textit{Fr.}} Set my mind at ease,'' he said, not changing his voice and tone, in which indifference and even mockery could be heard behind sympathy and politeness. %vasily

``How can one be healthy\ldots{}when one is suffering morally? Can one really remain calm in our time, if one has any feeling?'' Anna Pavlovna said. ``You are staying here the whole evening, I hope?'' %annapavlovna

``But what about the English ambassador's festival? Today is Wednesday. I have to put in an appearance there,'' the prince said. ``My daughter is coming for me and will take me there.'' %vasily

``I thought that today's festival was postponed. \textit{Je vous avoue que toutes ces f\^etes et tous ces feux d'artifice commencent \`a devenir insipides.}''\footnote{I confess that all these parties and all these fireworks are starting to become dull. \textit{Fr.}} %annapavlovna

``If they had known that you wished it, they would have postponed the festival,'' the prince said out of habit, like a wound clock, saying things that he did not expect others to believe. %vasily

\textit{``Ne me tourmentez pas. Eh bien, qu'a-t-on d\'ecid\'e par rapport \`a la d\'ep\^eche de Novosilzoff? Vous savez tout.''}\footnote{Stop tormenting me. Very well, what have they decided about the report on Novosiltsov's dispatch? You know everything. \textit{Fr.}} %annapavlovna

``What can I tell you?'' the prince said in a cold, bored
tone. \textit{``Qu'a-t-on d\'ecid\'e? On a d\'ecid\'e que Buonaparte a br\^ul\'e ses vaisseuax, et je crois que nous sommes en train de br\^uler les notres.''}\footnote{What have they decided? They have decided that Bonaparte has burned his ships, and I believe that we are about to burn ours. \textit{Fr.}} %vasily

Prince Vasily always spoke lazily, like an actor says his lines in an old play. Anna Pavlovna Scherer, on the contrary, despite her forty years, was brimming with liveliness and outburts of energy.

To be an enthusiast had become her place in society, and sometimes, when she did not even want to, she played the part of the enthusiast so as not to disappoint people's expectations. The restrained smile that constantly played on Anna Pavlovna's face, although it did not fit her aging features, expressed a constant awareness, like that of a spoiled child, of her endearing shortcoming, which she found neither desirable or necessary to correct.

In the middle of this conversation about political goings-on, Anna Pavlovna became heated.

``Ah, do not talk to me about Austria! Perhaps I do not understand anything, but Austria has never wanted war and does not want it now. They will betray us. Russia alone must be the savior of Europe. Our benefactor knows his high calling and will be true to it. That is one thing that I trust. The greatest role in the world awaits our kind and marvelous sovereign, and he is so beneficent and good that God will not leave his side, and he will fulfill his calling to crush the hydra of revolution, which is now so horribly present in the person of that villain and murderer. We alone must atone for the blood of that righteous man.\footnote{Anna Pavlovna is referring to the execution of the Duke of Enghien by Napoleon in 1804. --- Trans.} In whom can we place our hope, I ask you\ldots{}? England with its commercial spirit will not rise to the heights of Emperor Aleksandr's soul and never could rise that high. They refused to liberate Malta. They want to wait and see, they are looking for an ulterior motive for our actions. What did they say to Novosiltsov\ldots{}? Nothing. They do not understand, cannot understand the self-sacrifice of our emperor, who does not want anything for himself and does everything for the good of the world. And what have they promised? Nothing. And even what they promised, they will not do! Prussia has also declared that Bonaparte cannot be beaten and that all of Europe can do nothing against him\ldots{} And I do not believe even one word from Hardenburg or Haugwitz. \textit{Cette fameuse neutralit\'e prussienne, ce n'est qu'un pi\`ege.}\footnote{That famous Prussian neutrality, it is nothing but a trap. \textit{Fr.}} I trust in God alone and in the lofty fate of our emperor. He will save Europe\ldots{}!'' She stopped suddenly with a smile, mocking her own hot temper. %annapavlovna

``I think,'' the prince said, smiling, ``that had they sent you in place of our dear Wintzingerode, you would have seized the Prussian king's agreement in a day. You are so eloquent. Could you give me some tea?'' %vasily

``Just a moment. \textit{A propos,}'' she added, calming down again, ``two very interesting people will be here today, \textit{le vicomte de Mortemart, il est alli\'e aux Montmorency par les Rohans,} one of the best families of France.\footnote{The viscount of Mortemart, he is a relative by marriage of the Montmorencys through the Rohans\ldots{} \textit{Fr.}} He is one of the good emigrants, one of the true ones. And then \textit{l'abb\'e Morio:} do you know that deep mind? He has been received by the sovereign. Do you know him?'' %annapavlovna

``Ah! I will be happy to meet him,'' the prince said. ``Tell me,'' he added, as if he had just remembered something, and especially off-handedly at that, when, in fact, the thing he was asking about was the primary reason for his visit, ``is it true that \textit{l'imp\'eratrice-m\`ere} wishes to see Baron Funke named first secretary at Vienna? \textit{C'est un pauvre sire, ce baron, \`a ce qu'il para\^it.}''\footnote{The dowager empress\ldots{} He is a poor excuse for a man, that baron, or so it seems. \textit{Fr.}} Prince Vasily wished to see his son named to the post, which others were trying to obtain for the baron through Empress Maria Feodorovna. %vasily

Anna Pavlovna half-closed her eyes to show that neither she, nor anyone else, could pass judgment about what the empress did or did not like.

\textit{``Monsieur le baron de Funke a \'et\'e recommand\'e \`a l'imp\'eratrice-m\`ere par sa soeur,''} was all she would say, in a dry, sad tone.\footnote{Baron Funke was recommended to the dowager empress by his sister. \textit{Fr.}} When Anna Pavlovna referred to the empress, her face suddenly took on an expression of deep and earnest devotion and respect, joined with sadness, which happened every time she mentioned her lofty patroness in conversation. She told him that Her Majesty had been pleased to show Baron Funke \textit{beaucoup d'estime,} and her eyes again took on a sad look.\footnote{Great respect. \textit{Fr.}} %annapavlovna

The prince fell into an indifferent silence. Anna Pavlovna, with her characteristic courtly and feminine dexterity and quickness of tact, wanted to slap him on the wrist for daring to speak up about a person who had been recommended to the empress, and at the same time to comfort him.

\textit{``Mais \`a propos de votre famille,''} she said, ``did you know that your daughter, ever since she came out, \textit{fait les d\'elices de tout le monde. On la trouve belle, comme le jour.}''\footnote{But about your family\ldots{} [She] has become the delight of the whole world. People find her to be as beautiful as the light of day. \textit{Fr.}} %annapavlovna

The prince bowed to show his respect and appreciation.

``I often think,'' Anna Pavlovna continued after a moment's silence, moving closer to the prince and smiling at him affectionately, as if to show him that talk of politics and society was over and now an intimate conversation was beginning: ``I often think about how happiness in life is sometimes distributed unfairly. Why did fate give you two wonderful children---excluding Anatoly, your youngest, him I do not love,'' she put in peremptorily, raising her brows, ``such charming children? And indeed you value them less than anyone and for that reason are not worthy of them.'' %annapavlovna

And she smiled her ecstatic smile.

\textit{``Que voulez-vous? Lafater aurait dit que je n'ai pas la bosse de la paternit\'e,''} the prince said.\footnote{What would you have me do? Lavater would say that I do not have the bump of paternity. \textit{Fr.}} %vasily

``Stop making jokes. I wanted to speak with you seriously. Do you know, I am displeased with your youngest son. Let me say, just between us,'' her face took on a sad expression, ``they were talking about him before Her Highness and people pity you\ldots{}'' %annapavlovna

The prince did not answer, but she looked at him significantly, saying nothing and waiting for an answer. Prince Vasily frowned.

``What am I supposed to do?'' he said finally. ``You know that I have done everything for their upbringing that a father can do, and both of them turned out \textit{des imb\'eciles.}\footnote{Idiots. \textit{Fr.}} Ippolit is a fool at rest, but Anatoly is restless. That is the only difference,'' he said, smiling even more unnaturally and animatedly than usual, and at the same time showing very openly an aspect that was unexpectedly coarse and unpleasant in the wrinkles around his mouth. %vasily

``And why are children born to people like you? If you were not a father, I would be able to find no fault in you,'' Anna Pavlovna said, raising her eyes pensively. %annapavlovna

\textit{``Je suis votre} devoted slave, \textit{et \`a vous seule je puis l'avouer.} My children---\textit{ce sont les entraves de mon existence.} And my cross. That is how I explain it to myself. \textit{Que voulez vous\ldots{}?''}\footnote{I am your devoted slave, and I will confess it to you alone. My children are the bane of my existence\ldots{} What would you have me do? \textit{Fr.}} He fell silent, showing with a gesture his submission to cruel fate.

Anna Pavlovna was pensive.

``You have not thought to look for a wife for your prodigal son Anatoly. They say,'' she said, ``that old maids \textit{ont la manie des mariages.} I do not yet feel that weakness in myself, but I do have a \textit{petite personne} who is unhappy with her father, \textit{une parente \`a nous, une princesse} Bolkonskaya.''\footnote{\ldots{}have an obsession with making marriages. \ldots{}Little person\ldots{}related to us, a princess Bolkonskaya. \textit{Fr.}} Prince Vasily did not answer, although with a quickness of understanding and memory characteristic of society people, he showed with a movement of his head that he had taken in all this information. %annapavlovna

``No, did you know that that Anatoly costs me 40,000 a year,'' he said, obviously unable to hold back the aggrieved course of his thoughts. He was silent. %vasily

``How will things be in five years if this continues? \textit{Voil\`a l'avantage d'\^etre p\`ere.} Is she rich, this princess?''\footnote{The benefits of being a father. \textit{Fr.}}

``The father is very rich and stingy. He lives in the country. Do you know, it is the famous Prince Bolkonsky, who was dismissed from service under the late emperor and was called the ``Prussian king.'' He is a very intelligent man, but severe and idiosyncratic. \textit{La pauvre petite est malheureuse, comme les pierres.} She has a brother, the man who just married Lise Meinen, one of Kutuzov's adjutants. He will be here tonight.''\footnote{The poor little girl is as sad as a dog with no home. \textit{Fr.}} %annapavlovna

\textit{``Ecoutez, ch\`ere Annette,''} the prince said, suddenly taking her by the hand and pulling it downward for some reason. ``\textit{Arrangez-moi cette affaire et je suis votre} most devoted slave \textit{\`a tout jamais}---slafe \textit{comme mon} foreman \textit{m'ecrit des} dispatches: ay-eff-ee. She is from a good family and rich. That is all that I need.''\footnote{Listen, dear Annette. Arrange this matter for me and I will be your most devoted slave forever---slafe as my foreman writes to me in his dispatches. \textit{Fr.}} %vasily

And with the free and familiar, graceful movements that set him apart, he took the maid of honor by her hand, kissed it, and, having kissed it, waved the maid of honor's hand around, falling back against his armchair and looking to the side.

\textit{``Attendez,''} Anna Pavlovna said, weighing the
matter. ``Tonight I will say something to Lise (\textit{la femme du jeune} Bolkonsky). And maybe it will be settled. \textit{Ce sera dans votre famille, que je ferai mon apprentissage de vieille fille.}''\footnote{Listen. \ldots{}the young Bolkonsky's wife\ldots{} It will be with your family that I undergo my apprenticeship to become an old maid. \textit{Fr.}}

\section*{II} % Book One, Part One, Chapter Two

Anna Pavlovna's drawing room was slowly beginning to fill up. The highest nobility of Petersburg was arriving, people of the most diverse ages and characters, but identical in the circle of society in which they all lived; Prince Vasily's daughter arrived, the great beauty H\'el\`ene, who had come to collect her father and go with him to the ambassador's festival. She was wearing a ball gown and the monogram that marked her as a maid of honor. Arriving at the same time was the young little Princess Bolkonskaya, well-known as \textit{la femme la plus s\'eduisante de P\'etersbourg,} who had married the previous winter and no longer went out in `great' society due to the fact that she was pregnant, but who still went to smaller parties.\footnote{The most attractive woman in Petersburg. \textit{Fr.}} Prince Ippolit, the son of Prince Vasily, arrived with Mortemart, whom he presented to everyone; Abbot Morio and many others arrived as well.

``Have you seen \textit{ma tante} yet?'' or ``Do you know her?'' Anna Pavlovna said to guests as they arrived and very earnestly brought them over to a little old woman covered with large bows, who had surfaced from another room the moment that guests began to arrive.\footnote{My aunt. \textit{Fr.}} She introduced all the guests to her by name, slowly turning her eyes from them to \textit{ma tante,} and then left. %annapavlovna

All the guests performed this rite of greeting with the unknown, uninteresting, and unnecessary aunt. Anna Pavlovna followed their greetings with sad, solemn concern, silently endorsing them from afar. \textit{Ma tante} spoke with each one in the very same expressions about their health, about her own health, and about Her Majesty's health, which was now better, thank God. All those who were brought to her, trying not to show their haste out of politeness, finally walked away from her with a feeling of relief at having fulfilled a heavy obligation and never returned to her again the whole evening.

The young Princess Bolkonskaya arrived with her work in a velvet bag with gold embroidery. Her pretty upper lip, which was covered in fine black hairs, did not quite cover her teeth, but it was endearing when her lips parted, and was still more endearing when the upper one stretched to meet the lower. As is often the case with incredibly attractive women, her defects---the shortness of her lip and her half-open mouth---seemed to be the essence of her own beauty. Everyone enjoyed watching that pretty mother-to-be, full of health and liveliness, who was enduring her condition so well. Old men and bored, gloomy young men felt that they were becoming more like her from sitting and talking with her a little while. Whoever talked with her and saw her bright smile and white teeth gleaming at every word, which were constantly visible, thought that they must be particularly charming just then. And every one of them thought just that.

The little princess got up, went around the table with quick little steps, her little work bag in her hand, and, happily fixing her dress, she sat down on the sofa near the silver samovar, as if everything she did was \textit{partie de plaisir} for herself and everyone around her.\footnote{The purest pleasure. \textit{Fr.}} 

\textit{``J'ai apport\'e mon ouvrage,''} she said, opening her handbag and addressing everyone at once.\footnote{I have brought my work. \textit{Fr.}} %lise

``See here, Annette, \textit{ne me jouez pas un mauvais tour,}'' she said, addressing the hostess. \textit{``Vous m'avez \'ecrit, que c'\'etait une toute petite soir\'ee; voyez, comme je suis attif\'ee.''}\footnote{I hope you are not playing a trick on me. You wrote to me that this party would be quite small; look at how I am dressed. \textit{Fr.}} %lise

And she spread her arms to show her elegant grey dress, all in lace, tied with a broad ribbon just below her chest.

\textit{``Soyez tranquille, Lise, vous serez toujours la plus jolie,''} Anna Pavlovna answered.\footnote{Rest assured, Lise, you will always be the prettiest. \textit{Fr.}} %annapavlovna

\textit{``Vous savez, mon mari m'abandonne,''} she continued in the same tone, turning to a general. \textit{``Il va se faire tuer. Dites moi, pourqoui cette vilaine guerre,''} she said to Prince Vasily and, not waiting for an answer, turned to Prince Vasily's daughter, the beautiful H\'el\`ene.\footnote{You know, my husband is abandoning me. He is going to get himself killed. Tell me, why do we have this nasty war? \textit{Fr.}} %lise

\textit{``Quelle d\'elicieuse personne, que cette petite princesse!''}~Prince Vasily said quietly to Anna Pavlovna.\footnote{What a delightful person, this little princess! \textit{Fr.}} %vasily

Right behind the little princess, a massive, fat young man entered, his hair cut short, wearing glasses, light-colored trousers in the latest fashion, a high jabot, and a brown tail-coat. This fat young man was the illegitimate son of the famous Elizabethan-era magnate, Count Bezukhov, who was then dying in Moscow. He had still not entered service anywhere and had just arrived from abroad, where he had been educated, and was appearing in society for the first time. Anna Pavlovna greeted him with a bow she reserved for the lowest level of the hierarchy in her drawing room. But despite this low type of greeting, at the sight of Pierre entering the room, Anna Pavlovna's face showed anxiety and fear similar to what one expresses at the sight of something unnatural and much too huge. Although Pierre was indeed somewhat larger than the other men in the room, this fear could only have been connected with the intelligent but timid, searching and natural glance that distinguished him from everyone else in the drawing room.

\textit{``C'est bien aimable \`a vous, \emph{monsieur Pierre,} d'\^etre venu voir une pauvre malade,''} Anna Pavlovna said to him, exchanging fearful glances with her aunt, to whom she was leading him.\footnote{It is very kind of you, \emph{Mr.~Pierre,} to have come to see a poor sick woman. \textit{Fr.}} Pierre muttered something incomprehensible, his eyes continuing to search for something. He smiled happily, joyfully, bowing to the little princess as to a close acquaintance, and approached the aunt. Anna Pavlovna's fear was not in vain, because Pierre, not listening to the aunt's speech about Her Highness's health all the way through, walked away from her. Anna Pavlovna, fearful, spoke to stop him: %annapavlovna

``Do you know Abbot Morio? He is a very interesting person\ldots{}'' she said. %annapavlovna

``Yes, I've heard of his plan for eternal peace, and it is very interesting, but hardly possible\ldots{}'' %pierre

``Do you think so\ldots{}?'' Anna Pavlovna said, merely in order to have something to say, and turned again to her activities as hostess, but Pierre was now impolite in reverse. Before, he had walked away without listening to what his conversation partner had had to say; now, he stopped his partner when she needed to walk away. Head bent and long legs spread wide, he began to lay out for Anna Pavlovna why he considered the abbot's plan a chimera. %annapavlovna

``We can talk later,'' Anna Pavlovna said, smiling. %annapavlovna

And, ridding herself of the young man, who did not know how to conduct himself, she returned to her activities as hostess and continued to listen in and keep an eye on things, ready to help out at just the place where the conversation was flagging. Like the owner of a cotton mill who sets his workers in their places, walks around the establishment, noting a stopped machine or a spindle that is making an unusually loud screeching noise, moving hastily and stopping the machine or putting it on the right track---so Anna Pavlovna, walking around her drawing room, went up to a circle that had fallen silent or was talking too loudly and, with a single word or change of seating, again set the pleasant talking-machine in regular motion. But in the midst of all these cares, one could still see in her a particular fear about Pierre. She glanced at him cautiously when he went up to hear what was being said around Mortemart and then stepped away to the other circle, where the abbot was speaking. For Pierre, educated abroad, this party at Anna Pavlovna's was the first he had seen in Russia. He knew that all the intelligentsia of Petersburg was gathered here, and his head was spinning, like a child's in a toy store. He was afraid of missing anything of the intelligent conversations he could hear. Looking at the assured and refined expressions of the faces gathered here, he was waiting to hear something especially intelligent. Finally, he went up to Morio. The conversation seemed interesting to him, and he stopped, waiting for the chance to express his thoughts, as young people love to do.

\section*{III} % Book One, Part One, Chapter Three

Anna Pavlovna's party had been set in motion. The spindles ran precisely everywhere, never falling silent. Except for \textit{ma tante} --- who sat next to an elderly woman with a gaunt, tear-stained face who was somewhat foreign to the shining company assembled there --- the company had divided itself into three circles. In the first, more masculine one, the center was the abbot; in the next, younger one were the beautiful Princess H\'el\`ene, Prince Vasily's daughter, and the pretty, pink-faced Princess Bolkonskaya, too plump for her young age. In the third were Mortemart and Anna Pavlovna.

The viscount was nice-looking, with soft features and mannerisms, a young man, who obviously considered himself a person of some renown, but who, due to good breeding, humbly permitted himself to be enjoyed by the society in which he found himself. Anna Pavlovna was obviously treating her guests to him. Just as a good ma\^itre d'h\^otel can present as extraordinarily superb a piece of beef that no one would want to eat if it were seen in the dirty kitchen, Anna Pavlovna that evening served up first the viscount, then the abbot, to her guests as something extraordinarily refined. In Mortemart's circle, everyone immediately began talking about the killing of the Duke of Enghien. The viscount said that the Duke of Enghien was done in by his generosity of spirit, and that there had been particular reasons for Bonaparte's animosity.

\textit{``Ah!~voyons. Contez-nous, vicomte,''} Anna Pavlovna said, feeling with happiness that her phrase resounded with something \textit{\`a la Louis XV, ``contez-nous cela, vicomte.''}\footnote{Ah, yes! Tell us the story, viscount[\ldots{}]tell it to us, viscount. \textit{Fr.}} %annapavlovna

The viscount bowed to show his obedience and smiled politely. Anna Pavlovna formed a circle around the viscount and invited everyone to listen to his story.

\textit{``Le vicomte a \'et\'e personnellement connu de monseigneur,''} Anna Pavlovna whispered to one person. \textit{``Le vicomte est un parfait conteur,''} she said to another. \textit{``Comme on voit l'homme de la bonne compagnie,''} she said to a third; and the viscount was presented to the company in the most elegant and favorable light, like roast beef on a hot platter, garnished with greens.\footnote{The viscount was personally acquainted with the duke. The viscount is perfect at telling stories. Now you see what a man of good society looks like. \textit{Fr.}} %annapavlovna

The viscount was ready to begin his story and smiled thinly.

``Come this way, \textit{ch\`ere H\'el\`ene,}'' Anna Pavlovna said to the beautiful princess, who was sitting a little way off, forming the center of a different circle. %anapavlovna

Princess H\'el\`ene smiled; she stood up with the same unchanging smile she had had on when she entered the drawing room, the smile of a perfectly beautiful woman. With a slight rustle of her white ball gown, decorated in ivy and moss, her shoulders gleaming white and her hair and diamonds brilliant, she passed through the crowd of men, which parted, smiling at all of them but looking at no one, as if granting each one the right to admire the beauty of her figure, her round shoulders, her very exposed chest and back (following the fashion of the day), and seeming to carry with her all the luster of a ball, she went over to Anna Pavlovna. H\'el\`ene was so pretty that, not only did she not display even a shadow of coquetry, but she seemed to be ashamed of her unquestionable, overpowering, all-conquering beauty. She seemed to want to diminish the power of her beauty but could not.

\textit{``Quelle belle personne!''} said everyone who saw her.\footnote{What a beautiful woman! \textit{Fr.}} As if struck by something supernatural, the viscount shrugged his shoulders and lowered his eyes while she sat in front of him and turned that same changeless, illuminating smile on him as well.

\textit{``Madame, je crains pour mes moyens devant un pareil auditoire,''} he said, inclining his head with a smile.\footnote{Madame, an audience like this makes me uncertain of my abilities. \textit{Fr.}} %viscount

The princess leaned her bare, plump arm on a little table and did not find it necessary to say anything. She waited, smiling. During the entire story, she sat straight, looking from time to time at her plump, beautiful arm resting lightly on the table, or at her even more beautiful chest, adjusting her diamond necklace; several times, she adjusted the folds of her dress and, when the story made an impression on her, she looked over at Anna Pavlovna and took on the same expression that was on the maid of honor's face, and then returned to her beaming smile as she became calm again. The little princess, Lise, followed H\'el\`ene from the tea table.

\textit{``Attendez moi, je vais prendre mon ouvrage,''} she said. \textit{``Voyons, \`a quoi pensez-vous?''} she said, turning to Prince Ippolit, \textit{``apportez-moi mon ridicule.''}\footnote{Wait for me, I'm going to bring my work. Come now, what are you thinking about? Carry my handbag to me. \textit{Fr.}} %lise

The little princess, smiling and talking with everyone, suddenly repositioned herself and, having settled in her seat, smoothed her dress happily.

``I'm comfortable now,'' she said and, begging the viscount to begin, took up her work. %lise

Prince Ippolit brought her handbag to her, followed after her, and moved a chair close to her, sitting down next to her.

The resemblance between \textit{le charmant Hippolyte} and his beautiful sister was striking, all the more so because, despite that resemblance, he was strikingly ugly. The features of his face were the same as his sister's, but with her, everything was lit with that vivacious, self-satisfied, young, unchanging smile and the supernatural, classical beauty of her body; with her brother, on the other hand, the same face was clouded by imbecility and unchangingly expressed self-assured disgust, and his body was thin and weak. His eyes, nose, and mouth were all compressed into a vague, bored-looking grimace, and he always held his arms and legs in unnatural positions.

\textit{``Ce n'est pas une histoire de revenants?''} he said, having seated himself next to the little princess and hurriedly brought his spectacles up to his eyes, as if he could not begin speaking without the help of that instrument.\footnote{Is this a ghost story? \textit{Fr.}} %ippolit

\textit{``Mais non, mon cher,''} said the surprised story-teller, shrugging his shoulders.\footnote{Why no, my dear. \textit{Fr.}} %viscount

\textit{``C'est que je d\'eteste les histoires de revenants,''} Prince Ippolit said, in a tone that clearly showed he had said the words, and only afterward understood what they meant.\footnote{It's just that I hate ghost stories. \textit{Fr.}} %ippolit

Because of the self-assurance with which he spoke, no one could understand whether what he said was very smart or very stupid. He was in a dark-green tail-coat, in pants the color of \textit{cuisse de nymphe effray\'ee,} as he himself said, and in stockings and low boots.\footnote{A frightened nymph's thigh. \textit{Fr.}}

The \textit{vicomte} recounted very nicely a then-current anecdote about how the Duke of Enghien went to Paris in secret to see Mlle.~George, and how he met Bonaparte, who had also been enjoying the favors of the famous actress, and how, having met the duke there, Napoleon by chance had a fainting spell, which he was prone to do, and found himself in the duke's power, which the duke did not advantage of, but how Bonaparte subsequently repaid the duke's magnanimity by killing him.

The story was very nice and interesting, especially the part where the rivals suddenly recognize one another, and the ladies seemed to be very excited.

\textit{``Charmant,''} Anna Pavlvona said, looking at the little princess questioningly. %annapavlovna

\textit{``Charmant,''} the little princess whispered, sticking her needle in her work, as if to show that the charm and interest of the story prevented her from continuing her work. %lise

The viscount appreciated this unspoken praise and, smiling gratefully, started to continue; but at the same time, Anna Pavlovna, who had been keeping an eye on the frightening (to her) young man across the room, noticed that he was saying something to the abbot too loudly and heatedly, and she hurried toward the danger to help. Indeed, Pierre had managed to tie the abbot up in conversation about political equilibrium, and the abbot, apparently intrigued by the young man's simple-hearted fervor, was laying out his beloved idea for him. Both of them were too animated and were speaking and listening very naturally, and this Anna Pavlovna did not like.

``The means are European equilibrium and \textit{droit des gens,}'' the abbot said.\footnote{Law of nations. \textit{Fr.}} ``Only that one mighty government is required, like Russia, famed for its barbarism, that is willing to become, unselfishly, the head of a union with equilibrium in Europe as its goal, and it will save the world!'' %abbot

``Just how will you find that equilibrium?'' Pierre was beginning to say; but at the same time, Anna Pavlovna approached them and, looking severely at Pierre, asked the Italian how he was managing the local climate. The Italian's face suddenly changed and took on an offensively false, sweet-looking expression, which was apparently habitual for him in his conversations with women.

``I am so charmed by the delights of the education and intellect of the company in which I have the pleasure of being received, especially the feminine company, that I have not managed to think about the climate yet,'' he said. %abbot

No longer willing to leave the abbot and Pierre to themselves, Anna Pavlovna united them with the larger circle in order to supervise them more easily.

At this time, a new person appeared in the drawing room. The new person was the young Prince Andrei Bolkonsky, the husband of the little princess. Prince Bolkonsky was a very attractive young man of medium height with clearcut, dry features. Everything in his figure, starting from his tired, bored glance to his quiet, measured step, presented the starkest possible contrast with his young, vivacious wife. He was apparently not only acquainted with everyone in the drawing room, but already fed up with them, so much so that he was completely bored by having to look at them and listen to them. Of all the tiresome faces there, he seemed to be most fed up with seeing the face of his pretty wife. He turned away from her with a grimace that ruined his handsome face. He kissed Anna Pavlovna's hand and, squinting, looked over the entire company.

\textit{``Vous vous enr\^olez pour la guerre, mon prince?''} Anna Pavlovna said.\footnote{Are you signing up for the war, Prince? \textit{Fr.}} %annapavlovna

\textit{``Le g\'en\'eral Koutouzoff,''} Bolkonsky said, emphasizing the final syllable, \textit{zoff,} like a Frenchman, \textit{``a bien voulu de moi pour aide-de-camp\ldots''}\footnote{General Kutuzov has been pleased to make me one of his aides-de-camp\ldots{} \textit{Fr.}} %andrei

\textit{``Et Lise, votre femme?''}\footnote{And Lise, your wife? \textit{Fr.}} %annapavlovna

``She is going to the country.'' %andrei

``Aren't you ashamed of yourself for taking your delightful wife away from us?'' %annapavlovna

\textit{``Andr\'e,''} his wife said, speaking to her husband in the same coquettish tone with which she addressed strangers, ``what a story the viscount was just telling us about Mlle.~George and Bonaparte!'' %lise

Prince Andrei frowned and turned away. Pierre, who had not taken his friendly, joyful gaze off Prince Andrei since his entrance into the drawing room, approached him and took him by the arm. Prince Andrei, not looking around, grimaced, expressing his annoyance at whomever was grabbing him by the arm, but, having seen Pierre's smiling face, he smiled with unexpected kindness and pleasure.

``Well then! Here you are in great society!'' he said to Pierre. %andrei

``I knew that you would be here,'' Pierre answered. ``I'm coming over for dinner,'' he added quietly so as not to interrupt the viscount, who was continuing his story. ``Can I?'' %pierre

``No, absolutely not,'' Prince Andrei said, laughing, letting Pierre know by a squeeze of the hand that he did not even need to ask. He wanted to say something else, but at the same time, Prince Vasily and his daughter got up, and all the men stood to make way for them. %andrei

``Please forgive me, my dear viscount,'' Prince Vasily said to the Frenchman, gently pulling him back down into his chair by his sleeve to keep him from standing. ``This unfortunate festival at the ambassador's forces me to interrupt you and deny myself the pleasure of staying. I am very sad to leave your lovely party,'' he said to Anna Pavlovna. %vasily

His daughter, Princess H\'el\`ene, walked between the chairs, lightly holding the folds of her dress, and her smile shone even brighter on her beautiful face. Pierre watched with ecstatic, almost terrified eyes as the great beauty walked past him.

``Very beautiful,'' Prince Andrei said. %andrei

``Very,'' Pierre said. %pierre

Walking past, Prince Vasily grabbed Pierre by the arm and turned to Anna Pavlovna.

``Give this bear some culture for me,'' he said. ``Here he has been living with me for a month, and this is the first time I've seen him in society. Nothing is so necessary for a young man as the company of intelligent women.'' %vasily

\section*{IV} % Book One, Part One, Chapter Four

Anna Pavlovna smiled and promised to work with Pierre, who, she knew, was related to Prince Vasily through his father. The elderly woman, who earlier had been sitting with \textit{ma tante,} stood up hurriedly and caught Prince Vasily in the front room. All her earlier feigned interest had disappeared from her face. Her kind, tear-stained face expressed only anxiety and fear.

``What can you tell me, Prince, about my Boris?'' she said, catching him in the front room. (She spoke the name Boris with particular emphasis on the \textit{o}.) ``I cannot remain any longer in Petersburg. Tell me, what news can I bring to my poor little boy?'' %annamikhailovna

Despite the fact that Prince Vasily was listening reluctantly and almost rudely to the elderly woman and even showed impatience, she smiled at him affectionately and touchingly and took him by the hand to keep him from leaving.

``What would it cost you to say a word to the sovereign? Then he will be transferred straight to the Guards,'' she asked him. %annamikhailovna

``Trust me, I will do everything that I can, Princess,'' Prince Vasily answered, ``but it is difficult for me to ask the sovereign; I would recommend you appeal to Rumyantsev, through Prince Golytsin: that would be smarter.'' %vasily

The elderly woman carried the name of Princess Drubetskoy, the name of one of the best families of Russia, but she was poor, had long ago left society, and had lost her former connections. She had come that night to obtain a commission in the Guards for her only son. She had invited herself to the party and come to Anna Pavlovna's only to see Prince Vasily, only for that reason had she listened to the viscount's story. She was startled by Prince Vasily's words; her once-beautiful face showed bitterness, but that lasted only for a moment. She smiled again and grasped Prince Vasily more firmly by the hand.

``Listen, Prince,'' she said, ``I have never asked you for anything, will never ask you again, and have never brought up my father's friendship for you. But now, by God, I implore you, do this for my son, and I will consider you our benefactor,'' she added hurriedly. ``No, do not get angry, just promise me. I asked Golytsin, he refused. \textit{Soyez le bon enfant que vous avez \'et\'e,}'' she said, trying to smile when there were tears in her eyes.\footnote{Be the good boy you once were. \textit{Fr.}} %annamikhailovna

``Papa, we're going to be late,'' Princess H\'l\`ene said, turning her pretty head on her classical shoulders, waiting by the door. %helene

But influence in society is capital, which needs to be protected, or it will disappear. Prince Vasily knew this and, once he had understood that if he started asking for favors whenever anyone asked him, he would soon be unable to ask for himself, he rarely used his influence. On the matter of Princess Drubetskoy, however, he felt something like the prick of conscience after her new appeal: he was indebted to her father for his first steps in the civil service. Furthermore, he saw by her mannerisms that she was one of those women, mothers especially, who, once they had gotten something into their heads, would not leave it alone until their desires were realized, and, in the opposite case, were ready to harass and annoy every day, every moment, and even make scenes. Understanding this last part swayed him.

``\textit{Ch\`ere} Anna Mikhailovna,'' he said with the familiarity and boredom that was eternally in his voice, ``it is almost impossible for me to do what you want; but in order to prove to you how much I love you, and how much I honor the memory of your late father, I will do the impossible: your son will be transferred to the Guards, here is my hand. Are you satisfied?'' %vasily

``My dear, you are our benefactor! I did not expect anything else; I knew you were kind.'' %annamikhailovna

He tried to leave.

``Wait, just two words. \textit{Une fois pass\'e aux gardes\ldots{}}''\footnote{Once he has been transferred to the Guards\ldots{} \textit{Fr.}} She faltered. ``You are in well with Mikhail Ilarionovich Kutuzov, recommend Boris to be one of his adjutants. Then I could rest happy, and then\ldots{}'' %annamikhailovna

Prince Vasily smiled.

``That I will not promise. You know how besieged Kutuzov is since he was named commander-in-chief. He said to me himself that all the great ladies of Moscow are scheming to make all their children his adjutants.'' %vasily

``No, promise me, I will not let you go, my dear benefactor.''

``Pap\`a,'' the beauty repeated in the same tone, ``we're going to be late.'' %helene

``Well, \textit{au revoir,} farewell. You see?'' %vasily

``Then tomorrow you will report to the sovereign about this?'' %annamikhailovna

``Certainly, but I make no promises about Kutuzov.'' %vasily

``No, promise me, promise me, \textit{Basile,}'' Anna Mikhailovna said to his back, with the smile of a young coquette, something that must have been natural for her once upon a time, but which did not suit her venerable face.

She had apparently forgotten her years and had set in motion all her old, girlish methods, out of habit. But as soon as he left, her face again took on the same cold, feigned expression she had worn earlier. She turned back to the circle where the viscount was continuing his story, and she again pretended that she was listening, waiting until it was time to leave, since her business was now finished.

